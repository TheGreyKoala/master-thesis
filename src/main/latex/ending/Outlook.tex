\section{Ausblick}
    \label{section:endingOutlook}
    Das \gls{wccs} bietet sehr viel Potenzial um ausgebaut zu werden.

    Ein Erweiterung, von der die \gls{wccdl} profitieren würden,
    ist die Einführung eines Vererbungskonzeptes für Klassen,
    um Features und Selektoren wiederzuverwenden und die Ausdrucksstärke
    der Sprache damit zu erhöhen.
    Hierzu muss ein konsistentes Konzept für Vererbung und Polymorphie
    von Features und Selektoren geschaffen werden,
    welches sich syntaktisch gut einfügt.
    Außerdem erfordert es weitergehende semantische Prüfungen.

    Wichtig aber auch herausfordernd ist außerdem die Einbettung
    von Sprachen zur Unterstützung von CSS, XPath und regulären Ausdrücken,
    sodass die Definition von Selektoren syntaktisch und semantisch geprüft werden kann
    und ihre Formulierung durch die Entwicklungsumgebung unterstützt wird.

    Der Entwickler könnte außerdem durch ein Browser-Plugin
    deutlich bei der Formulierung von Selektoren unterstützt werden.
    Nämlich dann, wenn dieses Plugin den Browser und die Entwicklungsumgebung
    dahingehend verbindet, dass der Browser solche Elemente auf der aktuellen Seite
    hervorhebt, die durch einen bestimmten Selektor im Quelltext erfasst werden.
    Andersherum könnte der Entwickler den Browser nutzen,
    um einen Selektor eines \gls{html}-Elementes zu bestimmen und
    in die Entwicklungsumgebung zu übertragen.

    Eine Möglichkeit die beschriebenen Sonderfälle besser abzudecken ist die
    Einführung eines Skript-Selektors.
    Das heißt zur Auswahl eines HTMl-Knotens ein JavaScript Fragment ausgeführt wird,
    was prinzipiell dank der Browser-Automatisierung von Puppeteer möglich sein sollte.
    Programmatisch sollten sich auch deutlich komplexere Strukturen einfangen lassen.

    Eine weitere Maßnahme zu diesem Zweck ist die Einführung von Selektor-Ketten.
    Das heißt ein Feature wird durch eine Reihe von Selektoren
    ermittelt, wobei das Ergebnis eines Selektors als Kontext Knoten des nächsten fungiert.
    Dadurch lässt sich der Suchraum mit verschiedenen Selektor-Typen eingrenzen,
    ohne zusätzliche Features definieren zu müssen.

    In Kapitel \ref{section:discussionComparisonClassificationSystem}
    wurde das Thema maschinelles Lernen bereits kurz aufgegriffen.
    Im \gls{wccs} sind zwei Anwendungen dieser Technik denkbar:

    \begin{enumerate}
        \item   Das System wird anhand einiger Klassifikationen trainiert
                und versucht anschließend ohne exakte Vorgaben (Selektoren),
                sondern nur auf Basis des Trainings,
                neue Seiten zu erkennen und zu strukturieren.
        \item   Das System analysiert Korrekturen, die durch Nutzer durchgeführt werden
                und erzeugt Vorschläge für Änderungen in weiteren Klassifikationen.
    \end{enumerate}

    Darüber hinaus sind einige weitere kleinere Erweiterungen denkbar,
    die das Werkzeug verbessern.

    Der Nutzen des Annotator Plugins kann zum Beispiel deutlich erhöht werden,
    indem das Löschen und Anlegen von Annotationen unterstützt wird.
    Außerdem wäre es denkbar neben den vordefinierten Klassen auch
    die Angabe eigener neuer Klassen zu erlauben.

    Ein Grund der schwachen Ausdrucksstärke der Sprache ist die Notwendigkeit
    jede Klasse einzeln bekannt zu machen.

    Zur Steigerung der Ausdrucksstärke der Sprache wäre es außerdem denkbar
    einmalig genutzte Klassen ohne eigene Features nicht separat,
    sondern zusammen mit ihrer Verwendung definieren zu können.
    Das könnte zum Beispiel folgendermaßen aussehen:

    \begin{lstlisting}[style=wccdl,language=wccdl,inputencoding=utf8/latin1]
classifies myFeature as content class MyClass by css « p »
    \end{lstlisting}
