\section{Fazit}
    Die \gls{wccdl} ist eine gut lesbare und einfach zu erlernende Sprache,
    die sich im Sinne von domänenspezifischen Sprachen
    auf wenige Anwendungsfälle beschränkt.
    Ein besonderes Merkmal ist ihre ausführliche Syntax,
    wodurch Modelle zwar länger als unbedingt erforderlich sind,
    dafür aber fast wie Fließtext gelesen werden können.
    Mit der \gls{wccdl} lassen sich viele Strukturen auf Webseiten beschreiben.
    Einschränkungen sind hauptsächlich durch die implementierten Selektoren begründet.
    Ihre Syntax kann aber nahtlos erweitert werden, um neue Selektoren zu unterstützen.
    Generierte Modelle können auch unabhängig vom \gls{wccs} weiterverwendet werden,
    wodurch sie prinzipiell auch für andere Anwendungsfälle nutzbar ist.
    Insgesamt erscheint die \gls{wccdl} besser geeignet als andere Sprachen
    zur Beschreibung eines {\classificationModel}s für das \gls{wccs}.

    Mit dem Klassifizierungsansatz des \glspl{wccs} konnte gezeigt werden,
    dass Inhalte einer Webseite durch Software automatisch fachlich strukturiert werden können.
    Diese Klassifizierung auf Basis der \gls{html}-Repräsentation einer Webseite durchzuführen,
    macht dies oftmals sehr einfach, ist mit den verfügbaren Selektoren
    in manchen Sonderfällen aber nicht ausreichend.
    Die Idee der Selektoren ist kein revolutionärer neuer Ansatz,
    in diesem Fall aber dennoch eine ungewöhnliche Anwendung z. B. von {\cssSelector}en.
    Ihre praktische Anwendung kann schnell eine komplexe Herausforderung werden,
    wenn viele atomare Informationen innerhalb eines \gls{html}-Elementes getrennt werden müssen.
    D. h., wenn die Inhalte durch die \gls{html}-Repräsentation nur schwach strukturiert werden.
    Außerdem ist es schwierig Varianz in die Selektoren einzubauen,
    sodass sie auf vielen ähnlichen Webseiten richtig greifen,
    ohne die genaue Struktur jeder Seite in Betracht ziehen zu müssen.

    Insgesamt stellt das Ergebnis der Arbeit trotzdem ein funktionales System dar,
    welches im vorgesehenen Anwendungsfall gut funktioniert
    und Potenzial für weitere Arbeiten bietet.