\section{Zusammenfassung}
    Das Ziel dieser Arbeit ist die Konzeption und Entwicklung
    eines Systems zur automatischen Klassifizierung der Inhalte
    von Webseiten, was unter anderem eine domänenspezifische
    Sprache zur Instrumentierung des Systems umfasst.
    Der Titel dieses Systems ist
    "`Webpage Content Classification System"' (WCCS).
    Motiviert ist dieses Ziel durch die Modernisierung des
    Internetauftrittes der {\fernUni},
    die eine Migration von Inhalten von {\wordpress} zu
    {\imperia} beinhaltet.
    Die Herausforderung dabei ist die sehr schwache Strukturierung
    der Inhalte, wodurch sie nicht ohne Weiteres in
    Dokumente in {\imperia} transformiert werden können.
    Mithilfe des \glspl{wccs} sollen die Inhalte in einer vorbereitenden
    Maßnahme strukturiert werden, um diesem Problem zu begegnen.

    Eine wichtige Komponente des \glspl{wccs} ist eine auf Xtext basierende
    domänenspezifische Sprache namens "`Webpage Content Modeling Language"' (WCML).
    Diese erlaubt die Definition von fachlichen Seiten-, Inhalts- und
    Referenzklassen, in die Webseiten, Inhalte und Referenzen eingeordnet werden.
    Definierbare Features dieser Klassen erlauben die Schaffung feingranularer
    und hierarchischer Klassifikationen.
    Zur Einordnung von Inhalten in Klassen verwendet das System
    CSS-, XPath- und reguläre Ausdrücke, die in diesem Kontext als
    Selektoren bezeichnet werden und
    ebenfalls Teil der Klassendefinitionen sind.
    Wichtige Merkmale der \gls{wccdl} sind ihre deklarative Natur
    und ihre Lesbarkeit,
    die durch die klare Fokussierung auf eine kleine Domäne ermöglicht wird.
    Programme in dieser Sprache werden zu technischen
    Konfigurationsdateien für das Klassifizierungssystem übersetzt.

    Das \gls{wccs} klassifiziert eine Webseite auf Basis ihrer
    \gls{html}-Repräsentation und verwendet einen Webbrowser zur Auswertung der Selektoren.
    Klassifikationen können durch weitere Komponenten des \glspl{wccs}
    als Webannotationen auf der klassifizierten Webseite visualisiert werden,
    was einer Sichtung, ersten Prüfung und Durchführung kleinerer
    Korrekturen dient.

    Das \gls{wccs} speichert Klassifikationen in einer Graphdatenbank,
    wodurch die Referenzen zwischen Webseiten und anderen {\resources}
    im \gls{www} leicht abzubilden sind.
    Darüber hinaus bietet eine Graphdatenbank Vorteile bei der Modellierung,
    Speicherung und Abfrage von Klassifikationen.
    Durch die Möglichkeit Knoten des Graphen in mehreren Klassifikationen
    zu verwenden, entsteht das Potential weiterführende Erkenntnisse
    aus den Informationen zu ziehen.

    Anwendung hat das \gls{wccs} bei zwei Fallbeispielen der {\fernUni} gefunden,
    die gezeigt haben, dass das Konzept des Systems funktioniert,
    aber einige Sonderfälle noch nicht ausreichend abgedeckt sind.

    Ein Vergleich mit anderen System hat gezeigt,
    dass die Sprache aufgrund ihrer Spezialisierung für ihren eigenen
    Anwendungsfall einige Vorteile bietet.
    Der Klassifizierungsansatz erfüllt die an ihn gestellten Anforderungen,
    ist aber weniger mächtig als komplexe Web Mining Werkzeuge.
    Ein gewisses Alleinstellungsmerkmal besitzt das \gls{wccs}
    bei der Anwendung als Werkzeug zur Prüfung der Konformität
    einer Webseite bezüglich eines individuellen inhaltlichen und strukturellen Schemas.