\documentclass[abstract=on,parskip=half,titlepage,twoside,openright]{scrreprt}

% Direct input of special characters
\usepackage[utf8]{inputenc}

% Support west european fonts / languages
\usepackage[T1]{fontenc}

% Use german translations for titles etc. (e.g. Inhaltsverzeichnis instead of Table of contents)
\usepackage[ngerman]{babel}

% Koma page style
\usepackage[automark]{scrlayer-scrpage}
\pagestyle{scrheadings}

% Bibliography
\usepackage[backend=biber,natbib=true,style=numeric-comp]{biblatex}
\addbibresource{../resources/literature.bib}

% Trademark symbol
\usepackage{textcomp}

\usepackage{listingsutf8}
\usepackage{subcaption}
\usepackage{color}

\lstdefinelanguage{wccdl}{
	morekeywords={
		as,
		by,
		class,
		classifies,
		content,
		css,
		each,
		is,
		many,
		page,
		pattern,
		recognized,
		reference,
		url,
		xpath
	},
	morestring=[s]{^^ab}{^^bb},
	sensitive=true,
	morecomment=[l]{//},
	morecomment=[s]{/*}{*/}
}

\lstdefinestyle{Xtext} {
	language = wccdl,
	basicstyle = \ttfamily\small,
	numberstyle = \footnotesize,
	showspaces = false,
	showtabs = false,
	showstringspaces = false,
	frame = single,
	tabsize = 4,
	captionpos = b,
	breaklines = true,
	numbers = left,
	breakatwhitespace = true,
	upquote=true,
	keywordstyle=\color{blue},
	stringstyle=\color{green}
}

\lstdefinestyle{wccdl} {
	language = wccdl,
	basicstyle = \ttfamily\small,
	numberstyle = \footnotesize,
	showspaces = false,
	showtabs = false,
	showstringspaces = false,
	frame = single,
	tabsize = 4,
	captionpos = b,
	breaklines = true,
	numbers = left,
	breakatwhitespace = true,
	upquote=true,
	keywordstyle=\color{blue},
	stringstyle=\color{green},
	commentstyle=\color{red}
}

\lstdefinestyle{pseudo} {
	language = Java,
	basicstyle = \ttfamily,
	numberstyle = \footnotesize,
	showspaces = false,
	showtabs = false,
	showstringspaces = false,
	frame = single,
	tabsize = 4,
	captionpos = b,
	breaklines = true,
	numbers = left,
	breakatwhitespace = true,
	upquote=true,
	keywordstyle=\color{blue},
	stringstyle=\color{green}
}

\lstdefinestyle{html} {
	language = HTML,
	basicstyle = \ttfamily,
	numberstyle = \footnotesize,
	showspaces = false,
	showtabs = false,
	showstringspaces = false,
	frame = single,
	tabsize = 4,
	captionpos = b,
	breaklines = true,
	numbers = left,
	breakatwhitespace = true,
	upquote=true,
	keywordstyle=\color{blue},
	stringstyle=\color{green},
	emph={article},
	emphstyle={\color{blue}}
}

\lstset {
	basicstyle = \ttfamily\small,
	numberstyle = \footnotesize,
	showspaces = false,
	showtabs = false,
	showstringspaces = false,
	frame = none,
	tabsize = 4,
	captionpos = none,
	breaklines = true,
	numbers = none,
	breakatwhitespace = true,
	upquote=true,
	keywordstyle=\color{blue}
}

\usepackage{graphicx}
\usepackage[margin=2.5cm]{geometry}
%\usepackage{lmodern}

\usepackage[acronym]{glossaries}
\makeglossaries
\newacronym{ajax}{Ajax}{Asynchronous JavaScript and XML}
\newacronym{babw}{BaBw}{Bachelor-Studiengang Bildungswissenschaft}
\newacronym{cms}{CMS}{Content Management System}
\newacronym{css}{CSS}{Cascading Style Sheets}
\newacronym{dsl}{DSL}{domänenspezifische Sprache}
\newacronym{gpl}{GPL}{General Purpose Language}
\newacronym{html}{HTML}{Hypertext Markup Language}
\newacronym{ksw}{KSW}{Kultur- und Sozialwissenschaften}
\newacronym{mvc}{MVC}{Model-View-Controller}
\newacronym{uri}{URI}{Uniform Resource Identifier}
\newacronym{url}{URL}{Uniform Resource Locator}
\newacronym{uwe}{UWE}{UML-based Web Engineering}
\newacronym{w3c}{W3C}{World Wide Web Consortium}
\newacronym[plural=WCCS']{wccs}{WCCS}{Web Content Classification System}
\newacronym{wccdl}{WCCDL}{Web Content Class Definition Language}
\newacronym{webml}{WebML}{Web Modelling Language}
\newacronym{www}{WWW}{World Wide Web}
\newacronym{wysiwyg}{WYSIWYG}{What You See Is What You Get}

% Support for links in pdf files
\usepackage[
	pdftex,
	hypertexnames=false,
	colorlinks=true,
	ocgcolorlinks
]{hyperref}

\hypersetup{
	unicode = {true},
	pdftitle = {Erstellung einer domänenspezifischen Sprache zur automatische Klassifizierung der Inhalte von Webseiten},
	pdfauthor = {Tim Gremplewski}
	% TODO pdfsubject = {},
	pdfkeywords = {
		{Domain Specific Languages},
		{Website Migration},
		{FernUni in Hagen}
	}
}

\titlehead{FernUniversität in Hagen\newline
Fakultät für Mathematik und Informatik\newline
Lehrgebiet Programmiersysteme\newline
Prof. Dr. Friedrich Steimann\newline
Universitätsstraße 1\newline
58097 Hagen, Deutschland\newline}
\subject{Masterarbeit}
\title{Erstellung einer domänenspezifischen Sprache zur automatischen Klassifizierung der Inhalte von Webseiten}
\author{Tim Gremplewski\thanks{tim.gremplewski@gmail.com, Martrikel-Nr: 9514244}}
\date{01. Februar 2018}

\parindent0mm

\newcommand{\annotationService}{Annotation Service}
\newcommand{\annotatorPlugin}{Annotator Plugin}
\newcommand{\childFeature}{Child Feature}
\newcommand{\classificationModel}{Klassifizierungsmodell}
\newcommand{\classificationService}{Classification Service}
\newcommand{\classificationStorage}{Classification Storage}
\newcommand{\classificationStorageAPI}{Classification Storage API}
\newcommand{\contentFeature}{Content Feature}
\newcommand{\collectionFeature}{Collection Feature}
\newcommand{\cssSelector}{CSS-Selektor}
\newcommand{\editor}{Redakteur}
\newcommand{\editors}{{\editor}e}
\newcommand{\fernUni}{Fern\-Uni\-ver\-si\-tät in Hagen}
\newcommand{\imageScalingFactor}{0.15}
\newcommand{\imperia}{imperia CMS}
\newcommand{\parentFeature}{Parent Feature}
\newcommand{\referenceFeature}{Reference Feature}
\newcommand{\resource}{Ressource}
\newcommand{\resources}{Ressourcen}
\newcommand{\scalarFeature}{Scalar Feature}
\newcommand{\urlSelector}{URL-Pattern-Selektor} % TODO: URL-Selektor?
\newcommand{\webAppService}{Web App}
\newcommand{\wordpress}{WordPress}
\newcommand{\wordpressCrawler}{WordPress Crawler}
\newcommand{\xpathSelector}{XPath-Selektor}

\begin{document}
	\maketitle

	\begin{abstract}
		Motiviert durch die Anforderung der {\fernUni} im Zuge der Modernisierung ihrer
Internetpräsenz Inhalte aus {\wordpress} zu {\imperia} zu migrieren,
widmet sich diese Arbeit der Konzipierung und Entwicklung eines Systems
zur automatischen Klassifizierung der Inhalte von Webseiten
-- dem Webpage Content Classification System.
Teil dieses Systems ist eine domänenspezifische Sprache, die die einfache
Definition von hierarchischen Klassen und Kategorisierungsregeln ermöglicht.
Anhand dieser Definitionen wird der Inhalt einer Webseite auf Basis ihrer
HTML-Repräsentation strukturiert und das Ergebnis Drittsystemen bereitgestellt.
Eine Erprobung des Systems erfolgt anhand mehrerer Seiten der {\fernUni}.
Die Ergebnisse verdeutlichen den Nutzen der Automatisierung und
beweisen die Machbarkeit des Ansatzes.
Gleichzeitig zeigen sie die Einschränkungen des Systems und seine Erweiterungsmöglichkeiten auf.
	\end{abstract}

	\cleardoublepage

	\begingroup
		% No new page after table of contents
		\let\cleardoublepage\relax
		\tableofcontents
		\listoffigures
		\listoftables
		\lstlistoflistings
	\endgroup

	\newpage

	%\chapter{Modernisierung des Internetauftrittes der FernUniversität in Hagen}
    \label{chapter:FernUniRelaunch}
    Ein zeitgemäßer Internetauftritt mit aktuellen Inhalten
    ist ein wichtiger Bestandteil der Öffentlichkeitsarbeit jeder Organisation,
    da sie für viele Interessierte die erste Anlaufstelle zur Beschaffung von Informationen ist.
    
    Wichtige Gründe hierfür sind die ständige Verfügbarkeit sowie die Orts-
    und Geräteunabhängigkeit.
    Eine Webseite steht zu jeder Tageszeit zur Verfügung und kann
    dank moderner Endgeräte wie Smartphones und Tablets
    auch unterwegs aufgerufen werden.
    Für den Nutzer stellt der Internetauftritt deshalb ein Medium dar,
    das er einfach, spontan und flexibel verwenden kann.

    Wegen dieser Beliebtheit dient die Webseite neben der reinen Bereitstellung von Informationen
    auch der eigenen Werbung und Vermarktung und spielt eine wichtige Rolle im Erfolg einer Organisation.
    Zwei Eigenschaften des Webauftrittes sind laut \cite{sillence:onlineHealthSites} dabei
    entscheidend: Inhalt und Design.

    Ab einer gewissen Größe der Webseite erscheint es deshalb nachvollziehbar,
    dass zur inhaltlichen Pflege eine eigene Rolle geschaffen wird,
    die durch entsprechend ausgebildetes Personal besetzt wird.
    Ein gebräuchlicher Titel dieser Rolle ist \textit{\editor},
    der auch im Verlauf dieser Arbeit verwendet wird.

    \editors verwenden zur Pflege ihrer Inhalte üblicherweise ein \gls{cms}.
    Eine abstrakte Beschreibung solcher Systeme liefert \cite[][Seite 5,6]{barker:webCMS}:

    \begin{quote}
        A content management system (CMS) is a software package that provides
        some level of automation for the tasks required to effectively manage content.
        [...]

        A CMS allows editors to create new content, edit existing content,
        perform editorial processes on content, and ultimately make that content
        available to other people to consume it.
        [...]
    \end{quote}

    Des Weiteren beschreibt \cite[][Seite 9-12]{barker:webCMS} die Kernaufgaben eines \gls{cms}:
    \begin{enumerate}
        \item   Kontrolle der Inhalte über Rollen- und Rechtekonzepte,
                Versionierung, Abhängigkeitsmanagement, Such- und Strukturierungsmöglichkeiten
        \item   Wiederverwendung von Inhalten ermöglichen
        \item   Automatische Aggregation, Verarbeitung und Aufbereitung von Inhalten
        \item   Arbeit der \editors effizienter gestalten
    \end{enumerate}

    Wie zuvor beschrieben, ist neben dem Inhalt auch das Design,
    mit dem Inhalte präsentiert werden, relevant für den Erfolg einer Webseite.
    Die Entwicklung der Trends im Webdesign seit den frühen 1990er Jahren veranschaulicht
    \cite{work:webDesignEvolution}.
    Gezeigt wird die Entwicklung angefangen bei Seiten, die nur Text enthalten,
    über Tabellenlayouts, Flash-Applikationen bis zum Responsive Design und für
    Mobilgeräte optimierte Seiten.
    Webauftritte entwickeln sich demnach stetig weiter,
    was \cite{murphy:webDesignEvolution} an Beispielen wie der Internetseite der
    Fluggesellschaft Ryanair ebenfalls verdeutlicht.

    Folglich ist jede Organisation angehalten laufend ihre Webseite inhaltlich zu pflegen
    und ihr regelmäßig ein modernes Design zu geben.
    Andernfalls läuft sie Gefahr die Erwartungen ihrer Besucher nicht zu erfüllen,
    die sich deshalb schlecht angesprochen fühlen und womöglich zur Konkurrenz wechseln.
	%\chapter{Problemanalyse}
    In diesem Kapitel erfolgt
    eine ausführlichere Betrachtung der Problemstellung,
    die mit einer Erläuterung der Problemdomäne beginnt.
    Anschließend folgt eine Vorstellung der Content Management Systeme,
    die sich an der {\fernUni} im Einsatz befinden
    und wie eine manuelle Migration von einem System zum anderen
    ablaufen kann.
    Basierend auf diesen Ausführungen wird anschließend diskutiert,
    wie die Inhalte einer Webseite vorbereitet werden müssen,
    um diese Migration in einem späteren Schritt automatisieren zu können.

    \label{chapter:ProblemAnalysis}
    \section{Webseiten}
    \label{section:problemAnalysisWebpagesInTheWWW}
    Für ein besseres Verständnis der Aufgabenstellung sowie zur Findung
    einer geeigneten Lösung ist es sinnvoll, die technischen und fachlichen
    Grundlagen von Webseiten zu betrachten.
    Die folgenden Erläuterungen beinhalten auch Aspekte des \glspl{www},
    da Webseiten ein Teil des \glspl{www} sind.

    \subsection{Das World Wide Web}
        Das \gls{w3c} definiert den Begriff "`World Wide Web"' in \cite{w3c:wwwArch} wie folgt:

        \begin{quote}
            The \textit{\textbf{World Wide Web}} (\textit{\textbf{WWW}}, or simply \textit{\textbf{Web}})
            is an information space in which the items of interest, referred to as resources,
            are identified by global identifiers called Uniform Resource Identifiers (\textit{\textbf{URI}}).
        \end{quote}

        Darüber hinaus nennt \cite{w3c:wwwArch} die drei grundlegenden Komponenten des \glspl{www}:

        \begin{quote}
            They are identification of resources,
            representation of resource state, and the protocols
            that support the interaction between agents and resources in the space.
        \end{quote}

    \subsection{{\resources}}
        \label{section:problemAnalysisWebpagesInTheWWWResources}
        Aus der obigen Definition wird deutlich,
        dass das \gls{www} nicht ausschließlich zur Nutzung von Webseiten
        vorgesehen ist.
        Stattdessen kann es jede Art von {\resources} bereitstellen.
        Die Beschreibung des \glspl{w3c} stellt außerdem klar,
        dass jede {\resource} im \gls{www} eindeutig über einen \gls{uri} identifiziert wird.
        Im Fall von Webseiten geschieht dies über \glspl{url},
        die eine Teilmenge von \glspl{uri} darstellen
        \cite[Kapitel 1.1.3]{rfc:3986}.

        Das Schema einer \gls{uri} beschreibt laut \cite[Kapitel 3.1]{rfc:3986},
        wie die restlichen Teile der \gls{uri} zu interpretieren sind.
        Viele Schemata sind nach Protokollen benannt,
        weshalb sie in der Praxis häufig auch das Verfahren bestimmen,
        mit dem auf eine {\resource} zugegriffen werden kann
        \cite[Kapitel 3.1]{w3c:wwwArch}.
        Schemata, die im \gls{www} häufig Anwendung finden,
        sind zum Beispiel
        \texttt{http},
        \texttt{https},
        \texttt{mailto},
        \texttt{ftp} und
        \texttt{data}.
        Für Webseiten sind dabei vor allem \texttt{http} und \texttt{https}
        und die gleichnamigen Protokolle von Bedeutung,
        da \glspl{url} diese vorrangig verwenden.

    \subsection{Web Agents}
        Um {\resources} im \gls{www} anzusprechen, existieren verschiedene Werkzeuge.
        Das \gls{w3c} \cite[Kapitel 6]{w3c:wwwArch} bezeichnet jedes von ihnen als
        "`Web agent"' und beschreibt diese Rolle wie folgt:
        "`A person or a piece of software acting on the information
        space on behalf of a person, entity, or process."'.
        Eine Spezialisierung stellen die sogenannten "`User agents"' dar,
        die das \gls{w3c} \cite[Kapitel 6]{w3c:wwwArch} als "`One type of Web agent;
        a piece of software acting on behalf of a person."' beschreibt.
        Webbrowser sind User agents, aber zum Beispiel auch Kommandozeilenprogramme
        wie curl\footnote{\url{https://curl.haxx.se/}}.
        Ihre persönliche Kennung halten User Agents zum Beispiel im
        HTTP-Header "`User-Agent"' fest \cite[Kapitel 5.5.3, Seite 46]{rfc:7231}.

        Der Prozess, bei dem ein Agent über eine \gls{uri} auf eine {\resource}
        zugreift, heißt "`Dereferencing the URI"'.
        Wie dieser Prozess aussieht, hängt von der Funktion des User Agents und der \gls{uri} ab.
        Ein User Agent könnte bspw. eine
        Repräsentation einer {\resource} abrufen, wohingegen ein anderer Agent
        lediglich prüft, ob eine eben solche Repräsentation existiert
        \cite[Kapitel 3.1]{w3c:wwwArch}.
        Wenn ein Webbrowser oder ein anderer Agent eine Webseite aufruft,
        fragt er bei einem Webserver also eine Repräsentation
        einer {\resource} an, die er über eine \gls{url} spezifiziert.
        Falls der Server die \gls{url} der Anfrage auflösen kann,
        antwortet er auf die Anfrage mit einer Repräsentation der Webseite.
        Der Browser übernimmt dann das Rendering der Seite und die Ausführung
        von eingebetteten Skripten.

        \subsection{HTML-Repräsentation}
            Webseiten sind Dokumente, die die Auszeichnungssprache \gls{html} verwenden,
            die aktuell in der Version 5 \cite{w3c:html5} vorliegt.
            Bis zur Version 3.2 war \gls{html} eine Implementierung der Metasprache SGML \cite[Kapitel 3]{w3c:html401}.
            Zusammen mit der Version 4 \cite{w3c:xhtml} wurde eine Ausprägung spezifiziert,
            die \gls{html} in eine Implementierung der Sprache XML überführt,
            die den Namen XHTML trägt.
            \gls{html}5 \cite[Kapitel 1.6]{w3c:html5} definiert eine abstrakte Sprache,
            die sowohl mit \gls{html}- als auch mit XML-Syntax genutzt werden kann.
            Einem Agent kann über den MIME-Type mitgeteilt werden,
            ob ein Dokument \gls{html}- oder XML-Syntax verwendet \cite[Kapitel 1.6]{w3c:html5}.
            Diese Angabe ist wichtig, da XML-Syntax restriktiver als \gls{html}-Syntax ist.
            Der \gls{html}-Syntax erfordert zum Beispiel nicht,
            dass alle öffnenden Tags auch einen schließenden Gegenpart besitzen
            \cite[Kapitel 3.2.3]{w3c:html5}.

            \gls{html}-Dokumente bestehen in beiden Syntaxvarianten aus verschachtelten
            Elementen, die in 10 Kategorien \cite[Kapitel 4]{w3c:html5} eingeteilt werden.
            Diese sind in Tabelle \ref{table:htmlElements} inkl. einiger Vertreter aufgelistet.
            Die verschiedenen Elemente geben ihren Inhalten eine semantische Bedeutung.
            Text in einem \texttt{h1}-Element \cite[Kapitel 4.3.6]{w3c:html5} wird von einem Webbrowser zum Beispiel in die Anzeige gerendert,
            wohingegen er in einem \texttt{script}-Element \cite[Kapitel 4.11.1]{w3c:html5} als Code interpretiert und ausgeführt wird.

            \begin{table}[h]
                \centering
                \begin{tabular}{|l|l|}
                \hline
                \textbf{Kategorie} & \textbf{Elemente (Auswahl)} \\
                \hline
                Das Wurzelelement & \texttt{html} \\
                \hline
                Metadaten des Dokumentes & \texttt{head}, \texttt{title}, \texttt{style} \\
                \hline
                Absätze & \texttt{body}, \texttt{article}, \texttt{section}, \texttt{h1} \\
                \hline
                Gruppierungen & \texttt{p}, \texttt{ul}, \texttt{div} \\
                \hline
                Semantische Kennzeichnung von Text & \texttt{a}, \texttt{strong}, \texttt{code} \\
                \hline
                Kennzeichnung von Textänderungen & \texttt{ins}, \texttt{del} \\
                \hline
                Eingebetteter Inhalt & \texttt{img}, \texttt{iframe}, \texttt{object} \\
                \hline
                Tabellenelemente & \texttt{table}, \texttt{tr}, \texttt{td} \\
                \hline
                Formulare & \texttt{form}, \texttt{input}, \texttt{button} \\
                \hline
                Skripte & \texttt{script}, \texttt{canvas} \\
                \hline
                \end{tabular}
                \caption{Die Kategorisierung von \acrshort{html}-Elementen}
                \label{table:htmlElements}
            \end{table}
        
            Eine Besonderheit von \gls{html} ist die Möglichkeit
            andere {\resources} des \glspl{www} zu referenzieren,
            wodurch ein gerichtetes und zyklisches Netzwerk entsteht.
            Beispiele für häufig in Webseiten referenzierte {\resources}
            sind andere Webseiten, Skripte, Stylesheet-Dateien, Bilder und Videos.
            Zur Referenzierung dieser Beispiele dienen die \gls{html}-Elemente
            \texttt{a} \cite[Kapitel 4.5.1]{w3c:html5},
            \texttt{script} \cite[Kapitel 4.11.1]{w3c:html5},
            \texttt{link} \cite[Kapitel 4.2.4]{w3c:html5},
            \texttt{img} \cite[Kapitel 4.7.1]{w3c:html5} und
            \texttt{video} \cite[Kapitel 4.7.6]{w3c:html5},
            die auch die Semantik der Referenz festlegen.
            Ein Bild kann sowohl über das \texttt{img}-
            als auch das \texttt{a}-Element referenziert werden.
            Im ersten Fall wird ein Webbrowser das Bild direkt in die Anzeige rendern,
            wohingegen er im zweiten Fall nur einen klickbaren Link auf das Bild darstellt.

        \subsection{Trennung der Zuständigkeiten}
            \label{section:problemAnalysisWebpagesInTheWWWSeparationOfConcerns}
            Die Trennung der Zuständigkeiten \cite{huersch:SeparationOfConcerns}
            ist ein allgegenwärtiges Konzept der Informatik.
            Im Sinne dieses Konzeptes existieren im Kontext von Webseiten
            drei Belange, die es zu trennen gilt:

            \begin{enumerate}
                \item Struktur und Inhalt
                \item Darstellung
                \item Funktionalität
            \end{enumerate}

            Der Inhalt einer Webseite ist in den \gls{html}-Elementen enthalten,
            deren Hierarchie darüber hinaus die Struktur der Seite bestimmt.
            Das Aussehen einer Seite wird hingegen über \gls{css} \cite{w3c:css} festgelegt.
            Dabei handelt es sich um "`[...] eine Sprache zur Beschreibung des Renderings
            von strukturierten Dokumenten (wie HTML und XML) auf Bildschirmen, auf Papier,
            in Sprache, etc."' \cite{w3c:css}.
            Zu diesem Zweck werden mit \gls{css} Regeln definiert,
            die einer Menge von \gls{html}-Elementen Eigenschaften in der Darstellung (Styles) zuweisen
            \cite{w3c:cssSyntax}.
            Diese Menge von Elementen wird wiederum über Selektoren \cite{w3c:cssSelectors} festgelegt,
            mit denen sich verschiedene Eigenschaften der auszuwählenden Elemente beschreiben lassen.
            Falls zwei Regeln dasselbe Element ansprechen,
            werden die Styles beider Regeln zusammengefasst
            \cite{w3c:cssCascading}.
            Überschneidungen in den Styles werden in diesem Fall anhand der Selektoren der betroffenen Regeln aufgelöst.
            Verschiedene Selektoren sprechen einzelne Elemente nämlich verschieden stark an.
            Bezogen auf das betroffene Element lässt sich dadurch eine Priorität festlegen,
            bei der eine Regel mit starkem Selektor Vorrang vor einer Regel mit schwachem Selektor hat
            \cite{w3c:cssSelectors}. 
            Wichtig zu beachten ist, dass \gls{css} die angewandten Styles eines Elementes auf alle Unterelemente vererbt.
            Diese können Styles allerdings explizit überschreiben
            \cite{w3c:cssCascading}.
            Auf den verbliebenen Aspekt "`Funktionalität"' geht der folgende Abschnitt ein.

        \subsection{Dynamische Webseiten}
            Komplexe Webanwendungen sind auf die Möglichkeit angewiesen, Geschäftslogik auszuführen.
            Dafür kommen sowohl der Webserver als auch der User Agent (Client) infrage.
            Der Server kann zur Bearbeitung einer Anfrage mit dem Inhalt einer Datei auf seinem Dateisystem antworten.
            Das kann zum Beispiel eine \gls{html}-Datei oder ein Bild sein.
            In diesem Fall spricht man von statischen Inhalten.
            Es ist allerdings auch möglich,
            dass eine Webanwendung serverseitig Geschäftslogik zur Bearbeitung einer Anfrage ausführt.
            Die Antwort an den Client kann in diesem Fall weiterhin statisch oder
            durch die Webanwendung generiert worden sein.
            Im letzten Fall spricht man von dynamisch generierten bzw. erzeugten Inhalten.

            Ein \gls{html}-Dokument kann in \texttt{script}-Elementen \cite[Kapitel 4.11.1]{w3c:html5} ebenfalls Logik enthalten,
            zu deren Formulierung Entwickler häufig die Skriptsprache JavaScript verwenden.
            Dabei handelt es sich um eine Implementierung des Sprachstandards ECMAScript
            \cite{ecma:ecmaScript}.
            Nicht der Server, sondern der anfragende User Agent -- häufig ein Webbrowser -- führt diese Logik aus.
            Es existieren verschiedene Programmierschnittstellen \cite[Kapitel 8]{whatwg:html},
            die solchen Skripten erlauben,
            Elemente zu manipulieren, auf Nutzerinteraktion zu reagieren
            oder asynchrone HTTP-Anfragen \cite{whatwg:xhr} auszuführen.
            Falls eine Webseite solche clientseitigen Mittel verwendet,
            wird sie als dynamische Seite bezeichnet.
            
            Client- und serverseitige Programmteile funktionieren oftmals Hand in Hand,
            wie das folgende Beispiel illustriert:
            Ein Nutzer klickt auf eine Schaltfläche in einer Webseite.
            Dadurch wird eine JavaScript-Funktion aktiviert,
            die eine asynchrone HTTP-Anfrage an einen Webserver schickt.
            Aufgrund dieser Anfrage führt der Server Geschäftslogik aus
            und generiert auf Basis des Ergebnisses ein \gls{html}-Fragment,
            welches er in seine Antwort schreibt.
            Der Empfang der Antwort löst im Browser die Ausführung einer weiteren
            JavaScript-Funktion aus, die das \gls{html}-Fragment
            an eine definierte Stelle der Webseite einfügt,
            wodurch es dem Nutzer sichtbar wird.
            Der konzeptionelle Ablauf in diesem Szenario sowie die verwendeten Mittel
            werden unter dem Begriff \gls{ajax} \cite{garrett:ajax} zusammengefasst.

        \subsection{Websites}
            \label{section:problemAnalysisWebpagesInTheWWWWebsites}
            Der Internetauftritt einer Organisation besteht selten aus einer
            einzelnen Webseite.
            Stattdessen besteht er aus vielen Webseiten,
            die untereinander verlinkt sind und von der jede eine eigene
            \gls{url} besitzt.
            Neben "`Internetauftritt"' hat sich der Begriff "`Website"' zur
            Referenzierung dieser Gesamtheit aller Webseiten einer Organisation
            etabliert \cite{duden:Internetauftritt, oxford:Website}.
    \section{\imperia}
    % TODO: Wie werden Medien gespeichert?
    {\imperia} ist ein kommerzielles Enterprise Content Management und
    Web Content Management System, welches seit 1995 entwickelt
    \cite{imperia:about, imperia:historie} und von 
    der {\fernUni} zur Pflege ihrer Webseiten genutzt wird
    \cite{fernUni:imperia}.

    Dieses Kapitel geht auf die wichtigsten Merkmale und Konzepte
    dieses \gls{cms} ein.

    \subsection{Statische Generierung}
        \label{section:imperiaStaticGeneration}
        {\imperia} ist ein statisch generierendes \gls{cms}.
        Das bedeutet, dass es Inhalte basierend auf Vorlagen
        in Dateien generiert, die dann auf ein Zielsystem übertragen
        und dort beliebig und vor allem unabhängig von
        {\imperia} genutzt werden können
        \cite[Kapitel 1.1]{imperia:ecmd}.

        Die Generierung der Dateien und deren Übertragung auf ein Zielsystem
        werden zusammgengefasst als "`Publizierung"' bezeichnet.

        % TODO Vorteile

    \subsection{Dokumente}
        \label{section:imperiaDocuments}
        Die zentrale Datenstruktur in {\imperia} sind Dokumente,
        da sie zur Speicherung von Inhalten dienen
        \cite[Kapitel 1.1]{imperia:ecmd}.
        Dokumente werden beim Anlegen stets einer Kategorie zugewiesen,
        die festlegt auf welcher Voralage\footnote{vgl. Kapitel \ref{section:imperiaTemplates}}
        das Dokument basiert.
        Die Vorlage bestimmt wiederum welches Eingabeformular
        {\editors} zur Pflege der Inhalte des Dokumentes verwenden
        \cite[Kapitel 1.1.4]{imperia:ecmd}.
        
        {\imperia} speichert ein Dokument als simple Menge von
        Schlüssel-Wert-Paaren.
        Als Schlüssel -- auch Metavariablen genannt -- dienen die Namen der Felder des Eingabeformulares.
        Entsprechend übernimmt {\imperia} die Inhalte dieser Felder
        als Werte der Schlüssel-Wert-Paare.
        Diese allgemeine Datenstruktur ermöglicht die Nutzung der Inhalte
        in verschiedenen Ausgabeformaten, wie \gls{html}, XML, etc
        \cite[Kapitel 1.1.2]{imperia:ecmd}.

    \subsection{Vorlagen}
        \label{section:imperiaTemplates}
        Ein wichtiges Ziel von {\imperia} ist die Trennung von Inhalt
        und Layout einer Webseite.
        Zu diesem Zweck speichert es Inhalte layoutunabhängig
        in Dokumenten\footnote{vgl. Kapitel \ref{section:imperiaDocuments}}.

        Das Layout wird hingegen in Vorlagen festgehalten,
        die zwei Ziele verfolgen
        \cite[Kapitel 36]{imperia:ecmd}:

        \begin{enumerate}
            \item {\editors} ohne technische Kenntnisse eine einfache Pflege von Inhalten ermöglichen
            \item Inhalte in das Layout integrieren
        \end{enumerate}

        Zur Erfüllung des ersten Zieles kann jede Vorlage ein Eingabeformular
        und dessen Felder spezifizieren.
        Dazu stehen geläufige Komponenten wie Textfelder,
        aber auch {\imperia} eigene Elemente zur Verfügung
        \cite[Kapitel 1.1.4]{imperia:ecmd}.
        Jeder Kategorie wird eine Vorlage zugewiesen,
        wobei eine Vorlage von mehreren Kategorien genutzt werden kann.
        Durch die Einteilung der Dokumente in Kategorien
        \footnote{vgl. Kapitel \ref{section:imperiaDocuments}}
        kann jedes Dokument somit einer Vorlage zugeordnet werden.
        Das durch diese Vorlage definierte Eingabeformular
        wird {\editors n} zur Pflege der Inhalte des Dokumentes präsentiert
        \cite[Kapitel 1.1.4]{imperia:ecmd}.

        Das zweite Ziel erreichen Vorlagen,
        indem sie neben dem Eingabeformular auch ein Gerüst für das
        Layout definieren.
        Über eine spezielle Syntax können sie an beliebigen Positionen
        Metavariablen\footnote{vgl. Kapitel \ref{section:imperiaDocuments}} referenzieren,
        deren Werte im Ausgabedokument an der entsprechenden Stelle integriert werden
        \cite[Kapitel 36]{imperia:ecmd}.

        Vorlagen können in zwei Variaten vorliegen.
        Die erste vereint sowohl die Definition des Eingabeformulares
        als auch die des Layouts in einer Datei.
        Ein {\editor} kann Dokumente dadurch nach dem Prinzip \gls{wysiwyg} bearbeiten.
        Die Eingabefelder entfernt {\imperia} automatisch im Ausgabedokument.
        Die zweite Variante trennt beide Definitionen in unterschiedliche Dateien.
        Dieses Vorgehen ist sinnvoll, wenn die Inhalte eines Dokumentes mit
        verschiedenen Layouts genutzt werden sollen.
        In diesem Fall existiert also eine Vorlage,
        die das neutrale Eingabeformular bestimmt
        und mehrere Layout-Vorlagen, die dieselben Inhalte unterschiedlich darstellen
        \cite[Kapitel 36]{imperia:ecmd}.

    \subsection{Workflows}
        Ein Dokument durchläuft von seiner Generierung bis zur
        Publizierung mehrere Verarbeitungsschritte,
        die in {\imperia} sogenannte Workflows festlegen.
        Jeder Kategorie wird dazu ein Workflow zugewiesen,
        den die Dokumente durchlaufen müssen.
        Ein solcher Workflow legt unter anderem fest,
        welche Schritte und in welcher Reihenfolge
        zu durchlaufen sind
        \cite[Kapitel 1.1.5]{imperia:ecmd}.

        Ein typischer Workflow ist der Folgende
        \cite[Kapitel 1.1]{imperia:ecmd}:

        \begin{itemize}
            \item Erstellung des Dokumentes und Angabe allgemeiner Daten
            \item Inhaltliche Pflege des Dokumentes
            \item Prüfung der Inhalte durch einen zweiten berechtigten \editor
            \item Publizierung der Inhalte
        \end{itemize}

    \subsection{Architektur}
        \label{section:imperiaArch}
        {\imperia} basiert auf einer mehrschichtigen Client-Server-Architektur,
        die in Abbildung \ref{image:imperiaArchitektur} dargestellt ist.

        \begin{figure}
            \centering
            \includegraphics[width=\textwidth]{../resources/imperia/architektur.png}
            \caption{Architkektur von {\imperia} \cite{imperia:ecmd}}
            \label{image:imperiaArchitektur}
        \end{figure}

        Die wichtigste Komponente in dieser Architektur ist der imperia Server,
        der die zentralen Funktionen des Systems bereitstellt.
        Dazu gehören das Anlegen und Strukturieren von Projekten
        und Dokumenten sowie die Ausführung von Workflows.
        Nicht zuletzt verwaltet er die Datenhaltung.

        Die verschiedenen Nutzer des Systems wie {\editors} und Administratoren
        verwenden für ihre Arbeit eine Weboberfläche,
        die über HTTP(S) mit dem Server kommuniziert.
        Über das gleiche Protokoll können auch Drittsysteme den Server
        ansprechen und verschiedene Aktionen durchführen oder Daten abfragen.

        Sobald ein Dokument alle Workflow-Schritte durchlaufen hat,
        wird es durch eine automatische oder manuelle Publizierung
        in eine Datei generiert, die dann auf ein Zielsystem übertragen wird.
        Dieses System ist eigenständig und gehört nicht zu {\imperia}.
        Allerdings bietet es die Möglichkeit über Dienste wie (S)FTP
        die generierten Dateien zu empfangen.
        Ein Beispiel sind Webserver, auf die Webseiten publiziert werden,
        die sie dann Besuchern bereitstellen.
    \section{\wordpress}
    % TODO:
    % - Sites!!!
    \label{section:WordPress}
    Das Open-Source-Projekt {\wordpress} startete 2003
    mit dem Ziel eine Anwendung zur einfachen Pflege eines Weblogs
    (kurz Blog) zu schaffen \cite{wordpress:About}.
    Das Ergebnis ist die gleichnamige Software,
    die noch immer von der Community weiterentwickelt
    und von der Fakultät \gls{ksw} der {\fernUni} für ihren
    Internetauftritt genutzt wird.
    In diesem Kapitel werden grundlegende Konzepte dieses Systems vorgestellt.

    \subsection{Weblog-Software}
        \label{section:weblogSoftware}
        {\wordpress} ist im Kern eine Software zur Pflege eines Blogs,
        wobei es sich um eine spezielle Form einer Webseite handelt
        \cite[Kapitel "`Introduction to Blogging"']{wordpress:codex}:

        \begin{quote}
            "`Blog"' is an abbreviated version of "`weblog"',
            which is a term used to describe websites that maintain
            an ongoing chronicle of information.
            A blog features diary-type commentary and links to articles
            on other websites, usually presented as a list of entries in
            reverse chronological order.
            Blogs range from the personal to the political,
            and can focus on one narrow subject or a whole range of subjects.
        \end{quote}

        Da Blogs eine spezielle Form von Webseiten sind,
        kann man auch eine Anwendung zu ihrer Pflege als
        spezielle Form eines \glspl{cms} betrachten.
        Diese Aussage trifft {\wordpress} auch über sich selbst
        und vergleichbare Software \cite[Kapitel "`Introduction to Blogging"']{wordpress:codex}:

        \begin{quote}
            Many blogging software programs are considered a specific type of CMS.
            They provide the features required to create and maintain a blog,
            and can make publishing on the internet as simple as writing an article,
            giving it a title, and organizing it under (one or more) categories.
        \end{quote}

        Nicht zuletzt, weil auch Privatpersonen eine Zielgruppe
        solcher Anwendungen sind, vereinfachen sie Blogeinträge
        auf die zwei elementaren Elemente Titel und Inhalt.
        Mit dieser schwachen Strukturierung der Inhalte
        unterscheidet sich {\wordpress} deutlich von {\imperia}
        und seinen Dokumenten, die beliebig stark strukturiert
        werden können.
        Aufgrund der Anpassbarkeit durch Themes
        und Plugins sieht sich
        {\wordpress} trotzdem als vollwertiges \gls{cms}
        \cite{wordpress:About}.

    \subsection{Dynamische Generierung}
        \label{section:problemAnalysisWordPressDynamicGeneration}
        Anders als {\imperia} ist {\wordpress} nicht nur das Redaktionssystem,
        sondern gleichzeitig auch das ausliefernde Zielsystem.
        Inhalte generiert {\wordpress} deshalb nicht in statische Dateien,
        sondern erzeugt eine Webseite auf Anfrage dynamisch.

    \subsection{Posts und Pages}
        \label{section:wordpressPostsPages}
        {\wordpress} unterscheidet zwei Arten von Beiträgen \cite[Kapitel "`Pages"']{wordpress:codex},
        auf denen eine Webseite basieren kann: Posts
        und Pages.

        \paragraph*{Posts}
        Klassische Blogeinträge werden in {\wordpress} "`Posts"' genannt.
        Neben der Pflege des Titels und des Inhaltes eines Posts stehen dem
        Anwender noch weitere Optionen zur Verfügung.
        Ein Post kann in Kategorien einsortiert oder mit Schlagwörtern versehen werden.
        Dadurch kann {\wordpress} Übersichtsseiten generieren,
        die z. B. alle Posts einer Kategorie enthalten.
        Über den eigentlichen Inhalt hinausgehende Informationen können
        in sogenannten "`Custom Fields"' gespeichert werden.
        Davon machen z. B. Plugins Gebrauch
        und speichern Metadaten des Posts in ihnen.
        Jeder Post besitzt einen Post Type,
        der eine Aussage über die Art des Beitrages macht.
        {\wordpress} definiert einige Standardtypen \cite[Kapitel "`Post Types"']{wordpress:codex},
        lässt aber auch die Angabe eigener Typen zu.
        Die Standardtypen sind:

        \begin{itemize}
            \item Post,
            \item Page,
            \item Attachment,
            \item Revision,
            \item Navigation Menu,
            \item Custom CSS und
            \item Changesets.
        \end{itemize}

        Der Typ eines Posts ändert nichts an seinen redaktionellen Feldern.
        Das heißt, unabhängig vom Post Type besitzt ein Beitrag nur
        einen Titel und Inhalt
        \cite[Kapitel "`Posts"' \& "`Post Types"']{wordpress:codex}.
        Stattdessen machen Typen wie Revision und Custom CSS deutlich,
        dass {\wordpress} Posts auch zur Realisierung technischer Anforderung verwendet.

        \paragraph*{Pages}
        Neben Posts kennt {\wordpress} auch das Konzept einer Page \cite[Kapitel "`Pages"']{wordpress:codex},
        die sich in ihrem Zweck klar von einem Post unterscheidet:

        \begin{quote}
            In contrast, pages are generally for non-chronological,
            hierarchical content: pages like "`About"' or "`Contact"'
            would be common examples.
            [...]
            Pages live outside of the normal blog chronology,
            and are often used to present timeless information about
            yourself or your site -- information that is always relevant.
            You can use Pages to organize and manage the structure of your website content.
        \end{quote}

        Aus diesem Grund besitzen Pages zwar ebenfalls einen Titel und Inhalt,
        können aber lediglich mit Schlagwörtern versehen werden.
        Die Einordnung in eine Kategorie ist nicht möglich
        \cite[Kapitel "`Pages"']{wordpress:codex}.
        Wie aus der oben erfolgten Auflistung der Post Types hervorgeht,
        sind Pages technisch gesehen lediglich Posts mit dem Post Type "`Page"'.

        {\wordpress} speichert alle Inhalte in einer relationalen Datenbank \cite[Kapitel "`Database Description"']{wordpress:codex}.
        Posts und Pages teilen sich in dieser Datenbank eine Tabelle,
        was ebenfalls verdeutlicht, dass Pages lediglich Posts eines speziellen Typs sind.
        Der Inhalt eines Beitrages wird in der Datenbank als \gls{html}-Fragment abgelegt,
        welches während der dynamischen Generierung in die Webseite übernommen wird.
        Zusätzlich enthält ein Beitrag aber auch spezielles Anweisungen\footnote{vgl. Kapitel \ref{section:wordpressPlugins}},
        welches von {\wordpress} während der Generierung interpretiert wird.

    \subsection{Vorlagen und Themes}
        \label{section:wordpressTemplatesThemes}
        Wie {\imperia} strebt auch {\wordpress} eine Trennung von
        Inhalt und Layout an.
        Inhalte werden dazu in
        Posts und Pages
        unabhängig vom Layout der Webseite gespeichert.
        Das Layout bestimmen Vorlagen und Themes.

        \paragraph*{Vorlagen}
        {\wordpress} nutzt Vorlagen \cite[Kapitel "`Templates"']{wordpress:codex}, um Inhalte in eine Seite einzubinden
        und ihr Aussehen festzulegen.
        Dazu definieren sie das Gerüst der Webseite und enthalten Kommandos,
        um Inhalte aus der Datenbank auszulesen.
        Eine Webseite wird auf Basis einer Vorlage dynamisch von {\wordpress} generiert.
        Vorlagen sind bei genauerem Hinsehen nichts anderes als PHP-Dateien,
        die \gls{html}, allgemeinen PHP-Code und sogenannte
        "`Template-Tags"' enthalten.
        Dabei handelt es sich um Aufrufe von
        {\wordpress}-eigenen PHP-Funktionen,
        um Inhalte aus der Datenbank abzufragen.
        Während der Generierung -- technisch lediglich die Ausführung
        des PHP-Codes -- können Vorlagen andere Vorlagen inkludieren,
        wodurch wiederkehrende Elemente in eigene Vorlagen ausgelagert
        werden können \cite[Kapitel "`Template Files"']{wordpress:codex}.
        Anhand der \gls{url} der Anfrage und der darin enthaltenen
        Kennung eines Posts oder einer Page,
        entscheidet {\wordpress}, welche Vorlage es zur Generierung der Seite nutzt
        \cite[Kapitel "`Template Hierarchy"']{wordpress:codex}.

        \paragraph*{Themes}
        Eine Sammlung aller {\resources}, die notwendig sind, um
        eine Webseite und ihr Layout umzusetzen,
        wird im Kontext von {\wordpress} als "`Theme"' bezeichnet.
        Ein Theme enthält demnach Vorlagen, Bilder sowie
        JavaScript-, \gls{css}-, und PHP-Dateien.
        Themes können {\resources} anderer Themes wiederverwenden oder überschreiben,
        wodurch Anpassungen an einem vorhandenen Theme einfach umzusetzen sind
        \cite[Kapitel "`Using Themes"']{wordpress:codex}.

    \subsection{Plugins}
        \label{section:wordpressPlugins}
        {\wordpress} besitzt ein Plugin-System \cite[Kapitel "`Plugins"']{wordpress:codex},
        über das es beliebig funktional erweitert werden kann,
        ohne an {\wordpress}' eigenen Quellen Änderungen vorzunehmen.
        Es existiert eine große Anzahl an freien Plugins,
        die oft benötigte Funktionen implementieren.
        Meist sind dies Funktionen, die über eine reine Weblog-Software hinausgehen.
        Ein Beispiel ist das Plugin
        "`Form Maker"'\footnote{vgl. \url{https://wordpress.org/plugins/form-maker/}},
        welches die Definition von Webformularen erlaubt,
        die dann in beliebigen Beiträgen eingebettet werden können.
        Zur Nutzung eines Plugins wird meist eine spezielle Anweisung -- ein Shortcode \cite[Kapitel "`Shortcode API"']{wordpress:codex} --
        in den Inhalt eines Posts oder einer Page geschrieben,
        die zum Zeitpunkt der Generierung ausgewertet wird.
        Beim Form Maker Plugin wäre dies zum Beispiel \texttt{[Form id="1"]},
        wodurch das Formular mit der Kennung "`1"' in die Webseite integriert wird.

    \section{Eine Migration von {\wordpress} zu {\imperia}}
    An einem Beispiel soll erklärt werden,
    wie eine manuelle Migration der Inhalte einer Webseite von {\wordpress} zu {\imperia}
    vonstatten gehen könnte und wo die größte Herausforderung dabei liegt.
    Darauf basierend erklärt das nächste Kapitel,
    welchem Aspekte dieses Prozesses sich diese Arbeit widmet.

    Als Beispiel dient die Seite eines Mitarbeiters,
    die schon aus Kapitel \ref{section:fernUniChallenges} bekannt ist.
    Die Schritte, die ein {\editor} zur Migration dieser Seite durchführen muss,
    sind schnell beschrieben:

    \begin{enumerate}
        \item   Der {\editor} öffnet die Seite des Mitarbeiters in {\wordpress}.
        \item   Der {\editor} legt in {\imperia} ein neues Dokument an.
                Die Wahl einer Vorlage trifft er autonom oder nach festgelegten Kriterien.
        \item   \label{item:problemAnalysisManualMigrationSlectInfoStep}Der {\editor} kopiert den Namen, die Telefonnummer und die E-Mail-Adresse
                des Mitarbeiters einzeln aus dem gemeinsamen Formularfeld in {\wordpress}
                und fügt sie in die vorgesehenen Felder in {\imperia} ein.
                Welche dies sind entscheidet er wiedrum autonom oder aufgrund einer Regel.
        \item   Der {\editor} lädt das Bild des Mitarbeiters von {\wordpress} herunter
                und fügt es in {\imperia} ein.
        \item   Der {\editor} füllt bei Bedarf sonstige Felder in {\imperia},
                für die es in {\wordpress} keine Entsprechung gab.
        \item   Der {\editor} speichert das Dokument und gibt es frei.
    \end{enumerate}

    Prinzipiell kann jeder dieser Schritte über die Schnittstellen von {\wordpress}
    und {\imperia} durch passende Migrationsskripte automatisiert werden.
    Lediglich der \ref{item:problemAnalysisManualMigrationSlectInfoStep}. Schritt
    birgt eine konzeptionelle Herausforderung für eine Automatisierung.
    Nämlich, wie ein Migrationsskript entscheiden soll, was im Formularfeld in {\wordpress}
    der Name, was die Telefonnummer und was die E-Mail-Adresse des Mitarbeiters ist.
    Alles steht in einem gemeinsamen Formularfeld, weshalb die Informationen nicht
    einzeln angesprochen werden können.
    Vor allem unter dem Gesichtspunkt, dass das Feld ein Freitextfeld ist und prinzipiell beliebig gefüllt sein kann.
    Ein Mensch kann diese Zuordnung leicht treffen,
    eine Maschine muss hingegen instrumentiert werden.

    \section{Klassifizierung der Inhalte einer Webseite}
    %\url{http://www.fernuni-hagen.de/KSW/portale/babw/service/}

    %\begin{figure}
    %    \centering
    %    \includegraphics[width=\textwidth]{../resources/babw_service_faq.png}
    %    \caption{FAQ Seite des Studienportals B.A. Bildungswissenschaft}
    %    \label{image:BuildingBlocks}
    %\end{figure}

	%\chapter{Lösungskonzept}
    \label{chapter:SolutionConcept}
    Dieses Kapitel erläutert ein Lösungskonzept für die in Kapitel \ref{chapter:ProblemAnalysis} beschriebene Problemstellung.
    Zentrale Elemente dieser Lösung sind ein System zur automatischen Klassifizierung der Inhalte von Webseiten
    und eine \gls{dsl} zu dessen Instrumentierung.

    \section{Klassen, Features und Selektoren}
    \section{Eine domänenspezifische Sprache zur Spezifikation von Klassen}
    \section{Klassifizierungsalgorithmus}
        \lstinputlisting{../resources/classification.code}
    \section{Persistenz}
    \section{Visualisierung, Nachbesserung und Prüfung der Klassifikation}
    \section{Bereitstellung der Klassifikation zur Weiterverareitung}
    \section{Werkzeug zum Auffinden aller zu klassifizierenden Seiten}
    \section{Architektur}
	\chapter{Lösungsdetails}
    \label{chapter:SolutionDetails}
    Bisher erfolgte lediglich eine oberflächliche Beschreibung des \glspl{wccs},
    um einen übersichtlichen Blick auf die Konzepte und das System als Ganzes
    zu vermitteln.
    Dieses Kapitel beschreibt das \gls{wccs} nun detaillierter,
    um die Lösung und ihre Umsetzung vollständig zu präsentieren.

    % Spezifikation zur Beschreibung von REST Schnittstellen: https://www.openapis.org/

    \section{Sprache zur Klassendefinition}
    \label{section:solutionDetailsDSL}
    Dieses Kapitel beschreibt Details der entwickelten \gls{dsl}.
    Dazu werden zunächst einige Grundlagen zu DSLs sowie
    zur verwendeten Language Workbench Xtext vermittelt.
    Die anschließende Beschreibung der Sprache orientiert sich
    an den Design Dimensionen und Paradigmen von \cite{voelter:DslEngineering}.

    \subsection{Domänenspezifische Sprachen}
    Domänenspezifische Sprachen (engl. Domain specific languages (DSLs))
    sind spezielle Sprachen zum Ausdrücken der Programme einer speziellen
    Problemdomäne \cite[Kapitel 2.2]{voelter:DslEngineering}.

    Im Gegensatz zu \glspl{gpl} sind \glspl{dsl} nicht zwangsläufig Turing-Vollständig
    und deshalb nicht generell austauschbar.
    Stattdessen ist eine \gls{dsl} spezialisiert und optimiert auf eine gegebene Domäne,
    weshalb sie eng mit deren Konzepten und Abstraktionen verbunden ist
    und in Form von speziellen Sprachkonzepten wiederspiegelt.
    Technische und unnötige Details blendet sie aus und überträgt diese Verantwortung
    auf die Execution Engine, d.h. einen Interpreter oder einer Repräsentation in einer
    niedrigeren Sprache, zu der sie transformiert wird.
    Dadurch kann eine \gls{dsl} die Programme ihrer Domäne kürzer und mit einer besseren Semantik
    als \glspl{gpl} ausdrücken \cite[Kapitel 2.2]{voelter:DslEngineering}.

    Durch die Verwendung einer \gls{dsl} steigert sich sowohl die Produktivität
    als auch die Qualität des Ergebnisses,
    da der Entwickler sich stärker auf die eigentliche Problemstellung und weniger
    auf unterstützende technische Aspekte konzentrieren kann.
    Der Quelltext wird dadurch bei gleichbleibender Semantik kürzer und besser les- sowie wartbar.
    Dies wird auch durch die eingeschränkten Möglichkeiten des Entwicklers,
    durch die er weniger Fehler produzieren kann, unterstützt.
    Die stärkere Semantik erleichtert die Implementierung von Analysen und die Formulierung
    hilfreicher Fehlermeldungen.
    Sowohl der Bau als auch Verwendung einer \gls{dsl} fördert das Verständnis der Domäne
    der Entwickler, steigert die Kommunikation und erlaubt die Einbeziehung von Domänenexperten
    \cite[Kapitel 2.5]{voelter:DslEngineering}.
    Aspekte, die auch in \citet{evans:DomainDrivenDesign} Domain Driven Design eine zentrale Rolle spielen.

    % TODO: Vorteile aus SolutionConcept/DSL.tex hier hin? Oder zu Interpretation der Ergebnisse?

    \glspl{dsl} sind natürlich auch mit einigen Herausforderungen verbunden.
    Dazu zählt zunächst der zusätzliche Aufwand, der durch den Entwurf und Bau einer \gls{dsl} entsteht,
    der durch moderne Werkzeuge allerdings verringert werden kann.
    Gleichzeitig erfordert der Entwurf einer guten Sprache Erfahrung,
    da die Domäne nicht nur analysiert und modelliert werden muss,
    sondern auch entschieden werden muss, welche Konzepte in welcher Form in die Sprache übernommen werden.
    Iterative Vorgehen sind hierbei zu empfehlen.
    Die Implementierung einer oftmals nur schwer portiert werden,
    wodurch eine Abhängigkeit zum verwendet Werkzeug entsteht.
    Nicht zuletzt müssen die Nutzer der Sprache diese erlernen,
    was ebenfalls Aufwand erzeugt.
    Gleichzeitig kann sich das parallele Lernen der Domäne und der Sprache positiv auf einander auswirken
    \cite[Kapitel 2.6]{voelter:DslEngineering}.

    \glspl{dsl} sollten generell nicht eingesetzt werden,
    wenn ein kein gemeinsames Verständnis der Domäne und keine Möglichkeit dieses zu erlangen
    existiert.
    Außerdem sollte die Erfahrung der Entwickler in diesem Bereich ausreichend sein,
    um erfolgreich zu sein.

    Eine Umsetzung einer \gls{dsl} kann generell in zwei Formen geschehen:
    Als "`Internal \gls{dsl}"' oder als "`External \gls{dsl}"'.
    Internal \glspl{dsl} nutzen die Sprachmöglichkeiten einer anderen Sprache
    -- meist einer \gls{gpl} --, wodurch sie sich in diese einbetten und
    Programme schnell den Eindruck machen eine ganz neue Sprache zu verwenden.
    Technisch stellen sie aber nur eine geschickte Nutzung der Sprachmöglichkeiten
    einer anderen Sprache, kombiniert mit einigen die Domäne modellierenden Programmierschnittstellen, dar.
    Sie besitzen keinen eigenen Compiler oder Interpreter und stellen deshalb immer gültige Programme
    in der Wirtssprache dar.
    Ihr Vorteil ist ihre vergleichsweise leichte Umsetzung sowie die Möglichkeit das Umfeld der Sprache,
    wie Standardbibliotheken leicht nutzen zu können
    \cite[Kapitel 2.8.1]{voelter:DslEngineering}.

    External \glspl{dsl} stellen hingegen komplett eigenständige Programmiersprachen dar,
    bei denen der Syntax komplett frei definierbar ist,
    wodurch die Domäne oftmals simpler wiedergespiegelt werden kann.
    Sie erfordern allerdings mehr Aufwand, zum Beispiel bei der Definition der Grammatik
    und der Implementierung eines Generators oder Interpreters.

    Bei der entwickelten Sprache handelt es sich um eine externe \gls{dsl}.
    \subsection{Xtext}
    Xtext bla \cite{bettini:xtext}
    \subsection{Sprachkonzepte}
    \label{solutionDetails:dslConcepts}
    Die \gls{wccdl} spiegelt die Konzepte aus
    Kapitel \ref{section:conceptClassesFeaturesSelectors}
    durch Linguistic Abstractions wider.
    Die folgende Beschreibung orientiert sich an den
    Domänenkonzepten.
    
    \paragraph{Klassen}
    Abbildung \ref{image:dslClasses} modelliert,
    wie das Konzept der Klassen in der \gls{wccdl} abgebildet wird.

    \begin{figure}[htb]
        \centering
        \includegraphics[scale=\imageScalingFactor]{../resources/dsl/classes.png}
        \caption{Klassen in der \acrshort{wccdl}}
        \label{image:dslClasses}
    \end{figure}

    Zentraler Bestandteil der Sprache ist das abstrakte Konzept \texttt{Class},
    welches gemäß der Domänenkonzepte in drei konkrete Ausprägungen unterteilt wird:
    \texttt{PageClass}, \texttt{ContentClass} und \texttt{ReferenceClass}.
    Jedes in der \gls{wccdl} geschriebene Programm ist eine Sammlung
    von Klassendefinitionen und enthält deshalb genau ein \texttt{ClassificationModel},
    das wiedrum beliebig viele Instanzen der genannten Konzepte beinhalten kann.
    Klassen besitzen einen Namen und einen Selektor,
    der -- ausgenommen von Seitenklassen -- optional ist.
    Die Sprache enthält hierzu das entsprechende Konzept \texttt{Selector},
    welches weiter unten vorgestellt wird.
    Da nicht jeder Selektor für jede Klasse geeignet ist,
    verwenden die einzelnen Klassentypen verschiedene Selektortypen,
    wobei es sich um \texttt{PageSelector}, \texttt{ContentSelector}
    und \texttt{ReferenceSelector} handelt,
    die ebenfalls später genauer thematisiert werden.
    Wieso der Selektor Teil der Typinformation von \texttt{ContentClass}
    und \texttt{ReferenceClass} ist,
    wird bei der unteren Diskussion der Features ersichtlich.
    Lediglich Seiten- und Inhaltsklassen können Features besitzen,
    weshalb die Sprache das Konzept \texttt{FeatureCapableClass} einführt,
    um diese Eigenschaft auszudrücken.
    \texttt{PageClass} und \texttt{ContentClass} sind
    Ableitungen dieses Konzeptes, welches
    wiederum eine Spezialisierung von \texttt{Class} ist.
    Auf der anderen Seite sind nur Inhalts- und Referenzklassen als Klasse eines Features geeignet.
    Für diese Eigenschaft enthält die Sprache das Konzept \texttt{FeatureClass},
    welches ebenfalls von \texttt{Class} erbt und
    von dem sowohl \texttt{ContentClass} als auch \texttt{ReferenceClass} ableiten.

    \paragraph{Features}
    Die \gls{wccdl} repräsentiert Features über das gleichnamige Sprachkonzept,
    was Abbildung \ref{image:dslFeatures} veranschaulicht.
    Analog zum Domänenkonzept besitzt jedes Feature in der Sprache einen Namen
    und eine Klasse, welche vom oben vorgestellten Typ \texttt{FeatureClass} sein muss.
    Des Weiteren kann ein Feature einen Selektor besitzen,
    um den der Klasse für sich zu überschreiben.
    Da nicht jede Selektorart für Features geeignet ist,
    führt die Sprache das Konzept \texttt{FeatureSelector} ein,
    um eine Unterscheidung zwischen geeigneten und ungeeigneten Arten möglich zu machen.
    Auf \texttt{FeatureSelector} wird weiter unten genauer eingegangen.
    In Bezug auf den Selektor eines Features ist außerdem relevant,
    ob dieser für die Klasse des Features geeignet ist.
    Das heißt, ob der Selektor zum Beispiel ein \texttt{ContentSelector} ist,
    wenn es sich bei der Klasse um eine Inhaltsklasse handelt.
    Dies wird sichergestellt,
    indem der Typ des Selektors Teil der Typinformation von
    \texttt{Feature} ist und sowohl die Eigenschaften \texttt{class} als auch
    \texttt{selector} diesen verwenden und damit zueinanderpassen müssen.
    Ob es sich bei einem Feature um ein {\scalarFeature} oder ein {\collectionFeature} handelt,
    ist aus Sicht der Sprache lediglich eine boolesche Eigenschaft und erfordert
    keine weitere Unterscheidung.
    Das ist eine Abweichung vom Modell des Domänenkonzeptes, die dadurch begründet ist,
    dass die Sprache Features lediglich deklariert und nicht sicherstellen muss,
    dass in einer konkreten Klassifikation alle Elemente eines {\collectionFeature}s die gleiche Klasse verwenden.
    Eine Unterscheidung zwischen {\contentFeature}s und {\referenceFeature}s
    ist ebenfalls nicht notwendig,
    da diese Information klar aus der verwendeten \texttt{FeatureClass} hervorgeht.

    \begin{figure}[tb]
        \centering
        \includegraphics[scale=\imageScalingFactor]{../resources/dsl/features.png}
        \caption{Features in der \acrshort{wccdl}}
        \label{image:dslFeatures}
    \end{figure}

    \paragraph{Selektoren}
    Die \gls{wccdl} bietet des Weiteren Konzepte zur Definition von Selektoren,
    die vereinzelt schon angesprochen wurden.
    Eine vollständige Übersicht bietet Abbildung \ref{image:dslSelectors}.

    \begin{figure}[htb]
        \centering
        \includegraphics[scale=\imageScalingFactor]{../resources/dsl/selectors.png}
        \caption{Selektoren in der WCML}
        \label{image:dslSelectors}
    \end{figure}

    Das abstrakte Konzept \texttt{Selector} entspricht dem gleichnamigen Domänenkonzept
    und unterteilt sich in weitere abstrakte Konzepte,
    mit denen Selektoren nach ihrer Eignung für die verschiedenen Klassentypen getrennt werden.
    Diese Konzepte sind \texttt{PageSelector}, \texttt{ContentSelector} und \texttt{ReferenceSelector}.
    Die vom Klassifizierungssystem unterstützten
    Selektoren\footnote{vgl. Kapitel \ref{section:conceptSupportedSelectors}}
    spiegelt die \gls{wccdl} mit den konkreten Konzepten
    \texttt{CssSelector}, \texttt{XPathSelector} und \texttt{UrlPatternSelector} wider,
    die die zuvor genannten Konzepte spezialisieren.
    Dadurch wird eindeutig abgebildet, welche Selektorarten für welche Klassentypen geeignet sind.
    Ein weiterer relevanter Aspekt ist die Eignung der Selektoren für Features,
    wozu das bereits erwähnte Konzept \texttt{FeatureSelector} existiert.
    Da Features nur Inhalts- oder Referenzklassen verwenden können,
    leiten nur \texttt{ContentSelector} und \texttt{ReferenceSelector} von diesem ab.
    
    \subsection{Strukturelles Design}
    Dieses Kapitel beschreibt einige strukturelle Aspekte der \gls{wccdl},
    die im Wesentlichen auf den Ausführungen in \cite[Kapitel 4, 5.1]{voelter:DslEngineering}
    basieren.

    \paragraph{Separation of Concerns}
    Eine Domäne kann aus verschiedenen Aspekten bestehen,
    die verschiedene Bereiche der Domäne abdecken.
    Alle diese Belange müssen beim Design der Sprache berücksichtigt werden.
    \citet[Kapitel 4.1]{voelter:DslEngineering}
    stellen dazu zwei Ansätze vor: Eine Sprache, die alle Aspekte adressiert
    und mehrere Sprachen, die sich auf jeweils einen Aspekt fokussieren.
    Für den Anwendungsfall der \gls{wccdl} lassen sich die
    folgenden Aspekte identifizieren:

    \begin{enumerate}
        \item Die Deklaration von Klassen, d. h. sie lediglich bekannt zu machen.
        \item Die Definition von Klassen, d. h. die Angabe von Features.
        \item Die Definition von Selektoren für Klassen und Features.
    \end{enumerate}

    Wie aus den bisherigen Ausführungen bereits deutlich wurde,
    deckt die \gls{wccdl} alle diese Aspekte ab,
    da sie sehr eng gekoppelt sind und jeweils einen sehr geringen Umfang besitzen.

    \paragraph{Modularisierung und Sichtbarkeit}
    Die Modularisierung beschreibt die logische Strukturierung
    von Programmelementen und wird durch jede Sprache in irgendeiner Form adressiert.
    Ein Beispiel ist die Strukturierung der Elemente in Namensräume.
    Die Referenzierung eines Elementes sollte immer auf der logischen
    Struktur erfolgen, weshalb die Modularisierung eng mit der Sichtbarkeit
    von Programmelementen verbunden ist.
    Auch dieses Konzept ist in jeder Sprache vertreten und legt fest,
    von wo ein Element referenziert werden kann
    \cite[Kapitel 5.1.1]{voelter:DslEngineering}.
    In Java lässt sich dieser Raum z. B. über die Schlüsselwörter
    \texttt{private} und \texttt{protected} einschränken
    \cite[Kapitel 6.6]{oracle:javaSpec}.
    Die \gls{wccdl} verwendet ein sehr einfaches Konzept bez.
    der Modularisierung und Sichtbarkeit von Porgrammelementen.
    Klassen sind die einzigen Elemente, die in dieser Sprache das Ziel einer Referenz
    sein können, weshalb Features und Selektoren implizit nirgendwo sichtbar sind.
    Klassen sind hingegen global sichtbar,
    sodass sie für jedes Feature genutzt werden können.

    \paragraph{Partitionierung}
    Die Strukturierung eines Programmes in physische Einheiten (Dateien) heißt Partitionierung.
    Da die Referenzierung von Programmelementen nur auf Basis der logischen Struktur erfolgen sollte,
    muss die physische der logischen Strukturierung nicht entsprechen
    \cite[Kapitel 5.1.2]{voelter:DslEngineering}.
    Die \gls{wccdl} erlaubt die Klassendefinitionen auf verschiedenen Dateien
    aufzuteilen, wobei eine Klasse immer vollständig in einer Datei enthalten sein muss.
    Da alle Klassen logisch in einem globalen Namensraum gesammelt werden,
    sind auch Klassen anderer Dateien referenzierbar.

    \paragraph{Scoping und Linking}
    Der Scope eines Querverweises in einem Programm ist die Menge der
    gültigen Ziele dieser Referenz.
    Linking beschreibt den Transformationsprozess vom syntaktischen Baum des Programmes
    zum semantischen Graphen, bei dem Verweise basierend auf den Namen der referenzierten
    Elemente aufgelöst werden
    \cite[Kapitel 8]{voelter:DslEngineering}.
    Querverweise sind in der \gls{wccdl} nur zur Angabe der Klasse eines Features notwendig.
    Der Scope wird an dieser Stelle bereits durch das Sprachkonzept hinreichend
    eingeschränkt, da das Ziel vom Typ
    \texttt{FeatureClass}\footnote{vgl. Kapitel \ref{solutionDetails:dslConcepts}} sein muss.
    Das Linking wird vollständig durch Xtext übernommen
    \cite[Kapitel "`Language Implementation"']{xtext:documentation}.

    \paragraph{Spezifikation und Implementierung}
    Sprachen können die Trennung von Spezifikation und Implementierung von
    Porgrammelementen unterstützen, um eine bessere Entkopplung und verschiedene
    Implementierungen zu ermöglichen \cite[Kapitel 5.1.3]{voelter:DslEngineering}.
    Ein Beispiel aus Java sind Interfaces \cite[Kapitel 9]{oracle:javaSpec}.
    Eine denkbare Anwendung dieser Idee in der \gls{wccdl} ist Selektoren als
    Implementierung zu betrachten. Dann könnte man Klassen inkl. Features einmalig spezifizieren
    und z. B. für verschiedene Sites durch verschiedene Selektoren unterschiedlich "`implementieren"'.
    Die \gls{wccdl} unterstützt eine solche Trennung bisher allerdings nicht.

    \paragraph{Spezialisierung}
    In vielen Sprachen können Programmelemente andere erweitern und somit eine Spezialisierung formulieren.
    In diesen Fällen erbt das konkrete Element alle Eigenschaften des allgemeinen
    und kann deshalb an dessen Stelle verwendet werden
    \cite[Kapitel 5.1.4]{voelter:DslEngineering}.
    Für die \gls{wccdl} wäre dieses Konzept für Klassen denkbar,
    um ihre Selektoren, Features und deren Selektoren zu vererben.
    Sie unterstützt diese Funktion bisher allerdings nicht.

    \paragraph{Typen und Instanzen}
    Die in einer Programmiersprache definierten Typen können bei ihrer Instanziierung
    häufig Parameter entgegen nehmen, wodurch sich ihre Wiederverwendbarkeit steigert.
    Ein Beispiele sind die Parameter eines Konstruktors in objektorientierten Programmiersprachen
    \cite[Kapitel 5.1.5]{voelter:DslEngineering}.
    In der \gls{wccdl} können Features den Selektor ihrer Klasse überschreiben,
    was eine Anwendung dieses Konzeptes ist.

    \paragraph{Superposition und Aspekte}
    \citet[Kapitel 5.1.6]{voelter:DslEngineering} beschreiben zwei
    allgemeine Sprachkonzepte, durch die Programmfragmente zusammengeführt
    oder verändert werden können.
    Superposition vereint mehrere Fragmente anhand eines speziellen Operators.
    Aspekte bieten die Möglichkeit, durch eine Abfrage verschiedene Codestellen
    auszuwählen und aspektspezifisch zu modifizieren.
    Die \gls{wccdl} unterstützt keines dieser Konzepte,
    da sie für ihren Anwendungsfall zu komplex sind und kein sinnvoller Anwendungsfall existiert.

    \paragraph{Sprachmodularität}
    Dieses Konzept beschreibt die Möglichkeit (Teil-)Sprachen wiederzuverwenden
    und zu einer neuen zusammenzusetzen.
    Dadurch kann Konsistenz gewahrt und eine doppelte Implementierung vermieden werden.
    Die verschiedenen Ansätze die \citet[Kapitel 4.6]{voelter:DslEngineering} beschreiben
    sind Language Referencing, Extension, Reuse und Embedding.
    Für die \gls{wccdl} bietet sich die Einbettung anderer Sprachen zur Definition von Selektoren an,
    um Entwickler bei der Formulierung von CSS-, XPath und regulären Ausdrücken zu unterstützen
    und die syntaktische Korrektheit von Selektoren sicherzustellen.
    Bisher setzt die Sprache dies aber nicht um.

    \subsection{Statische Semantik}
    Die statische Semantik einer Sprache beschreibt alle Bedingungen,
    die zum Zeitpunkt der Kompilierung eines Programmes erfüllt sein müssen.
    \citet[Kapitel 4.3]{voelter:DslEngineering} unterteilt sie in zwei Kategorien:
    Constraints und Type System Rules.

    Die \gls{wccdl} besitzt kein komplexes Typsystem,
    da an keiner Stelle Typen berechnet werden müssen.
    Ein Großteil der Typbezogenen Regeln wird über Sprachkonzepte und Syntax (siehe unten)
    sichergestellt.
    Der Selektor einer Klasse hat speziellen Typ.
    Als Feature dürfen nur bestimmte Klassen.

    Allerdings gibt es einige Bedingungen, die nicht über Sprachkonzepte und Syntax gewährleistet werden können.
    Hierfür implementiert die Sprache einige semantische Validierungen.
    Dies sind:
    Sicherstellen, dass Klassennamen global eindeutig sind.
    Sicherstellen, dass die Namen der Features innerhalb einer Klasse eindeutig sind.
    Sicherstellen, dass für jedes Feature ein Selektor ableitbar ist. Entweder über Klasse oder direkt über Feature.
    Falls ein Feature selbst einen Selektor spezifiziert, muss dieser zur Klasse des Features passen.
    Ein Selektor ist nicht leer und besteht nicht nur aus Leerzeichen.
    \subsection{Dynamische Semantik}
    Die dynamische Semantik eines Programmes beschreibt sein Verhalten
    während der Laufzeit. Deshalb wird sie auch als Execution Semantics bezeichnet
    \cite[Kapitel 4.3]{voelter:DslEngineering}.

    Die \gls{wccdl} ist eine rein deklarative Sprache.
    Sie beschreibt keinen Kontrollfluss sondern lediglich
    Klassen, deren Struktur (Features) und Selektoren,
    anhand derer Klassifizierungswerkzeuge Klassen und Features erkennen.
    Wie diese Informationen verarbeitet werden, spielt für die Sprache keine Rolle.

    Entsprechend wird der Programmcode nicht in ein ausführbares Programm übersetzt,
    sondern lediglich in ein anderes Beschreibungsformat
    (siehe Kapitel \ref{section:conceptDslGeneration}).

    Bezüglich des Laufzeitverhaltens trägt die \gls{wccdl}
    deshalb lediglich die Verantwortung eine Klassendefinition
    korrekt zu übersetzen.
    \subsection{Generierung}
    \label{section:solutionDetailsDslGeneration}
    Ein {\classificationModel} wird in ein JSON-Dokument übersetzt,
    wozu ein entsprechender Generator implementiert wurde.

    \paragraph{Klassen}
    Im generierten Zustand werden die Klassendefinitionen weiterhin nach
    Seiten-, Inhalts- und Referenzklassen unterteilt,
    was im JSON-Dokument durch die Schlüssel \texttt{pageClasses},
    \texttt{contentClasses} und \texttt{referenceClasses} geschieht.
    Ihre Objekte enthalten wiederum Schlüssel-Wert-Paare,
    wobei der Schlüssel in diesem Fall der Name einer Klasse und der
    Wert die Klasse selbst ist.
    Klasse speichern ihren Namen zusätzlich in der Eigenschaft \texttt{name},
    sodass sie auch ohne Kontext identifizierbar sind.
    Listing \ref{listing:dlsGenerationPageClass} zeigt dies am Beispiel
    der Seitenklasse \texttt{Service}.
    Inhalts- und Referenzklassen sind sehr ähnlich aufgebaut.

    \lstinputlisting[
        label=listing:dlsGenerationPageClass,
        caption=Ein Beispiel einer Seitenklasse in einem generierten {\classificationModel},
        style=json
    ]{../resources/dsl/generation/page-class.json}

    \paragraph{Selektoren}
    Unter \texttt{selector} ist in Listing \ref{listing:dlsGenerationPageClass}
    der Selektor der Klasse gespeichert,
    was seinen Typ und seinen Wert umfasst.
    Der Selektor einer Klasse ist niemals undefiniert,
    kann im Falle von Inhtals- oder Referenzklassen aber ein leeres Objekt sein.
    Dann ist der Selektor von den Features zu beziehen.
    Der Typ eines Selektors ist immer \texttt{CssSelector},
    \texttt{UrlPatternSelector} oder \texttt{XPathSelector}.
    Im Beispiel ist zu sehen, dass der {\urlSelector} \verb+\/service\/?$+ codiert wurde.
    Andernfalls würden JSON-Parser z. B. die Rückwärtsschrägstriche als besondere Zeichen interpretieren.
    Der JSON-Standard \cite[Kapitel 7]{rfc:8259} spezifiziert einige Zeichen,
    die innerhalb von Zeichenketten so codiert werden müssen.
    Diese Aufgabe übernimmt der entwickelte Generator,
    sodass Nutzer der \gls{wccdl} sich darum nicht kümmern müssen.
    Eine Ausnahme ist allerdings der {\xpathSelector}.
    XPath 1.0 kennt keine Codierung für besondere Zeichen,
    wie z. B. \texttt{\textbackslash{n}} für einen Zeilenumbruch.
    Außerdem ist in dieser Version die Funktion
    \texttt{codepoints-to-string} noch nicht enthalten,
    mit der Unicode Codes in Zeichenketten umgewandelt werden können
    \cite{w3c:xpath}
    \cite[Kapitel 5.2.1]{w3c:xpathXquery}.
    Um trotzdem Inhalte mit besonderen Zeichen beschreiben zu können,
    wird in {\xpathSelector}en der Rückwärtsschrägstrich während der
    Generierung nicht codiert.
    Das heißt, die Sequenz \texttt{\textbackslash{n}} wird genau so in das JSON-Dokument übernommen.
    Ein JSON-Parser setzt das Zeichen zu einem Zeilenumbruch um,
    wodurch der XPath-Ausdruck einen tatsächlichen Umbruch enthält
    der als solcher interpretiert wird.
    Ein konkretes Anwendungsbeispiel für diese Ausnahme im Generator
    ist im ersten Fallbeispiel in Kapitel \ref{section:findingsTeachersClassificationModel} zu sehen.

    \paragraph{Features}
    Seiten- und Inhaltsklassen können in den Eigenschaften
    \texttt{contents} und \texttt{references} Features speichern.
    Die Struktur eines generierten Features zeigt Listing
    \ref{listing:dlsGenerationFeature} am Beispiel des {\contentFeature}s \texttt{pageHeading}.

    \lstinputlisting[
        label=listing:dlsGenerationFeature,
        caption=Ein Beispiel eines Features in einem generierten {\classificationModel},
        style=json
    ]{../resources/dsl/generation/page-heading-feature.json}

    Wie schon bei Klassen wird der Name des Features sowohl als Schlüssel des Objektes
    als auch innerhalb des Objektes selbst verwendet.
    Außerdem besitzt ein Feature die Eigenschaft \texttt{class},
    die den Namen der Klasse des Features speichert. Dabei handelt es sich immer
    um den Namen einer Klasse aus \texttt{contentClasses} oder \texttt{referenceClasses}.
    Das JSON-Dokument ist also in sich geschlossen.
    Darüber hinaus besitzt jedes Feature die Eigenschaft \texttt{isCollection},
    die entweder \texttt{true} oder \texttt{false} sein kann.
    Sie definiert,
    ob es sich um ein {\scalarFeature} oder ein {\collectionFeature} handelt.
    Wie Klassen enthält auch ein Feature einen Selektor.
    Ist dieser leer, ist der Selektor der Klasse zu verwenden.
    {\childFeature}s werden innerhalb eines Features ebenfalls unter \texttt{contens} bzw.
    \texttt{references} abgelegt.
    \subsection{Konkrete Syntax und deren Grammatik}
    Ein wichtiges Merkmal einer Programmiersprache ist ihre konkrete Syntax.
    Dieses Kapitel präsentiert die Syntax der \gls{wccdl} und erläutert ihre Grammatik.    
    Als Einstieg zeigt Listing \ref{listing:dlsExample}
    dazu ein Beispielhaftes {\classificationModel},
    welches in der \gls{wccdl} geschrieben wurde.

    \lstinputlisting[
        label=listing:dlsExample,
        caption=Ein {\classificationModel} in der \acrshort{wccdl},
        style=wccdl,
        inputencoding=utf8/latin1
    ]{../resources/dsl/example.wccd}

    Dieses Beispiel definiert die Seitenklasse \texttt{Service} inkl.
    ihres {\urlSelector}s und den beiden Features \texttt{pageHeading} und \texttt{images}.
    Erstere besitzt die Inhaltsklasse \texttt{PageHeading}, die ebenfalls im Beispiel definiert wird,
    und überschreibt deren Selektor mit einem individuellen {\cssSelector}.
    Das Feature \texttt{image} ist hingegen ein {\referenceFeature} vom Typ
    \texttt{Image} und überschreibt den Selektor ihrer Klasse ebenfalls.

    Die Intention der Syntax ist eine Klassendefinition
    als englischsprachigen Satz lesen zu können.
    Viele Schlüsselwörter -- im Listing blau dargestellt --
    sind deshalb beschreibend und verwenden
    den Indikativ anstelle des Imperativs.  

    Die Beschreibung der Grammatik dieser Syntax erfolgt schrittweise
    und orientiert sich wieder an den Konzepten der
    Domäne\footnote{vgl. Kapitel \ref{section:conceptClassesFeaturesSelectors}}.
    
    \paragraph{Klassen}
    Listing \ref{listing:dlsGrammarClasses} enthält die Parserregeln,
    die die allgemeine Syntax von Klassendefinitionen bestimmen.

    \lstinputlisting[
        label=listing:dlsGrammarClasses,
        caption=Klassen in der Grammatik der \acrshort{wccdl},
        inputencoding=utf8/latin1,
        style=Xtext
    ]{../resources/dsl/grammar/classes.xtext}

    Jede Klassendefinition beginnt mit einer Folge von Schlüsselwörtern,
    die festlegen, um welche Art von Klasse es sich handelt.
    Die Definition einer Seitenklassen beginnt demnach mit \texttt{page class},
    die einer Inhaltsklassen mit \texttt{content class} und die einer
    Referenzklassen mit \texttt{reference class}.
    Anschließend folgt der Name der Klasse, der eine Zeichenkette gemäß
    der Xtext-Terminal-Rule \texttt{ID} sein muss. % TODO: Referenz?
    Dem Namen kann die Definition des Selektors folgen,
    was durch die Schlüsselwortsequenz \texttt{is recognized by} geschieht.
    Jede Klasse referenziert dazu eine andere Parserregel,
    wodurch syntaktisch sichergestellt ist, dass nur semantisch korrekte Selektoren
    verwendet werden.
    Diese Regeln werden später erörtert.
    Bei Inhalts- und Referenzklassen ist der Selektor optional,
    was man im Beispiel an der Klasse \texttt{Image} sieht und
    die Grammatik durch die Kardinalität in den Regeln \texttt{ContentClass}
    und \texttt{ReferenceClass} erlaubt.
    Bei Seitenklassen ist der Selektor hingegen zwingend erforderlich.
    Im Falle von Seiten- und Inhaltsklassen kann ebenfalls optional die Deklaration von Features folgen.
    Bei Referenzen sieht die Grammatik das gemäß der Domäne nicht vor.

    \paragraph{Features}
    Die Deklaration von Features wird durch das Schlüsselwort \texttt{classifies} eingeleitet,
    dem mindestens eine Deklaration folgen muss.
    Der entsprechende Teil der Grammatik ist in Listing \ref{listing:dlsGrammarFeatures} zu sehen.

    \lstinputlisting[
        label=listing:dlsGrammarFeatures,
        caption=Features in der Grammatik der \acrshort{wccdl},
        inputencoding=utf8/latin1,
        style=Xtext
    ]{../resources/dsl/grammar/features.xtext}

    Eine Deklaration eines Features beginnt mit der Angabe des Namens,
    wozu die entsprechenden Parserregeln \texttt{ScalarFeature} und \texttt{CollectionFeature}
    wiederum auf die Regel \texttt{ID} zurückgreifen.    
    Anschließend folgt das Schlüsselwort \texttt{as}, welches der Angabe der Klasse
    des Features vorangeht.
    Soll das Feature ein {\collectionFeature} sein, folgt allerdings zunächst das
    Schlüsselwort \texttt{many} und anschließend erst der Name der Klasse.
    Dazu definieren die beiden Regeln einen Querverweis zu einer \texttt{FeatureClass},
    die ihrerseits zu den oben gezeigten Parserregeln \texttt{ContentClass} oder
    \texttt{ReferenceClass} delegiert.
    Dadurch ist sichergestellt, dass nur Inhalts- oder Referenzklassen für Features verwendet werden können.
    Der Klasse kann ein Selektor folgen,
    was bei {\scalarFeature}s durch \texttt{by} und bei
    {\collectionFeature}s durch \texttt{each by} begonnen wird.
    Der Selektor muss ein \texttt{FeatureSelector} sein,
    worunter alle Selektoren fallen, zu denen die Regeln \texttt{ContentSelector}
    und \texttt{ReferenceSelector} delegieren.
    Welche dies sind, erklärt der nächste Abschnitt.
    Es sei hier noch erwähnt, dass \texttt{FeatureSelector} nicht zu
    \texttt{ContentSelector} und \texttt{ReferenceSelector} delegieren kann,
    was semantisch korrekter wäre,
    weil dadurch eine unauflösbare Doppeldeutigkeit für den Parser entstünde.

    \paragraph{Selektoren}
    Die Syntax von Selektoren ist für Klassen und Features identisch
    und ist durch den Abschnitt der Grammatik in Listing \ref{listing:dlsGrammarSelectors} definiert.

    \lstinputlisting[
        label=listing:dlsGrammarSelectors,
        caption=Selektoren in der Grammatik der \acrshort{wccdl},
        inputencoding=utf8/latin1,
        style=Xtext
    ]{../resources/dsl/grammar/selectors.xtext}

    Die Definition eines Selektors beginnt mit einem Schlüsselwort,
    welches die Art des konkreten Selektors bestimmt.
    In der Grammatik ist das durch die Regeln
    \texttt{CssSelector}, \texttt{UrlPatternSelector} und \texttt{XPathSelector} definiert.
    Demnach beginnt ein Selektor mit \texttt{css}, \texttt{url pattern} oder \texttt{xpath}.
    Anschließend folgt in jedem Fall die Definition des Selektors,
    die durch {\flqq } und {\frqq } umschlossen wird.
    Dazwischen können beliebige Zeichen folgen, wobei die Definition mindestens ein Zeichen umfassen muss.
    Ausgenommen sind nur die Zeichen Backspace, Formfeed, Linebreak, Carriage Return und Tab.
    Anstelle der Xtext-Terminal-Rule \texttt{STRING} wird hier eine individuelle Regel verwendet.
    Die durch \texttt{STRING} definierten Zeichenketten müssen einige Zeichen codieren,
    wie den Rückwärtsschrägstrich oder die einfachen und doppelten Anführungszeichen.
    Das erschwert die Definition von Selektoren, die von diesen Zeichen gebrauch machen.
    Durch die Definition einer eigenen Regel kann die \gls{wccdl} ihren Nutzern diese Komplexität
    ersparen und im Generierungsprozess\footnote{vgl. Kapitel \ref{section:solutionDetailsDslGeneration}} behandeln.

    \subsection{IDE Features}
    Zu einer Sprache gehört auch guter Tool-Support.
    Das heißt Entwickler wollen eine professionelle
    integrierte Entwicklungsumgebung (IDE),
    die coole Features für die Sprache bietet.

    \citet{voelter:DslEngineering} beschreibt folgende Features:
    Code Completion, Syntax Coloring, Go-to-Definition und Find References,
    Pretty Printing, Quick Fixes, Refactoring,Labels und Icons, Outline
    Code Folding, Tooltips, Visualizations, Diff and Merge.
    Vieles davon setzt Xtext von Haus aus um.

    Zusätzlich implementiert wurde die farbliche Hervorhebung von Selektoren,
    sowie die Autovervollständigung für die Klammern, in die Selektoren geschrieben werden,
    weil es dafür keine Taste auf der Tastatur gibt.
    Außerdem sind die Fehlermeldungen für die verschiedenen Validitätsprüfungen aussagekräftig.
    % TODO: Ggf. Screenshots

    \section{Der Classification Service}
    \label{section:solutionDetailsClassificationService}
    Der Classification Service ist die Komponente im \gls{wccs},
    die die tatsächliche Klassifizierung durchführt.
    Dieses Kapitel widmet sich ein paar Details dieses Services
    und beschreibt seine Arbeitsweise.

    \subsection{Funktionen und Schnitstellen}
    \label{section:solutionDetailsClassificationServiceFunctions}
    Die Beschreibung des {\classificationService}s beginnt mit einem
    Überblick über seine Funktionen und Schnittstellen.
    Die Schnittstellenbeschreibungen beschränken sich auf die wichtigsten
    Parameter und betrachten keine Fehlerfälle,
    da eine entsprechende Beschreibung zu umfangreich wäre.

    \paragraph{Klassifizierung}
    Die Hauptaufgabe des {\classificationService}s ist die Klassifizierung einer Webseite
    und die Speicherung der resultierenden Klassifikation.
    Diese Funktion kann über die Schnittstelle in Tabelle \ref{table:startClassificationInterface} genutzt werden.

    \begin{table}[htb]
        \centering
        \begin{tabular}{|l|l|}
        \hline
        \textbf{\gls{url}} & \texttt{http://<HOST>:44284/classifications}\\
        \hline
        \textbf{Methode} & \texttt{POST}\\
        \hline
        \textbf{Eingabedokument} & \lstinputlisting[title=ClassificationServiceTasks]{../resources/classification-service/tasks.json}\\
        \hline
        \textbf{Statuscode} & \texttt{202 (Accepted)}\\
        \hline
        \end{tabular}
        \caption{Die Schnittstelle zum Starten einer Klassifizierung}
        \label{table:startClassificationInterface}
    \end{table}

    Das Eingabedokument spezifiziert die Webseiten,
    die vom {\classificationService} strukturiert werden sollen.
    Neben den \glspl{url} der Webseiten enthält es außerdem Informationen über ihre Site,
    damit in der Webanwendung\footnote{vgl. Kapitel \ref{section:solutionConceptWebApp}}
    eine Zuordnung der klassifizierten Seiten zu ihrer Site möglich ist.
    In einem Dokument können beliebig viele Sites und Webseiten spezifiziert werden.

    \paragraph{{\classificationModel} abfragen}
    Der Service bietet außerdem die Möglichkeit die ihm bekannten Klassen abzufragen,
    was das {\annotatorPlugin}\footnote{vgl. Kapitel \ref{section:conceptWebAnnotations}}
    für seine Korrekturfunktion nutzt.
    Zur Verwendung dieser Servicefunktion stehen drei Schnittstellen bereit,
    die sich nur in einem Teil ihrer \gls{url} unterscheiden und deshalb gemeinsam in
    Tabelle \ref{table:getClassesInterface} dokumentiert sind.

    \begin{table}[htb]
        \centering
        \begin{tabular}{|l|l|}
        \hline
        \textbf{\gls{url}} & \texttt{http://<HOST>:44284/[page | content | reference]-classes}\\
        \hline
        \textbf{Methode} & \texttt{GET}\\
        \hline
        \textbf{Statuscode} & \texttt{200 (OK)}\\
        \hline
        \textbf{Ausgabedokument} & \lstinputlisting[title=ClassificationServiceClasses]{../resources/classification-service/classes.json}\\
        \hline
        \end{tabular}
        \caption{Die Schnittstelle zum Abfragen des {\classificationModel}s}
        \label{table:getClassesInterface}
    \end{table}

    Das Ausgabedokument kapselt die Klassen in einem neuen Dokument,
    welches neben den eigentlichen Klassen in \texttt{classes}
    auch die Anzahl aller Klassen in \texttt{total} enthält.
    Dadurch kann die Schnittstelle zu einem späteren Zeitpunkt z. B. um eine
    seitenweise Abfrage erweitert werden,
    ohne das Ausgabedokument inkompatibel zu früheren Version zu verändern.
    Die Klassen verwenden das Format des generierten
    {\classificationModel}s.

    \subsection{Konfiguration}
    Als Konfigurationsdatei dient das von der \gls{wccdl} genereirte Klassifizierungsmodell.
    Dieses wird beim Start des Services einmalig eingelesen.
    Zum Laden einer neuen Konfiguration muss der Service neugestartet werden.
    
    Eine Alternative wäre gewesen einer Klassifizierungsanfrage das zu nutzende Modell anzufügen.
    Dies hätte allerdings bedeutet, dass Nutzer des Services diese Konfiguration kennen müssen,
    was womöglich nicht der Fall ist, wenn diese z. B. von einer anderen Abteilung geschrieben wurde.
    Außerdem hätte es jede Anfrage unnötig vergrößert.
    Der eigentliche Nutzen dieses Vorgehen ist verschiedene Konfigurationen nutzen zu können.
    Ob dieses Anwendungsfall häufig eintritt bleibt abzuwarten und kann ggf. sogar unerwünscht sein.
    
    Eine spätere Kombination beider Ansätze ist denkbar.
    Genauso wie ein Endpunkt, über den man die Konfiguration setzen kann.
    \section{Die Klassifizierung}
    \subsection{Klassifizierungsergebnis}
        \begin{figure}[htb]
            \centering
            \includegraphics[width=\textwidth]{../resources/concept/page.png}
            \caption{Datenmodell eines Klassifizierungsergebnisses}
            \label{image:conceptPageDataModel}
        \end{figure}
    \subsection{Technologien}
    Der Classification Service ist in JavaScript geschrieben und läuft auf der
    NodeJS Plattform\footnote{\url{https://nodejs.org/}}.
    Sie verwendet die Bibliotheken
    express\footnote{\url{http://expressjs.com/}},
    unirest\footnote{\url{http://unirest.io/nodejs.html}},
    object-hash\footnote{\url{https://github.com/puleos/object-hash}}
    und winston\footnote{\url{https://github.com/winstonjs/winston}}.

    \section{Der {\classificationStorage}}
    \label{section:solutionDetailsPersistence}
    Der {\classificationStorage} realisiert die Datenhaltung des \glspl{wccs}
    und speichert die erzeugten Klassifikationen.
    Er setzt dazu auf eine Graphdatenbank,
    weshalb in diesem Kapitel zunächst eine kurze Einführung
    in dieses Datenbankkonzept erfolgt.
    Anschließend folgen Ausführungen zum verwendeten Datenmodell,
    die durch ein Beispielt ergänzt werden.
    Zuletzt werden noch einige Details zum verwendeten System bereitgestellt.

    \subsection{Graphdatenbanken}
    Zur Familie der NoSQL-Datenbanken gehören auch Graphdatenbanken,
    die Informationen in Form eines gerichteten Graphens speichern.
    Anders als relationale Datenbanken besitzen diese Graphen kein Schema
    und basieren auf dem "`Property Graph Model"'.
    Bei diesem besteht der Graph klassisch aus Knoten und Kanten
    (in diesem Kontext auch "`Beziehungen"'),
    die aber beide eine beliebige Menge an Informationen in Form
    von Schlüssel-Wert-Paaren speichern können.
    Beziehungen sind außerdem benannt, stets gerichtet und haben immer einen
    Star- und einen Endknoten
    \cite[Kapitel 1]{robinson:graphdatabases}.
    In dem vom \gls{wccs} verwendeten Graphdatenbanksystem Neo4J sind Knoten ebenfalls benannt.
    Sie können dazu eine beliebige Menge von "`Labels"' besitzen
    \cite[Kapitel 1.2.1.4]{neo4j:documentation}.

    Anders als in relationalen und vielen anderen NoSQL-Datenbanken,
    sind Beziehungen in Graphdatenbanken also First-Class-Citiziens,
    wodurch ihre Abfrage und Auswertung ohne komplexe Aggregierungsfunktionen möglich ist.
    Dadurch eigenen sie sich besser für verbundene Daten,
    auf denen oft gemeinsam Abfragen geschehen
    \cite[Kapitel 2]{robinson:graphdatabases}
    \cite[Kapitel 11.2]{sadalage:nosql}.

    Weitere Stärken sind
    \cite[Kapitel 1]{robinson:graphdatabases}
    \cite[Kapitel 11.1]{sadalage:nosql}:
    
    \begin{enumerate}
        \item   Auch bei größer werdenden Datenmenge gleichbleibende Performanz,
                da Beziehungen nicht berechnet werden müssen.
        \item   Größere Flexibilität bei der Datenmodellierung da zum Beispiel
                neue Beziehungstypen einfach und ohne Risiko oder Anpassungen eingeführt werden können.
        \item   Gute Integration in agile Entwicklungsmethoden.
    \end{enumerate}
    
    Eine Herausforderung bei der Nutzung von Graphdatenbanken ist die Skalierung,
    da Knoten prinzipiell zu jedem anderen Knoten eine Beziehung erhalten können
    und das Aufteilen der Datenbank auf mehrere Server dadurch erschwert wird
    \cite[Kapitel 11.2.5]{sadalage:nosql}.

    Graphdatenbanken sind aus verschiedenen Gründen geeignet für die Anforderungen des \gls{wccs}.
    Innerhalb einer Datenbank werden verschiedene Klassifikationen gespeichert,
    die verschiedene Schema besitzen können, da sie verschiedene Seitenklassen besitzen können.
    Diese sind darüber hinaus frei definierbar und deshalb aus Sicht der Datenbank unvorhersehbar.
    Bei der Verwendung einer relationalen Datenbank hätte es deshalb zwei Alternativen gegeben:
    
    Pro Seiten-, Inhalts-, und Referenzklasse werden zur Laufzeit nach Bedarf Datenbanktabellen angelegt,
    die das Schema der jeweiligen Klasse wiederspiegeln und über Fremdschlüssel zum Beispiel die Beziehung
    zwischen Parent und Child Feature realisieren.
    Tiefe Klassenstrukturen erfordern bei diesem Ansatz aufwändige JOINS.
    Eine Änderung der Klasse eines Features hieße den Datensatz in eine andere Tabelle zu verschieben
    und alle Fremdschlüssel entsprechend zu aktualisieren.
    Vereinzelte Ausnahmen auf Seiten, wie zum Beispiel zusätzliche Informationen heißt bei diesem Ansatz
    eine Erweiterung der betroffenen Klasse und aller Datensätze.

    Eine Alternative bei relationalen Datenbank wäre die Speicherung der Daten in sehr abstrakten Tabellen
    wie "`Page"' und "`Feature"'.
    Die Beziehung zwischen Parent und Child Feature wäre hierbei in einer weiteren Tabelle gespeichert,
    die Paare aus Schlüsseln der Tabelle Feature speichert, wobei einer das Parent und der andere das
    Child Feature identifiziert.
    Die Auflösung dieser Beziehungen für eine ganze Seite würde sich sehr komplex erweisen.
    Beide Ansätze sind theoretisch denkbar, scheinen aber keine optimale Lösung darzustellen.
    
    Graphdatenbanken haben den Vorteil, dass sie das Netzwerk, welches aus den Verweisen zwischen
    Webseiten entsteht, sehr direkt und natürlich abbilden können.
    Außerdem sind Beziehungen, wie zum Beispiel zwischen Parent und Child Feature,
    sehr leicht auszuwerten, was oben bereits beschrieben wurde.
    Des Weiteren sind Ausnahmen in Seiten leicht zu realisieren,
    da der Graph nur um entsprechende Knoten und Beziehungen erweitert werden muss.

    Dies wird in den folgenden Kapiteln noch deutlicher.
    \subsection{Datenmodell}
    Dieses Kapitel beschreibt, wie eine Klassifikation in der Datenbank modelliert wird.
    Dabei orientiert es sich an dem Vorgehen in https://neo4j.com/developer/guide-data-modeling/
    und geht deshalb zunächst darauf ein, welche Entitäten durch Knoten dargestellt werden.
    Anschließend, wie die Beziehungen aussehen.

    Zur Veranschaulichung zeigt Abbildung \ref{image:dbDataModelOverview} eine Übersicht des Datenmodells.

    \begin{figure}
        \centering
        \includegraphics[width=\textwidth]{../resources/db-data-model/nodes.png}
        \caption{Übersicht der Nodes und ihrer Beziehungen}
        \label{image:dbDataModelOverview}
    \end{figure}

    Entitäten, die als Node umgesetzt werden, sind Seiten, Content Features,
    der Text eines Content Features, referenzierte {\resources} und Sites.
    Diese erhalten die entsprechenden Labels Page, Content, Text, Reference und Site.
    Eine Seite kann als {\resource} existieren, bevor sie selbst klassifiziert wird.
    In diesem Fall erhält der Knoten zum Zeitpunkt der Klassifizierung zusätzlich auch das Label Page.

    Ein Seitenknoten speichert die \gls{url} der Seite, die auch als eindeutiger Identifier innerhalb der Datenbank dient,
    und die Klasse der Seite.

    Der Knoten einer {\resource} enthält lediglich ihre \gls{url}, die wiederum als eindeutiger Schlüssel dient.
    Wie später deutlich wird, kann er für alle Referenzen, die diese {\resource} als Ziel haben, genutzt werden.

    Knoten mit dem Label "`Content"' enthalten eine Prüfsumme über sich selbst, die innerhalb der Datenbank als Schlüssel Verwendung findet.
    Im späteren Verlauf geht Kapitel \ref{section:solutionDetailsStorageAPIStoreClassification} darauf ein,
    wieso diese Eigenschaft des Weiteren benötigt wird.
    Für skalare Content Features existiert exakt ein Knoten.
    Für CollectionFeatures existiert ein Knoten pro Element der Liste.

    Der Text eines Content Featuers wird in einem separaten Knoten gespeichert,
    der neben dieser Information nichts speichert und deshalb auch eindeutig über den Text identifiziert wird.
    Zwischen diesen beiden Knoten existiert deshalb eine Beziehung, die das Label Reads besitzt.
    Jeder Content Knoten ist maximal mit einem Textknoten verbunden.
    Der Grund für die Auslagerung ist, dass es viele Features geben kann, die den gleichen textuellen Inhalt besitzen.
    Die Intention ist diese Eigenschaft im Graphen explizit zu machen,
    sodass sie leicht für komplexere Analysen genutzt werden kann.
    Ein Anwendungsfall ist z. B. Informationen aus einer Datenbank,
    die auf verschiedene Seiten ausgegeben werden.
    Ein Text Knoten hat in diesem Fall viele eingehende Beziehungen,
    sodass leicht herausgefunden werden kann, auf welchen Seiten er enthalten ist.
    Eine explizite Beziehung ist dabei semantisch ausdrucksstärker und effizienter,
    als ein einfacher String-Vergleich, der über mit allen Knoten gemacht wird.
    Durch die Beziehung hat man die Info direkt.
    String-Vergleich ist auch dann nicht mehr sinnvoll, wenn man alle Knoten haben möchte,
    die sich einen Text teilen.
    Dann müsste man sehr viele Vergleiche durchführen.
    Durch die Beziehung ist es einfach nur alle Text Nodes, die mehrere einkommende Kanten haben.

    Zu guter Letzt speichern Site Knoten die ID der Site, wodurch der Knoten eindeutig identifiziert wird.
    Jede Seite kann mit beliebig vielen Page Knoten verbunden sein.
    Diese Beziehungen haben das Label Owns.

    Sowohl Seiten als auch Content Features können Content Features enthalten.
    Eine Seite ist zu jedem Content Feature mit einer eigenen Beziehung verbunden,
    die ebenfalls das Label Owns besitzt.
    Das gleiche gilt für Contents und ihre eigenen Content Features.
    Eine genauere Darstellung dieser Beziehung bietet Abbildung \ref{image:dbDataModelContentRelationship}.

    \begin{figure}
        \centering
        \includegraphics[width=\textwidth]{../resources/db-data-model/content-relationship.png}
        \caption{Content Features}
        \label{image:dbDataModelContentRelationship}
    \end{figure}

    Eine Beziehung zwischen Page und Content bzw. Content und Content
    wird hier als FeatureRelationship bezeichnet.
    Eine solche Beziehung besitzt eine Reihe von Eigenschaften.
    Sie speichert den Namen des Features (name),
    ob es sich um ein Element eines CollectionFeatures handelt (isCollection)
    und den eindeutigen Selektor des Nodes, der zum Feature gehört
    (start-, endSelectorType; start-, endSelectorValue; start- endSelectorOffset, ).
    Bei Collection Features existieren viele ausgehende Kanten für ein Feature.
    Jede dieser Beziehungen speichert den gleichen Namen und hat isCollection auf true gesetzt.
    Es ist sinnvoll diese Informationen nicht im Content Knoten, sondern in der Beziehung zu speichern,
    um den Knoten besser wiederverwendbar zu machen.
    Details werden in Kapitel \ref{section:solutionDetailsStorageAPIStoreClassification} erklärt.

    Referenzen werden in der Datenbank durch eine Kombination aus
    {\resource} Knoten und Beziehungen zu diesen Knoten dargestellt.
    Page und Content Knoten können demanch eine ausgehende Beziehung zu einem
    {\resource} Knoten haben, die mit References markiert ist.
    Abbildung \ref{image:dbDataModelResourceRelationship} stellt diese Beziehung in den Fokus.

    \begin{figure}
        \centering
        \includegraphics[width=\textwidth]{../resources/db-data-model/resource-relationship.png}
        \caption{Reference Features}
        \label{image:dbDataModelResourceRelationship}
    \end{figure}

    Wie zu sehen ist, handelt es sich bei einer ReferenceRelationship ebenfalls
    um eine FeatureRelationship, weshalb sie ebenfalls die oben beschriebenen Informationen speichert.
    Zusätzlich enthält sie aber auch die Klasse der Referenz.
    Diese kann nicht im {\resource} Knoten gespeichert werden,
    da der Knoten für viele Referenzen dienen kann und die Klasse nicht überall identisch sein muss.
    \subsection{Beispiel}
    \label{section:solutionDetailPersistenceDataModelExample}
    Um das eben beschriebene Datenmodell greifbarer zu machen,
    präsentiert dieses Kapitel, wie eine konkrete Klassifikation als
    Graph im {\classificationStorage} gespeichert wird.
    Spätere Erklärungen nehmen auf dieses Beispiel ebenfalls Bezug.

    Eine Übersicht der Knoten des Beispielgraphs wurde aus Gründen
    der Übersichtlichkeit auf die Abbildungen \ref{image:dbDataModelExampleOverviewPart1}
    und \ref{image:dbDataModelExampleOverviewPart2} verteilt.
    Sie sind über den \texttt{Content}-Knoten \texttt{c5} verbunden
    und verzichten zunächst auf die ausführliche Darstellung der Eigenschaften der Beziehungen.

    \begin{figure}[htb]
        \centering
        \includegraphics[scale=\imageScalingFactor]{../resources/db-data-model/example/example_part1.png}
        \caption{Beispiel eines Graphs im {\classificationStorage} (1)}
        \label{image:dbDataModelExampleOverviewPart1}
    \end{figure}

    Die betroffene Webseite wurde als "`GettingStarted"' klassifiziert und besitzt drei skalare {\contentFeature}s,
    für die \texttt{Content}-Knoten existieren.
    Der \texttt{Page}-Knoten ist mit ihnen über \texttt{Owns}-Beziehungen verbunden.
    Die Inhalte \texttt{c1} und \texttt{c2} beinhalten Text und sind deshalb mit entsprechenden \texttt{Text}-Knoten verbunden.
    Hingegen hat \texttt{c3} zwei {\childFeature}s (\texttt{c4} und \texttt{c5}), die seinen Text feingranular speichern
    und deshalb ihrerseits mit \texttt{Text}-Knoten verbunden sind.

    In Abbildung \ref{image:dbDataModelExampleOverviewPart2} ist zu sehen,
    dass \texttt{c5} neben seinem Text auch zwei {\resources} referenziert,
    die durch entsprechende \texttt{Resource}-Knoten dargestellt werden.

    \begin{figure}[htb]
        \centering
        \includegraphics[scale=\imageScalingFactor]{../resources/db-data-model/example/example_part2.png}
        \caption{Beispiel eines Graphs im {\classificationStorage} (2)}
        \label{image:dbDataModelExampleOverviewPart2}
    \end{figure}

    Bei diesen Referenzen handelt es sich um ein {\collectionFeature},
    was aus Abbildung \ref{image:dbDataModelExampleRel10} hervorgeht,
    die eine Beziehung zu einem \texttt{Resource}-Knoten detailliert darstellt.

    \begin{figure}[htb]
        \centering
        \includegraphics[scale=\imageScalingFactor]{../resources/db-data-model/example/c5-r2.png}
        \caption{Beispiel einer Beziehung zu einer {\resource} im {\classificationStorage}}
        \label{image:dbDataModelExampleRel10}
    \end{figure}

    Die Eigenschaft \texttt{isCollection} besitzt nämlich den Wert \texttt{true}.
    Des Weiteren wird hier deutlich, wie die Informationen des eindeutigen Selektors gespeichert werden.
    Wie zuvor beschreiben\footnote{vgl. Kapitel \ref{section:solutionDetailsClassificationServiceClassification}}
    handelt es sich dabei um zwei XPath-Ausdrücke und um zwei Versatzangaben.

    Dem Datenmodell folgend ist eine konkrete Beziehung zu einem \texttt{Content}-Knoten
    sehr ähnlich aufgebaut, was aus Abbildung \ref{image:dbDataModelExampleRel1} hervorgeht.
    Sie verzichtet lediglich auf die Speicherung der Klasse.

    \begin{figure}[htb]
        \centering
        \includegraphics[scale=\imageScalingFactor]{../resources/db-data-model/example/p-c1.png}
        \caption{Beispiel einer Beziehung zu einem {\contentFeature} im {\classificationStorage}}
        \label{image:dbDataModelExampleRel1}
    \end{figure}

    \subsection{Teilen von Knoten in der Datenbank}
    Der erste Beweggrund für die Verwendung einer Graphdatenbank
    ist die natürliche Modellierung und Speicherung von Verweisen
    zwischen Seiten und auf sonstige {\resources}.
    Des Weiteren sollte durch die Vermeidung von Duplikaten in der Datenbank
    und der Möglichkeit einen Knoten mehrfach zu referenzieren,
    die Möglichkeit entstehen weiterführende Analysen auf den klassifizierten
    Inhalten auszuführen.

    Das zweite Beispiel hat gezeigt,
    wie Referenzen zwischen Seiten auf sehr einfache Art und Weise
    die Reihenfolge der Übersichtsseiten und Navigationspfade explizit machen.
    Eine Auswertung ist trivial, da lediglich ein- und ausgehenden Kanten gefolgt werden muss.
    Andere Datenbankmodelle hätten komplexere Konstrukte erfordert,
    um diese Informationen zu speichern oder hätten komplexere
    Aggregierungsschritte zur Auswertung erfordert.

    Betrachtet man Tabele \ref{table:findingsTeachersFiguresNodesByLabel} wird deutlich,
    dass Text- und Resource-Knoten verhältnismäßig am meisten von der
    Möglichkeit Knoten wiederzuverwenden profitieren.
    Das ist nicht verwunderlich, da sie jeweils genau einen Wert speichern,
    der sie identifiziert und keine ausgehenden Kanten besitzen,
    die seitenspezifische Informationen enthalten.
    Wie Tabelle \ref{table:findingsTeachersFiguresSharedNodes} belegt,
    steigt ihre Wiederverwendung bei mehreren Klassifikationen in einer Datenbank,
    was eine logische Konsequenz der größeren Informationsmenge ist.

    Für Resource-Knoten ist dies aber nicht sofort ersichtlich,
    zum Beispiel die Zahl der mehrfach referenzierten
    Lehrgebiets-Resource-Knoten im Falle einer gemeinsamer Datenbank niedriger ist,
    als die Summe der geteilten Knoten in einzelnen Datenbanken (30 vs. 36).
    Gleichzeitig ist die Zahl der geteilten SubjectArea-Knoten aber höher,
    die jeder einen Lehrgebietsknoten referenzieren.
    Es wird also lediglich ein größerer Teilbaum geteilt.
    Außerdem enthielten die einzelnen Datenbanken identische Resource-Knoten,
    die in der gemeinsamen natürlich zusammengefasst werden konnten.

    Resource-Knoten innerhalb einer Datenbank zu duplizieren
    macht aus semantischen Gründen keinen Sinn,
    da sie eine Entität der Domäne darstellen,
    die genau ein mal existiert, was die Datenbank wiederspiegeln sollte
    und außerdem die Möglichkeiten eines Graphens besser ausschöpft.

    Nicht so eindeutig ist dies allerdings bei Text-Knoten.
    Aus Tabelle \ref{table:findingsTeachersFiguresSharedNodes} geht hervor,
    dass im ersten Fallbeispiel niemals ein Text-Knoten geteilt wurde.
    Stattdessen konnte immer der Content-Knoten,
    der ihn referenziert geteilt werden.
    Deshalb ist die Zahl beider Knoten-Typen gesunken und die der geteilten
    Content Knoten dafür gesteigen.
    Das ist möglich, wenn die entsprechenden Content-Knoten
    sich sowohl in ihrer Klasse als auch in ihren Features nicht unterscheiden.
    Betrachtet man die Klassen der geteilten Content Knoten wird deutlich,
    dass diese keine Features haben.

    Anders ist das beim zweiten Fallbeispiel.
    Dort wurden Text-Knoten geteilt, weil es mehrere News mit derselben ÜBerschrift
    gab, die aber jeweils eine eigene Detailseite referenzieren.
    % TODO: Referenz auf TABELLE!

    
% Oder: Verwendung einer Graphdatenbank

    \section{Classification Storage API}
    \label{section:solutionDetailsStorageAPI}
    Die Classification Storage API ist eine fachliche Schnittstelle zur Datenhaltung des \gls{wccs}.
    Die folgenden Ausführungen thematisieren die Funktionen dieser Komponente und erläutern die wichtigste Umsetzungsdetails.

    \subsection{Funktionen und Schnitstellen}
    \label{section:solutionDetailsClassificationServiceFunctions}
    Die Beschreibung des {\classificationService}s beginnt mit einem
    Überblick über seine Funktionen und Schnittstellen.
    Die Schnittstellenbeschreibungen beschränken sich auf die wichtigsten
    Parameter und betrachten keine Fehlerfälle,
    da eine entsprechende Beschreibung zu umfangreich wäre.

    \paragraph{Klassifizierung}
    Die Hauptaufgabe des {\classificationService}s ist die Klassifizierung einer Webseite
    und die Speicherung der resultierenden Klassifikation.
    Diese Funktion kann über die Schnittstelle in Tabelle \ref{table:startClassificationInterface} genutzt werden.

    \begin{table}[htb]
        \centering
        \begin{tabular}{|l|l|}
        \hline
        \textbf{\gls{url}} & \texttt{http://<HOST>:44284/classifications}\\
        \hline
        \textbf{Methode} & \texttt{POST}\\
        \hline
        \textbf{Eingabedokument} & \lstinputlisting[title=ClassificationServiceTasks]{../resources/classification-service/tasks.json}\\
        \hline
        \textbf{Statuscode} & \texttt{202 (Accepted)}\\
        \hline
        \end{tabular}
        \caption{Die Schnittstelle zum Starten einer Klassifizierung}
        \label{table:startClassificationInterface}
    \end{table}

    Das Eingabedokument spezifiziert die Webseiten,
    die vom {\classificationService} strukturiert werden sollen.
    Neben den \glspl{url} der Webseiten enthält es außerdem Informationen über ihre Site,
    damit in der Webanwendung\footnote{vgl. Kapitel \ref{section:solutionConceptWebApp}}
    eine Zuordnung der klassifizierten Seiten zu ihrer Site möglich ist.
    In einem Dokument können beliebig viele Sites und Webseiten spezifiziert werden.

    \paragraph{{\classificationModel} abfragen}
    Der Service bietet außerdem die Möglichkeit die ihm bekannten Klassen abzufragen,
    was das {\annotatorPlugin}\footnote{vgl. Kapitel \ref{section:conceptWebAnnotations}}
    für seine Korrekturfunktion nutzt.
    Zur Verwendung dieser Servicefunktion stehen drei Schnittstellen bereit,
    die sich nur in einem Teil ihrer \gls{url} unterscheiden und deshalb gemeinsam in
    Tabelle \ref{table:getClassesInterface} dokumentiert sind.

    \begin{table}[htb]
        \centering
        \begin{tabular}{|l|l|}
        \hline
        \textbf{\gls{url}} & \texttt{http://<HOST>:44284/[page | content | reference]-classes}\\
        \hline
        \textbf{Methode} & \texttt{GET}\\
        \hline
        \textbf{Statuscode} & \texttt{200 (OK)}\\
        \hline
        \textbf{Ausgabedokument} & \lstinputlisting[title=ClassificationServiceClasses]{../resources/classification-service/classes.json}\\
        \hline
        \end{tabular}
        \caption{Die Schnittstelle zum Abfragen des {\classificationModel}s}
        \label{table:getClassesInterface}
    \end{table}

    Das Ausgabedokument kapselt die Klassen in einem neuen Dokument,
    welches neben den eigentlichen Klassen in \texttt{classes}
    auch die Anzahl aller Klassen in \texttt{total} enthält.
    Dadurch kann die Schnittstelle zu einem späteren Zeitpunkt z. B. um eine
    seitenweise Abfrage erweitert werden,
    ohne das Ausgabedokument inkompatibel zu früheren Version zu verändern.
    Die Klassen verwenden das Format des generierten
    {\classificationModel}s.


    \subsection{Klassifikation speichern}
        \label{section:solutionDetailsStorageAPIStoreClassification}
        \lstinputlisting[label=listing:storeClassification,caption=Algorithmus zum Speichern]{../resources/store-classification.code}

    \subsection{Technologien}
    \section{Annotation Service}
    \begin{table}[htb]
        \centering
        \begin{tabular}{|l|l|}
        \hline
        \textbf{Endpunkt}     & /pages/\{pageId\}\\
        \hline
        \textbf{Methode}      & GET\\
        \hline
        \textbf{Beschreibung} & Liefert Annotator Storage API version\\
        \hline
        \textbf{Status}       & 200\\
        \hline
        \textbf{Antwort}      & \{ ``name'': ``Annotator Store API'', ``version'': ``2.0.0'' \}\\
        \hline
        & \\
        \hline
        \textbf{Endpunkt}     & /pages/\{pageId\}/annotations\\
        \hline
        \textbf{Methode}      & GET\\
        \hline
        \textbf{Beschreibung} & Liefert alle Annotationen einer Seite\\
        \hline
        \textbf{Status}       & 200\\
        \hline
        \textbf{Antwort}      & \{ ``name'': ``Annotator Store API'', ``version'': ``2.0.0'' \}\\
        \hline
        \end{tabular}
        \caption{My caption}
        \label{my-label}
    \end{table}
    % Schnittstellenbeschreibung:
    % - Endpunkt
    % - Methoden
    % -- Input-Dokument
    % -- Status Codes der Antwort
    % -- Return Dokument für jede Antwort
    \section{Annotator Plugin}
    \label{section:solutionDetailsAnnotatorPlugin}
    Dieses Kapitel stellt das Plugin für die Web Annotationen Bibliothek
    Annotator\footnote{vgl. \url{http://annotatorjs.org/}} vor.
    Diese Bibliothek ist ein Open Source Produkt und besitzt ein Plugin System,
    durch das eine Erweiterung vereinfacht wird.
    Anders als vergleichbare Produkte bietet es außerdem den Vorteil
    On-Premises, das heißt selbst gehostet, betrieben werden zu können
    und nicht nur in der Cloud.

    Folgend werden die Funktionen des entwickelten Plugins
    sowie die seine Einbindung beschrieben.

    \subsection{Funktionen und Schnitstellen}
    \label{section:solutionDetailsClassificationServiceFunctions}
    Die Beschreibung des {\classificationService}s beginnt mit einem
    Überblick über seine Funktionen und Schnittstellen.
    Die Schnittstellenbeschreibungen beschränken sich auf die wichtigsten
    Parameter und betrachten keine Fehlerfälle,
    da eine entsprechende Beschreibung zu umfangreich wäre.

    \paragraph{Klassifizierung}
    Die Hauptaufgabe des {\classificationService}s ist die Klassifizierung einer Webseite
    und die Speicherung der resultierenden Klassifikation.
    Diese Funktion kann über die Schnittstelle in Tabelle \ref{table:startClassificationInterface} genutzt werden.

    \begin{table}[htb]
        \centering
        \begin{tabular}{|l|l|}
        \hline
        \textbf{\gls{url}} & \texttt{http://<HOST>:44284/classifications}\\
        \hline
        \textbf{Methode} & \texttt{POST}\\
        \hline
        \textbf{Eingabedokument} & \lstinputlisting[title=ClassificationServiceTasks]{../resources/classification-service/tasks.json}\\
        \hline
        \textbf{Statuscode} & \texttt{202 (Accepted)}\\
        \hline
        \end{tabular}
        \caption{Die Schnittstelle zum Starten einer Klassifizierung}
        \label{table:startClassificationInterface}
    \end{table}

    Das Eingabedokument spezifiziert die Webseiten,
    die vom {\classificationService} strukturiert werden sollen.
    Neben den \glspl{url} der Webseiten enthält es außerdem Informationen über ihre Site,
    damit in der Webanwendung\footnote{vgl. Kapitel \ref{section:solutionConceptWebApp}}
    eine Zuordnung der klassifizierten Seiten zu ihrer Site möglich ist.
    In einem Dokument können beliebig viele Sites und Webseiten spezifiziert werden.

    \paragraph{{\classificationModel} abfragen}
    Der Service bietet außerdem die Möglichkeit die ihm bekannten Klassen abzufragen,
    was das {\annotatorPlugin}\footnote{vgl. Kapitel \ref{section:conceptWebAnnotations}}
    für seine Korrekturfunktion nutzt.
    Zur Verwendung dieser Servicefunktion stehen drei Schnittstellen bereit,
    die sich nur in einem Teil ihrer \gls{url} unterscheiden und deshalb gemeinsam in
    Tabelle \ref{table:getClassesInterface} dokumentiert sind.

    \begin{table}[htb]
        \centering
        \begin{tabular}{|l|l|}
        \hline
        \textbf{\gls{url}} & \texttt{http://<HOST>:44284/[page | content | reference]-classes}\\
        \hline
        \textbf{Methode} & \texttt{GET}\\
        \hline
        \textbf{Statuscode} & \texttt{200 (OK)}\\
        \hline
        \textbf{Ausgabedokument} & \lstinputlisting[title=ClassificationServiceClasses]{../resources/classification-service/classes.json}\\
        \hline
        \end{tabular}
        \caption{Die Schnittstelle zum Abfragen des {\classificationModel}s}
        \label{table:getClassesInterface}
    \end{table}

    Das Ausgabedokument kapselt die Klassen in einem neuen Dokument,
    welches neben den eigentlichen Klassen in \texttt{classes}
    auch die Anzahl aller Klassen in \texttt{total} enthält.
    Dadurch kann die Schnittstelle zu einem späteren Zeitpunkt z. B. um eine
    seitenweise Abfrage erweitert werden,
    ohne das Ausgabedokument inkompatibel zu früheren Version zu verändern.
    Die Klassen verwenden das Format des generierten
    {\classificationModel}s.

    \subsection{Einbindung des Plugins}
    \label{section:solutionDetailsAnnotatorPluginIntegration}
    Das {\annotatorPlugin} kann in jede Webseite eingebunden werden,
    die Annotator verwendet.
    Dazu sind die folgenden Schritte notwendig.

    \paragraph*{Einbinden von Bibliotheken und Styledefinitionen}
    Zunächst muss gewährleistet sein, dass die
    Annotator Bibliothek
    und die Annotator Stylesheets
    \cite[Kapitel "`Getting started with Annotator"']{annotator:documentation}
    korrekt in die Seite eingebunden sind.
    Des Weiteren muss das Annotator Plugin über ein \texttt{script}-Element
    im \texttt{head}-Element der Webseite eingebunden werden.
    Anschließend sollten die Styledefinitionen aus Listing \ref{listing:annotatorCustomStyles}
    dem \texttt{head}-Element der Webseite hinzugefügt werden,
    da Annotator die Struktur der Seite bearbeitet und andernfalls Designprobleme auftreten können.

    \lstinputlisting[
        label=listing:annotatorCustomStyles,
        caption=Zusätzliche Styledefinitionen zur Nutzung des {\annotatorPlugin}s,
        style=css
    ]{../resources/annotator-plugin/custom-styles.css}

    \paragraph*{Annotator Initialisierung}
    Anschließend muss der Initialisierungsaufruf von Annotator ergänzt werden,
    sodass es das neue Plugin verwendet.
    Listing \ref{listing:annotatorInitialization} zeigt dazu ein Beispiel.

    \lstinputlisting[
        label=listing:annotatorInitialization,
        caption=Die Initialisierung von Annotator zur Nutzung des {\annotatorPlugin}s,
        style=js
    ]{../resources/annotator-plugin/annotator-initialize.js}

    Nachdem Annotator auf dem \texttt{body}-Element der Webseite initialisiert wurde,
    werden die verschiedenen Plugins registriert.
    Begonnen wird mit dem
    Storage Plugin \cite[Kapitel "`Plugins"']{annotator:documentation},
    welches für den {\annotationService} konfiguriert wird.
    Entsprechend der vorgestellten Schnittstelle dieser
    Komponente
    legt die Konfiguration des Plugins den Präfix von Anfrage-\glspl{url} fest.
    Anschließend wird das {\annotatorPlugin} über den Schlüssel \texttt{wccs} inkludiert.
    Zuletzt konfiguriert das Skript das
    Permissions Plugin \cite[Kapitel "`Plugins"']{annotator:documentation}.

    \section{Webanwendung}
    In diesem Kapitel steht die Webanwendung zur ausführlicheren Einsicht
    und Prüfungen von Klassifikationen im Mittelpunkt.

    \subsection{Funktionen und Schnitstellen}
    \label{section:solutionDetailsClassificationServiceFunctions}
    Die Beschreibung des {\classificationService}s beginnt mit einem
    Überblick über seine Funktionen und Schnittstellen.
    Die Schnittstellenbeschreibungen beschränken sich auf die wichtigsten
    Parameter und betrachten keine Fehlerfälle,
    da eine entsprechende Beschreibung zu umfangreich wäre.

    \paragraph{Klassifizierung}
    Die Hauptaufgabe des {\classificationService}s ist die Klassifizierung einer Webseite
    und die Speicherung der resultierenden Klassifikation.
    Diese Funktion kann über die Schnittstelle in Tabelle \ref{table:startClassificationInterface} genutzt werden.

    \begin{table}[htb]
        \centering
        \begin{tabular}{|l|l|}
        \hline
        \textbf{\gls{url}} & \texttt{http://<HOST>:44284/classifications}\\
        \hline
        \textbf{Methode} & \texttt{POST}\\
        \hline
        \textbf{Eingabedokument} & \lstinputlisting[title=ClassificationServiceTasks]{../resources/classification-service/tasks.json}\\
        \hline
        \textbf{Statuscode} & \texttt{202 (Accepted)}\\
        \hline
        \end{tabular}
        \caption{Die Schnittstelle zum Starten einer Klassifizierung}
        \label{table:startClassificationInterface}
    \end{table}

    Das Eingabedokument spezifiziert die Webseiten,
    die vom {\classificationService} strukturiert werden sollen.
    Neben den \glspl{url} der Webseiten enthält es außerdem Informationen über ihre Site,
    damit in der Webanwendung\footnote{vgl. Kapitel \ref{section:solutionConceptWebApp}}
    eine Zuordnung der klassifizierten Seiten zu ihrer Site möglich ist.
    In einem Dokument können beliebig viele Sites und Webseiten spezifiziert werden.

    \paragraph{{\classificationModel} abfragen}
    Der Service bietet außerdem die Möglichkeit die ihm bekannten Klassen abzufragen,
    was das {\annotatorPlugin}\footnote{vgl. Kapitel \ref{section:conceptWebAnnotations}}
    für seine Korrekturfunktion nutzt.
    Zur Verwendung dieser Servicefunktion stehen drei Schnittstellen bereit,
    die sich nur in einem Teil ihrer \gls{url} unterscheiden und deshalb gemeinsam in
    Tabelle \ref{table:getClassesInterface} dokumentiert sind.

    \begin{table}[htb]
        \centering
        \begin{tabular}{|l|l|}
        \hline
        \textbf{\gls{url}} & \texttt{http://<HOST>:44284/[page | content | reference]-classes}\\
        \hline
        \textbf{Methode} & \texttt{GET}\\
        \hline
        \textbf{Statuscode} & \texttt{200 (OK)}\\
        \hline
        \textbf{Ausgabedokument} & \lstinputlisting[title=ClassificationServiceClasses]{../resources/classification-service/classes.json}\\
        \hline
        \end{tabular}
        \caption{Die Schnittstelle zum Abfragen des {\classificationModel}s}
        \label{table:getClassesInterface}
    \end{table}

    Das Ausgabedokument kapselt die Klassen in einem neuen Dokument,
    welches neben den eigentlichen Klassen in \texttt{classes}
    auch die Anzahl aller Klassen in \texttt{total} enthält.
    Dadurch kann die Schnittstelle zu einem späteren Zeitpunkt z. B. um eine
    seitenweise Abfrage erweitert werden,
    ohne das Ausgabedokument inkompatibel zu früheren Version zu verändern.
    Die Klassen verwenden das Format des generierten
    {\classificationModel}s.

    \subsection{Technologien}
    Der Classification Service ist in JavaScript geschrieben und läuft auf der
    NodeJS Plattform\footnote{\url{https://nodejs.org/}}.
    Sie verwendet die Bibliotheken
    express\footnote{\url{http://expressjs.com/}},
    unirest\footnote{\url{http://unirest.io/nodejs.html}},
    object-hash\footnote{\url{https://github.com/puleos/object-hash}}
    und winston\footnote{\url{https://github.com/winstonjs/winston}}.
    \section{WordPress Crawler}
    \label{section:solutionDetailsCrawler}
    Der {\wordpress}-Crawler ist ein Kommandozeilenwerkzeug,
    welches die REST-API von {\wordpress} verwendet,
    % TODO: Referenz: https://developer.wordpress.org/rest-api/
    um die \glspl{url} aller Posts und Pages beliebig vieler Sites zu ermitteln.
    Für die gesammelten \glspl{url} initiiert er die Klassifizierung,
    indem er entsprechende Aufträge an den
    Classification Service\footnote{vgl. Kapitel \ref{section:solutionDetailsClassificationService}}
    richtet.
    
    Dieses Kapitel beschreibt, wie dieses Werkzeug konfiguriert und gestartet wird
    und vermittelt einen Überblick über die verwendeten Technologien.

    \subsection{Konfiguration und Aufruf}
    Der Crawler benötigt einige Informationen,
    die ihm beim Start in einer Konfigurationsdatei mitgeteilt werden.
    Die Konfigurationsparameter lassen sich in drei Kategorien einteilen:
    Classification Service, Sites und Crawling.

    \paragraph{Classification Service Parameter}
    Der Crawler soll die Klassifizierung der gefundenen Seiten anstoßen,
    weshalb er die \gls{url} des Classification Services benötigt.
    Die verfügbaren Parameter sind in Tabelle \ref{table:crawlerClassificationServiceParameter} aufgeführt.
    Jeder dieser Parameter ist optional und besitzt einen Standardwert,
    der genutzt wird, falls er nicht durch die Konfiguration überschrieben wird.

    \begin{table}[]
        \centering
        \begin{tabular}{|l|l|}
            \hline
            \textbf{Parameter} & \textbf{Standardwert}\\
            \hline
            scheme & http \\
            \hline
            host & localhost \\
            \hline
            port & 44284 \\
            \hline
            path & /classifications \\
            \hline
            \end{tabular}
        \caption{Konfigurationsparameter des Crawlers: Classification Service}
        \label{table:crawlerClassificationServiceParameter}
    \end{table}

    Der Crawler baut basierend auf den Angaben die \gls{url} des Classification Services nach
    folgendem Schema auf: \texttt{<scheme>://<host>:<port>:<path>}.

    \paragraph{Sites Parameter}
    Neben Informationen zum Classification Service benötigt der Crawler auch Informationen zu den Sites,
    die er durchsuchen soll.
    Es können deshalb beliebig viele Sites durch die Angabe einer Id,
    eines Namens und vor allem einer Basis-\gls{url} konfiguriert werden.

    \paragraph{Crawling Parameter}
    Die Kategorie "`Crawling"' enthält zwei Parameter:
    "`resultPageSize"' und "`maxConcurrentRequests"'.
    Der erste Parameter legt die Anzahl der Posts und Pages fest,
    die in einer Anfrage von {\wordpress} angefordert werden.
    Der Standardwert dieses Parameters ist 8.
    Der Crawler führt entsprechend viele Anfragen durch,
    um alle Seiten zu beziehen.
    Seiten die er ermittelt hat, leitet er direkt zum Classification Service weiter,
    anstatt erst alle Seiten zu sammeln.
    Der zweite Parameter bestimmt die Anzahl der gleichzeiten Anfragen,
    die an {\wordpress} gerichtet werden.
    Hier liegt der Standardwert bei 5.
    Beide Parameter dienen der Erzeugung eines guten Flusses zwischen Anfragen an
    {\wordpress} und den Classification Service.
    Außerdem stellen sie sicher, dass {\wordpress} durch die Anfragen des Crawlers
    nicht überlastet wird und keine anderen Anfragen mehr bearbeiten kann.

    \paragraph{Beispielkonfiguration}
    Eine Konfiguration muss in YAML\footnote{vgl. \url{http://yaml.org}} geschrieben.
    Ein Beispiel ist in Listing \ref{listing:crawlerConfiguration} zu sehen.

    \lstinputlisting[
        label=listing:crawlerConfiguration,
        caption=Beispielkonfiguration für den {\wordpress}-Crawler,
        style=pseudo
    ]{../resources/crawler/config.yaml}

    \paragraph{Aufruf}
    Der Aufruf erfolgt über automatisch generierte Start-Skripte,
    die dem Werkzeug beiliegen und denen als Argument
    der Pfad zur Konfigurationsdatei mitgeteilt werden muss.
    Ein beispielhafter Aufruf ist \texttt{./wccs-wordpress-crawler /home/wccs/crawler.yaml}.

    \subsection{Technologien}
    Der Classification Service ist in JavaScript geschrieben und läuft auf der
    NodeJS Plattform\footnote{\url{https://nodejs.org/}}.
    Sie verwendet die Bibliotheken
    express\footnote{\url{http://expressjs.com/}},
    unirest\footnote{\url{http://unirest.io/nodejs.html}},
    object-hash\footnote{\url{https://github.com/puleos/object-hash}}
    und winston\footnote{\url{https://github.com/winstonjs/winston}}.
    \section{Plattform}
    Die verschiedenen Services des \gls{wccs} sind als
    setzen auf der Docker Plattform\footnote{vgl. \url{https://www.docker.com}} auf,
    wodurch sie von einander isoliert betrieben werden sowie einfach einzeln
    gestoppt, gestartet und skaliert werden können.

    Konkret laufen
    der Annotation Service,
    der Classification Service,
    der Classification Storage,
    die Classification Storage API,
    und die Webanwendung in Docker Containern.
    Diese werden über das Tool Docker Compose\footnote{vgl. \url{https://docs.docker.com/compose}}
    gebaut, ausgeführt und über ein virtuelles Netzwerk verbunden.
    Die enthaltene Konfiguration mappt die Ports der Services
    1:1 auf die gleichen Ports des Host-Systems.


	%\chapter{Ergebnisse}
    \label{chapter:Findings}
    Das \gls{wccs} wurde im Rahmen dieser Arbeit auf zwei Fallbeispiele angewandt.
    Die Ergebnisse dieser Versucht werden in diesem Kapitel präsentiert.
   
    \section{Vorgehen}
    Vor den eigentlichen Ergebnissen soll an dieser Stelle
    das Vorgehen beschrieben werden.

    Einge Studienportale der Fakultät \gls{ksw} der {\fernUni}
    besitzen Übersichtsseiten aller Lehrenden und Betreuenden
    in dem jeweligen Portal.
    Das erste Fallbeispiel klassifiziert diese Seiten.
    Auf Basis der Seite des Studienportal "`B.A. Bildungswissenschaft"'
    wird dazu ein Modell der Seite erstellt,
    welches dann in eine Menge von Klassendefinitionen in der \gls{wccdl} überführt wird,
    sodass alle relevanten Inhalte dieser Seite erfasst werden.
    Ziel ist es zu zeigen, dass das System erfolgreich auf einer einzelnen Seite angewandt werden kann.
    Andernfalls wäre es zu nichts nutze.

    Anschließend wird die Klassendefinition einzeln auf die Pendants der restlichen Sites angewandt,
    ohne im Vorfeld irgendwelche Anpassungen vorzunehmen,
    da dies ein Vorgehen in der Praxis darstellt.
    Dieser Schritt legt dar, wie einfach sich Konfigurationen auf mehrere vermeintlich
    sehr ähnliche Webseiten anwenden lässt und welcher zusätzliche Aufwand für Anpassungen entsteht.
    Die Zahl der Unterschiede im Modell der Seiten und der unzutreffenden Selektoren ist ein Indikator hierfür.

    Der letzte Abschnitt des ersten Fallbeispiels wiederholt den vorherigen,
    unterscheidet sich aber dahingehend, dass alle Klassifikationen
    gemeinsam in einer Datenbank gespeichert werden.
    Die Anzahl der resultierenden Knoten und Kanten in der Datenbank
    gibt Aufschluss darüber, wie effektiv die Möglichkeit Knoten in mehreren Seiten
    zu verwenden, in der Praxis ist und welche Informationen daraus gezogen werden können.

    Das zweite Fallbeispiel klassifiziert Übersichtsseiten mit aktuellen Nachrichten
    der Fakultät und der einzelnen Studienportale.
    Wie später gezeigt wird, gibt es auf diesen Seiten im Vergleich zum ersten Beispiel
    wichtige Unterschiede in der HTML-Struktur -- auf der die Klassifizierung letztendlich basiert --,
    weshalb mit diesem Beispiel gezeigt werden soll, inwieweit das System mit dieser Struktur zurecht kommt.
    Gleichzeitig verwendet es einige Klassendefinitionen aus dem ersten Beispiel wieder,
    und definiert lediglich spezielle Klassen selbst.
    Nach der Klassifizierung einer einzelnen Seite des Portals "`B.A. Bildungswissenschaft"'
    werden alle Nachrichtenübersichtsseiten dieser Site gemeinsam klassifiziert,
    was wiederum neue Resultate ergeben wird.

    Die Ergebnisse bestehen in beiden Beispielen aus
    den entwickelten Klassendefinitionen,
    Kennzahlen über die angelegten Koten und Kanten in der Datenbank,
    Auffälligkeiten in der Klassifikation und der Visualisierung durch Annotationen
    inkl. einer Erklärung, wodurch diese begründet sind.

    \section{Fallbeispiel 1 -- Lehrende und Betreuende}
    \label{section:findingsCaseStudy1}
    In diesem Fallbeispiel werden Übersichtsseiten
    über die Lehrenden und Betreuenden (Mitarbeiter) eines Studienportals
    der Fakultät \gls{ksw} der {\fernUni} klassifiziert.
    Anhand der Seite des Portals \gls{babw} wird dazu zunächst ein
    konzeptionelles Modell dieser Seiten sowie ein {\classificationModel} erstellt.
    Nach der einzelnen Klassifizierung dieser Seite folgt die separate Klassifizierung
    der Übersichtsseiten der Portale

    \begin{itemize}
        \item \gls{bapvs}\footnote{vgl. \url{http://www.fernuni-hagen.de/KSW/portale/bapvs/einstieg/lehrende-und-betreuende-im-b-a-pvs/}},
        \item \gls{bscpsy}\footnote{vgl. \url{http://www.fernuni-hagen.de/KSW/portale/bscpsy/einstieg/lehrende-und-betreuende-im-b-sc-psychologie/}} und
        \item \gls{mabm}\footnote{vgl. \url{http://www.fernuni-hagen.de/KSW/portale/mabm/einstieg/lehrende-und-betreuende-im-m-a-eeducation/}}.
    \end{itemize}

    Jede Klassifikation wird dabei in separaten Datenbanken gespeichert,
    um die Graphen isoliert betrachten zu können.
    Anschließend werden die Seiten aller Sites erneut klassifiziert und in
    einer gemeinsamen Datenbank gespeichert,
    um zu sehen, welchen Einfluss das auf die Graphen hat.
    Zu jeder Klassifikation werden einige Kennzahlen der Datenbank präsentiert.
    Anschließend erfolgt eine Betrachtung von Auffälligkeiten in den Klassifikationen und in den Annotationen.

    \subsection{Modell der Seite}
    % TODO: UML Diagramm der Seite, dann ist es auch ein Modell
    Abbildung \ref{image:findingNewsModelOverview} zeigt einen
    Ausschnitt der zu klassifizierenden Seite,
    anhand dessen das Modelld er Seite erläutert wird.
    Eine schematische Darstellung ist außerdem in Abbildung
    \ref{image:findingNewsModelUml} zu sehen.
    Aufgrund der nachfolgend beschriebenen Überschneidungen zum
    ersten Fallbeispiel, verzichtet dieses auf die Wiederholung
    identischer Elemente.

    \begin{figure}[htb]
        \centering
        \includegraphics[width=\textwidth]{../resources/findings/case-study-2/news-overview.png}
        \caption{Ausschnitt aus einer Seite: Lehrende und Betreuende}
        \label{image:findingNewsModelOverview}
    \end{figure}

    Es ist leicht zu erkennen, dass es auf dieser Seite einige Überschneidungen
    mit der des vorherigen Beispiels gibt.
    Das betrifft den Kopfbereich, den Namen des Portals,
    die seitlichen Navigationspunkte und die Überschrift der Seite.
    Im Vergleich zum ersten Beispiel gibt es hier nur wenige inhaltliche,
    keine konzeptionellen Unterschide.

    Diese werden erst im mittleren Bereich ersichtlich,
    wo die Auflistung der einzelnen Nachrichten erfolgt.
    Jede Nachricht besitzt ein Datum, eine Überschrift und beliebig viele Absätze,
    die Text, Links etc. enthalten.
    Die Überschrift ist gleichzeitig auch ein Link auf die Einzelseite der Meldung.
    Diese Seite enthält nicht alle Meldungen.
    Stattdessen werden sie auf mehrere Seiten verteilt.
    Zwei Links, die in Abbildung \ref{image:findingNewsModelOverview} nicht zu sehen ist,
    führen einen Besucher der Webseite zur nächsten bzw. vorherigen Übersichtsseite.
    Dabei handelt es sich um eigenständige Seiten mit individuellen \glspl{url},
    die aber alle dem beschriebenen Modell folgen.

    \begin{figure}[htb]
        \centering
        \includegraphics[width=\textwidth]{../resources/findings/case-study-2/model.png}
        \caption{Schematische Darstellung einer Nachrichtenübersichtsseite}
        \label{image:findingNewsModelUml}
    \end{figure}
    \subsection{Klassifizierungsmodell}
    Aus der obigen Analyse der Seite ist ein Klassifizierungsmodell hervorgegangen,
    welches sich in allgemeine und spezielle Klassen aufteilen lässt.
    Die Definition wurde entsprechend auf zwei Dateien aufgeteilt,
    sodass sich die allgemeinen Klassen leicht übertragen lassen.
    Zur Definition der Selektoren wurde diente die HTML-Struktur
    eines Mitarbeiters, die beispielhaft in Listing
    \ref{listing:findingsTeachersHtmlSource} dargestellt ist.
    Konkrete Inhalte wurden aus Gründen der Übersichtlichkeit entfernt.
    % TODO: Erwähnen, dass min. 1 Kontakt b anstatt strong verwendet?
    Wie später ersichtlich wird, spielen die Zeilenumbrüche im p-Element
    im Quelltext eine relevante Rolle, weshalb sie unverändert übernommen wurden.

    \lstinputlisting[
        label=listing:findingsTeachersHtmlSource,
        caption=HTML-Struktur eines Lehrenden,
        language=HTML
    ]{../resources/findings/case-study-1/babw/teacher.html}

    Die Klassen der allgemeinen Bereiche sind in Listing
    \ref{listing:findingsTeachersCommon} aufgeführt.
    Es folgt eine kurze Erläuterung,
    welche Elemente der Seite die einzelnen Klassen darstellen.

    \lstinputlisting[
        label=listing:findingsTeachersCommon,
        caption=Klassendefinition für allgemeine Bereiche,
        language=wccdl,
        inputencoding=utf8/latin1,
        style=wccdl
    ]{../resources/findings/case-study-1/classification-model/Common.wctd}

    Die Klasse "`Header"' repräsentiert den Kopfbereich der Seite
    und besitzt entsprechend Features,
    die das Logo sowie die Links klassifizieren.
    Das Logo wird durch die Klase "`Brand"' modelliert,
    die sowohl ein Feature für das Bild selbst als auch den Link enthält.
    Für den Namen des Portals und den zugehörige Link existiert die Klasse "`Portal"'.
    Die Überschrift der Seite wird als "`PageHeading"' klassifiziert und
    der einleitende Absatz als "`Introduction"', die außerdem Links auf andere
    Seiten der {\fernUni} erkennt.
    Wie die Links im Kopfbereich werden sie als "`FernUniInternalLink"' klassifiziert.

    Zusätzlich wird in dieser Datei für Bilder die Klasse "`Image"' definiert,
    die in der zweiten Datei Verwendung findet.
    Aus dieser geht auch hervor, wie die Navigationspunkte im linken Bereich erfasst werden.
    Sie ist in Listing \ref{listing:findingsTeachersSpecial} zu sehen.

    \lstinputlisting[
        label=listing:findingsTeachersSpecial,
        caption=Klassendefinition für Lehrende und Betreuende,
        language=wccdl,
        inputencoding=utf8/latin1,
        style=wccdl
    ]{../resources/findings/case-study-1/classification-model/Teachers.wctd}

    Zunächst wird die Seitenklasse "`Teachers"' definiert,
    inklusive einiger Features für die allgemeinen Bereiche der Seite,
    wofür die Klassen der vorangegangenen Datei genutzt werden.
    Eines dieser Features ist "`sidebarNavigationLinks"',
    welches die Navigationslinks auf der linken Seite als "`FernUniInternalLink"'
    klassifiziert.
    Ein einzelner Mitarbeiter wird durch die Klasse "`Teacher"' dargestellt
    sowie sein Name durch "`TeacherName"'.
    Ein Lehrgebiet mit Namen und Link wird als "`SubjectArea"' bzw. "`SubjectAreaNem"' klassifiziert.
    Die Kontaktinformationen eines Mitarbeiters kapselt die Klasse "`ContactInformation"',
    wobei die einzelnen Angaben durch die Klassen "`Phone"', "`Email"', "`Fax"' und "`Room"'
    repräsentiert werden.

    % TODO: Erwähnen, dass die Klassen iterativ erstellt wurden?
    \subsection{Kennzahlen}
    Nach jeder Klassifizierung wurden einige Kennzahlen des
    resultierenden Graphs ermittelt.
    Dies geschah im Fall einzelner Datenbanken und im Fall
    einer gemeinsamen.
    Zum besseren Verständnis der präsentierten Zahlen wird
    zunächst die konkrete Struktur der Graphen beschrieben.

    \paragraph{Struktur eines Graphs}
    Abbildung \ref{image:findingTeachersFiguresDbModel1}
    und \ref{image:findingTeachersFiguresDbModel2} zeigen,
    wie der Graph einer Klassifikation in diesem Fallbeispielt aufgebaut ist.
    Beide sind durch den \texttt{Content}-Knoten mit der Klasse \texttt{Teacher} verbunden.
    Referenzen sind in den Abbildungen nicht zu sehen,
    da sie lediglich Kanten von einem Knoten zu einer {\resource} darstellen
    und die Darstellungen unnötig vergrößern würden.
    Anhand des {\classificationModel}s und der beiden Abbildungen ist
    offensichtlich, welche der zu sehenden Knoten Referenzen besitzen.
    Bei {\collectionFeature}s ist immer ein stellvertretendes Element zu sehen.

    \begin{figure}[htb]
        \centering
        \includegraphics[scale=\imageScalingFactor]{../resources/findings/case-study-1/dbmodel/dbmodel1.png}
        \caption{Struktur des Graphs einer Seite über Lehrende und Betreuende (1)}
        \label{image:findingTeachersFiguresDbModel1}
    \end{figure}

    \begin{figure}[htb]
        \centering
        \includegraphics[scale=\imageScalingFactor]{../resources/findings/case-study-1/dbmodel/dbmodel2.png}
        \caption{Struktur des Graphs einer Seite über Lehrende und Betreuende (2)}
        \label{image:findingTeachersFiguresDbModel2}
    \end{figure}

    \paragraph{Präsentation der Kennzahlen}
    Die folgenden Tabellen präsentieren die gesammelten Kennzahlen.
    Sie sind folgendermaßen aufgebaut:
    Die erste Spalte benennt die Kennzahl.
    Dann folgen vier Spalten, die den Wert der Kennzahl für die einzelnen Sites enthalten.
    Diese Zahlen werden in der Spalte "`Summe"' aufaddiert,
    um sie mit der letzten Spalte zu vergleichen.
    Diese gibt den Wert der Kennzahl für den Fall der gemeinsam genutzten Datenbank an.
    Eine ausführliche Interpretation dieser Zahlen geschieht in Kapitel \ref{section:findingsInterpretation}.
    Trotzdem wird hier schon die Bedeutung einiger Kennzahlen kurz hervorgehoben.

    Tabelle \ref{table:findingsTeachersFiguresNodesByLabel}
    gruppiert die Knoten der Datenbank nach ihren Labels und zeigt,
    wie oft jedes Label oder jede Kombination von Labels Verwendung fand.   
    Für die einzelnen Klassifikationen gibt "`Content"' außerdem an,
    wie viele {\contentFeature}s sie enthalten.

    \begin{table}[htb]
        \centering
        \begin{tabular}{|l|c|c|c|c|c|c|}
            \hline
            \multicolumn{1}{|c|}{\textbf{Label}} & \textbf{\gls{babw}} & \textbf{\gls{bapvs}} & \textbf{\gls{bscpsy}} & \textbf{\gls{mabm}} & \textbf{Summe} & \textbf{Alle} \\ \hline
            Content                                     & 270           & 275            & 176             & 128           & 849            & 824           \\ \hline
            Page + Resource                             & 1             & 1              & 1               & 1             & 4              & 4             \\ \hline
            Resource                                    & 133           & 159            & 89              & 71            & 452            & 430           \\ \hline
            Site                                        & 1             & 1              & 1               & 1             & 4              & 4             \\ \hline
            Text                                        & 105           & 69             & 75              & 56            & 305            & 284           \\ \hline
            \hline
            \textbf{Summe}                              & 510           & 505            & 342             & 257           & 1614           & 1546          \\ \hline
        \end{tabular}
        \caption{Knoten gruppiert nach Labels für Seiten über Lehrende und Betreuende}
        \label{table:findingsTeachersFiguresNodesByLabel}
    \end{table}

    Die Knoten mit dem Label "`Content"' lassen sich nach der ihnen zugewiesenen Klasse
    weiter aufschlüssen, was in Tabelle \ref{table:findingsTeachersFiguresContentNodesByClass} geschieht.
    Eine Kennzahl in dieser Tabelle ist demnach gleichzusetzen mit der Häufigkeit der Verwendung
    der genannten Klasse in der Klassifikation.

    \begin{table}[htb]
        \centering
        \begin{tabular}{|l|c|c|c|c|c|c|}
        \hline
            \textbf{Klasse}  & \multicolumn{1}{l|}{\textbf{\gls{babw}}} & \multicolumn{1}{l|}{\textbf{\gls{bapvs}}} & \multicolumn{1}{l|}{\textbf{\gls{bscpsy}}} & \multicolumn{1}{l|}{\textbf{\gls{mabm}}} & \multicolumn{1}{l|}{\textbf{Summe}} & \multicolumn{1}{l|}{\textbf{Alle}} \\ \hline
            Brand              & 1                                  & 1                                   & 1                                    & 1                                  & 4                                   & 4                                  \\ \hline
            ContactInformation & 55                                 & 67                                  & 33                                   & 24                                 & 179                                 & 178                                \\ \hline
            Fax                & 1                                  & 0                                   & 0                                    & 1                                  & 2                                   & 1                                  \\ \hline
            Header             & 1                                  & 1                                   & 1                                    & 1                                  & 4                                   & 4                                  \\ \hline
            Introduction       & 1                                  & 0                                   & 1                                    & 1                                  & 3                                   & 2                                  \\ \hline
            PageHeading        & 1                                  & 1                                   & 1                                    & 1                                  & 4                                   & 4                                  \\ \hline
            Phone              & 35                                 & 38                                  & 29                                   & 21                                 & 123                                 & 119                                \\ \hline
            Portal             & 1                                  & 1                                   & 1                                    & 1                                  & 4                                   & 4                                  \\ \hline
            Room               & 2                                  & 0                                   & 0                                    & 0                                  & 2                                   & 2                                  \\ \hline
            SubjectArea        & 53                                 & 67                                  & 33                                   & 22                                 & 175                                 & 172                                \\ \hline
            SubjectAreaName    & 9                                  & 18                                  & 11                                   & 7                                  & 45                                  & 39                                 \\ \hline
            Teacher            & 55                                 & 70                                  & 33                                   & 24                                 & 182                                 & 182                                \\ \hline
            TeacherName        & 55                                 & 11                                  & 32                                   & 24                                 & 122                                 & 113                                \\ \hline
            \hline
            \textbf{Summe}     & 270                                & 275                                 & 176                                  & 128                                & 849                                 & 824                                \\ \hline
        \end{tabular}
        \caption{Content Knoten aufgeteilt nach Klasse}
        \label{table:findingsTeachersFiguresContentNodesByClass}
    \end{table}

    Auch über die Kanten des Graphens lassen sich eine Zahlen ermitteln.
    Tabelle \ref{table:findingTeachersFiguresEdgesByLabel} beginnt dazu
    mit der Aufschlüsselung der Kanten nach ihrem Label.
    Die Kennzahl "`References"' spiegelt die Anzahl der Referenzen innerhalb der Klassifikation wieder.

    \begin{table}[htb]
        \centering
        \begin{tabular}{|l|c|c|c|c|c|c|}
            \hline
            \multicolumn{1}{|c|}{\textbf{Kanten-Label}} & \textbf{\gls{babw}} & \textbf{\gls{bapvs}} & \textbf{\gls{bscpsy}} & \textbf{\gls{mabm}} & \textbf{Summe} & \textbf{Alle} \\ \hline
            Reads                                       & 105           & 69             & 75              & 56            & 305            & 284           \\ \hline
            References                                  & 180           & 209            & 110             & 86            & 585            & 582           \\ \hline
            Owns                                        & 320           & 335            & 200             & 144           & 999            & 996           \\ \hline
            \hline
            \textbf{Summe}                              & 605           & 613            & 385             & 286           & 1889           & 1862          \\ \hline
        \end{tabular}
        \caption{Kanten nach Label}
        \label{table:findingTeachersFiguresEdgesByLabel}
    \end{table}

    Neben dem Label ist in Bezug auf Kanten auch die Frage interessant,
    welche Knoten sie verbinden.
    Tabelle \ref{table:findingsTeachersFiguresEdgesByStartEndNodeLabel}
    zeigt, welche Arten von Knoten wie oft verbunden wurden.
    Die Beziehung eines Content Knotens zu einem Text-Knoten ist nicht enthalten,
    da diese äquivalent zum oben gezeigten Reads-Label ist.
    Diese Tabelle liefert unter anderem Informationen darüber,
    wie viele Referenzen die Seite selbst hat und wie viele zu Content Features gehören.

    \begin{table}[htb]
        \centering
        \begin{tabular}{|l|c|c|c|c|c|c|}
            \hline
            \multicolumn{1}{|c|}{\textbf{Start-, Zielknoten-Label}} & \textbf{\gls{babw}} & \textbf{\gls{bapvs}} & \textbf{\gls{bscpsy}} & \textbf{\gls{mabm}} & \textbf{Summe} & \textbf{Alle} \\ \hline
            (:Content) $\rightarrow$ (:Content)                           & 260           & 260            & 162             & 115           & 797            & 794           \\ \hline
            (:Content) $\rightarrow$ (:Resource)                         & 172           & 201            & 102             & 78            & 553            & 550           \\ \hline
            (:Page) $\rightarrow$ (:Content)                              & 59            & 74             & 37              & 28            & 198            & 198           \\ \hline
            (:Page) $\rightarrow$ (:Resource)                             & 8             & 8              & 8               & 8             & 32             & 32            \\ \hline
            (:Site) $\rightarrow$ (:Page)                                 & 1             & 1              & 1               & 1             & 4              & 4             \\ \hline
            \hline
            \textbf{Summe}                                          & 500           & 544            & 310             & 230           & 1584           & 1578          \\ \hline
        \end{tabular}
        \caption{Kanten nach Start und Zielknoten}
        \label{table:findingsTeachersFiguresEdgesByStartEndNodeLabel}
    \end{table}

    Eine letzte zu beantwortende Frage ist,
    wie viele Knoten in der Datenbank mehr als eine eingehende Kante haben,
    d.h., wie oft sie an verschiedenen Stellen einer oder mehrerer Klassifikationen Verwendung finden.

    \begin{table}[htb]
        \centering
        \begin{tabular}{|l|c|c|c|c|c|c|}
            \hline
            \multicolumn{1}{|c|}{\textbf{Knoten}} & \textbf{\gls{babw}} & \textbf{\gls{bapvs}} & \textbf{\gls{bscpsy}} & \textbf{\gls{mabm}} & \textbf{Summe} & \textbf{Alle} \\ \hline
            Bild                                  & 0             & 1              & 1               & 0             & 2              & 2             \\ \hline
            ContactInformation                    & 0             & 3              & 0               & 0             & 3              & 3             \\ \hline
            E-Mail-Adresse                        & 0             & 1              & 1               & 0             & 2              & 14            \\ \hline
            Fax                                   & 0             & 0              & 0               & 0             & 0              & 1             \\ \hline
            Hauptseite "`Studium"'                & 1             & 0              & 0               & 0             & 1              & 1             \\ \hline
            Homepage der FU                       & 0             & 0              & 0               & 0             & 0              & 1             \\ \hline
            Introduction                          & 0             & 0              & 0               & 0             & 0              & 1             \\ \hline
            Lehrgebietsseiten                     & 7             & 16             & 8               & 5             & 36             & 30            \\ \hline
            Phone                                 & 4             & 1              & 0               & 0             & 5              & 9             \\ \hline
            SubjectArea                           & 0             & 2              & 0               & 0             & 2              & 5             \\ \hline
            SubjectAreaName                       & 8             & 16             & 7               & 5             & 36             & 31            \\ \hline
            Teacher                               & 0             & 1              & 0               & 0             & 1              & 1             \\ \hline
            TeacherName                           & 0             & 3              & 1               & 0             & 4              & 13            \\ \hline
            \hline
            \textbf{Summe}                        & 20            & 44             & 18              & 10            & 92             & 112           \\ \hline
        \end{tabular}
        \caption{Mehrfach referenzierte Knoten}
        \label{table:findingsTeachersFiguresSharedNodes}
    \end{table}

    \subsection{Visualisierung der Klassifikation durch Annotationen}
    Eine wichtige Funktion des \glspl{wccs}
    ist die Visualisierung einer Klassifikation durch Webannotationen
    auf der klassifizierten Webseite.
    Aus diesem Grund folgt eine Übersicht der Annotationen
    des Studienportals \gls{babw},
    welches stellvertretend auch für die restlichen klassifizierten Portale steht.
    Abbildung \ref{image:findingTeachersAnnotationsOverview}
    zeigt einen Ausschnitt der annotierten
    Seite\footnote{Die Darstellungsfehler oben rechts im Kopfbereich
    sowie am Anfang der Brotkrümelnavigation unter dem Portal
    sind dem in Kapitel \ref{section:findingsMethod} beschriebenen
    Annotation Viewer geschuldet.
    Durch die Zwischenschaltung dieser Komponente
    führt der Browser Cross-Origin-Requests durch,
    um die genutzte Bibliothek für Symbole zu beziehen.
    Diese Aufrufe werden allerdings unterbunden,
    weshalb die Symbole nicht korrekt dargestellt werden.
    Bei einer direkten Einbindung des Plugins wäre dies nicht der Fall.
    Der fünfte Link im Kopfbereich wurde richtig klassifiziert.}.
    Bis auf wenige Ausnahmen, die in
    Kapitel \ref{section:findingsTeachersAbnormalitiesBabw} besprochen werden,
    wurden alle klassifizierten Elemente korrekt hervorgehoben.
    Eine detaillierte Ansicht einer beispielhaften Annotation zeigt
    Abbildung \ref{image:findingTeachersSubjectAreaAnnotations}.
    Das Lehrgebiet besitzt korrekterweise zwei Annotationen,
    da das HTML-Element sowohl den Namen als auch den Link enthält
    und deshalb doppelt klassifiziert wurde.

    \begin{figure}[htb]
        \centering
        \includegraphics[width=\textwidth]{../resources/findings/case-study-1/babw/annotations/overview.png}
        \caption{Die annotierte Webseite über Mitarbeiter des Portals \acrshort{babw}}
        \label{image:findingTeachersAnnotationsOverview}
    \end{figure}

    \begin{figure}[htb]
        \centering
        \includegraphics[scale=\screenshotScaleFactor]{../resources/findings/case-study-1/babw/annotations/double-lg-annotation.png}
        \caption{Die Annotationen eines Lehrgebietes}
        \label{image:findingTeachersSubjectAreaAnnotations}
    \end{figure}

    \subsection{Unregelmäßigkeiten in der Klassifikation des Portals "`Babw"'}
    \label{section:findingsTeachersAbnormalitiesBabw}
    Dieses Kapitel stellt die Auffälligkeiten in der Klassifikation
    der Seite der Lehrenden und Betreuenden im Studienportal
    "`B.A. Bildungswissenschaft"' vor.
    Zusätzlich erfolgt eine Erklärung,
    wodurch die jeweilige Unregelmäßigkeit begründet ist.

    \paragraph{Zwei Mitarbeiter ohne Lehrgebiet}
    Die Klassifikation enthält für zwei Mitarbeiter kein Lehrgebiet.
    Im ersten Fall nennt die Webseite kein Lehrgebiet,
    weshalb die Klassifikation an dieser Stelle korrekt ist.
    Der zweite Kontakt ist ein Mitarbeiter einer fremden Universität,
    weshalb dessen Lehrgebiet kein Verweis auf eine andere Seite,
    sondern einfacher Text ist.
    Der verwendete Selektor "`div.team-member-des > p > a:first-child"'
    hat diesen Text nicht erfasst, da er einen Link sucht.

    \paragraph{Unvollständige Namen zweier Mitarbeiter}
    Zwei Kontakte besitzen laut Klassifikation den Namen "`Prof."' bzw. "`Dr."'.
    In diesen Fällen stehen Titel und Name in getrennten strong-Elementen.
    Da der Name eines Mitarbeiters ein skalares Feature ist,
    wurde nur das erste Element vom System erfasst.

    \paragraph{Falsche und fehlende Hervorhebungen durch Annotationen}
    Einige Elemente der Seite wurden durch das Annotator inkorrekt oder nicht hervorgehoben.
    Aus Abbildung \ref{image:findingTeachersAnnotationsOverview} geht bereits hervor,
    dass Bilder hiervon betroffen sind und Annotator diese markiert.
    Abbildung \ref{image:findingTeachersBaBwWrongAnnotations}
    veranschaulicht, dass Telefonnummern und Raumangaben oftmals verschoben annotiert werden.
    Der Grund ist die in Kapitel \ref{section:solutionDetailsClassificationServiceClassification}
    beschriebene Konflikt bei der Bestimmung eines eindeutigen Selektor.
    Allerdings ist auch zu beobachten, dass einige Angaben,
    wie die E-Mail-Adresse in Abbildung \ref{image:findingTeachersBaBwWrongAnnotations}
    überhaupt nicht markiert werden.

    \begin{figure}[htb]
        \centering
        \includegraphics[width=0.5\textwidth]{../resources/findings/case-study-1/babw/annotations/missing-annotation.png}
        \caption{Fehlerhafte Hervorhebung durch Annotationen}
        \label{image:findingTeachersBaBwWrongAnnotations}
    \end{figure}

    \subsection{Unregelmäßigkeiten in der Klassifikation des Portals "`\acrshort{bapvs}"'}
    \label{section:findingsTeachersAbnormalitiesBaPVS}
    Die Klassifikation des Lehrgebietes \gls{bapvs}
    weist ebenfalls einige Unregelmäßigkeiten auf,
    die dieses Kapitel beschreibt.

    \paragraph{Doppelt klassifizierte Mitarbeiter}
    Drei Mitarbeiter des Portals wurden doppelt klassifiziert.
    Der Grund ist eine abweichende HTML-Struktur bei diesen Mitarbeitern,
    die in Listing \ref{listing:findingsTeachersBaPVSHtmlSource} zu sehen ist.

    \lstinputlisting[
        label=listing:findingsTeachersBaPVSHtmlSource,
        caption=Abweichende HTML-Struktur eines Mitarbeiters im Portal BaPVS,
        style=html
    ]{../resources/findings/case-study-1/bapvs/teacher.html}

    Im Vergleich zur erwarteten Struktur\footnote{vgl. Listing \ref{listing:findingsTeachersHtmlSource}} wiederholt sich
    das \texttt{div}-Element mit der Klasse \texttt{grid},
    weshalb der verwendete Selektor \texttt{section\#content div.grid}
    betroffene Mitarbeiter doppelt erfasst.
    Im Fall von zwei Mitarbeitern kann das System bei der zweiten Klassifizierung
    (beim inneren \texttt{div}-Element) kein Bild finden,
    weshalb sie von der ersten abweicht und ein zweiter \texttt{Teacher}-Knoten angelegt wird.
    Diese teilen sich aber die restlichen Informationen,
    also Name, Lehrgebiet und Kontaktinformationen.
    Im dritten Fall besitzt der Mitarbeiter kein Bild,
    weshalb der \texttt{Teacher}-Knoten vollständig wiederverwendet werden kann.

    \paragraph{Mitarbeiter ohne Lehrgebiet}
    Das Lehrgebiet eines einzelnen Mitarbeiters wurde vom System nicht erfasst.
    Anders als beim Lehrgebiet \gls{babw} ist der Grund allerdings,
    dass vor dem Lehrgebiet der Text "`Auskunft erteilt auch:"' platziert ist.
    Der Selektor \texttt{div.team-member-des > p > a:first-child} findet
    kein Lehrgebiet, weil er nach einem \texttt{a}-Element sucht,
    welches das erste Kindelement seines Vaterelementes ist.
    Eine naheliegende Lösung ist die Änderung des Selektors,
    sodass er mit \texttt{a:first-of-type} endet.
    Bei der Erstellung des {\classificationModel}s,
    was auf Basis des Portals \gls{babw} geschah,
    wurde sich allerdings bewusst gegen diese Variante entschieden.
    Bei dem Mitarbeiter im \gls{babw}, für den kein Lehrgebiet erfasst
    wurde\footnote{vgl. Kapitel \ref{section:findingsTeachersAbnormalitiesBabw}},
    hätte dieser Selektor nämlich dazu geführt,
    dass seine E-Mail-Adresse als Lehrgebiet erkannt wird.

    \paragraph{Falscher Text als Name klassifiziert}
    Der Name eines gewissen Mitarbeiters ist laut Klassifikation
    "`Auskunft erteilt auch:"'.
    Der Grund ist, dass dieser Text in einem \texttt{strong}-Element steht,
    welches vor dem Namen auftaucht.
    Das System hat in diesem Fall erwartungskonform nur den ersten
    Treffer für das skalare Feature klassifiziert.
        
    \paragraph{28 Mitarbeiter ohne Telefonnummer}
    Einige Kontakte besitzen in der Klassifikation keine Telefonnummer.
    Bei neun ist auch auf der Webseite keine zu finden.
    Bei den restlichen ist der Nummer nicht "`Tel.: "'
    sondern "`Telefon: "' oder "`Tel:"' vorangestellt.
    Der Selektor hat die Telefonnummern deshalb nicht erkannt.
    
    \paragraph{56 Mitarbeiter ohne Namen}
    Des Weiteren wurde für eine Vielzahl der Mitarbeiter kein Name erkannt,
    da sie weder in einem \texttt{strong} noch einem \texttt{b} Element stehen.
    Stattdessen befinden sie sich wie die Telefonnummer als reiner Text im Element
    oder in einem \texttt{a}-Element.

    \paragraph{Inkorrekte Telefonnummern}
    Die Klassifikation enthält einige Telefonnummern,
    die neben der eigentlichen Nummer auch weitere Informationen enthalten.
    Ein Beispiel ist "`02331/987-4315 email: Lisa.Schaefer Sprechstunde: nach Vereinbarung via e-mail"'.
    Die Telefonnummer wird über einen {\xpathSelector} erfasst,
    der auf dem Seitenquelltext ausgeführt wird.
    Anders als beim Portal \gls{babw} existieren in \gls{bapvs} Mitarbeiter,
    bei denen die Telefonnummer weder die letzte Angabe ist
    noch durch einen physischen Zeilenumbruch im Quelltext gefolgt wird.
    Der Selektor der Telefonnummer erkennt sie deshalb
    nicht\footnote{vgl. Kapitel \ref{section:findingsTeachersClassificationModel}}.

    \paragraph{Unterschiede im konzeptionellen Modell der Seite}
    Im Vergleich zur klassifizierten Seite des Portals \gls{babw}
    sind auch zwei Unterschiede im konzeptionellen Modell der Seite deutlich geworden.
    Der Name eines Mitarbeiters ist in einigen Fällen
    auch ein Link auf eine Detailseite.
    Außerdem besitzen einige Mitarbeiter Sprechzeiten.

    \paragraph{Falsch annotierte Telefonnummern}
    Bei der Betrachtung der Annotationen des Portals \gls{babw}
    ist bereits aufgefallen, dass sie für Telefonnummern
    mehrmals verschoben sind.
    Dies ist im Fall des Portals \gls{bapvs} ebenfalls zu beobachten,
    allerdings in einem deutlicheren Ausmaß,
    wie Abbildung \ref{image:findingTeachersBaPVSWrongPhone} zeigt.

    \begin{figure}[htb]
        \centering
        \includegraphics[scale=\screenshotScaleFactor]{../resources/findings/case-study-1/bapvs/annotations/triple-annotation.png}
        \caption{Verschobene Annotation einer Telefonnummer im \acrshort{bapvs}}
        \label{image:findingTeachersBaPVSWrongPhone}
    \end{figure}

    Bei der Bestimmung des eindeutigen Selektors,
    wird die Position des klassifizierten Textes in der Eigenschaft
    \texttt{innerText} als Versatz im Startelement
    verwendet\footnote{vgl. Kapitel \ref{section:solutionDetailsClassificationServiceClassification}}.
    In den betroffenen Fällen befindet sich zwischen der Telefonnummer
    und der E-Mail-Adresse nur ein \texttt{br}-Element,
    aber kein physischer Zeilenumbruch im
    Seitenquelltext\footnote{vgl. Listing \ref{listing:findingsTeachersHtmlSource}}.
    Die Telefonnummer, die über einen {\xpathSelector} ermittelt wird,
    enthält deshalb zu viele Informationen (siehe oben),
    die wiederum nur durch ein Leerzeichen, aber nicht durch einen Zeilenumbruch getrennt sind.
    In \texttt{innerText} wird das \texttt{br}-Element aber zu einem Umbruch,
    weshalb der klassifizierte Text in dieser Eigenschaft nicht gefunden wird.
    Als Versatz wird deshalb $-1$ gespeichert.
    Annotator setzt die Annotation deshalb an den Anfang des umschließenden Elementes.

    \subsection{Unregelmäßigkeiten in der Klassifikation des Portals \acrshort{bscpsy}}
    Auch bei der Klassifizierung des Portals \gls{bscpsy}
    sind im Vergleich zu den vorherigen Portalen zwei neue
    Auffälligkeiten aufgetreten.

    \paragraph*{Leere Einleitung}
        Die Klassifikation der Webseite enthält eine Einleitung,
        die aber nur ein Leerzeichen enthält.
        Die Webseite besitzt einen Absatz,
        auf den der Selektor zutrifft.
        Dieser Absatz enthält aber nur das Zeichen \texttt{\&nbsp;}.

    \paragraph*{Zwei Mitarbeiter ohne E-Mail-Adresse}
        Des Weiteren wurden zwei Mitarbeiter ohne E-Mail-Adresse klassifiziert.
        Bei einem ist dies korrekt, da die Webseite ebenfalls keine Adresse enthält.
        Im anderen Fall wurde sie nicht erkannt,
        da die \gls{uri} kein vorangestelltes \texttt{mailto:} enthält
        und deshalb nicht vom Selektor erfasst wurde.
        
    \subsection{Unregelmäßigkeiten in der Klassifikation des Portals "`MaBm"'}
    Nicht zuletzt enthält auch die Klassifikation des Portals
    "`M.A. Bildung und Medien: eEducation"'
    zwei Auffälligkeiten,
    die hier kurz aufgezeigt werden.

    \paragraph{Zwei Lehrende ohne Lehrgebiet}
    Im Falle zweier Mitarbeiter wurde kein Lehrgebiet erkannt.
    Der Grund ist eine erneute andere Struktur des \glspl{html}.
    In diesen Fällen ist der Link auf das Lehrgebiet
    in ein strong Element eingebettet, weshalb es nicht erkannt wurde.

    \paragraph{Inkorrekte Nabem zweier Mitarbeiter}
    Bei den gleichen Mitarbeitern enthält ihr Name außerdem ihr Lehrgebiet.
    Wie beschrieben enthält das strong Element das Lehrgebiet,
    aber auch den Namen des Mitarbeiters.
    Der Selektor des Namens des Mitarbeiters erfasst genau dieses strong Element.


    \section{Fallbeispiel 2 -- Nachrichtenübersichtsseite}
    Dieses Fallbeispiel klassifiziert Übersichtsseiten mit aktuellen
    Meldungen der Fakultät \gls{ksw} der {\fernUni} klassifiziert.

    Wie im vorherigen Beispiel wird dazu zunächst eine dieser Seiten analysiert
    und das verwendete Klassifizierungsmodell vorgestellt.
    Anschließend folgt die Präsentation einiger Kennzahlen der Klassifikationen.

    Im Vergleich zum ersten Fallbeispiel in Kapitel \ref{section:findingsCaseStudy1}
    wurden keine neuen Auffälligkeiten ermittelt,
    weshalb ein entsprechendes Kapitel entfällt.

    \subsection{Modell der Seite}
    % TODO: UML Diagramm der Seite, dann ist es auch ein Modell
    Abbildung \ref{image:findingNewsModelOverview} zeigt einen
    Ausschnitt der zu klassifizierenden Seite,
    anhand dessen das Modelld er Seite erläutert wird.
    Eine schematische Darstellung ist außerdem in Abbildung
    \ref{image:findingNewsModelUml} zu sehen.
    Aufgrund der nachfolgend beschriebenen Überschneidungen zum
    ersten Fallbeispiel, verzichtet dieses auf die Wiederholung
    identischer Elemente.

    \begin{figure}[htb]
        \centering
        \includegraphics[width=\textwidth]{../resources/findings/case-study-2/news-overview.png}
        \caption{Ausschnitt aus einer Seite: Lehrende und Betreuende}
        \label{image:findingNewsModelOverview}
    \end{figure}

    Es ist leicht zu erkennen, dass es auf dieser Seite einige Überschneidungen
    mit der des vorherigen Beispiels gibt.
    Das betrifft den Kopfbereich, den Namen des Portals,
    die seitlichen Navigationspunkte und die Überschrift der Seite.
    Im Vergleich zum ersten Beispiel gibt es hier nur wenige inhaltliche,
    keine konzeptionellen Unterschide.

    Diese werden erst im mittleren Bereich ersichtlich,
    wo die Auflistung der einzelnen Nachrichten erfolgt.
    Jede Nachricht besitzt ein Datum, eine Überschrift und beliebig viele Absätze,
    die Text, Links etc. enthalten.
    Die Überschrift ist gleichzeitig auch ein Link auf die Einzelseite der Meldung.
    Diese Seite enthält nicht alle Meldungen.
    Stattdessen werden sie auf mehrere Seiten verteilt.
    Zwei Links, die in Abbildung \ref{image:findingNewsModelOverview} nicht zu sehen ist,
    führen einen Besucher der Webseite zur nächsten bzw. vorherigen Übersichtsseite.
    Dabei handelt es sich um eigenständige Seiten mit individuellen \glspl{url},
    die aber alle dem beschriebenen Modell folgen.

    \begin{figure}[htb]
        \centering
        \includegraphics[width=\textwidth]{../resources/findings/case-study-2/model.png}
        \caption{Schematische Darstellung einer Nachrichtenübersichtsseite}
        \label{image:findingNewsModelUml}
    \end{figure}
    \subsection{Klassifizierungsmodell}
    Aus der obigen Analyse der Seite ist ein Klassifizierungsmodell hervorgegangen,
    welches sich in allgemeine und spezielle Klassen aufteilen lässt.
    Die Definition wurde entsprechend auf zwei Dateien aufgeteilt,
    sodass sich die allgemeinen Klassen leicht übertragen lassen.
    Zur Definition der Selektoren wurde diente die HTML-Struktur
    eines Mitarbeiters, die beispielhaft in Listing
    \ref{listing:findingsTeachersHtmlSource} dargestellt ist.
    Konkrete Inhalte wurden aus Gründen der Übersichtlichkeit entfernt.
    % TODO: Erwähnen, dass min. 1 Kontakt b anstatt strong verwendet?
    Wie später ersichtlich wird, spielen die Zeilenumbrüche im p-Element
    im Quelltext eine relevante Rolle, weshalb sie unverändert übernommen wurden.

    \lstinputlisting[
        label=listing:findingsTeachersHtmlSource,
        caption=HTML-Struktur eines Lehrenden,
        language=HTML
    ]{../resources/findings/case-study-1/babw/teacher.html}

    Die Klassen der allgemeinen Bereiche sind in Listing
    \ref{listing:findingsTeachersCommon} aufgeführt.
    Es folgt eine kurze Erläuterung,
    welche Elemente der Seite die einzelnen Klassen darstellen.

    \lstinputlisting[
        label=listing:findingsTeachersCommon,
        caption=Klassendefinition für allgemeine Bereiche,
        language=wccdl,
        inputencoding=utf8/latin1,
        style=wccdl
    ]{../resources/findings/case-study-1/classification-model/Common.wctd}

    Die Klasse "`Header"' repräsentiert den Kopfbereich der Seite
    und besitzt entsprechend Features,
    die das Logo sowie die Links klassifizieren.
    Das Logo wird durch die Klase "`Brand"' modelliert,
    die sowohl ein Feature für das Bild selbst als auch den Link enthält.
    Für den Namen des Portals und den zugehörige Link existiert die Klasse "`Portal"'.
    Die Überschrift der Seite wird als "`PageHeading"' klassifiziert und
    der einleitende Absatz als "`Introduction"', die außerdem Links auf andere
    Seiten der {\fernUni} erkennt.
    Wie die Links im Kopfbereich werden sie als "`FernUniInternalLink"' klassifiziert.

    Zusätzlich wird in dieser Datei für Bilder die Klasse "`Image"' definiert,
    die in der zweiten Datei Verwendung findet.
    Aus dieser geht auch hervor, wie die Navigationspunkte im linken Bereich erfasst werden.
    Sie ist in Listing \ref{listing:findingsTeachersSpecial} zu sehen.

    \lstinputlisting[
        label=listing:findingsTeachersSpecial,
        caption=Klassendefinition für Lehrende und Betreuende,
        language=wccdl,
        inputencoding=utf8/latin1,
        style=wccdl
    ]{../resources/findings/case-study-1/classification-model/Teachers.wctd}

    Zunächst wird die Seitenklasse "`Teachers"' definiert,
    inklusive einiger Features für die allgemeinen Bereiche der Seite,
    wofür die Klassen der vorangegangenen Datei genutzt werden.
    Eines dieser Features ist "`sidebarNavigationLinks"',
    welches die Navigationslinks auf der linken Seite als "`FernUniInternalLink"'
    klassifiziert.
    Ein einzelner Mitarbeiter wird durch die Klasse "`Teacher"' dargestellt
    sowie sein Name durch "`TeacherName"'.
    Ein Lehrgebiet mit Namen und Link wird als "`SubjectArea"' bzw. "`SubjectAreaNem"' klassifiziert.
    Die Kontaktinformationen eines Mitarbeiters kapselt die Klasse "`ContactInformation"',
    wobei die einzelnen Angaben durch die Klassen "`Phone"', "`Email"', "`Fax"' und "`Room"'
    repräsentiert werden.

    % TODO: Erwähnen, dass die Klassen iterativ erstellt wurden?
    \subsection{Kennzahlen}
    Nach jeder Klassifizierung wurden einige Kennzahlen des
    resultierenden Graphs ermittelt.
    Dies geschah im Fall einzelner Datenbanken und im Fall
    einer gemeinsamen.
    Zum besseren Verständnis der präsentierten Zahlen wird
    zunächst die konkrete Struktur der Graphen beschrieben.

    \paragraph{Struktur eines Graphs}
    Abbildung \ref{image:findingTeachersFiguresDbModel1}
    und \ref{image:findingTeachersFiguresDbModel2} zeigen,
    wie der Graph einer Klassifikation in diesem Fallbeispielt aufgebaut ist.
    Beide sind durch den \texttt{Content}-Knoten mit der Klasse \texttt{Teacher} verbunden.
    Referenzen sind in den Abbildungen nicht zu sehen,
    da sie lediglich Kanten von einem Knoten zu einer {\resource} darstellen
    und die Darstellungen unnötig vergrößern würden.
    Anhand des {\classificationModel}s und der beiden Abbildungen ist
    offensichtlich, welche der zu sehenden Knoten Referenzen besitzen.
    Bei {\collectionFeature}s ist immer ein stellvertretendes Element zu sehen.

    \begin{figure}[htb]
        \centering
        \includegraphics[scale=\imageScalingFactor]{../resources/findings/case-study-1/dbmodel/dbmodel1.png}
        \caption{Struktur des Graphs einer Seite über Lehrende und Betreuende (1)}
        \label{image:findingTeachersFiguresDbModel1}
    \end{figure}

    \begin{figure}[htb]
        \centering
        \includegraphics[scale=\imageScalingFactor]{../resources/findings/case-study-1/dbmodel/dbmodel2.png}
        \caption{Struktur des Graphs einer Seite über Lehrende und Betreuende (2)}
        \label{image:findingTeachersFiguresDbModel2}
    \end{figure}

    \paragraph{Präsentation der Kennzahlen}
    Die folgenden Tabellen präsentieren die gesammelten Kennzahlen.
    Sie sind folgendermaßen aufgebaut:
    Die erste Spalte benennt die Kennzahl.
    Dann folgen vier Spalten, die den Wert der Kennzahl für die einzelnen Sites enthalten.
    Diese Zahlen werden in der Spalte "`Summe"' aufaddiert,
    um sie mit der letzten Spalte zu vergleichen.
    Diese gibt den Wert der Kennzahl für den Fall der gemeinsam genutzten Datenbank an.
    Eine ausführliche Interpretation dieser Zahlen geschieht in Kapitel \ref{section:findingsInterpretation}.
    Trotzdem wird hier schon die Bedeutung einiger Kennzahlen kurz hervorgehoben.

    Tabelle \ref{table:findingsTeachersFiguresNodesByLabel}
    gruppiert die Knoten der Datenbank nach ihren Labels und zeigt,
    wie oft jedes Label oder jede Kombination von Labels Verwendung fand.   
    Für die einzelnen Klassifikationen gibt "`Content"' außerdem an,
    wie viele {\contentFeature}s sie enthalten.

    \begin{table}[htb]
        \centering
        \begin{tabular}{|l|c|c|c|c|c|c|}
            \hline
            \multicolumn{1}{|c|}{\textbf{Label}} & \textbf{\gls{babw}} & \textbf{\gls{bapvs}} & \textbf{\gls{bscpsy}} & \textbf{\gls{mabm}} & \textbf{Summe} & \textbf{Alle} \\ \hline
            Content                                     & 270           & 275            & 176             & 128           & 849            & 824           \\ \hline
            Page + Resource                             & 1             & 1              & 1               & 1             & 4              & 4             \\ \hline
            Resource                                    & 133           & 159            & 89              & 71            & 452            & 430           \\ \hline
            Site                                        & 1             & 1              & 1               & 1             & 4              & 4             \\ \hline
            Text                                        & 105           & 69             & 75              & 56            & 305            & 284           \\ \hline
            \hline
            \textbf{Summe}                              & 510           & 505            & 342             & 257           & 1614           & 1546          \\ \hline
        \end{tabular}
        \caption{Knoten gruppiert nach Labels für Seiten über Lehrende und Betreuende}
        \label{table:findingsTeachersFiguresNodesByLabel}
    \end{table}

    Die Knoten mit dem Label "`Content"' lassen sich nach der ihnen zugewiesenen Klasse
    weiter aufschlüssen, was in Tabelle \ref{table:findingsTeachersFiguresContentNodesByClass} geschieht.
    Eine Kennzahl in dieser Tabelle ist demnach gleichzusetzen mit der Häufigkeit der Verwendung
    der genannten Klasse in der Klassifikation.

    \begin{table}[htb]
        \centering
        \begin{tabular}{|l|c|c|c|c|c|c|}
        \hline
            \textbf{Klasse}  & \multicolumn{1}{l|}{\textbf{\gls{babw}}} & \multicolumn{1}{l|}{\textbf{\gls{bapvs}}} & \multicolumn{1}{l|}{\textbf{\gls{bscpsy}}} & \multicolumn{1}{l|}{\textbf{\gls{mabm}}} & \multicolumn{1}{l|}{\textbf{Summe}} & \multicolumn{1}{l|}{\textbf{Alle}} \\ \hline
            Brand              & 1                                  & 1                                   & 1                                    & 1                                  & 4                                   & 4                                  \\ \hline
            ContactInformation & 55                                 & 67                                  & 33                                   & 24                                 & 179                                 & 178                                \\ \hline
            Fax                & 1                                  & 0                                   & 0                                    & 1                                  & 2                                   & 1                                  \\ \hline
            Header             & 1                                  & 1                                   & 1                                    & 1                                  & 4                                   & 4                                  \\ \hline
            Introduction       & 1                                  & 0                                   & 1                                    & 1                                  & 3                                   & 2                                  \\ \hline
            PageHeading        & 1                                  & 1                                   & 1                                    & 1                                  & 4                                   & 4                                  \\ \hline
            Phone              & 35                                 & 38                                  & 29                                   & 21                                 & 123                                 & 119                                \\ \hline
            Portal             & 1                                  & 1                                   & 1                                    & 1                                  & 4                                   & 4                                  \\ \hline
            Room               & 2                                  & 0                                   & 0                                    & 0                                  & 2                                   & 2                                  \\ \hline
            SubjectArea        & 53                                 & 67                                  & 33                                   & 22                                 & 175                                 & 172                                \\ \hline
            SubjectAreaName    & 9                                  & 18                                  & 11                                   & 7                                  & 45                                  & 39                                 \\ \hline
            Teacher            & 55                                 & 70                                  & 33                                   & 24                                 & 182                                 & 182                                \\ \hline
            TeacherName        & 55                                 & 11                                  & 32                                   & 24                                 & 122                                 & 113                                \\ \hline
            \hline
            \textbf{Summe}     & 270                                & 275                                 & 176                                  & 128                                & 849                                 & 824                                \\ \hline
        \end{tabular}
        \caption{Content Knoten aufgeteilt nach Klasse}
        \label{table:findingsTeachersFiguresContentNodesByClass}
    \end{table}

    Auch über die Kanten des Graphens lassen sich eine Zahlen ermitteln.
    Tabelle \ref{table:findingTeachersFiguresEdgesByLabel} beginnt dazu
    mit der Aufschlüsselung der Kanten nach ihrem Label.
    Die Kennzahl "`References"' spiegelt die Anzahl der Referenzen innerhalb der Klassifikation wieder.

    \begin{table}[htb]
        \centering
        \begin{tabular}{|l|c|c|c|c|c|c|}
            \hline
            \multicolumn{1}{|c|}{\textbf{Kanten-Label}} & \textbf{\gls{babw}} & \textbf{\gls{bapvs}} & \textbf{\gls{bscpsy}} & \textbf{\gls{mabm}} & \textbf{Summe} & \textbf{Alle} \\ \hline
            Reads                                       & 105           & 69             & 75              & 56            & 305            & 284           \\ \hline
            References                                  & 180           & 209            & 110             & 86            & 585            & 582           \\ \hline
            Owns                                        & 320           & 335            & 200             & 144           & 999            & 996           \\ \hline
            \hline
            \textbf{Summe}                              & 605           & 613            & 385             & 286           & 1889           & 1862          \\ \hline
        \end{tabular}
        \caption{Kanten nach Label}
        \label{table:findingTeachersFiguresEdgesByLabel}
    \end{table}

    Neben dem Label ist in Bezug auf Kanten auch die Frage interessant,
    welche Knoten sie verbinden.
    Tabelle \ref{table:findingsTeachersFiguresEdgesByStartEndNodeLabel}
    zeigt, welche Arten von Knoten wie oft verbunden wurden.
    Die Beziehung eines Content Knotens zu einem Text-Knoten ist nicht enthalten,
    da diese äquivalent zum oben gezeigten Reads-Label ist.
    Diese Tabelle liefert unter anderem Informationen darüber,
    wie viele Referenzen die Seite selbst hat und wie viele zu Content Features gehören.

    \begin{table}[htb]
        \centering
        \begin{tabular}{|l|c|c|c|c|c|c|}
            \hline
            \multicolumn{1}{|c|}{\textbf{Start-, Zielknoten-Label}} & \textbf{\gls{babw}} & \textbf{\gls{bapvs}} & \textbf{\gls{bscpsy}} & \textbf{\gls{mabm}} & \textbf{Summe} & \textbf{Alle} \\ \hline
            (:Content) $\rightarrow$ (:Content)                           & 260           & 260            & 162             & 115           & 797            & 794           \\ \hline
            (:Content) $\rightarrow$ (:Resource)                         & 172           & 201            & 102             & 78            & 553            & 550           \\ \hline
            (:Page) $\rightarrow$ (:Content)                              & 59            & 74             & 37              & 28            & 198            & 198           \\ \hline
            (:Page) $\rightarrow$ (:Resource)                             & 8             & 8              & 8               & 8             & 32             & 32            \\ \hline
            (:Site) $\rightarrow$ (:Page)                                 & 1             & 1              & 1               & 1             & 4              & 4             \\ \hline
            \hline
            \textbf{Summe}                                          & 500           & 544            & 310             & 230           & 1584           & 1578          \\ \hline
        \end{tabular}
        \caption{Kanten nach Start und Zielknoten}
        \label{table:findingsTeachersFiguresEdgesByStartEndNodeLabel}
    \end{table}

    Eine letzte zu beantwortende Frage ist,
    wie viele Knoten in der Datenbank mehr als eine eingehende Kante haben,
    d.h., wie oft sie an verschiedenen Stellen einer oder mehrerer Klassifikationen Verwendung finden.

    \begin{table}[htb]
        \centering
        \begin{tabular}{|l|c|c|c|c|c|c|}
            \hline
            \multicolumn{1}{|c|}{\textbf{Knoten}} & \textbf{\gls{babw}} & \textbf{\gls{bapvs}} & \textbf{\gls{bscpsy}} & \textbf{\gls{mabm}} & \textbf{Summe} & \textbf{Alle} \\ \hline
            Bild                                  & 0             & 1              & 1               & 0             & 2              & 2             \\ \hline
            ContactInformation                    & 0             & 3              & 0               & 0             & 3              & 3             \\ \hline
            E-Mail-Adresse                        & 0             & 1              & 1               & 0             & 2              & 14            \\ \hline
            Fax                                   & 0             & 0              & 0               & 0             & 0              & 1             \\ \hline
            Hauptseite "`Studium"'                & 1             & 0              & 0               & 0             & 1              & 1             \\ \hline
            Homepage der FU                       & 0             & 0              & 0               & 0             & 0              & 1             \\ \hline
            Introduction                          & 0             & 0              & 0               & 0             & 0              & 1             \\ \hline
            Lehrgebietsseiten                     & 7             & 16             & 8               & 5             & 36             & 30            \\ \hline
            Phone                                 & 4             & 1              & 0               & 0             & 5              & 9             \\ \hline
            SubjectArea                           & 0             & 2              & 0               & 0             & 2              & 5             \\ \hline
            SubjectAreaName                       & 8             & 16             & 7               & 5             & 36             & 31            \\ \hline
            Teacher                               & 0             & 1              & 0               & 0             & 1              & 1             \\ \hline
            TeacherName                           & 0             & 3              & 1               & 0             & 4              & 13            \\ \hline
            \hline
            \textbf{Summe}                        & 20            & 44             & 18              & 10            & 92             & 112           \\ \hline
        \end{tabular}
        \caption{Mehrfach referenzierte Knoten}
        \label{table:findingsTeachersFiguresSharedNodes}
    \end{table}



	%\chapter{Diskussion der Ergebnisse}
    \label{chapter:FindingsDiscussion}
    Die gewonnenen Ergebnisse\footnote{vgl. Kapitel \ref{chapter:Findings}}
    werden nun einer Evaluation unterzogen.
    Dies beginnt mit einigen Schlüssen,
    die man aus den Ergebnissen ableiten kann.
    Anschließend erfolgt eine Bewertung der entwickelten Sprache
    und des Klassifizierungsansatzes.
    Abschließend erfolgt außerdem ein Vergleich mit anderen Arbeiten,
    die die gleichen oder ähnliche Ziele verfolgen.

    \section{Interpretation}
    \label{section:findingsInterpretation}
    Beide Fallbeispiele haben die prinzipielle Funktionsfähigkeit des \gls{wccs} bewiesen,
    da sowohl die domänenspezifische Sprache als auch der Klassifizierungsansatz
    erfolgreich angewandt werden konnten.
    Aufgrund der hohen Zahl an Lehrenden und Betreuenden,
    verdeutlicht das erste Beispielt außerdem,
    dass eine Automatisierung der Klassifizierung sinnvoll ist.

    Die Ergebnisse lassen aber selbstverständlich noch wietergehende Schlüsse zu,
    von denen einige in den folgenden Unterkapiteln diskutiert werden.

    \subsection{Eignung der domänenspezifischen Sprache}
    % TODO: Vorteile aus SolutionConcept/DSL.tex hier hin? Oder zu Grundlagen über DSLs bzw. im Solution Details Abschnitt der DSL?
    \label{section:discussionInterpretationLanguage}
    Die Analyse der \gls{wccdl} orientiert sich
    an den Design Dimensionen von \cite{voelter:DslEngineering}.

    \paragraph{Ausdrucksstärke}
    Die Ausdrucksstärke einer Sprache beschreibt,
    wie kompakt Programme einer Problemdomäne in dieser Sprache
    formuliert werden können.
    Zur Vergleichbarkeit zweier Sprachen bezüglich ihrer Ausdrucksstärke
    stellen \citet[Kapitel 4.1]{voelter:DslEngineering} die folgende
    Definition auf, wobei
    $P_D$ die Menge aller Programme der Domäne,
    $P_L$ die Menge aller in einer Sprache formulierbaren Programme und
    $p_L$ ein konkretes Programm in einer Sprache bezeichnet.

    \begin{quote}
        A language $L_1$ is \textit{more expressive in domain D}
        than a language $L_2 (L_1 {\prec}_D L_2)$,
        if for each $p \in P_D \cap {P_L}_1 \cap {P_L}_2, |{p_L}_1| < |{p_L}_2|$.
    \end{quote}

    Eine Aussage über die Ausdrucksstärke der \gls{wccdl} kann also
    nur relativ zu anderen Sprachen gemacht werden,
    weshalb zusätzlich zu den gesammelten Ergebnissen und
    Kapitel \ref{section:discussionComparisonLanguage}
    an dieser Stelle ein kurzer Vergleich stattfindet.

    Die obige Definition ist schwer anzuwenden,
    wenn eine Sprache textuell und die andere graphischer Natur ist.
    Viele Modellierungssprachen sind deshalb schwer mit der \gls{wccdl} zu vergleichen.
    Allerdings konnten zwei geeignete Kandidaten gefunden werden.
    Der erste ist PlantUML\footnote{vgl. \url{http://plantuml.com/}},
    ein Werkzeug und eine Sprache zur textuellen Beschreibung und anschließender
    Generierung graphischer UML-Diagramme.
    Die zweite Sprache wurde bei Recherchen auf
    GitHub\footnote{vgl. \url{https://github.com/sabrams/web-miner}} gefunden
    und verfolgt ein sehr ähnliches Ziel wie die \gls{wccdl}.
    Sie ist eine interne Ruby DSL, die trotz ihrer etwaigen Unvollständigkeit sowie der fehlenden
    offiziellen wissenschaftlichen Arbeit oder Dokumentation
    für einen Vergleich mit der \gls{wccdl} geeignet ist.
    Sie vermittelt nämlich, wie das Vorhaben der \gls{wccdl}
    in einer anderen Sprache aussehen kann.
    Als Vergleich dient ein Beispiel im
    Testcode\footnote{vgl. \url{https://github.com/sabrams/web-miner/blob/master/features/mine_according_to_command_file.feature}}
    dieser DSL, welches in Listing \ref{listing:discussionExpressivityExampleRubyDsl}
    zu sehen ist.

    \lstinputlisting[
        label=listing:discussionExpressivityExampleRubyDsl,
        caption=Programm in einer Ruby DSL zum Vergleich der Ausdrucksstärke,
        style=pseudo,
        language=Ruby
    ]{../resources/discussion/interpretation/language/expressivity-example.rb}

    Ein äquivalentes Konstrukt in PlantUML zeigt Listing
    \ref{listing:discussionExpressivityExamplePlantUml}.

    \lstinputlisting[
        label=listing:discussionExpressivityExamplePlantUml,
        caption=Programm in PlantUML zum Vergleich der Ausdrucksstärke,
        style=plantuml
    ]{../resources/discussion/interpretation/language/expressivity-example.plantuml}

    Nicht zuletzt zeigt Listing \ref{listing:discussionExpressivityExampleWccs}
    wie das Beispiel in der \gls{wccdl} aussieht.

    \lstinputlisting[
        label=listing:discussionExpressivityExampleWccs,
        caption=Programm in der \acrshort{wccdl} zum Vergleich der Ausdrucksstärke,
        inputencoding=utf8/latin1,
        style=wccdl
    ]{../resources/discussion/interpretation/language/expressivity-example.wctd}

    Obwohl sich auch die \gls{wccdl} in diesem Beispiel kompakt ausdrückt,
    braucht sie mehr Code als die anderen beiden Sprachen.
    Auch ohne genaue Analyse sollte diese geringere Ausdrucksstärke
    allgemeingültig für die \gls{wccdl} sein.
    Anders als ihre Kontrahenten benötigt sie z. B. für jedes Feature eine definierte Klasse
    (im Beispiel \texttt{Object}).
    Außerdem ist die Definition einer einzelnen Klasse,
    eines einzelnen Features oder eines Selektors in etwa gleich lang oder länger,
    als in der Vergleichssprachen.

    Der Grund sind die langen Schlüsselwortsequenzen,
    die zum Beispiel auch im Selektor der Seitenklasse \texttt{Teacher}
    erkennbar sind.
    An dieser Klasse und ihrem Pendant \texttt{News}
    wird außerdem deutlich, dass der Sprache ein Vererbungskonzept fehlt.
    Identische Features müssen nämlich wiederholt werden,
    wodurch sich die Ausdrucksstärke der Sprache verringert.

    Eine Kritik der Ausdrucksstärke der \gls{wccdl} ist berechtigt,
    allerdings muss beachtet werden,
    dass dieses Merkmal in Konflikt mit einigen nun folgenden
    Eigenschaften steht, die positiver bewertet werden.

    % TODO: Namespace?

    \paragraph{Abdeckung der Domäne}
    Die Abdeckung der Domäne einer Sprache beschreiben
    \citet[Kapitel 4.2]{voelter:DslEngineering} als den
    prozentualen Anteil an allen Programmen der Domäne,
    die durch die Sprache formuliert werden können.
    Die formale Definition lautet

    \begin{quote}
        $C_D(L) = \frac{number of P_D programs expressable by L}{number of programs in domain D}$.
    \end{quote}

    Eine volle Abdeckung ist demnach gegeben,
    wenn jedes Programm der Domäne in der Sprache ausgedrückt werden kann.

    Im Falle der \gls{wccdl} kann zumindest die Frage beantwortet werden,
    ob sie ihre Domäne voll abdeckt, das heißt, ob jede beliebige Struktur auf einer Webseite
    durch sie erfasst werden kann.
    Wie das zweite Fallbeispiel zeigte, ist das nicht der Fall,
    da kein allgemeiner Selektor gefunden werden konnte,
    der die Absätze einer Nachricht korrekt erfasst.
    Spezielle Klassen pro Meldung können dieses Problem zwar lösen,
    sind aber nicht im eigentlichen Sinne des \glspl{wccs}.
    Es wurden aber auch einige nicht triviale Strukturen durch die Sprache abgedeckt.
    Selbstverständlich spielen bei dieser Frage auch die Fähigkeiten der Selektoren
    eine Rolle\footnote{vgl. Kapitel \ref{section:discussionInterpretationSelectors}}.

    \paragraph{Vollständigkeit}
    Eine Sprache ist nach \citet[Kapitel 4.5]{voelter:DslEngineering}
    vollständig wenn ihr Generierungsergebnis keine manuell geschriebenen Codefragmente in niedrigeren Sprachen,
    Konfigurationsdateien oder ähnliches benötigt,
    um die gleiche Semantik zu besitzen wie das Programm in der \gls{dsl}.
    Formaler beschreiben sie diese Eigenschaft wie folgt:

    \begin{quote}
        Let us introduce a function $G$ ("code generator") that transforms
        a program $p$ in $L_D$ to a program $q$ in $L_{D-1}$.
        For a complete language, $p$ and $q$ have the same semantics, i.e.
        $OB(p) == OB(G(p)) == OB(q)$ [...]. For incomplete languages
        where $OB(G(p)) \subset OB(p)$ we have to write additional
        code in $L_{D-1}$ to obtain a program in $D_{-1}$ that has the same semantics
        as intended by the original program in $L_D$.
    \end{quote}

    Die \gls{wccdl} ist vollständig, da das generierte JSON-Dokument eines
    {\classificationModel}s mit diesem semantisch übereinstimmt.

    \paragraph{Konkreter Syntax}
    Für den konkreten Syntax einer Sprache schlagen \citet[Kapitel 4.7]{voelter:DslEngineering}
    verschiedene Bewertungskriterien vor: Writeability, Readability, Learnability, Effectivness.

    Der Nachteil der langen Sequenzen von Schlüsselwörtern wird beim Schreiben des Codes
    deutlich, da auch einfache Konzepte vergleichsweise viel Code erfordern.
    Allerdings relativiert sich dieser Nachteil durch die Unterstützung der Entwicklungsumgebung.
    Ein weiterer Nachteil beim Schreiben des Codes ist die fehlende Prüfung der
    Selektoren, wodurch ihre Formulierung unnötig schwer und fehleranfällig wird.
    Das gilt vor allem bei komplexeren Selektoren.
    Die Zeichen zur Einklammerung von Selektoren sind auf vielen Tastaturlayouts nicht zu finden.
    Das ist nachteilig, weil man auf die Autovervollständigung der Entwicklungsumgebung angewiesen ist.
    Gleichzeitig erlauben sie die Nutzung vieler Zeichen ohne sie speziell Kennzeichnen zu müssen,
    wie z. B. Anführungszeichen.
    Im Vergleich zu anderen Sprachen ist das ein wichtiger Vorteil.

    Die vielen Schlüsselwörter sind aber auch ein Vorteil,
    da sich Programme nahezu wie ein Fließtext lesen lassen
    und ihre Verständlichkeit dadurch steigt.
    Da Quelltext in der Regel häufiger gelesen als geschrieben wird,
    ist das sehr positiv zu bewerten.
    Beide Fallbeispiele zeigen aber auch Konstrukte,
    in denen die Lesbarkeit besser sein könnte.
    Zum Beispiel bei der Klasse \texttt{Portal},
    bei der nicht offensichtlich ist,
    dass sie selbst den Namen des Studienportals speichert,
    aber der Link in einem {\childFeature} gespeichert wird.
    Eingeschränkt wird die Lesbarkeit aber durch die Selektoren,
    die in den Fallbeispielen schnell eine hohe Komplexität angenommen haben,
    wie zum Beispiel bei der Klasse \texttt{Phone}.
    Bezüglich der Selektoren und ihrer Lesbarkeit ist aber nochmals
    die fehlende Notwendigkeit, spezielle Zeichen zu kodieren, positiv hervorzuheben.

    Die Konzepte der Sprache sind leicht zu verstehen
    und durch die gebotene Autovervollständigung und einige
    semantische Prüfungen, ist das Erlernen der Sprache einfach.
    Hinzu kommen allerdings die benötigten Kenntnisse von CSS, XPath und regulären Ausdrücken,
    wodurch das formulieren von Selektoren erschwert wird.

    Für eine fundierte Bewertung der Effektivität der Sprache,
    also der Frage, wie gut typische Probleme der Domäne ausgedrückt werden können,
    sind weitere Anwendungen auf noch mehr Webseiten notwendig.
    Das durch die Fallbeispiele erhaltene Bild ist zweiseitig:
    Viele Features und Klassen lassen sich durch simple Selektoren erfassen,
    einige erfordern jedoch komplexe Selektoren.
    Dabei dürfte zumindest das bei der Klasse \texttt{Portal} angewandte Konstrukt,
    um ein HTML-Element sowohl als {\contentFeature} als auch als {\referenceFeature} zu nutzen,
    eine häufige Anforderung sein und sollte besser unterstützt werden.

    \paragraph{Struktur}
    Beim Entwurf der \gls{wccdl} wurde sich bewusst gegen
    eine logische Strukturierung der Klassen in Namensräume,
    dafür aber für eine globale Sichtbarkeit aller Klassen
    in allen Quelldateien eines Projektes entschieden.
    Die hier vorgestellten Fallbeispiele profitieren von diesen Entscheidungen,
    da die Klassen leicht aufgeteilt und wiederverwendet werden konnten.
    Als Argument gegen Namensräume wurde die höhere Komplexität für
    Nicht-Programmierer genannt, was ein valider Punkt ist.
    Allerdings steigt die Einstiegshürde für diese Gruppe bereits
    durch die Selektoren. Namensräume würden diese Hürde womöglich nicht noch weiter anheben.
    Es ist außerdem nicht auszuschließen, dass in größeren Projekten
    Namensräume und eine eingeschränkte Sichtbarkeit von Vorteil wären.
    Anders als die Beispielklassen \texttt{NewsDate} und \texttt{TeacherName},
    müssten Klassen mit abstrakten Namen dann nicht mit einem Präfix versehen werden,
    was durchaus aus schlechter Stil angesehen werden kann.

    Insgesamt ist festzuhalten, dass die \gls{wccdl} viele Problemstellungen der Domäne
    mit einfachen Konzepten abdeckt, was wichtiger ist als jeden Sonderfall zu berücksichtigen
    und deshalb positiv zu bewerten ist.
    \subsection{Übertragbarkeit eines {\classificationModel}s}
    Ein Ziel der Fallbeispiele ist herauszufinden,
    wie gut sich ein anhand einer einzelnen Webseite
    erstelltes {\classificationModel} auf andere Seiten anwenden lässt,
    die augenscheinlich identisch aufgebaut sind.
    Je besser diese Übertragbarkeit ist,
    desto höher ist der Nutzen der Automatisierung.
    Eine erste naheliegende Erkenntnis ist,
    das Klassen iterativ zu definieren sind,
    was schon bei der Entwicklung des {\classificationModel}s
    für das Portal \gls{babw} deutlich wurde.
    Ausnahmen innerhalb einer Webseite werden oft erst
    bei der Prüfung der Klassifikation entdeckt,
    woraufhin die Ausnahme analysiert und das Modell
    entsprechend angepasst werden muss.
    Ein Beispiel ist die Faxnummer, die nur bei einem Mitarbeiter auftritt,
    deren Erfassung aber auch eine Anpassung des Selektors der Telefonnummer
    erforderlich machte.
    Diese Erkenntnis wiegt natürlich noch schwerer,
    wenn eine Menge an Klassen auf viele Webseiten angewandt werden soll.
    Wie das erste Fallbeispiel zeigt,
    können auch vermeintlich sehr ähnliche Seiten deutliche Unterschiede
    im konzeptionellen Modell und in der \gls{html}-Struktur besitzen.
    Diese können nicht alle vorausgedacht werden.
    Es müssen demnach mehrere Webseiten bei der Entwicklung der Klassendefinitionen
    betrachtet werden.
    Andernfalls werden Informationen nicht klassifiziert oder es kommt zu unerwarteten Ergebnissen.
    Beispiele hierfür sind der Verweis auf die Detailseite eines Mitarbeiters
    und die doppelt klassifizierten Mitarbeiter im Portal \gls{bapvs}.
    Deutlich wurde aber auch, dass viele und zu große Unterschiede
    die Auflösung eines Konfliktes sehr komplex machen.
    Diese Überlegungen lassen die Schlussfolgerung zu,
    dass die Erstellung eines allgemeingültigen {\classificationModel}s
    aufgrund der großen Unterschiede eine komplexe Aufgabe ist.
    Eine kritische, aber berechtigte Frage ist deshalb,
    wie groß der Nutzen durch die Automatisierung tatsächlich ist.
    Vor allem dann, wenn zur Erfassung einiger Inhalte Anpassungen an der Seite notwendig sind.
    Dem ist entgegenzusetzen, dass die Effektivität durch ein anderes Vorgehen beim
    Schreiben der Klassendefinitionen gesteigert werden kann.
    Eine solche Alternative zum Versuch alle Seiten durch ein Modell abzudecken
    besteht aus mehreren Schritten.
    Zunächst wird ein Modell für eine Webseite geschrieben,
    sodass es diese korrekt und vollständig klassifiziert.
    Dieser Schritt ist einfach und bringt zweifelsfrei Vorteile,
    was bei der initialen Klassifizierung der Mitarbeiter des Portals \gls{babw} ersichtlich wurde.
    Dann sollte das {\classificationModel} auf eine weitere Seite angewandt werden und
    das Ergebnis geprüft und angepasst werden,
    bis diese zweite Seite korrekt klassifiziert wird.
    Die Klassifizierung findet in diesem Schritt ausschließlich auf der zweiten Seite statt,
    sodass die erste Klassifikation unberührt bleibt.
    Dieser zweite Schritt wird für alle Seiten wiederholt.
    Dieses Vorgehen erlaubt die Wiederverwendung global gültiger Klassendefinitionen
    und erfordert nur die Anpassung einzelner Klassen an die jeweiligen Anforderungen.
    Insgesamt dürfte der Prozess dadurch deutlich effektiver werden.
    Es sollte außerdem nicht die Möglichkeit missachtet werden,
    bewusst einzelne falsche Klassifizierungen in Kauf zu nehmen
    und diese über das {\annotatorPlugin} im Nachhinein zu korrigieren.
    Nicht erfasste Features und unzutreffende Selektoren können
    aber auch auf Fehler in der Webseite hinweisen.
    Wie z. B. die E-Mail-Adresse ohne vorangestelltes \texttt{mailto}
    im Portal \gls{bscpsy}.
    Das \gls{wccs} fungiert indirekt also als ein Werkzeug
    zur Validierung eines individuell definierbaren Schemas für Webseiten.
    \subsection{Mächtigkeit der Selektoren}
    \label{section:discussionInterpretationSelectors}
    Zusammen mit der Schachtelung von Features und dem
    damit verbundenen eingeschränkten Kontexts bei der
    Auswertung von Selektoren,
    spielen deren Fähigkeiten eine zentrale Rolle bei den
    Möglichkeiten der Klassifizierung, die das \gls{wccs} bietet.
    Betrachtet man die {\classificationModel}e der beiden Fallbeispiele wird klar,
    dass ohne den {\xpathSelector} die Möglichkeiten weitaus eingeschränkter wären.
    Die Klasse \texttt{Portal}, durch die ein Feature das Kontextelement
    als {\referenceFeature} \texttt{homepage} klassifiziert,
    wäre beispielsweise weder mit dem {\cssSelector}
    noch mit dem {\urlSelector} möglich.
    Beide können ausschließlich untergeordnete Elemente erfassen.
    Die Selektoren und die Art ihrer Verwendung hat allerdings auch Grenzen.
    Ein Beispiel sind die Mitarbeiter, deren Namen laut Klassifikation
    "`Prof."' bzw. "`Dr."' sind.
    Ein ähnliches Problem gibt es im zweiten Beispiel bei den Absätzen einer Meldung,
    von denen es unterschiedlich viele geben kann.
    In beiden Fällen lässt sich argumentieren,
    dass mehrere HTML-Elemente zusammen eine Einheit darstellen
    und dass ihr textueller Inhalt deshalb
    in einem skalaren Feature gespeichert werden sollten.
    Anders also als {\collectionFeature}s,
    bei denen jedes Element für sich eine eigenständige Einheit darstellt.
    Die beschriebene Anforderung lässt sich bisher nur umsetzen,
    wenn die übergeordnete Einheit der Elemente durch ein gemeinsames
    Vaterelement dargestellt wird, welches nur genau diese Element enthält.
    Für das erste Beispiele hieße die Anforderung, dass der Selektor des Namens eines Mitarbeiters
    alle relevanten Elemente erfasst und
    das \gls{wccs} die textuellen Werte dieser Elemente
    in einem skalaren Feature speichert.
    Der eindeutige Selektor dieses Features würde alle Elemente umfassen,
    was prinzipiell schon jetzt möglich ist,
    da der Selektor Start- und Endelemente unterscheidet.

    Das zweite Fallbeispiel zeigt eine weitere Grenze des Systems,
    da es nicht möglich ist einen allgemeinen Selektor zu formulieren,
    der die Absätze pro Nachricht korrekt
    erfasst.
    Allgemein ausgedrückt ist eine sinnvolle Klassifizierung nicht möglich,
    wenn ein Element eines {\collectionFeature}s selbst ein {\collectionFeature} besitzt
    und weder das Parent- noch das {\childFeature} alleine in einem Element gekapselt ist.
    Diese Einschränkung ist durch die verwendeten Selektortypen begründet.
    Eine Erweiterung des Systems durch andere Selektortypen, kann diese Lücke
    füllen\footnote{vgl. Kapitel \ref{section:endingOutlook}}.

    Die Entscheidung als Ergebnis eines {\xpathSelector}s auch reinen Text und nicht nur
    HTML-Elemente zu akzeptieren, bringt deutliche Vorteile für das \gls{wccs}.
    Es erlaubt z. B. die Telefonnummer eines Mitarbeiters zu selektieren,
    was sonst nicht möglich wäre, da sie nicht alleine in einem Element steht.
    Allerdings darf in diesem Fall die Bestimmung der Versatzangabe des Startelementes nicht auf
    Basis der Eigenschaft \texttt{innerText} geschehen.
    Wie bei der falsch annotierten Telefonnummer im Portal \gls{bapvs} deutlich
    wird,
    muss der durch XPath ermittelte Text nicht in derselben Form im \texttt{innerText} eines Elementes auftauchen.
    Der zu durchsuchende Text muss deshalb ebenfalls über XPath ermittelt werden.
    Für Drittsystemen wird dadurch die Auswertung eines eindeutigen Selektors allerdings komplexer.

    \subsection{Teilen von Knoten in der Datenbank}
    Der erste Beweggrund für die Verwendung einer Graphdatenbank
    ist die natürliche Modellierung und Speicherung von Verweisen
    zwischen Seiten und auf sonstige {\resources}.
    Des Weiteren sollte durch die Vermeidung von Duplikaten in der Datenbank
    und der Möglichkeit einen Knoten mehrfach zu referenzieren,
    die Möglichkeit entstehen weiterführende Analysen auf den klassifizierten
    Inhalten auszuführen.

    Das zweite Beispiel hat gezeigt,
    wie Referenzen zwischen Seiten auf sehr einfache Art und Weise
    die Reihenfolge der Übersichtsseiten und Navigationspfade explizit machen.
    Eine Auswertung ist trivial, da lediglich ein- und ausgehenden Kanten gefolgt werden muss.
    Andere Datenbankmodelle hätten komplexere Konstrukte erfordert,
    um diese Informationen zu speichern oder hätten komplexere
    Aggregierungsschritte zur Auswertung erfordert.

    Betrachtet man Tabele \ref{table:findingsTeachersFiguresNodesByLabel} wird deutlich,
    dass Text- und Resource-Knoten verhältnismäßig am meisten von der
    Möglichkeit Knoten wiederzuverwenden profitieren.
    Das ist nicht verwunderlich, da sie jeweils genau einen Wert speichern,
    der sie identifiziert und keine ausgehenden Kanten besitzen,
    die seitenspezifische Informationen enthalten.
    Wie Tabelle \ref{table:findingsTeachersFiguresSharedNodes} belegt,
    steigt ihre Wiederverwendung bei mehreren Klassifikationen in einer Datenbank,
    was eine logische Konsequenz der größeren Informationsmenge ist.

    Für Resource-Knoten ist dies aber nicht sofort ersichtlich,
    zum Beispiel die Zahl der mehrfach referenzierten
    Lehrgebiets-Resource-Knoten im Falle einer gemeinsamer Datenbank niedriger ist,
    als die Summe der geteilten Knoten in einzelnen Datenbanken (30 vs. 36).
    Gleichzeitig ist die Zahl der geteilten SubjectArea-Knoten aber höher,
    die jeder einen Lehrgebietsknoten referenzieren.
    Es wird also lediglich ein größerer Teilbaum geteilt.
    Außerdem enthielten die einzelnen Datenbanken identische Resource-Knoten,
    die in der gemeinsamen natürlich zusammengefasst werden konnten.

    Resource-Knoten innerhalb einer Datenbank zu duplizieren
    macht aus semantischen Gründen keinen Sinn,
    da sie eine Entität der Domäne darstellen,
    die genau ein mal existiert, was die Datenbank wiederspiegeln sollte
    und außerdem die Möglichkeiten eines Graphens besser ausschöpft.

    Nicht so eindeutig ist dies allerdings bei Text-Knoten.
    Aus Tabelle \ref{table:findingsTeachersFiguresSharedNodes} geht hervor,
    dass im ersten Fallbeispiel niemals ein Text-Knoten geteilt wurde.
    Stattdessen konnte immer der Content-Knoten,
    der ihn referenziert geteilt werden.
    Deshalb ist die Zahl beider Knoten-Typen gesunken und die der geteilten
    Content Knoten dafür gesteigen.
    Das ist möglich, wenn die entsprechenden Content-Knoten
    sich sowohl in ihrer Klasse als auch in ihren Features nicht unterscheiden.
    Betrachtet man die Klassen der geteilten Content Knoten wird deutlich,
    dass diese keine Features haben.

    Anders ist das beim zweiten Fallbeispiel.
    Dort wurden Text-Knoten geteilt, weil es mehrere News mit derselben ÜBerschrift
    gab, die aber jeweils eine eigene Detailseite referenzieren.
    % TODO: Referenz auf TABELLE!

    
% Oder: Verwendung einer Graphdatenbank
    \subsection{Visualisierung der Klassifikation durch Annotationen}
    Eine wichtige Funktion des \glspl{wccs}
    ist die Visualisierung einer Klassifikation durch Webannotationen
    auf der klassifizierten Webseite.
    Aus diesem Grund folgt eine Übersicht der Annotationen
    des Studienportals \gls{babw},
    welches stellvertretend auch für die restlichen klassifizierten Portale steht.
    Abbildung \ref{image:findingTeachersAnnotationsOverview}
    zeigt einen Ausschnitt der annotierten
    Seite\footnote{Die Darstellungsfehler oben rechts im Kopfbereich
    sowie am Anfang der Brotkrümelnavigation unter dem Portal
    sind dem in Kapitel \ref{section:findingsMethod} beschriebenen
    Annotation Viewer geschuldet.
    Durch die Zwischenschaltung dieser Komponente
    führt der Browser Cross-Origin-Requests durch,
    um die genutzte Bibliothek für Symbole zu beziehen.
    Diese Aufrufe werden allerdings unterbunden,
    weshalb die Symbole nicht korrekt dargestellt werden.
    Bei einer direkten Einbindung des Plugins wäre dies nicht der Fall.
    Der fünfte Link im Kopfbereich wurde richtig klassifiziert.}.
    Bis auf wenige Ausnahmen, die in
    Kapitel \ref{section:findingsTeachersAbnormalitiesBabw} besprochen werden,
    wurden alle klassifizierten Elemente korrekt hervorgehoben.
    Eine detaillierte Ansicht einer beispielhaften Annotation zeigt
    Abbildung \ref{image:findingTeachersSubjectAreaAnnotations}.
    Das Lehrgebiet besitzt korrekterweise zwei Annotationen,
    da das HTML-Element sowohl den Namen als auch den Link enthält
    und deshalb doppelt klassifiziert wurde.

    \begin{figure}[htb]
        \centering
        \includegraphics[width=\textwidth]{../resources/findings/case-study-1/babw/annotations/overview.png}
        \caption{Die annotierte Webseite über Mitarbeiter des Portals \acrshort{babw}}
        \label{image:findingTeachersAnnotationsOverview}
    \end{figure}

    \begin{figure}[htb]
        \centering
        \includegraphics[scale=\screenshotScaleFactor]{../resources/findings/case-study-1/babw/annotations/double-lg-annotation.png}
        \caption{Die Annotationen eines Lehrgebietes}
        \label{image:findingTeachersSubjectAreaAnnotations}
    \end{figure}


    \section{Vergleich mit anderen Arbeiten}
    Die Diskussion der Ergebnisse wird in diesem Kapitel
    durch einen Vergleich der Sprache und des Klassifizierungssystems
    mit vergleichbaren Arbeiten abgeschlossen.

    \subsection{Sprache}
        \label{section:discussionComparisonLanguage}
        Es existieren eine Vielzahl von textuellen und graphischen Sprachen,
        die zur Datenmodellierung geeignet sind.
        Bei eigenen Recherchen wurde allerdings keine offizielle wissenschaftliche Arbeit gefunden,
        die eine Sprache mit einem identischen Anwendungsfall und einer identischen Domäne behandelt.
        
        % TODO: Wollen wir diesen Absatz überhaupt?
        % Auch wenn nirgendwo erwähnt, trotzdem interessanter Vergleich der Ansätze
        % Viele Unterschiede sind aber auf die Frage externe vs. interne DSL zurückzuführen.
        Es wurde lediglich das in Kapitel \ref{section:discussionInterpretationLanguage}
        erwähnte GitHub Projekt\footnote{vgl. \url{https://github.com/sabrams/web-miner}} entdeckt,
        welches die Implementierung einer Sprache mit einer sehr ähnlichen Intention enthält.
        Diese Sprache ist eine in Ruby umgesetzte interne DSL,
        die mittels XPath Informationen aus HTML-Dokumenten bezieht
        und in flachen Datenstrukturen speichert.
        Wie die \gls{wccdl} erfasst die sich wiederholende Elemente in Listen,
        unterstützt offensichtlich aber nicht die Erzeugung komplexer Datenstrukturen.
        Neben der in Kapitel \ref{section:discussionInterpretationLanguage} aufgezeigten besseren
        Ausdrucksstärke, stehen ihr als interne DSL prinzipielle sämtliche Sprachmittel und Funktion
        zur Verfügung, die Ruby bietet.
        Gewonnene Informationen können dadurch beliebig weiterverarbeitet werden.
        Neben den Vorteilen interner DSLs besitzt die Sprache natürlich auch dessen Nachteile,
        wie zum Beispiel die eingeschränkten syntaktischen Möglichkeiten,
        wodurch der Syntax bereits jetzt technischer und komplexer als der der \gls{wccdl} ist.
        Wie bereits angesprochen bietet sie außerdem keine komplexen Strukturen und verwendet lediglich XPath.

        Erweitert man die Domäne, findet man Sprachen,
        die durchaus Überschneidungen mit der \gls{wccdl} besitzen.
        Ein Beispiel ist die \gls{webml} \cite{ceri:webML}.
        Diese Sprache erlaubt die graphische Modellierung von Webseiten
        und bietet dazu fünf verschiedene Modelltypen:
        Structure, Composition, Navigation, Presentation und Personalization.
        Vor allem die beiden zuerst genannten Modelltypen eignen sich
        als Alternative zur \gls{wccdl}.

        Ein weiteres Beispiel ist \gls{uwe},
        ein Software Engineering Ansatz für Webseiten \cite{koch:uwe},
        welcher ein speziell für die Entwicklung von Webanwendung entwickeltes
        UML Profil bietet.
        Es verwendet unter anderem Anwendungsfalldiagramme zur Beschreibung von Anforderungen,
        Klassendiagramme zur Modellierung der Domäne
        und mit speziellen Stereotypen versehende Klassendiagramme zur Modellierung der Navigation
        und Präsentation.
        Klassendiagramme sind ebenfalls geeignet, um die Problemstellungen des \gls{wccs} zu beschreiben.

        Im Vergleich zur \gls{webml} und zum \gls{uwe} besitzt die \gls{wccdl} dennoch stärken,
        wenn es um die Nutzung als Sprache zur Instrumentierung des \gls{wccs} geht.
        Der liegt vor allem in ihrer begrenzteren Domäne,
        wodurch sie deren Konzepte in geeigneteren Sprachkonzepte umsetzen kann.
        Die anderen beiden Sprachen besitzen zum Beispiel kein konkretes Konzept für
        Selektoren, weshalb dazu zum Beispiel ein Attribut in einem Klassendiagramm herhalten müsste.
        Aus der Sicht eines Entwicklers ist dies weniger sprechend als das Konzept der \gls{wccdl}.
        Da sie eine kleinere Domäne abdeckt, besitzt die weniger Konzepte
        und ist dadurch leichter zu erlernen.
        Die stärken von \gls{webml} und \gls{uwe} liegen dagegen in der vollständigen Konzeption
        ganzer Webseiten und Webanwendungen.
    
    \subsection{Klassifizierungssystem}
        \label{section:discussionComparisonClassificationSystem}
        % TODO: Paragraphen ggf. als subsection
        Ein Vergleich des Klassifizierungssystems und seines Ansatzes
        hängt von seiner Einordnung ab.
        Dieses Kapitel führt einen Vergleich in drei denkbaren Kategorien vor.

        \paragraph{\gls{cms}-Migrationen}
        Das \gls{wccs} wurde aufgrund des konkreten Falles der {\fernUni},
        entwickelt bei der eine Migration von Inhalten aus {\wordpress} zu {\imperia} notwendig ist.
        Die Intention des \gls{wccs} für diesen Fall ist die vorbereitende Strukturierung der
        Inhalte aus {\wordpress}.
        Es ist deshalb legitim das \gls{wccs} als ein Bestandteil eines Werkzeugkastens
        für \gls{cms}-Migrationen zu kategorisieren und vergleichebaren Produkten gegenüberzustellen.
        Im Internet finden sich mehrere Dienste die eine einfache Migration versprechen.
        Ein interessanter Anbieter ist cms2cms\footnote{vgl. \url{https://cms2cms.com/}}.
        Dieser wirbt damit Migrationen innerhalb 15 Minuten abzuschließen.
        Dazu muss sowohl die Art eines Quellsystem als auch eines Zielsystems ausgewählt
        und auf beiden ein auf das jeweilige System ausgelegte Plugin installiert werden.
        Die Migration kann anschließend über cms2cms gestartet werden,
        welches anschließend die Inhalte transformiert.
        Wie es genau dabei vor geht und wie es die verschiedenen Konzepte und Strukturen
        der Systeme auf einander abbildet, konnte ohne eine entsprechenden Test nicht ermittelt werden.
        Die Vorteile dieses Systems werden dennoch deutlich:
        Eine aufwändige Vorbereitung der Inhalte ist nicht notwendig.
        Die speziellen Plugins erlauben prinzipiell außerdem eine Behandlung spezieller
        Eigenarten jedes unterstützten \glspl{cms}.
        Die Nachteile treten allerdings ebenso hervor: Ohne eine vorangegangene Strukturierung,
        müssen die Inhalte womöglich später im Zielsystem passend strukturiert werden,
        was, falls notwendig, ebenso aufwendig ist.
        Außerdem unterstützt cms2cms nur gewisse Systeme,
        wozu {\imperia} beispielsweise nicht zählt.
        Eine Transformation basierend auf dem \gls{html} einer Webseite wird zwar angeboten,
        aber nicht vollständig unterstützt.
        Genauso können die Inhalte der Systeme nicht durch Drittsysteme abgerufen werden.
        Beides sind elementare Ideen des \gls{wccs}.

        \paragraph{Web Mining}
        Das \gls{wccs} betreibt im weitesten Sinne eine Form des Web Minings.
        Diese Disziplin ist in die Bereiche Web Structure Mining,
        Web Usage Mining und Web Content Mining unterteilt\cite{markov:webMining}.

        Links zwischen Webseiten erzeugen ein Netzwerk,
        welches im Fokus des Web Structure Mining steht.
        Basierend auf den eingehenden Verweisen kann dabei zum Beispiel
        ein Rang einer Seite berechnet werden,
        den Suchmaschinen zur Sortierung ihrer Ergebnisse verwenden
        \cite[Part I]{markov:webMining}.
        Ein bekannter Vertreter ist ist der von Google verwendete PageRank-Algorithmus
        \cite{page:pageRank}.

        Das \gls{wccs} ist selbst kein Werkzeug zum Web Structure Mining,
        da es weder einen Ranking-Algorithmus implementiert,
        noch einen Webseitennutzer imitiert, der verschiedenen Links folgt.
        Stattdessen analysiert es nur vorgegebene Seite.
        Da es das Netzwerk der klassifizierten Seiten in seiner Graphdatenbank speichert,
        taugt seine Datenbasis prinzipiell aber für die Ermittlung eines Ranks
        der klassifizierten Seiten.

        Beim Web Usage Mining wird das Verhalten eines Webseitenbesuchers analysiert,
        um darin Muster zu finden, Nutzer-Cluster zu bilden und darauf basierend
        zum Beispiel spezielle Inhalte auf einer Webseite anzuzeigen.
        Dazu werden zum Beispiel Log-Dateien analysiert oder spezielle JavaScript-Funktionen
        in Seiten eingebaut, die das Verhalten des Besuchers aufzeichnen.
        \cite[Part III]{markov:webMining}
        Das \gls{wccs} betreibt diese Form des Web Minings in keinster Weise
        und bietet durch seine Klassifikationen auch keine relevanten Informationen,
        da dieser Aspekt beim Design des Systems keine Rolle spielte.

        Das Web Content Mining ist die größte Herausforderung.
        Es versucht Korrelationen zwischen den Inhalten von Webseiten zu finden
        und aus ihnen Informationen zu beziehen,
        sodass Informationen im Internet besser gefunden werden können
        und Suchergebnisse auf mehr als dem Vorhandensein eines Suchbegriffes basieren.
        Dazu wendet es Techniken des Data Minings,
        des Text Minings und des Knowledge Discoveries auf Webdokumente an,
        wobei oftmals Systeme zum maschinellen Lernen zum Einsatz kommen.
        Zwei wichtige Ansätze sind Classification und Clustering.
        Beim ersten werden dem System klassifizierte Objekt präsentiert,
        woraufhin das System versucht nicht klassifizierte Objekte einzuordnen.
        Beim Clustering versucht das System hingegen eigenständig
        Übereinstimmungen und gemeinsame Muster unter Objekten zu finden
        und so Gruppen zu bilden
        \cite[Part II]{markov:webMining}.
        
        In gewissen Grenzen betreibt das \gls{wccs} eine simple Form des Web Content Minings,
        da es basierend auf einer vorhandenen Klassifizierungsregel
        Muster auf einer Webseite aufdeckt und Inhalte strukturiert.
        Durch die Struktur der Datenhaltung sind außerdem weiterführende Informationen ableitbar.
        Die Eignung des \gls{wccs} als vollständiges Web Content Mining Werkzeug ist
        aber explizit nicht gegeben.
        Zum einen findet es nur Muster, die den Regeln exakt entsprechen.
        Des Weiteren ist es weniger mächtig als Werkzeuge,
        die auf maschinelles Lernen setzen.
        
        Beides schränkt seine Möglichkeiten ein,
        was aber durch den angedachten Einsatzzweck begründet ist.
        Die Idee ist für eine definierte Menge an Seiten ein Modell zu entwickeln,
        anhand dessen das \gls{wccs} dann eine Klassifizierung dieser Seiten durchführt.
        Das "`Training"' des System findet also für eine sehr kleine Teilmenge des \glspl{www} statt,
        wie zum Beispiel einigen Seiten der Fakultät \gls{ksw}.
        Aber explizit nicht für das gesamte \gls{www}.
        Dadurch ist es nur auf wenigen Seiten anwendbar,
        kann aber sehr genau für die speziellen Anforderungen dieser Seiten instrumentiert werden.      

        \paragraph{Schemavalidierung von Webseiten}
        Das \gls{wccs} stellt außerdem eine Möglichkeit dar,
        Webseiten auf die Konformität bezüglich eines definierten
        inhaltlichen und strukturellen Schemas zu prüfen.
        Nämlich dann, wenn das Fehlen eines Features als Verletzung
        dieses Schemas interpretiert wird.
        Im ersten Fallbeispielen hätte diese Verwendungsart zum Beispielt aufgedeckt,
        dass bei zwei Kontakten die E-Mail-Adresse inkorrekt ist,
        da sie kein vorangestelltes "`mailto:"' enthalten.
        Eine solche Auswertung muss bisher noch manuell anhand der Klassifikation
        oder durch Datenbankabfragen geschehen.
        Eine Erweiterung des \gls{wccs} und seiner Schnittstellen,
        sodass diese Funktion allgemein nutzbar ist,
        ist einfach zu realisieren.

        Andere Werkzeuge zur Validierung von \gls{html}-Dokumenten
        beschränken sich auf die Einhaltung von Standards der
        \glspl{w3c}\footnote{vgl. \url{https://validator.w3.org/} und \url{https://jigsaw.w3.org/css-validator/}}.
        Vorausgesetzt die Webseite ist ein valides XML-Dokument,
        ließe sich ein individuelles Schema auch über XSD formulieren und validieren.
        Über XSD lassen sich komplexere Bedingungen formulieren und es existieren
        zahlreise Applikationen zur Auswertung.
        Da Webseiten selten valide XML-Dokumente sind,
        besitzt das \gls{wccs} an dieser Stelle durchaus
        ein Alleinstellungsmerkmal.

	%\chapter{Schluss}
    \label{chapter:SummaryAndOutlook}
    Zum Abschluss fasst dieses Kapitel diese Arbeit zusammen
    und zeigt einige Möglichkeiten zur Erweiterung des \glspl{wccs} auf.
    Danach zieht es ein abschließendes Fazit.

    \section{Zusammenfassung}
    Das Ziel dieser Arbeit ist die Konzeption und Entwicklung
    eines Systems zur automatischen Klassifizierung der Inhalt
    von Webseiten, was unter anderem eine domänenspezifische
    Sprache zur Instrumentierung des Systems umfasst.
    Der Titel dieses Systems ist
    "`Web Content Classification System"' (WCCS).

    Motiviert ist dieses Ziel durch die Modernisierung des
    Internetauftrittes der {\fernUni},
    die eine Migration von Inhalten von {\wordpress} zu
    {\imperia} beinhaltet.
    Die Herausforderung dabei ist die sehr schwache Strukturierung
    dieser Inhalte, wodurch sie nicht ohne weiteres in
    Dokumente in {\imperia} transformierbar sind.
    Mithilfe des \gls{wccs} sollen die Inhalte in einer vorbereitenden
    Maßnahme strukturiert werden, um diesem Problem zu begegnen.

    Eine wichtige Komponente des Systems ist eine auf Xtext basierende
    domänenspezifische Sprache namens "`Web Content Class Definition Language"' (WCCDL).
    Diese erlaubt die Definition von fachlichen Seiten-, Inhalts- und
    Referenzklassen, in die die Inhalte eingeordnet werden.
    Definierbare Features dieser Klassen erlauben die Schaffung feingranularer
    und hierarchischen Klassifikationen.
    Zur Einordnung von Inhalten in Klassen verwendet das System
    CSS-, XPath und reguläre Ausdrücke, die in diesem Kontext als
    Selektoren bezeichnet werden und
    ebenfalls Teil der Klassendefinitionen sind.
    Wichtige Merkmale der \gls{wccdl} sind ihre deklarative Natur
    und ihr sehr lesbarer Text,
    der durch die klare Fokussierung auf eine kleine Domäne ermöglicht wird.
    Programme in dieser Sprache werden zu technischen
    Konfigurationsdateien für das Klassifizierungssystem übersetzt.

    Dieses klassifiziert eine Webseite auf Basis ihrer
    \gls{html}-Repräsentation und verwendet Browser-Automatisierung
    zur Auswertung der Selektoren.

    Klassifikationen können durch weitere Komponenten des \gls{wccs}
    als Webannotationen auf der jeweiligen Webseite visualisiert werden,
    was einer Sichtung, ersten Prüfung und Durchführung kleinerer
    Korrekturen der Klassifikation dient.

    Das \gls{wccs} speichert eine Klassifikation in einer Graphdatenbank,
    wodurch die Referenzen zwischen Seiten und zu sonstigen {\resources}
    im \gls{www} leicht abzubilden sind.
    Durch die Möglichkeit Knoten des Graphens in mehreren Klassifikationen
    zu verwenden, entsteht das Potential leicht weiterführende Erkenntnisse
    aus den Informationen zu ziehen.

    Anwendung hat das \gls{wccs} bei zwei Fallbeispielen der {\fernUni} gefunden,
    die gezeigt haben, dass das Konzept des Systems funktioniert,
    aber einige Sonderfälle noch nicht ausreichend abgedeckt sind.

    Ein vergleich mit anderen System hat gezeigt,
    dass die Sprache aufgrund ihrer Spezialisierung für ihren eigenen
    Anwendungsfall einige Vorteile bietet.
    Der Klassifizierungsansatz erfüllt die an ihn gestellten Anforderungen,
    ist aber weniger mächtig als komplexe Web Mining Werkzeuge.
    Ein gewisses Alleinstellungsmerkmal besitzt das \gls{wccs}
    bei der Anwendung als Werkzeug zur Prüfung der Konformität
    einer Webseite bezüglich eines individuellen Schemas.
    \section{Ausblick}
    \label{section:endingOutlook}
    Eine Erweiterung, von der die \gls{wccdl} profitieren würden,
    ist die Einführung eines Vererbungskonzeptes für Klassen,
    um Features und Selektoren wiederzuverwenden und die Ausdrucksstärke
    der Sprache zu erhöhen.
    Hierzu muss ein konsistentes Konzept für Vererbung und Polymorphie
    von Features und Selektoren geschaffen werden,
    welches sich syntaktisch gut einfügt.
    Außerdem erfordert es weitergehende semantische Prüfungen.

    Interessant ist außerdem die Einbettung
    von Sprachen zur Unterstützung von CSS-, XPath- und regulären Ausdrücken,
    sodass Selektoren syntaktisch überprüft werden können
    und ihre Formulierung durch die Entwicklungsumgebung unterstützt wird.

    Der Entwickler könnte außerdem durch ein Browser-Plugin
    deutlich bei der Formulierung von Selektoren unterstützt werden.
    Nämlich dann, wenn dieses Plugin den Browser und die Entwicklungsumgebung
    dahingehend verbindet, dass der Browser solche Elemente auf der aktuellen Webseite
    hervorhebt, die durch einen bestimmten Selektor im {\classificationModel} erfasst werden.
    Andersherum könnte der Entwickler den Browser nutzen,
    um einen Selektor eines \gls{html}-Elementes zu bestimmen und
    in die Entwicklungsumgebung zu übertragen.

    Eine Möglichkeit die beschriebenen Sonderfälle besser abzudecken ist die
    Einführung eines Skript-Selektors,
    da sich programmatisch auch deutlich komplexere Strukturen klassifizieren lassen sollten.
    Zur Auswahl eines \gls{html}-Elementes oder von Text würde also ein Stück JavaScript ausgeführt werden.
    Dank der Browserautomatisierung durch Puppeteer ist dies möglich.

    Eine weitere Maßnahme zu diesem Zweck ist die Einführung von Selektorketten.
    Ein Feature kann dann durch eine sequentiell ausgeführte Reihe von Selektoren
    ermittelt werden, wobei das Ergebnis eines Selektors als Kontextelement des nächsten fungiert.
    Dadurch lässt sich der Suchraum mit verschiedenen Selektortypen eingrenzen,
    ohne zusätzliche Features definieren zu müssen.

    In Kapitel \ref{section:discussionComparisonClassificationSystem}
    wurde das Thema maschinelles Lernen bereits kurz aufgegriffen.
    Im \gls{wccs} sind zwei Anwendungen dieser Technik denkbar:

    \begin{enumerate}
        \item   Das System wird anhand einiger Klassifikationen trainiert
                und versucht anschließend ohne exakte Vorgaben (Selektoren),
                sondern nur auf Basis des Trainings,
                neue Seiten zu erkennen und zu strukturieren.
        \item   Das System analysiert Korrekturen, die durch Nutzer durchgeführt werden
                und erzeugt Vorschläge für Änderungen in weiteren Klassifikationen.
    \end{enumerate}

    Darüber hinaus sind einige weitere kleinere Erweiterungen denkbar,
    die das Werkzeug verbessern.
    Der Nutzen des {\annotatorPlugin}s kann zum Beispiel deutlich erhöht werden,
    indem das Löschen und Anlegen von Annotationen unterstützt wird.
    Außerdem wäre es denkbar, neben den vordefinierten Klassen auch
    die Angabe eigener neuer Klassen zu erlauben.

    Ein Grund der schwachen Ausdrucksstärke der Sprache ist die Notwendigkeit,
    jede Klasse einzeln bekannt machen zu müssen.
    Dem könnte durch die Einführung eingebetteter Klassendefinitionen begegnet werden.
    Einmalig genutzte Klassen ohne Features können dadurch bei ihrer Verwendung
    definiert werden.
    Die Definition eines Features könnte dann folgendermaßen aussehen:

    \lstinputlisting[
        caption=Ausblick auf eingebettete Klassendefinitionen in der \acrshort{wccdl},
        label=listing:outlookEmbeddedClasses,
        inputencoding=utf8/latin1,
        style=wccdl
    ]{../resources/outlook/embedded-classes.wctd}

    \section{Fazit}
    Die \gls{wccdl} ist eine gut lesbare und einfach zu erlernende Sprache,
    die sich im Sinne von domänenspezifischen Sprachen
    auf wenige Anwendungsfälle beschränkt.
    Ein besonderes Merkmal ist ihre ausführliche Syntax,
    wodurch Modelle zwar länger als unbedingt erforderlich sind,
    dafür aber fast wie Fließtext gelesen werden können.
    Mit der \gls{wccdl} lassen sich viele Strukturen auf Webseiten beschreiben.
    Einschränkungen sind hauptsächlich durch die implementierten Selektoren begründet.
    Ihre Syntax kann aber nahtlos erweitert werden, um neue Selektoren zu unterstützen.
    Generierte Modelle können auch unabhängig vom \gls{wccs} weiterverwendet werden,
    wodurch sie prinzipiell auch für andere Anwendungsfälle nutzbar ist.
    Insgesamt erscheint die \gls{wccdl} besser geeignet als andere Sprachen
    zur Beschreibung eines {\classificationModel}s für das \gls{wccs}.

    Mit dem Klassifizierungsansatz des \glspl{wccs} konnte gezeigt werden,
    dass Inhalte einer Webseite durch Software automatisch fachlich strukturiert werden können.
    Diese Klassifizierung auf Basis der \gls{html}-Repräsentation einer Webseite durchzuführen,
    macht dies oftmals sehr einfach, ist mit den verfügbaren Selektoren
    in manchen Sonderfällen aber nicht ausreichend.
    Die Idee der Selektoren ist kein revolutionärer neuer Ansatz,
    in diesem Fall aber dennoch eine ungewöhnliche Anwendung z. B. von {\cssSelector}en.
    Ihre praktische Anwendung kann schnell eine komplexe Herausforderung werden,
    wenn viele atomare Informationen innerhalb eines \gls{html}-Elementes getrennt werden müssen.
    D. h., wenn die Inhalte durch die \gls{html}-Repräsentation nur schwach strukturiert werden.
    Außerdem ist es schwierig Varianz in die Selektoren einzubauen,
    sodass sie auf vielen ähnlichen Webseiten richtig greifen,
    ohne die genaue Struktur jeder Seite in Betracht ziehen zu müssen.

    Insgesamt stellt das Ergebnis der Arbeit trotzdem ein funktionales System dar,
    welches im vorgesehenen Anwendungsfall gut funktioniert
    und Potenzial für weitere Arbeiten bietet.
	\appendix
	%\chapter{Architektur des WCCS}
    Abbildung \ref{image:wccsExternalArchitecture} zeigt die vollständige Architektur des \glspl{wccs},
    wie sie in Kapitel \ref{section:Architecture} vorgestellt wurde.

    \begin{figure}[htb]
        \centering
        \includegraphics[width=\textwidth]{../resources/architecture/complete_architecture.png}
        \caption{Architkektur des \glspl{wccs}}
        \label{image:wccsExternalArchitecture}
    \end{figure}

\chapter{Modell der DSL}
    \begin{figure}[htb]
        \centering
        \includegraphics[width=\textwidth]{../resources/dsl/model.png}
        \caption{Modell der DSL}
        \label{image:dslCompleteModel}
    \end{figure}
	\sloppy
	\printbibliography[heading=bibintoc]
\end{document}
