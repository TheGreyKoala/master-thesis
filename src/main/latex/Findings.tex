\chapter{Ergebnisse}
    \label{chapter:Findings}
    % http://www.fernuni-hagen.de/KSW/portale/babw/einstieg/lehrende-und-betreuende-im-b-a-bildungswissenschaft/
    % Seite, wo Automatisierung viel bringt
    % Telefonnummer schwierig: XPath: substring-before(substring-after(p, 'Tel.: '),'\n')
    % Dazu müsste System noch erweitert werden:
    % - Result Type ist hier anders. Es gibt kein iterateNext. D.h. nur den String auslesen
    % - Node wäre der gleiche wie Parent Node
    % - Offset müsste genau berechnet werden.

    % http://www.fernuni-hagen.de/KSW/portale/lehre/bachelor-und-master-studiengaenge/
    % Da kann man gut Xpath nutzen:
    % Studiengang (als Ganzes) über h5 Selektieren
    % Name des Studienganges über XPath (self): .
    % Text zum Studiengang über XPath (nächster div): following-sibling::div[1]/p[1]
    % Hier sind inhaltliche Strukturen anders als HTML-Strukturen
    % Kann umgangen werden, da Text über einen Node angesprochen werden kann

    % http://www.fernuni-hagen.de/KSW/portale/babw/service/aktuelles/
    % Neuer Artikel über hr
    % Datum: following-sibling::p[@class='datum'][1]
    % Heading: following-sibling::h4[@class='entry-title'][1]
    % Text: Lässt sich über XPath oder CSS nicht auswählen. Man braucht "alle folgenden Nodes bis zum nächsten hr)


    % Wie viel wird auf diesen Seiten geteilt?


    % -- 3 Seitentypen in KSW --> WELCHE SIND DAS?
    % -- 1 externe Seitentypen --> WELCHE IST DAS?
    % -- Für alle Konfiguration schreiben und Klassifizierung laufen lassen
    % -- Ergebnis in Arbeit besteht aus
    % --- DSL Dateien
    % --- Klassifikation (Page Objekt)
    % --- Screenshot der Anntationen?
    % --- Screenshot der WebApp?
    % --- Wie viele Nodes werden in der Datenbank für FernUni Seiten geteilt? Schließlich sind es immer mehrere eines Typs?