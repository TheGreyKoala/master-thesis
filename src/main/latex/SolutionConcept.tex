\chapter{Lösungskonzept}
    \label{chapter:SolutionConcept}
    Dieses Kapitel erläutert ein Lösungskonzept für die in Kapitel \ref{chapter:ProblemAnalysis} beschriebene Problemstellung.
    Zentrale Elemente dieser Lösung sind ein System zur automatischen Klassifizierung der Inhalte von Webseiten
    und eine \gls{dsl} zu dessen Instrumentierung.

    XSLT kann nicht genutzt werden, da HTML kein XML ist.

    % REST-Schnittstelle für Abfrage des Contents für Klassifizierung doof,
    % weil Kontext nicht bekannt. --> Braucht man die Info überhaupt oder ist die Option aufgrund
    % anderer Anforderungen eh egal?

    % Fachliche Klassen reichen alleine nicht aus.
    % System muss instrumentiert werden, wie Inhalte als spezielle Klasse identifiziert werden.

    % Wieso speichert das System nicht nur den Selektor, sondern auch den Inhalt?

    % Allgemeine Nutzung der Daten --> Auch Grund für Graph-DB, falls abstraktes JSON nicht reicht

    % Anforderung nach Nachbesserung ist Grund für AnnotatorJS und WebApp und Seiten als obsolet zu markieren.

    % Unabhängigkeit von FernUni
    % Abstraktion von \gls{cms} Eigenheiten, z.B. Markup der Plugins
    % --> Deshalb HTML scannen

    \section{Klassifizierungssystem}