\section{Visualisierung, Nachbesserung und Prüfung der Klassifikation}
    \label{section:conceptVisualization}

    \subsection{Web Annotationen}
        \label{section:conceptWebAnnotations}
        Die Visualisierung der Klassifikation geschieht über Web Annotationen.
        % TODO: LINK AUF W3C Standard.
        Dadurch werden die klassifizierten Inhalte direkt in der Seite selbst hervorgehoben.

        Eine Annotation ist dabei eine Referenz oder ein Content Feature ohne eigene Content Features.
        Andernfalls würden sich die Annotationen ggf. mehrfach überlagern und man würde nichts mehr erkennen.
        Eine Überlagerung ist zwar nicht ausgeschlossen, ist durch diese Maßnahme aber unwahrscheinlicher.
        Außerdem ist die speziellste Klasse eines Features eh am interessantesten.
        Der Inhalt der Annotation ist die Klasse des jeweiligen Features.

        Neben der Visualisierung dienen Annotationen auch der Nachbesserung der Klassifikation.
        Bearbeitet ein Nutzer eine Annotation, verändert er damit die Klasse des Features,
        was dann so auch persistiert wird.

        Das \gls{wccs} stellt ein entsprechende Erweiterung der JavaScript-Bibliothek Annotator bereit.

        % TODO: Noch erwähnen, dass Klassifizierung nicht getriggert wird?
        % TODO: SCREENSHOT?
    
    \subsection{Webanwendung}
        Über Web Annotationen ist es nicht möglich alle Details einer Klassifikation darzustellen.
        Oder einen Überblick über alle klassifizierten Seiten zu erhalten.
        Zu diesem Zweck stellt das \gls{wccs} deshalb auch eine Webanwendung bereit,
        die diese Informationen liefert.

        % TODO: SCREENSHOT?