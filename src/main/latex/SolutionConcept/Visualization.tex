\section{Visualisierung, Nachbesserung und Prüfung der Klassifikation}
    \label{section:conceptVisualization}
    Es ist möglich, dass eine Klassifizierung ein teilweise unerwartetes Ergebnis liefert,
    weil zum Beispiel Selektoren im Klassifizierungsmodell unzutreffend sind.
    Das \gls{wccs} bietet deshalb Möglichkeiten,
    um eine Klassifikation zu visualisieren und zu prüfen.

    \subsection{Webannotationen}
        \label{section:conceptWebAnnotations}
        % TODO: LINK AUF W3C Standard (https://www.w3.org/annotation/), aber den verwenden wir gar nicht?
        Die Visualisierung der Klassifikation geschieht über Webannotationen,
        wodurch die klassifizierten Inhalte und Referenzen direkt in der Webseite hervorgehoben werden.
        Der Inhalt der Annotation ist die Klasse des annotierten Elementes.
        Beispiele annotierter Webseiten zeigen die Abbildungen
        \ref{image:findingTeachersAnnotationsOverview},
        \ref{image:annotatorPluginViewer} und \ref{image:annotatorPluginEditor}.

        Eine Annotation spiegelt ein {\referenceFeature} oder ein
        {\contentFeature} ohne untergeordnete {\contentFeature}s wieder.
        Wäre jedes {\contentFeature} durch eine Annotation hervorgehoben,
        überlagerten sich die Annotationen unter Umständen mehrfach,
        wodurch sich die Übersichtlichkeit
        und damit der Nutzen der Annotationen deutlich verschlechtern würde.
        Durch Annotationen wird somit die Klassifikation auf der feingranularsten Ebene visualisiert.
        Eine Überlagerung ist aber weiterhin möglich,
        wenn ein {\contentFeature} ausschließlich untergeordnete {\referenceFeature}s besitzt,
        weil dann die Annotation des textuellen Inhalts teilweise durch die
        der Referenzen überlagert wird.
        Eine Unterscheidung ist trotzdem möglich\footnote{vgl. Kapitel \ref{section:solutionDetailsAnnotatorPlugin}}.
        
        Neben der Visualisierung dienen Annotationen auch der Korrektur der Klassifikation.
        Bearbeitet ein Nutzer nämlich eine Annotation,
        ändert er damit die Klasse des Features.

        Diese Funktion realisiert das \gls{wccs} über eine Erweiterung der JavaScript-Bibliothek
        Annotator\footnote{vgl. \ref{section:solutionDetailsAnnotatorPlugin}}.
    
    \subsection{Webanwendung}
        Über Webannotationen ist es nicht möglich anschaulich alle Details einer Klassifikation darzustellen
        oder einen Überblick über alle klassifizierten Seiten zu geben.
        Zu diesem Zweck stellt das \gls{wccs} deshalb auch eine Webanwendung bereit,
        die diese Lücken schließt.
        Die Oberflächen dieser Anwendung sind in Kapitel \ref{section:solutionDetailsWebAppFunctions}
        enthalten.
