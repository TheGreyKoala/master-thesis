\section{Klassifizierung einer Webseite}
    \label{section:conceptClassification}
    Dieses Kapitel beschreibt die konzeptionellen Grundlagen der Klassifizierung.
    % TODO: Irgendwo erwähnen, dass mit Spezifikation konfiguriert

    \subsection{Datenquelle}
        \label{section:conceptClassificationDataSource}
        Erste Frage der Klassifizierung ist, was als Datenquelle genommen wird.
        Wordpress bietet zwei Schnittstellen, um Daten über Posts und Pages abzufragen:
        Direkter Zugriff auf die Datenbank und die Tabelle wp\_posts
        \cite{wordpress:Database}.
        Oder über die REST-API\cite{wordpress:RestAPI}.

        Bei beiden Quellen gibt es Probleme.
        Die Datenbank enthält das spezielles Markup, zum Beispiel von Plugins.
        Das müsste erst ausgewertet werden, um die resultierenden Inhalte zu erlangen.
        Das ist sehr komplex und bei der Fülle an Plugins nicht praktikabel.
        Die REST-API übernimmt diese Auswertung hat aber ein anderes Problem.
        Beide liefern nur den Inhalt des speziellen Posts / Page.
        D.h. Inhalte, die über Templates kommen sind nicht direkt enthalten,
        wie z. B. Header, Footer, Navigation.
        Je nachdem, wie stark diese von der Seite abhängig sind,
        kann das ok (wenn sie immer gleich sind) oder nicht ok (falls sie abhängig sind),
        sein. Im konkreten Fall sind sie immer gleich, aber der Vollständigkeit halber sollte man das
        behalten.
        Allerdings fehlt dadurch der Kontext. D.h. man weiß nicht, wo die Inhalte auf der Seite sind,
        weil die Templates nicht bekannt sind (kann überall in \gls{html}-Hierarchie stehen).
        Das macht es sehr schwer die Klassifikation verlässlich auf der Seite zu visualisieren
        (siehe \ref{section:conceptVisualization}).
        Des Weiteren sind beide Quellen {\wordpress}-spezifisch, wodurch das System weniger
        allgemein einsatzbar wäre.

        Eine allgemeine Lösung, die auch die oben beschriebenen Probleme umgeht ist die Nutzung
        der Live-Seite als Datenquelle.
        Dort hat man die Inhalte ohne spezielles Markup und mit Kontext, da man das komplette \gls{html}-Dokument hat.
        Deshalb sind auch allgemeine Parts, wie Header, Footer enthalten und können bei Bedarf klassifiziert werden.

    \subsection{Allgemeiner Ablauf}
        Der allgemeine Ablauf der Klassifizierung lässt sich leicht und in wenigen Sätzen beschreiben.
        Da als Quelle die Live-Seite herangezogen wird, ist Eingabe die \gls{url} dieser Seite.

        Das System bestimmt zunächst die Klasse der Seite, indem die verschiedenen Selektoren der bekannten
        Seitenklassen gegen sie geprüft werden.
        Dann werden die Features dieser Klasse auf der Seite ermittelt,
        indem die Selektoren der Features auf der Seite angewandt werden.
        Für jedes Content Feature wird die Ermittlung der Features rekursiv wiederholt.
        Anschließend wird die gesamte Klassifikation persistiert.

    \subsection{Unterstützte Selektoren}
        \label{section:conceptSupportedSelectors}
        Das Klassifizierungssystem unterstützt drei Selektoren,
        die durch die gewählte Datenquelle begründet sind.
        Eine andere Quelle hätte womöglich zu anderen Selektoren geführt.

        Die Selektoren sind ein CSS-Selektor, ein XPath-Selektor und ein URL-Selektor
        und werden jetzt vorgestellt.

        % TODO: CSS wählt immer einen Knoten, Xpath Knoten oder Text

        \paragraph{CSS-Selektor}
        Der CSS-Selektor kann für Features verwendet werden und benutzt
        die durch das \gls{w3c} standardisierten CSS-Selektoren
        (siehe Kapitel \ref{section:wwwTechnicalFoundations} und \cite{w3c:cssSelectors}),
        um einen oder mehrere Nodes im \gls{html}-Dokument auszuwählen.
        Da diese Selektoren auf mehrere Nodes zutreffen können,
        sind Collection Featues direkt unterstützt.

        Das \gls{wccs} folgt bei der Auswertung eines CSS-Selektors der Logik des \glspl{w3c}
        \cite{w3c:selectorsAPI}.
        Das bedeutet, dass der Selektor für das gesamte Dokument ausgeführt,
        aber nur Elemente innerhalb des Parent Features im Ergebnis landen.
        Für Features der ersten Ebene wird document als Kontext-Node genommen.
        Dadurch ist sichergestellt, dass sich wiederholende Strukturen richtig zugeordnet werden können.

        Dazu ein Beispiel. Angenommen es gibt ein CollectionFeature mit dem CSS-Selektor
        "`article"'. Dieser wählt die drei article-Nodes aus.
        Die Klasse dieses Features hat wiederum ein Feature "`heading"' mit dem Selektor
        "`h3"'.
        Da dieser im Kontext des jeweiligen article-Nodes ausgeführt wird,
        ist klar, welche Überschrift zu welchem Artikel gehört.

        \begin{lstlisting}
<article>
    <h3>Article 1 Heading<h3>
</article>
<article>
    <h3>Article 2 Heading<h3>
</article>
<article>
    <h3>Article 3 Heading<h3>
</article>
        \end{lstlisting}

        \paragraph{XPath-Selektor}
        Der XPath-Selektor ist ebenfalls für Features geeignet und wertet auf dem
        HTML-Dokument XPath-Ausdrücke aus, um Nodes zu selektieren.
        Im Vergleich zu CSS ist XPath in manchen Belangn mächtiger.
        Zum Beispiel erlaubt es den aktuellen Node über "`self"' zu matchen,
        Siblings des aktuellen Nodes zu selektieren
        oder Nodes in anderen Teilen des Dokumentes zu selektieren \cite{w3c:xpath}.
        Ein XPath-Selektor verwendet ebenfalls document bzw. den Node des Parent Features als
        Kontext Node.
        Die über CSS-Selektoren garantierte Relativität kann vom Nutzer mit den oben genannten Mitteln
        durch XPath-Selektoren allerdings umgangen werden,
        wodurch er mehr Möglichkeiten, aber auch mehr Verantwrotung erhält.

        \paragraph{\gls{url}-Pattern-Selektor}
        Ein \gls{url}-Pattern-Selektor ist für Seiten und Referenzen geeignet.
        Bei diesem Selektor handelt es sich um einen regulären Ausdruck,
        der im Falle von Seiten auf deren \gls{url} angewandt wird.
        Gibt es ein Match, wird die Seite entsprechend klassifiziert.
        
        Zur Klassifizierung von Referenzen wird der Selektor auf die \gls{url} aller Knoten
        innerhalb des Parent Features angewandt, die eine {\resource} referenzieren.
        Gibt es ein Match, wird der Node als entsprechendes Referenz Feature erkannt.
        Da alle Nodes geprüft werden, sind Collection Features kein Problem.

        Dieser Selektor ist sinnvoll, um zum Beispiel externe Links anders als interne zu klassifizieren.

    \subsection{Datenmodell des Klassifizierungsergebnisses}
        \label{section:conceptPageDataModel}
        % TODO: Muss das ggf. höher, vllt. direkt nach Klassen, Features, Selektoren?
        % TODO: Muss der RangeSelector vielleicht doch enthalten sein?
        Dieses Kapitel beschreibt anhand Abbildung \ref{image:conceptPageDataModel}
        das Modell einer Klassifikation einer Seite.

        \begin{figure}[htb]
            \centering
            \includegraphics[scale=\imageScalingFactor]{../resources/concept/page.png}
            \caption{Datenmodell eines Klassifizierungsergebnisses}
            \label{image:conceptPageDataModel}
        \end{figure}

        Oben in der Hierarchie steht die Seite,
        zu der ihre \gls{url} und ihre Klasse gespeichert wird.
        Außerdem hat sie einen Status, der nach der Klassifizierung auf
        "`Classified"' gesetzt ist.
        Des weiteren speichert eine Seite in zwei getrennten Listen ihre
        Content- und Reference Features.
        Elemente dieser Listen können Content oder Reference Objekte oder Listen
        dieser Objekte sein. Listen spiegeln CollectionFeatures wieder.
        Jedes Feature speichert seinen Namen und seine Klasse.
        Content Features darüber hinaus ihren textuellen Inhalt % TODO WARUM?!
        und Reference Features die \gls{url} der referenzierten {\resource}.
        Features speichern außerdem einen Selektor, bei dem es sich nicht um den Selektor
        handelt, der während der Klassifizierung zu ihrer Ermittlung verwendet wurde.
        Stattdessen identifiziert dieser Selektor den zum Feature gehörenden Node eindeutig im HTML-Dokument.
        Das ist später z. B. für die Visualisierung der Klassifikation hilfreich.
        Außerdem bietet es Drittsystemen die Möglichkeit leicht festzustellen,
        ob sich der Inhalt eines Contents oder das Ziel einer Referenz seit der Klassifizierungssystem
        geändert hat.
