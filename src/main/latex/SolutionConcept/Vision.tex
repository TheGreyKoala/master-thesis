\section{Vision}
    An dieser Stelle erfolgt zunächst ein sehr abstrakter Blick auf das System,
    um die Idee der Gesamtlösung zu vermitteln.
    Dazu wird die Vision des angestrebten Klassifizierungsprozesses beschrieben.

    Ein Anwender entwickelt ein Modell zur Klassifizierung einer oder mehrerer Webseiten,
    welches er mit Hilfe einer \gls{dsl} ausdrückt.
    Das Modell wird übersetzt und fungiert zur Instrumentierung des Klassifizierungssystems.
    Der Anwender startet anschließend ein geeignetes Werkzeug,
    welches alle relevanten Webseiten ermittelt und ihre Klassifizierung beauftragt.
    Das \gls{wccs} führt die Klassifizierung auf Basis des bereitgestellten Modells durch
    und persistiert das Ergebnis dauerhaft.
    Eine erste Überprüfung der Klassifikation und kleinere Korrekturen kann der Anwender anschließend auf der
    Webseite selbst vornehmen.
    Dazu dient eine Visualisierung über Webannotationen.
    Eine detailliertere Einsicht erhält er über eine spezielle Webanwendung.
    Nach der Klassifizierung steht Drittsystemen das Ergebnis über eine definierte Schnittstelle
    zur Verfügung.
