\section{Auffindung zu klassifizierender Webseiten}
    \label{section:conceptCrawler}
    Dem \gls{wccs} muss mitgeteilt werden, welche Seiten es klassifizieren soll.
    Das können unter Umständen sehr viele Seiten sein,
    sodass es unpraktikabel ist sie manuell aufzufinden
    und dem \gls{wccs} mitzuteilen.
    Motiviert durch den konkreten Fall der {\fernUni} enthält das \gls{wccs}
    deshalb ein Kommandozeilenwerkzeug, welches alle öffentlichen Webseiten einer
    {\wordpress}-Installation ermittelt und die Klassifizierung für diese Seiten initiiert.
    Dazu verwendet es {\wordpress}' RESTful Webservice \cite{wordpress:RestAPI},
    weil diese Schnittstelle fester Bestandteil von {\wordpress} ist und nicht die
    Installation eines Plugins erfordert.
    Außerdem verwenden ihre Ergebnisse ein standardisiertes Format,
    welches leicht verarbeitet werden kann.
