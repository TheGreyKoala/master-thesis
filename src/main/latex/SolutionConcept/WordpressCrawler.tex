\section{Auffinden der zu klassifizierenden Seiten}
    Das Klassifizierungssystem muss wissen, welche Seiten es klassifizieren soll.
    Unter Umständen sehr viele Seiten, die man nicht manuell ermitteln und dem System
    mitteilen möchte.

    Motiviert durch den konkreten Fall der {\fernUni} enthält das \gls{wccs}
    ein Kommandozeilenwerkzeug, welches alle öffentlichen Seiten einer
    {\wordpress}-Installation ermittelt und die Klassifizierung anstößt.
    Dazu verwendet es {\wordpress}' REST-Schnittstelle \cite{wordpress:RestAPI}.

    Das ist sinnvoll, weil die immer da ist und man kein Plugin installieren muss,
    wie z. B. https://de.wordpress.org/plugins/google-sitemap-generator/.
    Außerdem ist es leicht als Links auf Seiten zu folgen.

    % TODO: Muss man hier noch erklären, warum Sitemap im Front-End doof ist?