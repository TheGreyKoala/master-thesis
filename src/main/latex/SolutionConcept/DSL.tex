\section{Eine domänenspezifische Sprache zur Spezifikation von Klassen}
    Zur Spezifikation von Klassen, Features und Selektoren bietet das
    \gls{wccs} eine domänenspezifische Sprache
    -- die \gls{wccdl} --,
    deren Rolle in diesem Kapitel dargelegt wird.

    \subsection{Funktionen}
        Die Sprache greift die bereits beschriebenen
        Konzepte
        auf und bietet alle Mittel, um konkrete Klassendefinitionen zu erstellen.
        Da diese Definitionen dem Klassifizierungssystem als Modell dienen,
        wird ein Programm in der \gls{dsl} auch als "`{\classificationModel}"' bezeichnet.
        Die \gls{wccdl} erlaubt die Definition von benannten Klassen,
        wobei sie die Unterscheidung zwischen Seiten-, Inhalts und Referenzklassen unterstützt.
        Des Weiteren bietet sie die Möglichkeit, einen Selektor für eine Klasse zu definieren,
        was im Falle von Seitenklassen zwingend erforderlich ist.
        Entwickler sind außerdem in der Lage, sowohl die {\scalarFeature}s
        als auch die {\collectionFeature}s einer Klasse zu deklarieren.
        Neben dem Namen und der Klasse des Features gehört dazu auch ein optionaler Selektor.
        Ob ein Feature ein {\contentFeature} oder ein {\referenceFeature} ist,
        ermittelt die Sprache selbstständig anhand der verwendeten Klasse.

    \subsection{Generierung}
        \label{section:conceptDslGeneration}
        Die \gls{dsl} übersetzt ein {\classificationModel} nicht in ein ausführbares Programm,
        sondern in eine technische Konfigurationsdatei für das Klassifizierungssystem.
        Der Grund ist, dass eine Klassifizierung unabhängig vom verwendeten Modell immer
        dem gleichen Algorithmus
        folgt\footnote{vgl. Kapitel \ref{section:solutionConceptClassificationAlg}
        und \ref{section:solutionDetailsClassificationServiceClassification}},
        weshalb es sinnvoll erscheint, diese Logik in eine separate Komponente auszulagern,
        von der die \gls{wccdl} unabhängig ist.

    % \subsection{Vorteile}
    %     Die Verwendung einer \gls{dsl} hat mehrere Vorteile.

    %     \paragraph*{Erleichterte Konfiguration des Systems}
    %     Das technische Format der Konfigurationsdatei wird verborgen
    %     und die Spezifikation in einem lesbaren Format gespeichert.
    %     Das erlaubt die Formulierung der Klassendefinitionen prinzipiell auch
    %     Nicht-Programmierern.
    %     Selektoren müssen beispielsweise nicht für das technische Format manuell escaped werden,
    %     was ihre Formulierung erleichtert.

    %     \paragraph*{Fehlervermeidung}
    %     Dadurch, dass wir mit einer Sprache arbeiten, die in ein Artefakt generiert wird,
    %     können Fehler besser abgefangen werden.
    %     Z. B. durch syntaktische Korrektheit oder semantische Validitätsprüfungen.
    %     Ein Beispiel sind die zur Verfügung stehenden Selektoren und ihre semantisch korrekte Verwendung.
    %     Genauso kann sichergestellt werden, dass für jedes Feature ein Selektor ableitbar ist.

    %     \paragraph*{Leichtere Wiederverwendbarkeit}
    %     Durch die logische und physische Aufteilung können Klassendefinitionen leichter
    %     wiederverwendet oder übertragen werden, als dies bei der direkten Verwendung eines
    %     technischen Formates wäre.

    %     \paragraph*{Unabhängigkeit vom konkreten Klassifizierungssystem}
    %     Die Sprache ist unabhängig von der konkreten Implementierung des Klassifizierungssystem.
    %     Es ist deshalb möglich für verschiedene Implementierungen verschiedene Dateien zu generieren.