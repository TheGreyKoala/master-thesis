\section{\wordpress}
    Das Open-Source-Projekt {\wordpress} startete 2003
    mit dem Ziel eine Anwendung zur einfachen Pflege eines Weblogs
    (kurz Blog) zu schaffen \cite{wordpress:About}.
    Das Ergebnis ist die gleichnamige Software,
    die noch immer von der Community weiterentwickelt wird
    und von der Fakultät \gls{ksw} der {\fernUni} für ihren
    Internetauftritt genutzt wird.

    In diesem Kapitel werden grundlegende Konzepte dieses Systems vorgestellt.

    \subsection{Weblog-Software}
        \label{section:weblogSoftware}
        {\wordpress} ist im Kern eine Software zur Pflege eines Blogs,
        wobei es sich um eine spezielle Form einer Webseite handelt
        \cite{wordpress:Blogging}:

        \begin{quote}
            "`Blog"' is an abbreviated version of "`weblog"',
            which is a term used to describe websites that maintain
            an ongoing chronicle of information.
            A blog features diary-type commentary and links to articles
            on other websites, usually presented as a list of entries in
            reverse chronological order.
            Blogs range from the personal to the political,
            and can focus on one narrow subject or a whole range of subjects.
        \end{quote}

        Da Blogs eine spezielle Form von Webseiten sind,
        kann man auch eine Anwendung zu ihrer Pflege als
        spezielle Form eines \gls{cms} betrachten.
        Diese Aussage trifft {\wordpress} auch über sich selbst
        und vergleichbare Software \cite{wordpress:Blogging}:

        \begin{quote}
            Many blogging software programs are considered a specific type of CMS.
            They provide the features required to create and maintain a blog,
            and can make publishing on the internet as simple as writing an article,
            giving it a title, and organizing it under (one or more) categories.
        \end{quote}

        Nicht zuletzt weil auch Privatpersonen eine Zielgruppe
        solcher Anwendungen sind, vereinfachen sie demnach Blog-Einträge
        auf die zwei elementaren Elemente Titel und Inhalt.
        Mit dieser schwachen Strukturierung der Inhalte
        unterscheidet sich {\wordpress} deutlich von {\imperia}
        und seinen Dokumenten, die beliebig start strukturiert
        werden können.

        Aufgrund der Anpassbarkeit durch Themes\footnote{vgl. Kapitel \ref{section:wordpressTemplatesThemes}}
        und Plugins\footnote{vgl. Kapitel \ref{section:wordpressPlugins}} sieht sich
        {\wordpress} trotzdem als vollwertiges \gls{cms}
        \cite{wordpress:About}.

    \subsection{Dynamische Generierung}
        Anders als {\imperia} ist {\wordpress} nicht nur das Redaktionssystem,
        sondern gleichzeitig auch das ausliefernde System.
        Verglichen mit der in Kapitel \ref{section:imperiaArch}
        beschriebenen Aufgabenteilung übernimmt {\wordpress} selbst
        also auch die Rolle des Zielsystems.

        Inhalte genereirt {\wordpress} deshalb nicht in statische Dateien,
        sondern erzeugt eine Webseite auf Anfrage dynamisch.

    \subsection{Posts und Pages}
        \label{section:wordpressPostsPages}
        {\wordpress} erlaubt die Pflege von Inhalten in Posts
        und in Pages \cite{wordpress:Pages}.

        \paragraph*{Posts}
        Blog-Einträge und somit die Kernkompetenz von
        {\wordpress} werden über Posts abgebildet.
        Wie in Kapitel \ref{section:weblogSoftware} beschrieben werden sie
        auf der Webseite nach ihrem Erstellungszeitpunkt absteigend sortiert dargestellt
        und besitzen zwei Felder, die mit redaktionellen Inhalten gefüllt werden können:
        Titel und Inhalt \cite{wordpress:Posts}.
        
        Darüber Hinaus stehen dem Anwender aber noch weitere Optionen zur
        Verfügung, wie die Zuweisung einer oder mehrerer Kategorien oder Tags.
        Über "`Custom Fields"' kann er zusätzliche Informationen an einem Post
        speichern, die zum Beispiel von Plugins\footnote{vgl. Kapitel \ref{section:wordpressPlugins}}
        genutzt werden.
        In der Praxis speichern sie Metadaten des Posts und dienen nicht zur Strukturierung
        der Inhalte des Posts \cite{wordpress:Posts}.
        Aufgrund der Tatsache, dass jeder Post über dieselben Custom Fields verfügt,
        ist ihre Eignung dafür auch fraglich.

        Posts können verschiedene Post Types besitzen,
        die auch durch eigene Typen ergänzt werden können.
        Die standardmäßigen Post Types in {\wordpress} sind
        \cite{wordpress:PostTypes}:

        \begin{itemize}
            \item Post
            \item Page
            \item Attachment
            \item Revision
            \item Navigation Menu
            \item Custom CSS
            \item Changesets
        \end{itemize}

        Der Typ eines Posts ändert nichts an seinen redaktionellen Feldern.
        Das heißt, es ist nicht möglich für einen eigenen neuen Post Type
        neue Felder für die Strukturierung der Inhalte zu spezifizieren
        \cite{wordpress:Posts, wordpress:PostTypes}.

        \paragraph*{Pages}
        Neben Posts kennt {\wordpress} auch das Konzept einer Page,
        die sich in ihrem Zweck klar von einem Post unterscheidet
        \cite{wordpress:Pages}:

        \begin{quote}
            In contrast, pages are generally for non-chronological,
            hierarchical content: pages like "`About"' or "`Contact"'
            would be common examples.
            [...]
            Pages live outside of the normal blog chronology,
            and are often used to present timeless information about
            yourself or your site -- information that is always relevant.
            You can use Pages to organize and manage the structure of your website content.
        \end{quote}

        Aus diesem Grund besitzen Pages zwar ebenfalls einen Titel und Inhalt,
        können aber lediglich mit Tags versehen werden.
        Die Einordnung in eine Kategorie ist nicht möglich
        \cite{wordpress:Pages}.

        Technisch gesehen ist eine Page nur ein Post,
        dessen Post Type "`Page"' ist.

    \subsection{Vorlagen und Themes}
        \label{section:wordpressTemplatesThemes}
        Wie {\imperia} strebt auch {\wordpress} eine Trennung von
        Inhalt und Layout an.
        Inhalte werden dazu in
        Posts und Pages\footnote{vgl. Kapitel \ref{section:wordpressPostsPages}}
        unabhängig vom Layout der Webseite gespeichert.
        Das Layout bestimmen Vorlagen und Themes.

        \paragraph*{Vorlagen}
        {\wordpress} nutzt Vorlagen, um Inhalte in eine Seite einzubinden
        und ihr Aussehen festzulegen.
        Dazu definieren sie das Gerüst der Webseite und enthalten Kommandos,
        um Inhalte aus der Datenbank auszulesen.
        Sie werden für eine konkrete Seite dynamisch von {\wordpress} generiert,
        wodurch eine vollständige Webseite entsteht
        \cite{wordpress:Templates}.

        Vorlagen sind bei genauerem Hinsehen nichts anderes als PHP-Dateien,
        die \gls{html}, allgemeinen PHP-Code und sogenannte
        "`Template-Tags"' enthalten.
        Dabei handelt es sich lediglich um Aufrufe von
        {\wordpress} eigenen PHP-Funktionen,
        um Inhalte aus der Datenbank abzufragen.
        Während der Generierung -- technisch lediglich die Ausführung
        des PHP-Codes -- können Vorlagen andere Vorlagen inkludieren,
        wodurch wiederkehrende Elemente in eigene Vorlagen ausgelagert
        werden können \cite{wordpress:TemplateFiles}.

        Anhand des \glspl{url} der Anfrage und der darin enthaltenen
        Kennung eines Posts oder einer Page,
        entscheidet {\wordpress} welche Vorlage es zur Generierung der Seite nutzt
        \cite{wordpress:TemplateHierarchy}.

        \paragraph*{Themes}
        Eine Sammlung aller {\resources}, die notwendig sind um
        eine Webseite und ihr Layout umzusetzen,
        wird im Kontext von {\wordpress} als "`Theme"' bezeichnet.
        Ein Theme enthält demnach Vorlagen, Bilder und
        JavaScript-, CSS-, oder PHP-Dateien.
        Themes können {\resources} anderer Themes wiederverwenden oder überschreiben,
        wodurch Anpassungen an einem vorhandenen Theme einfach umzusetzen sind
        \cite{wordpress:Themes}.

    \subsection{Plugins}
        \label{section:wordpressPlugins}
        Es ist möglich {\wordpress} über sein Plugin-System funktional
        zu erweitern \cite{wordpress:Plugins}.
        Nicht selten schreiben Erweiterung Anweisungen in den Inhalt
        eines Beitrages, die sie zum Generierungszeitpunkt erkennen und auswerten.
        Das heißt, dass diese Anweisungen in {\wordpress'} Datenbank stehen.

        Ein Beispiel ist das Plugin
        "`Form Maker"'\footnote{\url{https://wordpress.org/plugins/form-maker/}},
        welches Anweisungen wie \texttt{[Form id="1"]} in den Inhalt eines Beitrages
        einfügt, um ein zuvor konfiguriertes Formular an die entsprechende Stelle einzubinden.