\chapter{Einführung}
    \label{chapter:introduction}
    Ein zeitgemäßer Internetauftritt mit aktuellen Inhalten
    ist ein wichtiger Bestandteil der Öffentlichkeitsarbeit jeder Organisation,
    da sie für viele Interessierte die erste Anlaufstelle zur Beschaffung von Informationen ist.

    Wichtige Gründe hierfür sind die ständige Verfügbarkeit sowie die Orts-
    und Geräteunabhängigkeit.
    Eine Webseite steht zu jeder Tageszeit zur Verfügung und kann
    dank moderner Endgeräte wie Smartphones und Tablets
    auch unterwegs aufgerufen werden.
    Für den Nutzer stellt der Internetauftritt deshalb ein Medium dar,
    das er einfach, spontan und flexibel verwenden kann.

    Wegen dieser Beliebtheit dient die Webseite neben der reinen Bereitstellung von Informationen
    auch der eigenen Werbung und Vermarktung und spielt eine wichtige Rolle im Erfolg einer Organisation.
    Zwei Eigenschaften des Webauftrittes sind laut \cite{sillence:onlineHealthSites} dabei
    entscheidend: Inhalt und Design.

    Dieses Kapitel geht zunächst auf diese beiden Aspekte ein,
    um anschließend die Herausforderungen bei der Modernisierung der Webseite der {\fernUni} zu erläutern.
    Basierend darauf wird abschließend das Thema dieser Arbeit vorgestellt.
    
    \section{Inhaltliche Pflege und das Design einer Webseite}
        \label{section:ContentManagementAndDesign}
        Ab einer gewissen Größe der Webseite erscheint es nachvollziehbar,
        dass zur inhaltlichen Pflege eine eigene Rolle geschaffen wird,
        die durch entsprechend ausgebildetes Personal besetzt wird.
        So können stets aktuelle, relevante und qualitativ hochwertige Inhalte erzielt werden.
        Ein gebräuchlicher Titel dieser Rolle ist \textit{\editor},
        der auch im Verlauf dieser Arbeit verwendet wird.

        {\editors} verwenden zur Pflege ihrer Inhalte üblicherweise ein \gls{cms},
        welches ihnen für diese Aufgabe zum Beispiel formularbasierte Eingabemasken bereitstellt.
        Eine abstrakte Beschreibung solcher Systeme liefert \cite[][Seite 5,6]{barker:webCMS}:

        \begin{quote}
            A content management system (CMS) is a software package that provides
            some level of automation for the tasks required to effectively manage content.
            [...]

            A CMS allows editors to create new content, edit existing content,
            perform editorial processes on content, and ultimately make that content
            available to other people to consume it.
            [...]
        \end{quote}

        Des Weiteren beschreibt \cite[][Seite 9-12]{barker:webCMS} die Kernaufgaben eines \gls{cms}:
        \begin{enumerate}
            \item   Kontrolle der Inhalte über Rollen- und Rechtekonzepte,
                    Versionierung, Abhängigkeitsmanagement, Such- und Strukturierungsmöglichkeiten
            \item   Wiederverwendung von Inhalten ermöglichen
            \item   Automatische Aggregation, Verarbeitung und Aufbereitung von Inhalten
            \item   Arbeit der {\editors} effizienter gestalten
        \end{enumerate}

        Wie zuvor beschrieben, ist neben dem Inhalt auch das Design,
        mit dem Inhalte präsentiert werden, relevant für den Erfolg einer Webseite.
        Die Entwicklung der Trends im Webdesign seit den frühen 1990er Jahren veranschaulicht
        \cite{work:webDesignEvolution}.
        Gezeigt wird die Entwicklung angefangen bei Seiten, die nur Text enthalten,
        über Tabellenlayouts, Flash-Applikationen bis zum Responsive Design und für
        Mobilgeräte optimierte Seiten.
        Webauftritte entwickeln sich demnach stetig weiter,
        was \cite{murphy:webDesignEvolution} an Beispielen wie der Internetseite der
        Fluggesellschaft Ryanair ebenfalls verdeutlicht.

        Folglich ist jede Organisation angehalten, laufend ihre Webseite inhaltlich zu pflegen
        und ihr regelmäßig ein modernes Design zu geben.
        Andernfalls läuft sie Gefahr die Erwartungen ihrer Besucher nicht zu erfüllen,
        die sich deshalb schlecht angesprochen fühlen und womöglich zur Konkurrenz wechseln.

        Eine Modernisierung des Designs stellt den Inhaber der Webseite zwangsläufig vor die Frage,
        was mit seinen bestehenden Inhalten geschehen soll.
        Eine kurze Übersicht der Möglichkeiten und ihrer Argumente erwähnt \cite{kahl:contentMigration}.
        Zum einen könnten Inhalte 1:1 übernommen und lediglich im neuen Design dargestellt werden.
        Diese Option biete sich zum Beispiel für Pressemitteilungen an, die nie mehr verändert werden.
        Alternativ sollte man in Erwägung ziehen alte Inhalte zu verwerfen und durch neuere
        zu ersetzen, um einer neuen Marketingstrategie besser gerecht zu werden.
        Welche Variante angemessen ist, hinge immer von den konkreten Inhalten ab.

    \section{Modernisierung der Webseite der \fernUni}
        \label{chapter:FernUniRelaunch}
        Die \fernUni\footnote{\url{http://www.fernuni-hagen.de/}}
        steht vor einer Modernisierung ihres Internetauftrittes\footnote{Stand 01.02.2018},
        wodurch die Seite unter anderem ein responsives Design erhält und damit besser für
        mobile Endgeräte geeignet sein wird.

        Dieses Kapitel geht auf die \gls{cms}-Landschaft der {\fernUni} ein und zeigt auf,
        welche Herausforderung bei der Modernisierung ihres Internetauftrittes dadurch resultiert. 

        \subsection{Content Management Systeme an der \fernUni}
            \label{section:fernUniCMS}
            An der Universität arbeiten {\editors} mit dem \gls{cms} \textit{{\imperia} 9}
            \cite{fernUni:imperia}
            der \textit{pirobase imperia GmbH}\footnote{https://www.pirobase-imperia.com}.
            Eine Ausnahme stellt allerdings die Fakultät für \gls{ksw} dar.
            Hier ist das Open-Source-Produkt \textit{\wordpress}\footnote{https://wordpress.org/} im Einsatz.
            Für die Fakultät ist ein Vorteil dieser Abspaltung,
            dass sie frei über das Design ihrer Seiten bestimmen kann
            und deshalb schon jetzt ein responsives Design verwendet.
            Für die {\fernUni} als Ganzes hat dies allerdings unter anderem folgende Nachteile:

            \begin{enumerate}
                \item   Es existiert kein durchgehendes Corporate Design
                \item   Es existiert keine zentrale Datenhaltung, da beide Systeme ihre Inhalte getrennt speichern
                \item   Es entsteht ein erhöhter Administrationsaufwand
                \item   {\editors} verschiedener Fakultäten arbeiten mit verschiedenen Systemen und müssen entsprechend geschult sein.
                        Ein Wechsel eines {\editors} in eine andere Fakultät ist unter Umständen mit weiteren Schulungsmaßnahmen verbunden.
            \end{enumerate}

        \subsection{Herausforderungen}
            \label{section:fernUniChallenges}
            Die in Kapitel \ref{section:fernUniCMS} beschriebenen negativen Auswirkungen 
            begründen die Entscheidung, dass die Seiten der Fakultät \gls{ksw}
            in Zukunft ebenfalls das neue Design der {\fernUni} verwenden und
            über {\imperia} gepflegt werden sollen.

            Ihre vorhandenen Inhalte möchte die {\fernUni} nicht verwerfen, sondern übernehmen.
            Für die Inhalte der Fakultät \gls{ksw}, die bisher in {\wordpress} gepflegt werden,
            bedeutet dies, dass sie zu {\imperia} migriert werden müssen.

            Vergleicht man die Eingabemasken beider Systeme, fällt auf,
            dass sie unterschiedlichen Herangehensweisen folgen.
            Eine Eingabemaske in {\imperia} besteht aus einem feingranularen Formular,
            in dem verschiedene inhaltliche Elemente einer Seite in verschiedenen Formularfeldern
            gespeichert sind.
            In {\wordpress} hingegen besteht das Formular lediglich aus zwei Feldern:
            eines für den Titel der Seite und eines für den Inhalt der Seite.
            Für eine Migration stehen generell zwei Möglichkeiten zur Auswahl,
            um mit dieser Differenz umzugehen:

            \begin{enumerate}
                \item   Für die aus {\wordpress} stammenden Seiten werden in {\imperia} Formulare geschaffen,
                        die ebenfalls nur Felder für den Titel und den Inhalt einer Seite enthalten.
                        Die Inhalte können dann 1:1 übertragen werden.
                \item   Die Inhalte der aus {\wordpress} stammenden Seiten werden vor der Migration strukturiert,
                        sodass sie in die feingranularen Formulare in {\imperia} übertragen werden können.
            \end{enumerate}

            Beide Ansätze besitzen Vor- und Nachteile.
            Die erste Methode ist vergleichsweise einfach zu realisieren,
            hätte allerdings zur Folge, dass für die Inhalte der Fakultät \gls{ksw} Sonderbehandlungen
            in {\imperia} geschaffen werden, da die restlichen Fakultäten ihre bekannten feingranularen
            Formulare beibehalten möchten.
            Das hieße zum Beispiel, dass die Fakultät \gls{ksw} andere Formulare für die Pflege aktueller
            Nachrichten verwendet, als die restlichen Fakultäten.
            Da das Design beider Seiten am Ende aber identisch sein soll, bedeuten unterschiedliche Formulare
            auch Sonderbehandlungen in der Erzeugung der Seiten.
            Solche Sonderbehandlungen erzeugen erhöhte Aufwände sowohl im initialen Modernisierungsprojekt
            als auch in der späteren Wartung.
            Nicht auszuschließen ist außerdem, dass {\editors} der Fakultät \gls{ksw} aufgrund der groben Formularstruktur
            weniger redaktionelle Möglichkeiten hätten, als die anderer Fakultäten.
            % TODO: Technische Herausforderungen? (Markup der Plugins in DB)

            Die Vor- und Nachteile der zweiten Methode sind komplementär zu denen der ersten Methode.
            Das Resultat ihrer Umsetzung wäre universitätsweit einheitliche Formulare
            ohne Sonderbehandlungen für die ehemaligen Seiten aus {\wordpress}.
            Gleichzeitig ist die Strukturierung der Inhalte aber keine triviale Aufgabe.
            Unter anderem stellen sich dabei die folgenden Fragen:

            \begin{itemize}
                \item Welche Arten von Seiten liegen vor (Aktuelle Nachrichten, Prüfungen, Module, etc.)?
                \item Welche Inhalte besitzen diese Seitentypen?
                \item Wie müssen diese Inhalte strukturiert werden?
            \end{itemize}

            Die Beantwortung dieser Fragen erfordert eine genaue Analyse der Seiten der Fakultät \gls{ksw}
            und einen Vergleich mit den geplanten Seiten und Formularen in {\imperia}.
            
            Neben dieser konzeptionellen Herausforderung entstehen auch nennenswerte operative Aufwände bei der Durchführung.
            Die tatsächlichen Inhalte jeder einzelnen Seite der Fakultät \gls{ksw}  müssen
            mit den neuen Strukturen abgeglichen und in diese überführt werden.
            Bei den über 4000 Seiten\footnote{Stand 01.02.2018} der Fakultät
            würde sich diese Aufgabe ohne Automatismus als zeitaufwendig und anfällig für Fehler erweisen.

    \section{Inhalt und Aufbau der Arbeit}
        Die in Kapitel \ref{section:fernUniChallenges} festgestellte Herausforderung
        wirft die Frage auf, wie mittels Software die Inhalte aller Webseiten der Fakultät \gls{ksw}
        automatisch in definierte Strukturen überführt werden können.
        
        Ausgehend von der konkreten Anforderung an der {\fernUni} stellt diese Arbeit eine
        \gls{dsl} vor, die zur Instrumentierung eines Softwaresystems dient,
        welches die Inhalte von Webseiten automatisch klassifiziert.
        Dieses System ist ebenfalls Teil dieser Arbeit.

        Zur Veranschaulichung wird dabei immer wieder auf Beispiele der {\fernUni}
        zurückgegriffen, was nicht bedeutet, dass das vorgestellte System
        speziell für die Bedürfnisse dieser Universität zugeschnitten ist.
        Vielmehr ist eine allgemeine Lösung angestrebt.

        Die vorliegende Arbeit ist folgendermaßen aufgebaut:

        In \textit{Kapitel \ref{chapter:ProblemAnalysis}} wird die Problemstellung ausführlicher
        analysiert und spezifiziert, was auch die Betrachtung der Problemdomäne beinhaltet.

        Aufbauend auf den dann vorliegenden Erkenntnissen erarbeitet
        \textit{Kapitel \ref{chapter:SolutionConcept}}
        ein Lösungskonzept und erläutert die Architektur des Systems.

        Eine detaillierte Betrachtung der einzelnen Komponenten des Systems
        erfolgt in \textit{Kapitel \ref{chapter:SolutionDetails}}.
        Dazu geht das Kapitel auf die Anforderungen an jede Komponente,
        konzeptionelle Details, Schnittstellen sowie Aspekte der Implementierung ein.
        
        In \textit{Kapitel \ref{chapter:Findings}} findet das System in zwei Fallstudien
        Anwendung, deren Ergebnisse präsentiert werden.
        Bei der ersten Fallstudie handelt es sich um einige Seiten der Fakultät \gls{ksw}
        der \fernUni.
        Die zweite Fallstudie % TODO Zweite Fallstudie?!

        Eine kritische Diskussion der Ergebnisse der Fallstudien erfolgt anschließend in
        \textit{Kapitel \ref{chapter:FindingsDiscussion}}.

        Abschließend fasst \textit{Kapitel \ref{chapter:SummaryAndOutlook}}
        diese Arbeit zusammen und zeigt Einschränkungen und Erweiterungsmöglichkeiten
        des vorgestellten Systems auf.

        Einige Kapitel dieser Arbeit setzen gewisse Kenntnisse des Lesers voraus.
        Notwendige Grundlagen werden nicht zentral in einem separaten Kapitel vermittelt,
        sondern in den Kapiteln, in denen sie benötigt werden.