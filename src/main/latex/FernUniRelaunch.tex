\chapter{Modernisierung des Internetauftrittes der FernUniversität in Hagen}
    \label{chapter:FernUniRelaunch}
    Ein zeitgemäßer Internetauftritt mit aktuellen Inhalten
    ist ein wichtiger Bestandteil der Öffentlichkeitsarbeit jeder Organisation,
    da sie für viele Interessierte die erste Anlaufstelle zur Beschaffung von Informationen ist.
    
    Wichtige Gründe hierfür sind die ständige Verfügbarkeit sowie die Orts-
    und Geräteunabhängigkeit.
    Eine Webseite steht zu jeder Tageszeit zur Verfügung und kann
    dank moderner Endgeräte wie Smartphones und Tablets
    auch unterwegs aufgerufen werden.
    Für den Nutzer stellt der Internetauftritt deshalb ein Medium dar,
    das er einfach, spontan und flexibel verwenden kann.

    Wegen dieser Beliebtheit dient die Webseite neben der reinen Bereitstellung von Informationen
    auch der eigenen Werbung und Vermarktung und spielt eine wichtige Rolle im Erfolg einer Organisation.
    Zwei Eigenschaften des Webauftrittes sind laut \cite{sillence:onlineHealthSites} dabei
    entscheidend: Inhalt und Design.

    Ab einer gewissen Größe der Webseite erscheint es deshalb nachvollziehbar,
    dass zur inhaltlichen Pflege eine eigene Rolle geschaffen wird,
    die durch entsprechend ausgebildetes Personal besetzt wird.
    Ein gebräuchlicher Titel dieser Rolle ist \textit{\editor},
    der auch im Verlauf dieser Arbeit verwendet wird.

    \editors verwenden zur Pflege ihrer Inhalte üblicherweise ein \gls{cms}.
    Eine abstrakte Beschreibung solcher Systeme liefert \cite[][Seite 5,6]{barker:webCMS}:

    \begin{quote}
        A content management system (CMS) is a software package that provides
        some level of automation for the tasks required to effectively manage content.
        [...]

        A CMS allows editors to create new content, edit existing content,
        perform editorial processes on content, and ultimately make that content
        available to other people to consume it.
        [...]
    \end{quote}

    Des Weiteren beschreibt \cite[][Seite 9-12]{barker:webCMS} die Kernaufgaben eines \gls{cms}:
    \begin{enumerate}
        \item   Kontrolle der Inhalte über Rollen- und Rechtekonzepte,
                Versionierung, Abhängigkeitsmanagement, Such- und Strukturierungsmöglichkeiten
        \item   Wiederverwendung von Inhalten ermöglichen
        \item   Automatische Aggregation, Verarbeitung und Aufbereitung von Inhalten
        \item   Arbeit der \editors effizienter gestalten
    \end{enumerate}

    Wie zuvor beschrieben, ist neben dem Inhalt auch das Design,
    mit dem Inhalte präsentiert werden, relevant für den Erfolg einer Webseite.
    Die Entwicklung der Trends im Webdesign seit den frühen 1990er Jahren veranschaulicht
    \cite{work:webDesignEvolution}.
    Gezeigt wird die Entwicklung angefangen bei Seiten, die nur Text enthalten,
    über Tabellenlayouts, Flash-Applikationen bis zum Responsive Design und für
    Mobilgeräte optimierte Seiten.
    Webauftritte entwickeln sich demnach stetig weiter,
    was \cite{murphy:webDesignEvolution} an Beispielen wie der Internetseite der
    Fluggesellschaft Ryanair ebenfalls verdeutlicht.

    Folglich ist jede Organisation angehalten laufend ihre Webseite inhaltlich zu pflegen
    und ihr regelmäßig ein modernes Design zu geben.
    Andernfalls läuft sie Gefahr die Erwartungen ihrer Besucher nicht zu erfüllen,
    die sich deshalb schlecht angesprochen fühlen und womöglich zur Konkurrenz wechseln.