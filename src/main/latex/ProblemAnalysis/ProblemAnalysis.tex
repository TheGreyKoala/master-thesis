\chapter{Problemanalyse}
    Kapitel \ref{chapter:introduction} ist nur einleitend
    auf die Problemstellung dieser Arbeit eingegangen.
    Eine ausführlichere Betrachtung folgt in diesem Kapitel.
    Dazu wird zunächst die Problemdomäne erläutert.
    Es folgt die Vorstellung der sich an der {\fernUni}
    im Einsatz befindenden \gls{cms}.
    Basierend auf diesen Ausführungen wird anschließend
    die Herausforderung und die resultierende Problemstellung
    detailliert diskutiert und formuliert.

    \label{chapter:ProblemAnalysis}
    \section{Webseiten}
    \label{section:problemAnalysisWebpagesInTheWWW}
    Für ein besseres Verständnis der Aufgabenstellung sowie zur Findung
    einer geeigneten Lösung ist es sinnvoll, die technischen und fachlichen
    Grundlagen von Webseiten zu betrachten.
    Die folgenden Erläuterungen beinhalten auch Aspekte des \glspl{www},
    da Webseiten ein Teil des \glspl{www} sind.

    \subsection{Das World Wide Web}
        Das \gls{w3c} definiert den Begriff "`World Wide Web"' in \cite{w3c:wwwArch} wie folgt:

        \begin{quote}
            The \textit{\textbf{World Wide Web}} (\textit{\textbf{WWW}}, or simply \textit{\textbf{Web}})
            is an information space in which the items of interest, referred to as resources,
            are identified by global identifiers called Uniform Resource Identifiers (\textit{\textbf{URI}}).
        \end{quote}

        Darüber hinaus nennt \cite{w3c:wwwArch} die drei grundlegenden Komponenten des \glspl{www}:

        \begin{quote}
            They are identification of resources,
            representation of resource state, and the protocols
            that support the interaction between agents and resources in the space.
        \end{quote}

    \subsection{{\resources}}
        \label{section:problemAnalysisWebpagesInTheWWWResources}
        Aus der obigen Definition wird deutlich,
        dass das \gls{www} nicht ausschließlich zur Nutzung von Webseiten
        vorgesehen ist.
        Stattdessen kann es jede Art von {\resources} bereitstellen.
        Die Beschreibung des \glspl{w3c} stellt außerdem klar,
        dass jede {\resource} im \gls{www} eindeutig über einen \gls{uri} identifiziert wird.
        Im Fall von Webseiten geschieht dies über \glspl{url},
        die eine Teilmenge von \glspl{uri} darstellen
        \cite[Kapitel 1.1.3]{rfc:3986}.

        Das Schema einer \gls{uri} beschreibt laut \cite[Kapitel 3.1]{rfc:3986},
        wie die restlichen Teile der \gls{uri} zu interpretieren sind.
        Viele Schemata sind nach Protokollen benannt,
        weshalb sie in der Praxis häufig auch das Verfahren bestimmen,
        mit dem auf eine {\resource} zugegriffen werden kann
        \cite[Kapitel 3.1]{w3c:wwwArch}.
        Schemata, die im \gls{www} häufig Anwendung finden,
        sind zum Beispiel
        \texttt{http},
        \texttt{https},
        \texttt{mailto},
        \texttt{ftp} und
        \texttt{data}.
        Für Webseiten sind dabei vor allem \texttt{http} und \texttt{https}
        und die gleichnamigen Protokolle von Bedeutung,
        da \glspl{url} diese vorrangig verwenden.

    \subsection{Web Agents}
        Um {\resources} im \gls{www} anzusprechen, existieren verschiedene Werkzeuge.
        Das \gls{w3c} \cite[Kapitel 6]{w3c:wwwArch} bezeichnet jedes von ihnen als
        "`Web agent"' und beschreibt diese Rolle wie folgt:
        "`A person or a piece of software acting on the information
        space on behalf of a person, entity, or process."'.
        Eine Spezialisierung stellen die sogenannten "`User agents"' dar,
        die das \gls{w3c} \cite[Kapitel 6]{w3c:wwwArch} als "`One type of Web agent;
        a piece of software acting on behalf of a person."' beschreibt.
        Webbrowser sind User agents, aber zum Beispiel auch Kommandozeilenprogramme
        wie curl\footnote{\url{https://curl.haxx.se/}}.
        Ihre persönliche Kennung halten User Agents zum Beispiel im
        HTTP-Header "`User-Agent"' fest \cite[Kapitel 5.5.3, Seite 46]{rfc:7231}.

        Der Prozess, bei dem ein Agent über eine \gls{uri} auf eine {\resource}
        zugreift, heißt "`Dereferencing the URI"'.
        Wie dieser Prozess aussieht, hängt von der Funktion des User Agents und der \gls{uri} ab.
        Ein User Agent könnte bspw. eine
        Repräsentation einer {\resource} abrufen, wohingegen ein anderer Agent
        lediglich prüft, ob eine eben solche Repräsentation existiert
        \cite[Kapitel 3.1]{w3c:wwwArch}.
        Wenn ein Webbrowser oder ein anderer Agent eine Webseite aufruft,
        fragt er bei einem Webserver also eine Repräsentation
        einer {\resource} an, die er über eine \gls{url} spezifiziert.
        Falls der Server die \gls{url} der Anfrage auflösen kann,
        antwortet er auf die Anfrage mit einer Repräsentation der Webseite.
        Der Browser übernimmt dann das Rendering der Seite und die Ausführung
        von eingebetteten Skripten.

        \subsection{HTML-Repräsentation}
            Webseiten sind Dokumente, die die Auszeichnungssprache \gls{html} verwenden,
            die aktuell in der Version 5 \cite{w3c:html5} vorliegt.
            Bis zur Version 3.2 war \gls{html} eine Implementierung der Metasprache SGML \cite[Kapitel 3]{w3c:html401}.
            Zusammen mit der Version 4 \cite{w3c:xhtml} wurde eine Ausprägung spezifiziert,
            die \gls{html} in eine Implementierung der Sprache XML überführt,
            die den Namen XHTML trägt.
            \gls{html}5 \cite[Kapitel 1.6]{w3c:html5} definiert eine abstrakte Sprache,
            die sowohl mit \gls{html}- als auch mit XML-Syntax genutzt werden kann.
            Einem Agent kann über den MIME-Type mitgeteilt werden,
            ob ein Dokument \gls{html}- oder XML-Syntax verwendet \cite[Kapitel 1.6]{w3c:html5}.
            Diese Angabe ist wichtig, da XML-Syntax restriktiver als \gls{html}-Syntax ist.
            Der \gls{html}-Syntax erfordert zum Beispiel nicht,
            dass alle öffnenden Tags auch einen schließenden Gegenpart besitzen
            \cite[Kapitel 3.2.3]{w3c:html5}.

            \gls{html}-Dokumente bestehen in beiden Syntaxvarianten aus verschachtelten
            Elementen, die in 10 Kategorien \cite[Kapitel 4]{w3c:html5} eingeteilt werden.
            Diese sind in Tabelle \ref{table:htmlElements} inkl. einiger Vertreter aufgelistet.
            Die verschiedenen Elemente geben ihren Inhalten eine semantische Bedeutung.
            Text in einem \texttt{h1}-Element \cite[Kapitel 4.3.6]{w3c:html5} wird von einem Webbrowser zum Beispiel in die Anzeige gerendert,
            wohingegen er in einem \texttt{script}-Element \cite[Kapitel 4.11.1]{w3c:html5} als Code interpretiert und ausgeführt wird.

            \begin{table}[h]
                \centering
                \begin{tabular}{|l|l|}
                \hline
                \textbf{Kategorie} & \textbf{Elemente (Auswahl)} \\
                \hline
                Das Wurzelelement & \texttt{html} \\
                \hline
                Metadaten des Dokumentes & \texttt{head}, \texttt{title}, \texttt{style} \\
                \hline
                Absätze & \texttt{body}, \texttt{article}, \texttt{section}, \texttt{h1} \\
                \hline
                Gruppierungen & \texttt{p}, \texttt{ul}, \texttt{div} \\
                \hline
                Semantische Kennzeichnung von Text & \texttt{a}, \texttt{strong}, \texttt{code} \\
                \hline
                Kennzeichnung von Textänderungen & \texttt{ins}, \texttt{del} \\
                \hline
                Eingebetteter Inhalt & \texttt{img}, \texttt{iframe}, \texttt{object} \\
                \hline
                Tabellenelemente & \texttt{table}, \texttt{tr}, \texttt{td} \\
                \hline
                Formulare & \texttt{form}, \texttt{input}, \texttt{button} \\
                \hline
                Skripte & \texttt{script}, \texttt{canvas} \\
                \hline
                \end{tabular}
                \caption{Die Kategorisierung von \acrshort{html}-Elementen}
                \label{table:htmlElements}
            \end{table}
        
            Eine Besonderheit von \gls{html} ist die Möglichkeit
            andere {\resources} des \glspl{www} zu referenzieren,
            wodurch ein gerichtetes und zyklisches Netzwerk entsteht.
            Beispiele für häufig in Webseiten referenzierte {\resources}
            sind andere Webseiten, Skripte, Stylesheet-Dateien, Bilder und Videos.
            Zur Referenzierung dieser Beispiele dienen die \gls{html}-Elemente
            \texttt{a} \cite[Kapitel 4.5.1]{w3c:html5},
            \texttt{script} \cite[Kapitel 4.11.1]{w3c:html5},
            \texttt{link} \cite[Kapitel 4.2.4]{w3c:html5},
            \texttt{img} \cite[Kapitel 4.7.1]{w3c:html5} und
            \texttt{video} \cite[Kapitel 4.7.6]{w3c:html5},
            die auch die Semantik der Referenz festlegen.
            Ein Bild kann sowohl über das \texttt{img}-
            als auch das \texttt{a}-Element referenziert werden.
            Im ersten Fall wird ein Webbrowser das Bild direkt in die Anzeige rendern,
            wohingegen er im zweiten Fall nur einen klickbaren Link auf das Bild darstellt.

        \subsection{Trennung der Zuständigkeiten}
            \label{section:problemAnalysisWebpagesInTheWWWSeparationOfConcerns}
            Die Trennung der Zuständigkeiten \cite{huersch:SeparationOfConcerns}
            ist ein allgegenwärtiges Konzept der Informatik.
            Im Sinne dieses Konzeptes existieren im Kontext von Webseiten
            drei Belange, die es zu trennen gilt:

            \begin{enumerate}
                \item Struktur und Inhalt
                \item Darstellung
                \item Funktionalität
            \end{enumerate}

            Der Inhalt einer Webseite ist in den \gls{html}-Elementen enthalten,
            deren Hierarchie darüber hinaus die Struktur der Seite bestimmt.
            Das Aussehen einer Seite wird hingegen über \gls{css} \cite{w3c:css} festgelegt.
            Dabei handelt es sich um "`[...] eine Sprache zur Beschreibung des Renderings
            von strukturierten Dokumenten (wie HTML und XML) auf Bildschirmen, auf Papier,
            in Sprache, etc."' \cite{w3c:css}.
            Zu diesem Zweck werden mit \gls{css} Regeln definiert,
            die einer Menge von \gls{html}-Elementen Eigenschaften in der Darstellung (Styles) zuweisen
            \cite{w3c:cssSyntax}.
            Diese Menge von Elementen wird wiederum über Selektoren \cite{w3c:cssSelectors} festgelegt,
            mit denen sich verschiedene Eigenschaften der auszuwählenden Elemente beschreiben lassen.
            Falls zwei Regeln dasselbe Element ansprechen,
            werden die Styles beider Regeln zusammengefasst
            \cite{w3c:cssCascading}.
            Überschneidungen in den Styles werden in diesem Fall anhand der Selektoren der betroffenen Regeln aufgelöst.
            Verschiedene Selektoren sprechen einzelne Elemente nämlich verschieden stark an.
            Bezogen auf das betroffene Element lässt sich dadurch eine Priorität festlegen,
            bei der eine Regel mit starkem Selektor Vorrang vor einer Regel mit schwachem Selektor hat
            \cite{w3c:cssSelectors}. 
            Wichtig zu beachten ist, dass \gls{css} die angewandten Styles eines Elementes auf alle Unterelemente vererbt.
            Diese können Styles allerdings explizit überschreiben
            \cite{w3c:cssCascading}.
            Auf den verbliebenen Aspekt "`Funktionalität"' geht der folgende Abschnitt ein.

        \subsection{Dynamische Webseiten}
            Komplexe Webanwendungen sind auf die Möglichkeit angewiesen, Geschäftslogik auszuführen.
            Dafür kommen sowohl der Webserver als auch der User Agent (Client) infrage.
            Der Server kann zur Bearbeitung einer Anfrage mit dem Inhalt einer Datei auf seinem Dateisystem antworten.
            Das kann zum Beispiel eine \gls{html}-Datei oder ein Bild sein.
            In diesem Fall spricht man von statischen Inhalten.
            Es ist allerdings auch möglich,
            dass eine Webanwendung serverseitig Geschäftslogik zur Bearbeitung einer Anfrage ausführt.
            Die Antwort an den Client kann in diesem Fall weiterhin statisch oder
            durch die Webanwendung generiert worden sein.
            Im letzten Fall spricht man von dynamisch generierten bzw. erzeugten Inhalten.

            Ein \gls{html}-Dokument kann in \texttt{script}-Elementen \cite[Kapitel 4.11.1]{w3c:html5} ebenfalls Logik enthalten,
            zu deren Formulierung Entwickler häufig die Skriptsprache JavaScript verwenden.
            Dabei handelt es sich um eine Implementierung des Sprachstandards ECMAScript
            \cite{ecma:ecmaScript}.
            Nicht der Server, sondern der anfragende User Agent -- häufig ein Webbrowser -- führt diese Logik aus.
            Es existieren verschiedene Programmierschnittstellen \cite[Kapitel 8]{whatwg:html},
            die solchen Skripten erlauben,
            Elemente zu manipulieren, auf Nutzerinteraktion zu reagieren
            oder asynchrone HTTP-Anfragen \cite{whatwg:xhr} auszuführen.
            Falls eine Webseite solche clientseitigen Mittel verwendet,
            wird sie als dynamische Seite bezeichnet.
            
            Client- und serverseitige Programmteile funktionieren oftmals Hand in Hand,
            wie das folgende Beispiel illustriert:
            Ein Nutzer klickt auf eine Schaltfläche in einer Webseite.
            Dadurch wird eine JavaScript-Funktion aktiviert,
            die eine asynchrone HTTP-Anfrage an einen Webserver schickt.
            Aufgrund dieser Anfrage führt der Server Geschäftslogik aus
            und generiert auf Basis des Ergebnisses ein \gls{html}-Fragment,
            welches er in seine Antwort schreibt.
            Der Empfang der Antwort löst im Browser die Ausführung einer weiteren
            JavaScript-Funktion aus, die das \gls{html}-Fragment
            an eine definierte Stelle der Webseite einfügt,
            wodurch es dem Nutzer sichtbar wird.
            Der konzeptionelle Ablauf in diesem Szenario sowie die verwendeten Mittel
            werden unter dem Begriff \gls{ajax} \cite{garrett:ajax} zusammengefasst.

        \subsection{Websites}
            \label{section:problemAnalysisWebpagesInTheWWWWebsites}
            Der Internetauftritt einer Organisation besteht selten aus einer
            einzelnen Webseite.
            Stattdessen besteht er aus vielen Webseiten,
            die untereinander verlinkt sind und von der jede eine eigene
            \gls{url} besitzt.
            Neben "`Internetauftritt"' hat sich der Begriff "`Website"' zur
            Referenzierung dieser Gesamtheit aller Webseiten einer Organisation
            etabliert \cite{duden:Internetauftritt, oxford:Website}.
    \section{\imperia}
    % TODO: Wie werden Medien gespeichert?
    {\imperia} ist ein kommerzielles Enterprise Content Management und
    Web Content Management System, welches seit 1995 entwickelt
    \cite{imperia:about, imperia:historie} und von 
    der {\fernUni} zur Pflege ihrer Webseiten genutzt wird
    \cite{fernUni:imperia}.

    Dieses Kapitel geht auf die wichtigsten Merkmale und Konzepte
    dieses \gls{cms} ein.

    \subsection{Statische Generierung}
        \label{section:imperiaStaticGeneration}
        {\imperia} ist ein statisch generierendes \gls{cms}.
        Das bedeutet, dass es Inhalte basierend auf Vorlagen
        in Dateien generiert, die dann auf ein Zielsystem übertragen
        und dort beliebig und vor allem unabhängig von
        {\imperia} genutzt werden können
        \cite[Kapitel 1.1]{imperia:ecmd}.

        Die Generierung der Dateien und deren Übertragung auf ein Zielsystem
        werden zusammgengefasst als "`Publizierung"' bezeichnet.

        % TODO Vorteile

    \subsection{Dokumente}
        \label{section:imperiaDocuments}
        Die zentrale Datenstruktur in {\imperia} sind Dokumente,
        da sie zur Speicherung von Inhalten dienen
        \cite[Kapitel 1.1]{imperia:ecmd}.
        Dokumente werden beim Anlegen stets einer Kategorie zugewiesen,
        die festlegt auf welcher Voralage\footnote{vgl. Kapitel \ref{section:imperiaTemplates}}
        das Dokument basiert.
        Die Vorlage bestimmt wiederum welches Eingabeformular
        {\editors} zur Pflege der Inhalte des Dokumentes verwenden
        \cite[Kapitel 1.1.4]{imperia:ecmd}.
        
        {\imperia} speichert ein Dokument als simple Menge von
        Schlüssel-Wert-Paaren.
        Als Schlüssel -- auch Metavariablen genannt -- dienen die Namen der Felder des Eingabeformulares.
        Entsprechend übernimmt {\imperia} die Inhalte dieser Felder
        als Werte der Schlüssel-Wert-Paare.
        Diese allgemeine Datenstruktur ermöglicht die Nutzung der Inhalte
        in verschiedenen Ausgabeformaten, wie \gls{html}, XML, etc
        \cite[Kapitel 1.1.2]{imperia:ecmd}.

    \subsection{Vorlagen}
        \label{section:imperiaTemplates}
        Ein wichtiges Ziel von {\imperia} ist die Trennung von Inhalt
        und Layout einer Webseite.
        Zu diesem Zweck speichert es Inhalte layoutunabhängig
        in Dokumenten\footnote{vgl. Kapitel \ref{section:imperiaDocuments}}.

        Das Layout wird hingegen in Vorlagen festgehalten,
        die zwei Ziele verfolgen
        \cite[Kapitel 36]{imperia:ecmd}:

        \begin{enumerate}
            \item {\editors} ohne technische Kenntnisse eine einfache Pflege von Inhalten ermöglichen
            \item Inhalte in das Layout integrieren
        \end{enumerate}

        Zur Erfüllung des ersten Zieles kann jede Vorlage ein Eingabeformular
        und dessen Felder spezifizieren.
        Dazu stehen geläufige Komponenten wie Textfelder,
        aber auch {\imperia} eigene Elemente zur Verfügung
        \cite[Kapitel 1.1.4]{imperia:ecmd}.
        Jeder Kategorie wird eine Vorlage zugewiesen,
        wobei eine Vorlage von mehreren Kategorien genutzt werden kann.
        Durch die Einteilung der Dokumente in Kategorien
        \footnote{vgl. Kapitel \ref{section:imperiaDocuments}}
        kann jedes Dokument somit einer Vorlage zugeordnet werden.
        Das durch diese Vorlage definierte Eingabeformular
        wird {\editors n} zur Pflege der Inhalte des Dokumentes präsentiert
        \cite[Kapitel 1.1.4]{imperia:ecmd}.

        Das zweite Ziel erreichen Vorlagen,
        indem sie neben dem Eingabeformular auch ein Gerüst für das
        Layout definieren.
        Über eine spezielle Syntax können sie an beliebigen Positionen
        Metavariablen\footnote{vgl. Kapitel \ref{section:imperiaDocuments}} referenzieren,
        deren Werte im Ausgabedokument an der entsprechenden Stelle integriert werden
        \cite[Kapitel 36]{imperia:ecmd}.

        Vorlagen können in zwei Variaten vorliegen.
        Die erste vereint sowohl die Definition des Eingabeformulares
        als auch die des Layouts in einer Datei.
        Ein {\editor} kann Dokumente dadurch nach dem Prinzip \gls{wysiwyg} bearbeiten.
        Die Eingabefelder entfernt {\imperia} automatisch im Ausgabedokument.
        Die zweite Variante trennt beide Definitionen in unterschiedliche Dateien.
        Dieses Vorgehen ist sinnvoll, wenn die Inhalte eines Dokumentes mit
        verschiedenen Layouts genutzt werden sollen.
        In diesem Fall existiert also eine Vorlage,
        die das neutrale Eingabeformular bestimmt
        und mehrere Layout-Vorlagen, die dieselben Inhalte unterschiedlich darstellen
        \cite[Kapitel 36]{imperia:ecmd}.

    \subsection{Workflows}
        Ein Dokument durchläuft von seiner Generierung bis zur
        Publizierung mehrere Verarbeitungsschritte,
        die in {\imperia} sogenannte Workflows festlegen.
        Jeder Kategorie wird dazu ein Workflow zugewiesen,
        den die Dokumente durchlaufen müssen.
        Ein solcher Workflow legt unter anderem fest,
        welche Schritte und in welcher Reihenfolge
        zu durchlaufen sind
        \cite[Kapitel 1.1.5]{imperia:ecmd}.

        Ein typischer Workflow ist der Folgende
        \cite[Kapitel 1.1]{imperia:ecmd}:

        \begin{itemize}
            \item Erstellung des Dokumentes und Angabe allgemeiner Daten
            \item Inhaltliche Pflege des Dokumentes
            \item Prüfung der Inhalte durch einen zweiten berechtigten \editor
            \item Publizierung der Inhalte
        \end{itemize}

    \subsection{Architektur}
        \label{section:imperiaArch}
        {\imperia} basiert auf einer mehrschichtigen Client-Server-Architektur,
        die in Abbildung \ref{image:imperiaArchitektur} dargestellt ist.

        \begin{figure}
            \centering
            \includegraphics[width=\textwidth]{../resources/imperia/architektur.png}
            \caption{Architkektur von {\imperia} \cite{imperia:ecmd}}
            \label{image:imperiaArchitektur}
        \end{figure}

        Die wichtigste Komponente in dieser Architektur ist der imperia Server,
        der die zentralen Funktionen des Systems bereitstellt.
        Dazu gehören das Anlegen und Strukturieren von Projekten
        und Dokumenten sowie die Ausführung von Workflows.
        Nicht zuletzt verwaltet er die Datenhaltung.

        Die verschiedenen Nutzer des Systems wie {\editors} und Administratoren
        verwenden für ihre Arbeit eine Weboberfläche,
        die über HTTP(S) mit dem Server kommuniziert.
        Über das gleiche Protokoll können auch Drittsysteme den Server
        ansprechen und verschiedene Aktionen durchführen oder Daten abfragen.

        Sobald ein Dokument alle Workflow-Schritte durchlaufen hat,
        wird es durch eine automatische oder manuelle Publizierung
        in eine Datei generiert, die dann auf ein Zielsystem übertragen wird.
        Dieses System ist eigenständig und gehört nicht zu {\imperia}.
        Allerdings bietet es die Möglichkeit über Dienste wie (S)FTP
        die generierten Dateien zu empfangen.
        Ein Beispiel sind Webserver, auf die Webseiten publiziert werden,
        die sie dann Besuchern bereitstellen.
    \section{\wordpress}
    % TODO:
    % - Sites!!!
    \label{section:WordPress}
    Das Open-Source-Projekt {\wordpress} startete 2003
    mit dem Ziel eine Anwendung zur einfachen Pflege eines Weblogs
    (kurz Blog) zu schaffen \cite{wordpress:About}.
    Das Ergebnis ist die gleichnamige Software,
    die noch immer von der Community weiterentwickelt
    und von der Fakultät \gls{ksw} der {\fernUni} für ihren
    Internetauftritt genutzt wird.
    In diesem Kapitel werden grundlegende Konzepte dieses Systems vorgestellt.

    \subsection{Weblog-Software}
        \label{section:weblogSoftware}
        {\wordpress} ist im Kern eine Software zur Pflege eines Blogs,
        wobei es sich um eine spezielle Form einer Webseite handelt
        \cite[Kapitel "`Introduction to Blogging"']{wordpress:codex}:

        \begin{quote}
            "`Blog"' is an abbreviated version of "`weblog"',
            which is a term used to describe websites that maintain
            an ongoing chronicle of information.
            A blog features diary-type commentary and links to articles
            on other websites, usually presented as a list of entries in
            reverse chronological order.
            Blogs range from the personal to the political,
            and can focus on one narrow subject or a whole range of subjects.
        \end{quote}

        Da Blogs eine spezielle Form von Webseiten sind,
        kann man auch eine Anwendung zu ihrer Pflege als
        spezielle Form eines \glspl{cms} betrachten.
        Diese Aussage trifft {\wordpress} auch über sich selbst
        und vergleichbare Software \cite[Kapitel "`Introduction to Blogging"']{wordpress:codex}:

        \begin{quote}
            Many blogging software programs are considered a specific type of CMS.
            They provide the features required to create and maintain a blog,
            and can make publishing on the internet as simple as writing an article,
            giving it a title, and organizing it under (one or more) categories.
        \end{quote}

        Nicht zuletzt, weil auch Privatpersonen eine Zielgruppe
        solcher Anwendungen sind, vereinfachen sie Blogeinträge
        auf die zwei elementaren Elemente Titel und Inhalt.
        Mit dieser schwachen Strukturierung der Inhalte
        unterscheidet sich {\wordpress} deutlich von {\imperia}
        und seinen Dokumenten, die beliebig stark strukturiert
        werden können.
        Aufgrund der Anpassbarkeit durch Themes
        und Plugins sieht sich
        {\wordpress} trotzdem als vollwertiges \gls{cms}
        \cite{wordpress:About}.

    \subsection{Dynamische Generierung}
        \label{section:problemAnalysisWordPressDynamicGeneration}
        Anders als {\imperia} ist {\wordpress} nicht nur das Redaktionssystem,
        sondern gleichzeitig auch das ausliefernde Zielsystem.
        Inhalte generiert {\wordpress} deshalb nicht in statische Dateien,
        sondern erzeugt eine Webseite auf Anfrage dynamisch.

    \subsection{Posts und Pages}
        \label{section:wordpressPostsPages}
        {\wordpress} unterscheidet zwei Arten von Beiträgen \cite[Kapitel "`Pages"']{wordpress:codex},
        auf denen eine Webseite basieren kann: Posts
        und Pages.

        \paragraph*{Posts}
        Klassische Blogeinträge werden in {\wordpress} "`Posts"' genannt.
        Neben der Pflege des Titels und des Inhaltes eines Posts stehen dem
        Anwender noch weitere Optionen zur Verfügung.
        Ein Post kann in Kategorien einsortiert oder mit Schlagwörtern versehen werden.
        Dadurch kann {\wordpress} Übersichtsseiten generieren,
        die z. B. alle Posts einer Kategorie enthalten.
        Über den eigentlichen Inhalt hinausgehende Informationen können
        in sogenannten "`Custom Fields"' gespeichert werden.
        Davon machen z. B. Plugins Gebrauch
        und speichern Metadaten des Posts in ihnen.
        Jeder Post besitzt einen Post Type,
        der eine Aussage über die Art des Beitrages macht.
        {\wordpress} definiert einige Standardtypen \cite[Kapitel "`Post Types"']{wordpress:codex},
        lässt aber auch die Angabe eigener Typen zu.
        Die Standardtypen sind:

        \begin{itemize}
            \item Post,
            \item Page,
            \item Attachment,
            \item Revision,
            \item Navigation Menu,
            \item Custom CSS und
            \item Changesets.
        \end{itemize}

        Der Typ eines Posts ändert nichts an seinen redaktionellen Feldern.
        Das heißt, unabhängig vom Post Type besitzt ein Beitrag nur
        einen Titel und Inhalt
        \cite[Kapitel "`Posts"' \& "`Post Types"']{wordpress:codex}.
        Stattdessen machen Typen wie Revision und Custom CSS deutlich,
        dass {\wordpress} Posts auch zur Realisierung technischer Anforderung verwendet.

        \paragraph*{Pages}
        Neben Posts kennt {\wordpress} auch das Konzept einer Page \cite[Kapitel "`Pages"']{wordpress:codex},
        die sich in ihrem Zweck klar von einem Post unterscheidet:

        \begin{quote}
            In contrast, pages are generally for non-chronological,
            hierarchical content: pages like "`About"' or "`Contact"'
            would be common examples.
            [...]
            Pages live outside of the normal blog chronology,
            and are often used to present timeless information about
            yourself or your site -- information that is always relevant.
            You can use Pages to organize and manage the structure of your website content.
        \end{quote}

        Aus diesem Grund besitzen Pages zwar ebenfalls einen Titel und Inhalt,
        können aber lediglich mit Schlagwörtern versehen werden.
        Die Einordnung in eine Kategorie ist nicht möglich
        \cite[Kapitel "`Pages"']{wordpress:codex}.
        Wie aus der oben erfolgten Auflistung der Post Types hervorgeht,
        sind Pages technisch gesehen lediglich Posts mit dem Post Type "`Page"'.

        {\wordpress} speichert alle Inhalte in einer relationalen Datenbank \cite[Kapitel "`Database Description"']{wordpress:codex}.
        Posts und Pages teilen sich in dieser Datenbank eine Tabelle,
        was ebenfalls verdeutlicht, dass Pages lediglich Posts eines speziellen Typs sind.
        Der Inhalt eines Beitrages wird in der Datenbank als \gls{html}-Fragment abgelegt,
        welches während der dynamischen Generierung in die Webseite übernommen wird.
        Zusätzlich enthält ein Beitrag aber auch spezielles Anweisungen\footnote{vgl. Kapitel \ref{section:wordpressPlugins}},
        welches von {\wordpress} während der Generierung interpretiert wird.

    \subsection{Vorlagen und Themes}
        \label{section:wordpressTemplatesThemes}
        Wie {\imperia} strebt auch {\wordpress} eine Trennung von
        Inhalt und Layout an.
        Inhalte werden dazu in
        Posts und Pages
        unabhängig vom Layout der Webseite gespeichert.
        Das Layout bestimmen Vorlagen und Themes.

        \paragraph*{Vorlagen}
        {\wordpress} nutzt Vorlagen \cite[Kapitel "`Templates"']{wordpress:codex}, um Inhalte in eine Seite einzubinden
        und ihr Aussehen festzulegen.
        Dazu definieren sie das Gerüst der Webseite und enthalten Kommandos,
        um Inhalte aus der Datenbank auszulesen.
        Eine Webseite wird auf Basis einer Vorlage dynamisch von {\wordpress} generiert.
        Vorlagen sind bei genauerem Hinsehen nichts anderes als PHP-Dateien,
        die \gls{html}, allgemeinen PHP-Code und sogenannte
        "`Template-Tags"' enthalten.
        Dabei handelt es sich um Aufrufe von
        {\wordpress}-eigenen PHP-Funktionen,
        um Inhalte aus der Datenbank abzufragen.
        Während der Generierung -- technisch lediglich die Ausführung
        des PHP-Codes -- können Vorlagen andere Vorlagen inkludieren,
        wodurch wiederkehrende Elemente in eigene Vorlagen ausgelagert
        werden können \cite[Kapitel "`Template Files"']{wordpress:codex}.
        Anhand der \gls{url} der Anfrage und der darin enthaltenen
        Kennung eines Posts oder einer Page,
        entscheidet {\wordpress}, welche Vorlage es zur Generierung der Seite nutzt
        \cite[Kapitel "`Template Hierarchy"']{wordpress:codex}.

        \paragraph*{Themes}
        Eine Sammlung aller {\resources}, die notwendig sind, um
        eine Webseite und ihr Layout umzusetzen,
        wird im Kontext von {\wordpress} als "`Theme"' bezeichnet.
        Ein Theme enthält demnach Vorlagen, Bilder sowie
        JavaScript-, \gls{css}-, und PHP-Dateien.
        Themes können {\resources} anderer Themes wiederverwenden oder überschreiben,
        wodurch Anpassungen an einem vorhandenen Theme einfach umzusetzen sind
        \cite[Kapitel "`Using Themes"']{wordpress:codex}.

    \subsection{Plugins}
        \label{section:wordpressPlugins}
        {\wordpress} besitzt ein Plugin-System \cite[Kapitel "`Plugins"']{wordpress:codex},
        über das es beliebig funktional erweitert werden kann,
        ohne an {\wordpress}' eigenen Quellen Änderungen vorzunehmen.
        Es existiert eine große Anzahl an freien Plugins,
        die oft benötigte Funktionen implementieren.
        Meist sind dies Funktionen, die über eine reine Weblog-Software hinausgehen.
        Ein Beispiel ist das Plugin
        "`Form Maker"'\footnote{vgl. \url{https://wordpress.org/plugins/form-maker/}},
        welches die Definition von Webformularen erlaubt,
        die dann in beliebigen Beiträgen eingebettet werden können.
        Zur Nutzung eines Plugins wird meist eine spezielle Anweisung -- ein Shortcode \cite[Kapitel "`Shortcode API"']{wordpress:codex} --
        in den Inhalt eines Posts oder einer Page geschrieben,
        die zum Zeitpunkt der Generierung ausgewertet wird.
        Beim Form Maker Plugin wäre dies zum Beispiel \texttt{[Form id="1"]},
        wodurch das Formular mit der Kennung "`1"' in die Webseite integriert wird.

    % \section{Eine Migration von {\wordpress} zu {\imperia}}
        % Wie würde eine manuelle Migration laufen?
        % Ein Redakteur erhält Auftrag eine Seite zu migrieren
        % Heißt, dass jemand einen Überblick hat, welche Seiten existieren, welche mirgriert werden sollen und welche nicht
        % Redakteur muss Inhalte kopieren
        % Macht er von Live-Seite oder WP-Vorschau. Im WP-Formular steht spezielles Markup von Plugins etc., dass er nicht versteht
        % Öffnet Imperia
        % Entscheidet selbst oder anhand von Regeln, zu welcher Kategorie die neue Seite gehört
        % Oder ob WP-Seite auf mehrere Imperia Seiten aufgeteilt wird
        % Kopiert Inhalte von WP-Seite in Imperia Formulare
        % Entscheided selbst oder anhand von Regeln, wie Formulare mit Inhalten zu füllen sind
        % Kopiert ggf. Medien und bindet sie ein
        % Setzt Links auf andere Seiten
        % Wenn Arbeit beendet, prüft ein zweiter Redakteur (4-Augen-Prinzip)

        % Wie kann dieser Prozess automatisiert werden?
        % - Tool kann WP-API nutzen, um WP-Seiten abzufragen (keine Herausforderung)
        % - Tool kann für jede Seite n Dokumente in Imperia anlegen und füllen (keine Herausforderung)
        % - Tool kann z. B. anhand der Kategorien einer WP-Seite bestimmen, welche Imperia Kategorien richtig sind
        % - Woher soll Tool wissen, wie die Inhalte zu strukturieren sind, um das Dokument zu füllen?
        % Es wird eine Lösung vorgestellt, die diese Strukturierung der Inhalte automatisiert.
        % D.h., wie die Inhalte von Webseiten klassifiziert werden können.
        % Review sollte weiterhin durch einen Menschen durchgeführt werden

        % Dieses Kapitel zeigt also, dass die Klassifizierung der Inhalte eine Herausforderung ist
        % Das nächste beleutet diese Frage dann etwas genauer,
        % muss aber leider etwas angepasst werden.

        % Das Kapitel "Anforderungen" sollte dann auch eine Ebene hoch gezogen werden,
        % was auch eine _inhaltliche_ Anpassung erfordert.

    \section{Klassifizierung der Inhalte einer Webseite}
    %\url{http://www.fernuni-hagen.de/KSW/portale/babw/service/}

    %\begin{figure}
    %    \centering
    %    \includegraphics[width=\textwidth]{../resources/babw_service_faq.png}
    %    \caption{FAQ Seite des Studienportals B.A. Bildungswissenschaft}
    %    \label{image:BuildingBlocks}
    %\end{figure}
