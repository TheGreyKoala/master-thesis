\section{\wordpress}
    % TODO:
    % - Sites!!!
    \label{section:WordPress}
    Das Open-Source-Projekt {\wordpress} startete 2003
    mit dem Ziel eine Anwendung zur einfachen Pflege eines Weblogs
    (kurz Blog) zu schaffen \cite{wordpress:About}.
    Das Ergebnis ist die gleichnamige Software,
    die noch immer von der Community weiterentwickelt
    und von der Fakultät \gls{ksw} der {\fernUni} für ihren
    Internetauftritt genutzt wird.
    In diesem Kapitel werden grundlegende Konzepte dieses Systems vorgestellt.

    \subsection{Weblog-Software}
        \label{section:weblogSoftware}
        {\wordpress} ist im Kern eine Software zur Pflege eines Blogs,
        wobei es sich um eine spezielle Form einer Webseite handelt
        \cite[Kapitel "`Introduction to Blogging"']{wordpress:codex}:

        \begin{quote}
            "`Blog"' is an abbreviated version of "`weblog"',
            which is a term used to describe websites that maintain
            an ongoing chronicle of information.
            A blog features diary-type commentary and links to articles
            on other websites, usually presented as a list of entries in
            reverse chronological order.
            Blogs range from the personal to the political,
            and can focus on one narrow subject or a whole range of subjects.
        \end{quote}

        Da Blogs eine spezielle Form von Webseiten sind,
        kann man auch eine Anwendung zu ihrer Pflege als
        spezielle Form eines \glspl{cms} betrachten.
        Diese Aussage trifft {\wordpress} auch über sich selbst
        und vergleichbare Software \cite[Kapitel "`Introduction to Blogging"']{wordpress:codex}:

        \begin{quote}
            Many blogging software programs are considered a specific type of CMS.
            They provide the features required to create and maintain a blog,
            and can make publishing on the internet as simple as writing an article,
            giving it a title, and organizing it under (one or more) categories.
        \end{quote}

        Nicht zuletzt, weil auch Privatpersonen eine Zielgruppe
        solcher Anwendungen sind, vereinfachen sie Blogeinträge
        auf die zwei elementaren Elemente Titel und Inhalt.
        Mit dieser schwachen Strukturierung der Inhalte
        unterscheidet sich {\wordpress} deutlich von {\imperia}
        und seinen Dokumenten, die beliebig stark strukturiert
        werden können.
        Aufgrund der Anpassbarkeit durch Themes
        und Plugins sieht sich
        {\wordpress} trotzdem als vollwertiges \gls{cms}
        \cite{wordpress:About}.

    \subsection{Dynamische Generierung}
        \label{section:problemAnalysisWordPressDynamicGeneration}
        Anders als {\imperia} ist {\wordpress} nicht nur das Redaktionssystem,
        sondern gleichzeitig auch das ausliefernde Zielsystem.
        Inhalte generiert {\wordpress} deshalb nicht in statische Dateien,
        sondern erzeugt eine Webseite auf Anfrage dynamisch.

    \subsection{Posts und Pages}
        \label{section:wordpressPostsPages}
        {\wordpress} unterscheidet zwei Arten von Beiträgen \cite[Kapitel "`Pages"']{wordpress:codex},
        auf denen eine Webseite basieren kann: Posts
        und Pages.

        \paragraph*{Posts}
        Klassische Blogeinträge werden in {\wordpress} "`Posts"' genannt.
        Neben der Pflege des Titels und des Inhaltes eines Posts stehen dem
        Anwender noch weitere Optionen zur Verfügung.
        Ein Post kann in Kategorien einsortiert oder mit Schlagwörtern versehen werden.
        Dadurch kann {\wordpress} Übersichtsseiten generieren,
        die z. B. alle Posts einer Kategorie enthalten.
        Über den eigentlichen Inhalt hinausgehende Informationen können
        in sogenannten "`Custom Fields"' gespeichert werden.
        Davon machen z. B. Plugins Gebrauch
        und speichern Metadaten des Posts in ihnen.
        Jeder Post besitzt einen Post Type,
        der eine Aussage über die Art des Beitrages macht.
        {\wordpress} definiert einige Standardtypen \cite[Kapitel "`Post Types"']{wordpress:codex},
        lässt aber auch die Angabe eigener Typen zu.
        Die Standardtypen sind:

        \begin{itemize}
            \item Post,
            \item Page,
            \item Attachment,
            \item Revision,
            \item Navigation Menu,
            \item Custom CSS und
            \item Changesets.
        \end{itemize}

        Der Typ eines Posts ändert nichts an seinen redaktionellen Feldern.
        Das heißt, unabhängig vom Post Type besitzt ein Beitrag nur
        einen Titel und Inhalt
        \cite[Kapitel "`Posts"' \& "`Post Types"']{wordpress:codex}.
        Stattdessen machen Typen wie Revision und Custom CSS deutlich,
        dass {\wordpress} Posts auch zur Realisierung technischer Anforderung verwendet.

        \paragraph*{Pages}
        Neben Posts kennt {\wordpress} auch das Konzept einer Page \cite[Kapitel "`Pages"']{wordpress:codex},
        die sich in ihrem Zweck klar von einem Post unterscheidet:

        \begin{quote}
            In contrast, pages are generally for non-chronological,
            hierarchical content: pages like "`About"' or "`Contact"'
            would be common examples.
            [...]
            Pages live outside of the normal blog chronology,
            and are often used to present timeless information about
            yourself or your site -- information that is always relevant.
            You can use Pages to organize and manage the structure of your website content.
        \end{quote}

        Aus diesem Grund besitzen Pages zwar ebenfalls einen Titel und Inhalt,
        können aber lediglich mit Schlagwörtern versehen werden.
        Die Einordnung in eine Kategorie ist nicht möglich
        \cite[Kapitel "`Pages"']{wordpress:codex}.
        Wie aus der oben erfolgten Auflistung der Post Types hervorgeht,
        sind Pages technisch gesehen lediglich Posts mit dem Post Type "`Page"'.

        {\wordpress} speichert alle Inhalte in einer relationalen Datenbank \cite[Kapitel "`Database Description"']{wordpress:codex}.
        Posts und Pages teilen sich in dieser Datenbank eine Tabelle,
        was ebenfalls verdeutlicht, dass Pages lediglich Posts eines speziellen Typs sind.
        Der Inhalt eines Beitrages wird in der Datenbank als \gls{html}-Fragment abgelegt,
        welches während der dynamischen Generierung in die Webseite übernommen wird.
        Zusätzlich enthält ein Beitrag aber auch spezielles Anweisungen\footnote{vgl. Kapitel \ref{section:wordpressPlugins}},
        welches von {\wordpress} während der Generierung interpretiert wird.

    \subsection{Vorlagen und Themes}
        \label{section:wordpressTemplatesThemes}
        Wie {\imperia} strebt auch {\wordpress} eine Trennung von
        Inhalt und Layout an.
        Inhalte werden dazu in
        Posts und Pages
        unabhängig vom Layout der Webseite gespeichert.
        Das Layout bestimmen Vorlagen und Themes.

        \paragraph*{Vorlagen}
        {\wordpress} nutzt Vorlagen \cite[Kapitel "`Templates"']{wordpress:codex}, um Inhalte in eine Seite einzubinden
        und ihr Aussehen festzulegen.
        Dazu definieren sie das Gerüst der Webseite und enthalten Kommandos,
        um Inhalte aus der Datenbank auszulesen.
        Eine Webseite wird auf Basis einer Vorlage dynamisch von {\wordpress} generiert.
        Vorlagen sind bei genauerem Hinsehen nichts anderes als PHP-Dateien,
        die \gls{html}, allgemeinen PHP-Code und sogenannte
        "`Template-Tags"' enthalten.
        Dabei handelt es sich um Aufrufe von
        {\wordpress}-eigenen PHP-Funktionen,
        um Inhalte aus der Datenbank abzufragen.
        Während der Generierung -- technisch lediglich die Ausführung
        des PHP-Codes -- können Vorlagen andere Vorlagen inkludieren,
        wodurch wiederkehrende Elemente in eigene Vorlagen ausgelagert
        werden können \cite[Kapitel "`Template Files"']{wordpress:codex}.
        Anhand der \gls{url} der Anfrage und der darin enthaltenen
        Kennung eines Posts oder einer Page,
        entscheidet {\wordpress}, welche Vorlage es zur Generierung der Seite nutzt
        \cite[Kapitel "`Template Hierarchy"']{wordpress:codex}.

        \paragraph*{Themes}
        Eine Sammlung aller {\resources}, die notwendig sind, um
        eine Webseite und ihr Layout umzusetzen,
        wird im Kontext von {\wordpress} als "`Theme"' bezeichnet.
        Ein Theme enthält demnach Vorlagen, Bilder sowie
        JavaScript-, \gls{css}-, und PHP-Dateien.
        Themes können {\resources} anderer Themes wiederverwenden oder überschreiben,
        wodurch Anpassungen an einem vorhandenen Theme einfach umzusetzen sind
        \cite[Kapitel "`Using Themes"']{wordpress:codex}.

    \subsection{Plugins}
        \label{section:wordpressPlugins}
        {\wordpress} besitzt ein Plugin-System \cite[Kapitel "`Plugins"']{wordpress:codex},
        über das es beliebig funktional erweitert werden kann,
        ohne an {\wordpress}' eigenen Quellen Änderungen vorzunehmen.
        Es existiert eine große Anzahl an freien Plugins,
        die oft benötigte Funktionen implementieren.
        Meist sind dies Funktionen, die über eine reine Weblog-Software hinausgehen.
        Ein Beispiel ist das Plugin
        "`Form Maker"'\footnote{vgl. \url{https://wordpress.org/plugins/form-maker/}},
        welches die Definition von Webformularen erlaubt,
        die dann in beliebigen Beiträgen eingebettet werden können.
        Zur Nutzung eines Plugins wird meist eine spezielle Anweisung -- ein Shortcode \cite[Kapitel "`Shortcode API"']{wordpress:codex} --
        in den Inhalt eines Posts oder einer Page geschrieben,
        die zum Zeitpunkt der Generierung ausgewertet wird.
        Beim Form Maker Plugin wäre dies zum Beispiel \texttt{[Form id="1"]},
        wodurch das Formular mit der Kennung "`1"' in die Webseite integriert wird.
