\section{Klassifizierung der Inhalte einer Webseite}
    \label{section:WebpageClassification}
    Kapitel \ref{chapter:FernUniRelaunch}
    ist bereits auf das Vorhaben der {\fernUni}
    eingegangen, die {\wordpress}-basierten Seiten der Fakultät
    \gls{ksw} zu {\imperia} zu migrieren.
    Ein Vergleich der in den Kapiteln
    \ref{section:Imperia} und \ref{section:WordPress} vorgestellten
    Datenstrukturen dieser Systeme verdeutlicht die Herausforderung bei diesem
    Vorhaben.
    {\wordpress} speichert die Inhalte einer Seite unstrukturiert in einem einzigen Feld,
    wohingegen {\imperia} eine höhere Strukturierung der Inhalte vorsieht.
    Eine Überführung der Inhalte in die feineren Strukturen von
    {\imperia}, wie es die Universität beabsichtigt
    \footnote{vgl. Kapitel \ref{section:fernUniChallenges}},
    ist also nicht ohne Weiteres möglich.

    Aus diesem Grund soll die Möglichkeit einer automatischen
    Klassifizierung der Inhalte untersucht werden,
    deren Ergebnis die beschriebene Migration vereinfachen würde.
    Dieses Kapitel verfeinert diese Aufgabenstellung und klärt folgende Fragen:

    \begin{enumerate}
        \item Welche Elemente einer Webseite sind für eine Klassifizierung interessant?
        \item Nach welchen Kriterien findet eine Klassifizierung statt?
        \item Welche allgemeinen Anforderungen ergeben sich daraus?
    \end{enumerate}

    Die in den vorangegangenen Kapiteln gewonnenen Erkenntnisse ermöglichen
    die Beantwortung dieser Fragen.
    Zur Veranschaulichung wird unter anderem eine Seite der Fakultät \gls{ksw}
    \footnote{\url{http://www.fernuni-hagen.de/KSW/portale/babw/service/}, Stand 01.02.2018}
    herangezogen.
    Abbildung \ref{image:BaBwFAQ} zeigt einen Ausschnitt dieser
    Seite, der den Inhalt des Beitrages in {\wordpress} widerspiegelt.

    \begin{figure}
        \centering
        \includegraphics[width=\textwidth]{../resources/babw_service_faq.png}
        \caption{F.A.Q. Seite des Studienportals "`B.A. Bildungswissenschaft"'}
        \label{image:BaBwFAQ}
    \end{figure}

    Diese Seite beinhaltet Anworten zu häufig gestellten Fragen
    zum \gls{babw}.

    \subsection{Relevante Entitäten}
        Die Anforderung die Inhalte einer Seite zu klassifizieren
        ist allgemein und lässt die Frage offen,
        welche Bestandteile einer Seite hierfür von Bedeutung sind.
        Dieser Frage soll sich über die konkrete Anforderung
        der {\fernUni} genähert werden.

        Würde man manuell eine aus {\wordpress} stammende Seite zu
        {\imperia} migrieren, stellten sich mindestens drei Fragen:

        \begin{enumerate}
            \item   Auf welcher Vorlage soll die Seite basieren?
                    Das heißt zu welcher Kategorie gehört sie?
            \item   Mit welchen Inhalten sind die unterschiedlichen Felder
                    des Formulares zu füllen?
            \item   Welcher Bilder, Videos und Links auf andere Seiten existieren,
                    und wie müssen sie in die neue Seite übernommen werden?
        \end{enumerate}

        Am leichtesten ist die zweite Frage zu beantworten.
        Die Felder werden selbstverständlich mit dem Inhalt des Beitrages
        aus {\wordpress} gefüllt.
        Diese textuellen Inhalte müssen folglich einer Klassifizierung unterzogen werden.
        
        Im einfachen Fall basiert jede migrierte Seite in {\imperia} auf derselbe Vorlage.
        Es ist allerdings nicht unwahrscheinlich,
        dass verschiedene Layouts und damit verschiedene Vorlagen
        notwendig werden.
        Die Wahl der Vorlage wird dann auf der originalen Seite basieren,
        weshalb auch die Klassifizierung einer Seite als Ganzes notwendig ist.
        Jeder Klasse wird anschließend eine Kategorie in {\imperia}
        zugewiesen, die wiederum die zu nutzende Vorlage bestimmt.
        Eine Klassifizierung der gesamten Seite kann außerdem die
        Klassifizierung der Inhalte in einen Kontext setzen.
        Die Klasse "`Frage"' auf der F.A.Q-Seite des \gls{babw}
        ist zum Beispiel aussagekräftiger, wenn bekannt ist,
        dass die Seite zur Klasse "`F.A.Q."' gehört.

        Die dritte Frage besteht aus zwei Teilen.
        Der erste Teil fragt allgemein gesprochen nach einer Auflistung referenzierter {\resources},
        deren Bereitstellung keine Herausforderung darstellt,
        da sie direkt aus dem \gls{html}-Dokument hervorgeht.
        Der zweite Teil zielt hingegen darauf ab,
        wie sich die jeweilige {\resource} und die Referenz selbst einordnen lassen.
        Bilder, Videos und Links werden in der neuen Seite selbstverständlich
        unterschiedlich eingebunden.
        Das Banner einer Seite -- technisch ebenfalls nur ein Bild --
        besitzt womöglich ein dediziertes Eingabefeld,
        weshalb auch innerhalb der Gruppe "`Bilder"' weiter unterteilt werden muss.
        Das gilt zum Beispiel auch für Links.
        Referenzen auf externe Seiten erhalten häufig ein anderes Layout als Links
        auf Seiten innerhalb der eigenen Site.
        Referenzen müssen demnach ebenfalls klassifiziert werden.

        Zusammengefasst lassen sich also drei Entitäten ermitteln,
        die für eine Klassifizierung relevant sind:
        Die Seite als Ganzes, textueller Inhalt und Referenzen.

    \subsection{Kriterium der Klassifizierung}
        Die Wahl der Vorlage könnte auf dem Typ des Beitrages basieren,
        also ob er ein Post oder eine Page ist.
        In der Praxis wird diese Unterteilung aber nicht ausreichen,
        da verschiedene Pages im Zielsystem verschiedene Layouts erhalten werden.

    % Inhalte müssen strukturiert werden
    % Genauere Betrachtung
    % Basierend auf gewonnenen Erkenntnissen und Beispiel

    % Was ist für die Migration notwendig?
    % Auswahl eines Templates
    % Füllen der Felder
    % Referenzieren von Ressourcen
    % --> Bsp.

    % Wie kann der Inhalt klassifiziert werden? / Wie sollen die Klassen aussehen?
    % Klasse spiegelt Node im HTML wieder
    % Klasse spiegelt Layout-Element wieder
    % Klasse spiegelt Inhalt wieder
    % --> Bsp.

    % Spezielle Anforderung bei WP: Plugin-Unabhängigkeit
    % Man braucht eine Schnittstelle zur Abfrage

    %

    