\section{Die Klassifizierung der Inhalte einer Webseite}
    \label{section:WebpageClassification}
    Eine Automatisierung der Migration scheint machbar,
    wenn die Inhalte aus {\wordpress} eine ähnlich starke Strukturierung aufweisen,
    wie die Zieldokumente in {\imperia}.
    Diese Arbeit hat zum Ziel diese Strukturierung
    durch eine automatische Klassifizierung der Inhalte einer Webseite zu erzeugen.
    Dieses Kapitel verfeinert dieses Ziel und klärt dazu folgende Fragen:

    \begin{enumerate}
        \item Welche Elemente einer Webseite sind für eine Klassifizierung interessant?
        \item Nach welchen Kriterien findet eine Klassifizierung statt?
        \item Was ist eine Klasse in diesem Kontext?
        \item Welche Anforderungen ergeben sich daraus?
    \end{enumerate}

    Die in den vorangegangenen Kapiteln gewonnenen Erkenntnisse ermöglichen
    die Beantwortung dieser Fragen.
    Zur Veranschaulichung wird unter anderem wieder das Beispiel aus Kapitel
    \ref{section:fernUniChallenges} herangezogen.

    \subsection{Relevante Entitäten}
        \label{section:classificationEntities}
        Die Anforderung die Inhalte einer Seite zu klassifizieren
        ist allgemein und lässt die Frage offen,
        welche Bestandteile einer Seite hierfür von Bedeutung sind.
        Aus der beispielhaften manuellen Migration
        ergeben sich drei allgemeine Fragen,
        deren Beantwortung Aufschluss geben über die relevanten Entitäten
        bei einer Klassifizierung.

        \begin{enumerate}
            \item   Auf welcher Vorlage soll die neue Seite basieren?
            \item   Mit welchen Inhalten sind die unterschiedlichen Felder
                    des Formulars zu füllen?
            \item   Welcher Bilder, Videos und Links auf andere Seiten existieren,
                    und wie müssen sie in die neue Seite übernommen werden?
        \end{enumerate}

        \paragraph{Seiten}
        Im einfachen Fall basiert jede migrierte Seite in {\imperia} auf derselbe Vorlage.
        Es ist allerdings nicht unwahrscheinlich,
        dass verschiedene Layouts und damit verschiedene Vorlagen
        notwendig werden.
        Die Wahl der Vorlage wird dann auf der originalen Seite basieren,
        weshalb auch die Klassifizierung einer Webseite als Ganzes notwendig ist.
        Jeder Klasse wird anschließend einer Kategorie in {\imperia}
        zugewiesen, die wiederum die zu nutzende Vorlage bestimmt.

        \paragraph{Textuelle Inhalte}
        Die zweite Frage ist leicht zu beantworten.
        Die Formularfelder werden selbstverständlich mit dem Inhalt des Beitrages
        aus {\wordpress} gefüllt.
        Diese textuellen Inhalte müssen folglich einer Klassifizierung unterzogen werden.

        \paragraph{Referenzen}
        Die dritte Frage besteht aus zwei Teilen.
        Der erste Teil fragt allgemein gesprochen nach einer Auflistung
        referenzierter {\resources}\footnote{vgl. Kapitel \ref{section:problemAnalysisWebpagesInTheWWWResources}}.
        Deren Bereitstellung stellt keine Herausforderung dar,
        da sie direkt aus dem \gls{html}-Dokument hervorgehen.
        Der zweite Teil zielt hingegen darauf ab,
        wie sich die jeweilige {\resource} und die Referenz selbst einordnen lassen.
        Das Bild eines Mitarbeiters und seine E-Mail-Adresse sind im Beispiel beides Referenzen.
        In {\imperia} sind für sie aber unterschiedliche Formularfelder vorgesehen.
        Referenzen müssen demnach ebenfalls klassifiziert werden.

        Zusammengefasst lassen sich also drei Entitäten ermitteln,
        die für die Klassifizierung einer Seite relevant sind:
        Die Seite als Ganzes, textueller Inhalt und Referenzen.

    \subsection{Kriterium der Klassifizierung}
        \label{section:ClassificationCriteria}
        Nachdem nun geklärt ist, welche Entitäten einer Webseite für
        eine Klassifizierung relevant sind\footnote{vgl. Kapitel \ref{section:classificationEntities}},
        widmet sich dieses Kapitel der Frage, nach welchen
        Kriterien diese Einordnung erfolgen muss
        und welche Bedeutung die resultierenden Klassen besitzen.

        \paragraph{Klassifizierung von Seiten}
        Die Klassifikation der Seite hat direkten Einfluss auf die Wahl der Vorlage im neuen \gls{cms}.
        Die Klasse der Seite könnte auf dem Typ des Beitrages basieren,
        also ob er ein Post oder eine Page ist.
        In der Praxis wird diese Unterteilung aber nicht ausreichen,
        da verschiedene {\wordpress} Pages im Zielsystem verschiedene Layouts erhalten werden.
        Eine fachliche Unterteilung der Seiten anstatt einer technischen ist deshalb sinnvoller.
        Für die Seite eines Mitarbeiters hieße das zum Beispiel,
        dass sie als "`Mitarbeiterseite"' klassifiziert wird und nicht als "`Page"'.
        Eine fachliche Unterscheidung nach inhaltlicher Bedeutung macht die
        gewonnene Klassifikation außerdem unabhängiger vom konkreten
        Migrationsvorhaben der FernUniversität und somit für andere Anwendungsfälle nutzbar.

        \paragraph{Klassifizierung von Inhalten}
        Die Einteilung von Referenzen und textuellen Inhalten kann ebenfalls
        mit einer technischen oder fachlichen Ausrichtung erfolgen.
        Technisch lassen sich Inhalte zum Beispiel anhand ihres \gls{html}-Elementes einteilen.
        So entstehen Klassen wie "`p"' und "`h1"'.
        Name und Telefonnummer eines Mitarbeiters finden sich bei diesem Ansatz in der Klasse "`p"' wieder.
        In {\imperia} existieren für beide aber getrennte Felder.
        Basierend auf dieser Klassifikation kann ein Automatismus
        nicht entscheiden, welches Feld mit welchem Inhalt zu füllen ist.
        Abhilfe bringt wiederum nur eine fachliche Klassifizierung,
        bei der die inhaltliche Bedeutung im Vordergrund steht.
        Das heißt anstelle von allgemeinen Klassen wie "`p"',
        sind spezifischere fachliche Klassen wie
        "`Name"' und "`Telefonnummer"' notwendig.
        Nur so lässt sich entscheiden, wie die Inhalte in das neue \gls{cms}
        zu überführen sind.

        \paragraph{Klassifizierung von Referenzen}
        Diese Schlussfolgerung gilt auch für Referenzen,
        weil immer die Möglichkeit besteht,
        dass aufgrund einer technischen Einteilung nicht klar ist,
        wie die Referenzen der Seite zu übertragen sind.
        Der Link auf das Lehrgebiet und die E-Mail-Adresse eines Mitarbeiters
        können bspw. nicht beide als "`a"' klassifiziert werden,
        weil eine Unterscheidung möglich sein muss.
        Besser geeignet wären deshalb die Klassen "`Lehrgebietsseite"'
        und "`E-Mail-Adresse"'.

    \subsection{Anforderungen}
        \label{section:requirements}
        Die vorliegenden Informationen erlauben
        die Ableitung ausreichend konkreter Anforderungen an ein System
        zur automatischen Klassifizierung der Inhalte von Webseiten.

        \paragraph*{Automatische fachliche Klassifizierung von Seiten, Inhalten und Referenzen}
        Es ist ein System zu entwickeln, welches automatisiert Webseiten,
        deren Inhalte und Referenzen auf andere {\resources} im \gls{www}
        nach fachlichen und inhaltlichen Kriterien klassifiziert.
        Eine Klassifikation ermöglicht im vorliegenden konkreten Fall
        der {\fernUni} eine \gls{cms}-Migration von {\wordpress} zu {\imperia}.

        \paragraph*{Abbildung inhaltlicher Strukturen}
        Eine fachliche Klassifizierung bedeutet nicht,
        dass das Ergebnis aus einelementigen Klassen besteht.
        Eine Übersichtsseite aller Mitarbeiter enthält viele Einträge,
        die jeweils als "`Mitarbeiter"' klassifiziert werden können.
        Ein Mitarbeiter besitzt offensichtlich mehrere Eigenschaften,
        wie einen Namen und eine E-Mail-Adresse.
        Aus der Klassifikation muss hervorgehen,
        welche Informationen zusammengehören,
        d.h. welche Informationen einen speziellen Mitarbeiter ausmachen.
        Nutzer der Klassifikation können diese Zuordnung im Nachhinein nicht sicher vornehmen.
        Daraus ergibt sich folgende allgemeine Anforderung:

        Eine Klassifikation muss die fachliche Struktur
        und Hierarchie der Inhalte widerspiegeln,
        um diese Zuordnung vor allem bei sich wiederholenden Strukturen
        nicht zu verlieren.
        Das heißt, besteht ein Inhalt $I$ fachlich aus den Inhalten $I_1 \ldots I_n$
        und den Referenzen $R_1 \ldots R_m$,
        muss die Klassifikation $I'$ von $I$ auch aus den Klassifikationen
        $I'_1 \ldots I'_n$ und $R'_1 \ldots R'_m$ bestehen.

        \paragraph*{Allgemeine Klassendefinition}
        Es muss möglich sein die fachlichen Klassen allgemein zu
        spezifizieren, sodass eine beliebige Anzahl an gegebenen
        Webseiten (z. B. die Seiten einer Website)
        auf Basis dieser Spezifikation klassifiziert werden kann.

        \paragraph*{Schnittstelle zum Starten einer Klassifizierung}      
        Das System muss programmatisch angewiesen werden können,
        eine oder mehrere gegebene Webseiten zu klassifizieren,
        sodass ein Drittsystem die relevanten Webseiten auffinden
        und weitergeben kann.
        Der Aufwand einer manuellen Instrumentierung des Systems ist
        vor allem bei vielen Webseiten nicht zu rechtfertigen.

        \paragraph*{Schnittstelle zur Abfrage der Klassifikation}
        Die Klassifikation muss über eine definierte allgemeine
        Schnittstelle abrufbar sein, sodass Drittsysteme die Ergebnisse
        nutzen können.
        Im Fall der {\fernUni} könnte dies ein Werkzeug sein,
        welches die Migration von {\wordpress} zu {\imperia}
        durchführt und dafür auf die Klassifikation zurückgreift.

        \paragraph*{Möglichkeit der Nachbesserung}
        Das System muss die Prüfung sowie die manuelle Nachbesserung
        einer Klassifikation erlauben,
        um Fehler zu erkennen und zu beheben.
        Das entsprechende Werkzeug soll eine graphische Benutzeroberfläche
        besitzen, um seine Nutzung möglichst einfach zu gestalten.
