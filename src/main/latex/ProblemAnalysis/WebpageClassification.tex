\section{Klassifizierung der Inhalte einer Webseite}
    \label{section:WebpageClassification}
    Kapitel \ref{chapter:FernUniRelaunch}
    ist bereits auf das Vorhaben der {\fernUni}
    eingegangen, die {\wordpress}-basierten Seiten der Fakultät
    \gls{ksw} zu {\imperia} zu migrieren.
    Ein Vergleich der in den Kapiteln
    \ref{section:Imperia} und \ref{section:WordPress} vorgestellten
    Datenstrukturen dieser Systeme verdeutlicht die Herausforderung bei diesem
    Vorhaben.
    {\wordpress} speichert die Inhalte einer Seite unstrukturiert in einem einzigen Feld,
    wohingegen {\imperia} eine höhere Strukturierung der Inhalte vorsieht.
    Eine Überführung der Inhalte in die feineren Strukturen von
    {\imperia}, wie es die Universität beabsichtigt
    \footnote{vgl. Kapitel \ref{section:fernUniChallenges}},
    ist also nicht ohne Weiteres möglich.

    Aus diesem Grund soll die Möglichkeit einer automatischen
    Klassifizierung der Inhalte untersucht werden,
    deren Ergebnis die beschriebene Migration vereinfachen würde.
    Dieses Kapitel verfeinert diese Aufgabenstellung und klärt folgende Fragen:

    \begin{enumerate}
        \item Welche Elemente einer Webseite sind für eine Klassifizierung interessant?
        \item Nach welchen Kriterien findet eine Klassifizierung statt?
        \item Welche allgemeinen Anforderungen ergeben sich daraus?
    \end{enumerate}

    Die in den vorangegangenen Kapiteln gewonnenen Erkenntnisse ermöglichen
    die Beantwortung dieser Fragen.
    Zur Veranschaulichung wird unter anderem eine Seite der Fakultät \gls{ksw}
    \footnote{\url{http://www.fernuni-hagen.de/KSW/portale/babw/service/}, Stand 01.02.2018}
    herangezogen.
    Abbildung \ref{image:BaBwFAQ} zeigt einen Ausschnitt dieser
    Seite, der den Inhalt des Beitrages in {\wordpress} widerspiegelt.

    \begin{figure}
        \centering
        \includegraphics[width=\textwidth]{../resources/babw_service_faq.png}
        \caption{F.A.Q. Seite des Studienportals "`B.A. Bildungswissenschaft"'}
        \label{image:BaBwFAQ}
    \end{figure}

    Diese Seite beinhaltet Anworten zu häufig gestellten Fragen
    zum \gls{babw}.

    \subsection{Relevante Entitäten}
        \label{section:classificationEntities}
        Die Anforderung die Inhalte einer Seite zu klassifizieren
        ist allgemein und lässt die Frage offen,
        welche Bestandteile einer Seite hierfür von Bedeutung sind.
        Dieser Frage soll sich über die konkrete Anforderung
        der {\fernUni} genähert werden.

        Würde man manuell eine aus {\wordpress} stammende Seite zu
        {\imperia} migrieren, stellten sich mindestens drei Fragen:

        \begin{enumerate}
            \item   Auf welcher Vorlage soll die Seite basieren?
                    Das heißt zu welcher Kategorie gehört sie?
            \item   Mit welchen Inhalten sind die unterschiedlichen Felder
                    des Formulares zu füllen?
            \item   Welcher Bilder, Videos und Links auf andere Seiten existieren,
                    und wie müssen sie in die neue Seite übernommen werden?
        \end{enumerate}

        Am leichtesten ist die zweite Frage zu beantworten.
        Die Felder werden selbstverständlich mit dem Inhalt des Beitrages
        aus {\wordpress} gefüllt.
        Diese textuellen Inhalte müssen folglich einer Klassifizierung unterzogen werden.
        
        Im einfachen Fall basiert jede migrierte Seite in {\imperia} auf derselbe Vorlage.
        Es ist allerdings nicht unwahrscheinlich,
        dass verschiedene Layouts und damit verschiedene Vorlagen
        notwendig werden.
        Die Wahl der Vorlage wird dann auf der originalen Seite basieren,
        weshalb auch die Klassifizierung einer Seite als Ganzes notwendig ist.
        Jeder Klasse wird anschließend eine Kategorie in {\imperia}
        zugewiesen, die wiederum die zu nutzende Vorlage bestimmt.
        Eine Klassifizierung der gesamten Seite kann außerdem die
        Klassifikation der Inhalte in einen Kontext setzen.
        Die Klasse "`Frage"' auf der F.A.Q-Seite des \gls{babw}
        ist zum Beispiel aussagekräftiger, wenn bekannt ist,
        dass die Seite zur Klasse "`F.A.Q."' gehört.

        Die dritte Frage besteht aus zwei Teilen.
        Der erste Teil fragt allgemein gesprochen nach einer Auflistung referenzierter {\resources},
        deren Bereitstellung keine Herausforderung darstellt,
        da sie direkt aus dem \gls{html}-Dokument hervorgeht.
        Der zweite Teil zielt hingegen darauf ab,
        wie sich die jeweilige {\resource} und die Referenz selbst einordnen lassen.
        Bilder, Videos und Links werden in der neuen Seite selbstverständlich
        unterschiedlich eingebunden.
        Das Banner einer Seite -- technisch ebenfalls nur ein Bild --
        besitzt womöglich ein dediziertes Eingabefeld,
        weshalb auch innerhalb der Gruppe "`Bilder"' weiter unterteilt werden muss.
        Das gilt zum Beispiel auch für Links.
        Referenzen auf externe Seiten erhalten häufig ein anderes Layout als Links
        auf Seiten innerhalb der eigenen Site.
        Referenzen müssen demnach ebenfalls klassifiziert werden.

        Zusammengefasst lassen sich also drei Entitäten ermitteln,
        die für eine Klassifizierung relevant sind:
        Die Seite als Ganzes, textueller Inhalt und Referenzen.

    \subsection{Kriterium der Klassifizierung}
        \label{section:ClassificationCriteria}
        Nachdem nun geklärt ist, welche Entitäten einer Webseite für
        eine Klassifizierung relevant sind\footnote{vgl. Kapitel \ref{section:classificationEntities}},
        widmet sich dieses Kapitel der Frage, nach welchen
        Kriterien diese Einordnung erfolgen muss
        und welche Bedeutung die resultierenden Klassen besitzen.

        Die Klassifikation der Seite hat direkten Einfluss auf die Wahl der Vorlage im neuen \gls{cms}.
        Die Klasse der Seite könnte auf dem Typ des Beitrages basieren,
        also ob er ein Post oder eine Page ist.
        In der Praxis wird diese Unterteilung aber nicht ausreichen,
        da verschiedene Pages im Zielsystem verschiedene Layouts erhalten werden.
        Eine fachliche Unterteilung der Seiten anstatt einer technischen ist deshalb sinnvoller.
        Für die F.A.Q.-Seite des \gls{babw} hieße das zum Beispiel,
        dass sie als "`FAQ-Seite"' klassifiziert wird und nicht als "`Page"'.
        Eine fachliche Unterscheidung nach inhaltlicher Bedeutung macht die
        gewonnene Klassifikation außerdem unabhängiger vom konkreten
        Migrationsvorhaben und somit für andere Anwendungsfälle nutzbar.

        Die Einteilung von Referenzen und textueller Inhalte kann ebenfalls
        mit einer technischen oder fachlichen Ausrichtung erfolgen.
        Technisch ließen sich Inhalte zum Beispiel anhand ihres \gls{html}-Elementes einteilen.
        Es entstünden also Klassen wie "`p"', "`h1"' usw.
        Auf der F.A.Q.-Seite des \gls{babw} fände sich der einleitende Text vor dem
        ersten Themenbereich in der Klasse "`p"' wieder.
        Das gleiche gilt für den abschließenden Textabsatz nach dem letzten Themenbereich.
        Falls in {\imperia} zu diesen Absätzen
        korrespondierende Felder existieren, kann ein Automatismus basierend auf dieser
        Klassifikation nicht entscheiden, welches Feld mit welchem Inhalt zu füllen ist.
        
        Eine Einteilung nach Layoutelementen ist ebenso wenig hilfreich.
        Im beschriebenen Fall hieße die Klasse dann nämlich nicht "`p"',
        sondern "`Text"', was die Füllung der Felder nicht vereinfacht.
        Ein weiteres Beispiel sind die aufklappbaren Elemente auf der Seite,
        die sich nach diesem Konzept
        der Klasse "`Accordion"'\footnote{vgl. \url{https://getbootstrap.com/docs/3.3/javascript/\#collapse-example-accordion}}
        zuweisen ließen.
        Die Information, dass Inhalte ein gewisses Layout verwenden,
        ist unbrauchbar, wenn nicht gewährleistet ist,
        dass alle diese Inhalte wieder dieses Layout verwenden werden.

        Abhilfe bringt wiederum nur eine fachliche Klassifizierung,
        bei der die inhaltliche Bedeutung im Vordergrund steht.
        Das heißt anstelle von allgemeinen Klassen wie "`p"', "`Text"'
        oder "`Accordion"', sind spezifischere fachliche Klassen wie
        "`Einleitung"', "`Schluss"', "`FAQ-Themenbereich"', "`Frage"' und "`Antwort"'
        notwendig.
        Nur so lässt sich entscheiden, wie die Inhalte in das neue \gls{cms}
        zu überführen sind.

        Diese Schlussfolgerung gilt auch für Referenzen.
        Das Banner einer Seite kann nicht als "`img"'
        oder als "`Carousel"'\footnote{vgl. \url{https://getbootstrap.com/docs/3.3/javascript/\#carousel}}
        klassifiziert werden, weil immer die Möglichkeit besteht,
        dass aufgrund dieser technischen Einteilung nicht klar wird,
        wie die Inhalte der Seite zu übertragen sind.
        Benötigt wird stattdessen die fachliche Information,
        dass die referenzierte {\resource} das Banner der Seite darstellt,
        weshalb die Referenz auch als "`Banner"' klassifiziert werden muss.

    \subsection{Anforderungen}
        \label{section:requirements}
        Die vorliegenden Informationen erlauben
        die Ableitung ausreichend konkreter Anforderungen,
        um die Problemstellung zu formulieren, der sich diese Arbeit widmet.

        \paragraph*{Automatisierte fachliche Klassifizierung von Seiten, Inhalten und Referenzen}
        Es ist ein System zu entwickeln, welches automatisiert Webseiten,
        deren Inhalte und Referenzen auf andere {\resources} im \gls{www}
        nach fachlichen und inhaltlichen Kriterien klassifiziert.
        Eine Klassifikation ermöglicht im vorliegenden konkreten Fall
        der {\fernUni} eine \gls{cms}-Migration von {\wordpress} zu {\imperia}.
        Allgemein erlaubt sie die Verarbeitung und Analyse der Webseiten.

        \paragraph*{Abbildung inhaltlicher Strukturen}
        Eine fachliche Klassifizierung bedeutet nicht,
        dass das Ergebnis aus einelementigen Klassen besteht.
        Die F.A.Q.-Seite des \gls{babw} enthält mehrere aufklappbare
        Themenbereiche (Inhaltliche Ausrichtung, Berufsfeldorientierung, etc.),
        die jeweils als "`FAQ-Themenbereich"' klassifiziert werden können.
        Des Weiteren enthält jeder Themenbereich mehrere Paare aus Frage und Antwort.
        Aus der Klassifikation muss entsprechend hervorgehen,
        welche Frage-Antwort-Paare zu welchem Themenbereich gehören,
        da diese Zuordnung Nutzer der Klassifikation nicht sicher vornehmen können.
        Daraus ergibt sich folgende allgemeine Anforderung:

        Eine Klassifikation muss die fachliche Struktur
        und Hierarchie der Inhalte widerspiegeln,
        um diese Zuordnung vor allem bei sich wiederholenden Strukturen
        nicht zu verlieren.
        Das heißt, besteht ein Inhalt $I$ fachlich aus den Inhalten $I_1 \ldots I_n$
        und den Referenzen $R_1 \ldots R_m$,
        muss die Klassifikation $I'$ von $I$ auch aus den Klassifikationen
        $I'_1 \ldots I'_n$ und $R'_1 \ldots R'_m$ bestehen.

        \paragraph*{Allgemeine Klassendefinition}
        Es muss möglich sein die fachlichen Klassen allgemein zu
        spezifizieren, sodass eine beliebige Anzahl an gegebenen
        Webseiten auf Basis dieser Spezifikation klassifiziert wird.
        Die Syntax einer solchen Spezifikation soll auch von Nutzern ohne
        Programmierkenntnisse verstanden werden können.

        \paragraph*{Schnittstelle zum Starten einer Klassifizierung}      
        Das System muss programmatisch angewiesen werden können,
        eine oder mehrere gegebene Webseiten zu klassifizieren,
        sodass ein Drittsystem die relevanten Webseiten auffinden
        und weitergeben kann.
        Der Aufwand einer manuellen Instrumentierung des Systems ist
        vor allem bei vielen Webseiten nicht zu rechtfertigen.

        \paragraph*{Schnitstelle zur Abfrage der Klassifikation}
        Die Klassifikation muss über eine definierte allgemeine
        Schnittstelle abrufbar sein, sodass Drittsysteme die Ergebnisse
        nutzen können.
        Im Fall der {\fernUni} könnte dies ein Werkzeug sein,
        welches die Migration von {\wordpress} zu {\imperia}
        durchführt und dafür auf die Klassifikation zurückgreift.

        \paragraph*{Möglichkeit der Nachbesserung}
        % TODO: Was ist die allgemeine Anforderung für:
        % Seiten sollen als "veraltet" markiert werden können, um sie von der Migration auszuschließen.
        Das System muss die Prüfung sowie die manuelle Nachbesserung
        des Klassifizierungsergebnisses erlauben,
        um Fehler zu erkennen und zu beheben.
        Das entsprechende Werkzeug soll eine graphische Benutzeroberfläche
        besitzen, um seine Nutzung möglichst einfach zu gestalten.

        \paragraph*{Allgemeingültigkeit}
        Trotz dem konkreten Anwendungsfall an der {\fernUni}
        soll ein allgemein nutzbares System entstehen,
        welches nicht auf die speziellen Anforderungen dieser Universität
        zugeschnitten ist.
        Das bedeutet, dass eine Klassifizierung von Webseiten außerhalb
        der {\fernUni} möglich sein muss.
        Außerdem, dass das System nicht annehmen darf,
        dass eine zu klassifizierende Webseite auf {\wordpress} basiert.
