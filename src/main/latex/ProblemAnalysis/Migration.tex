\section{Eine Migration von {\wordpress} zu {\imperia}}
    An einem Beispiel soll erklärt werden,
    wie eine manuelle Migration der Inhalte einer Webseite von {\wordpress} zu {\imperia}
    vonstatten gehen könnte und wo die größte Herausforderung dabei liegt.
    Darauf basierend erklärt das nächste Kapitel,
    welchem Aspekte dieses Prozesses sich diese Arbeit widmet.

    Als Beispiel dient die Seite eines Mitarbeiters,
    die schon aus Kapitel \ref{section:fernUniChallenges} bekannt ist.
    Die Schritte, die ein {\editor} zur Migration dieser Seite durchführen muss,
    sind schnell beschrieben:

    \begin{enumerate}
        \item   Der {\editor} öffnet die Seite des Mitarbeiters in {\wordpress}.
        \item   Der {\editor} legt in {\imperia} ein neues Dokument an.
                Die Wahl einer Vorlage trifft er autonom oder nach festgelegten Kriterien.
        \item   \label{item:problemAnalysisManualMigrationSlectInfoStep}Der {\editor} kopiert den Namen, die Telefonnummer und die E-Mail-Adresse
                des Mitarbeiters einzeln aus dem gemeinsamen Formularfeld in {\wordpress}
                und fügt sie in die vorgesehenen Felder in {\imperia} ein.
                Welche dies sind entscheidet er wiedrum autonom oder aufgrund einer Regel.
        \item   Der {\editor} lädt das Bild des Mitarbeiters von {\wordpress} herunter
                und fügt es in {\imperia} ein.
        \item   Der {\editor} füllt bei Bedarf sonstige Felder in {\imperia},
                für die es in {\wordpress} keine Entsprechung gab.
        \item   Der {\editor} speichert das Dokument und gibt es frei.
    \end{enumerate}

    Prinzipiell kann jeder dieser Schritte über die Schnittstellen von {\wordpress}
    und {\imperia} durch passende Migrationsskripte automatisiert werden.
    Lediglich der \ref{item:problemAnalysisManualMigrationSlectInfoStep}. Schritt
    birgt eine konzeptionelle Herausforderung für eine Automatisierung.
    Nämlich, wie ein Migrationsskript entscheiden soll, was im Formularfeld in {\wordpress}
    der Name, was die Telefonnummer und was die E-Mail-Adresse des Mitarbeiters ist.
    Alles steht in einem gemeinsamen Formularfeld, weshalb die Informationen nicht
    einzeln angesprochen werden können.
    Vor allem unter dem Gesichtspunkt, dass das Feld ein Freitextfeld ist und prinzipiell beliebig gefüllt sein kann.
    Ein Mensch kann diese Zuordnung leicht treffen,
    eine Maschine muss hingegen instrumentiert werden.
