\chapter{Diskussion der Ergebnisse}
    \label{chapter:FindingsDiscussion}

    % DSL
    % Abdeckung der Comäne (Coverage)
    % - Hängt davon ab, was Domäne ist
    % - Kann mit der Sprache jedes beliebige Struktur auf einer Webseite klassifiziert werden
    % - Siehe Formeln in Zusammenfassung
    % - Einige Spezialfälle nicht abgedeckt, siehe Ergebnisse

    % Vollständigkeit (Completeness)
    % - Sprache ist komplett
    % - Alles kann direkt in Sprache geschrieben werden.
    % - Kein D-1 Code notwendig
    % - D-1 wäre JSON. Sprache braucht das nicht.

    % -- 1. Sieht alles gut aus! Hier ist es stark.
    % -- 2. Einschränkung: Z. B. mehrere p Elemente können nicht in ein Feature gepackt werden, maximal als Collection. Bissl doof, aber nicht kritisch.
    % -- 3. Wiederholungen ohne umschließendes Objekt nicht unterstützt.
    % -- Wie geeignet ist die DSL
    % --- Wie gut lässt sie sich lesen?
    % --- Wie gut lässt sie sich schreiben?
    % -- Wie sinnvoll ist das Sharen der Nodes tatsächlich?

    % Es ist schwierig einen Selektor zu finden, der auf alles auf EINER Seite und auf ALLE Portale passt:
    % - Telefonnumer in Lehrende:
    % -- Was im eclipse Projekt steht, passt nicht auf Nils Arne (http://www.fernuni-hagen.de/KSW/portale/bapvs/einstieg/lehrende-und-betreuende-im-b-a-pvs/),
    % -- weil im Quelltext nach der Nummer kein Zeilenumbruch steht.
    % -- Das passende Statement wäre:
    % -- substring-before(concat(substring-after(normalize-space(.//text()[contains(., 'Tel')]), 'Tel.: '), '\n'), '\n')
    % - Außerdem werden die Namen auf einer Seite fett, auf der anderen als Link. Macht matching schwer.
    % - Bei Christopher Dorn ist die Node Struktur (team-member-des) plötzlich anders
    % - Bei manchen heißt es Telefon anstatt Tel