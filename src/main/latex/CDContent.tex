\documentclass[parskip=half]{scrartcl}

% Direct input of special characters
\usepackage[utf8]{inputenc}

% Support west european fonts / languages
\usepackage[T1]{fontenc}

% Use german translations for titles etc. (e.g. Inhaltsverzeichnis instead of Table of contents)
\usepackage[ngerman]{babel}

% Koma page style
\usepackage[automark]{scrlayer-scrpage}
\pagestyle{scrheadings}

\usepackage{listingsutf8}
\usepackage{color}
\definecolor{my-string-color}{RGB}{0,128,7}
\definecolor{my-comment-color}{RGB}{64,128,128}


\usepackage[margin=0.5cm]{geometry}
\usepackage{nopageno}
\usepackage{titling}

\parindent0mm

\begin{document}
    \section*{Inhalt und Kurzanleitung}
    Dieses Dokument erläutert den Inhalt der CD-ROM und
    die wichtigsten Schritte zur Nutzung der
    Webpage Content Modeling Language und des
    Webpage Content Classification Systems.

    \subsection*{Inhalt der CD-ROM}
        Auf dieser CD-ROM finden Sie:

        \begin{itemize}
            \item \textit{Masterarbeit-Gremplewski\_Tim-9514244.pdf} -- Eine digitale Version der Ausarbeitung
            \item \textit{WCCS-Quellcode.zip} -- Der vollständige Quellcode des Webpage Content Classification Systems
            \item \textit{WCCS-Auslieferung.zip} -- Eine installier- und ausführbare Version des Webpage Content Classification Systems
        \end{itemize}

    \subsection*{Installation und Nutzung der WCML}
        Die Webpage Content Modeling Language wird als Eclipse Feature ausgeliefert
        und kann als solches über den Installationsmechanismus von Eclipse eingespielt werden.
        Die zu installierende Feature-Zip-Datei befindet sich im
        Ordner \textit{wccs-web-content-modeling-language-1.0.0}
        der Auslieferung.
        Nach der Installation kann in Eclipse ein Projekt und darin \textit{.wcm}-Dateien angelegt werden.

    \subsection*{Starten des WCCS'}
        Zum Starten des WCCS' gehen Sie folgendermaßen vor:

        \begin{enumerate}
            \item   Entpacken Sie die Datei \textit{WCCS-Auslieferung.zip}
            \item   Installieren Sie Docker\footnote{vgl. \url{https://docs.docker.com/engine/installation/}}
                    und Docker Compose\footnote{vgl. \url{https://docs.docker.com/compose/install/}}
            \item   Öffnen Sie die Datei \textit{docker-compose.yml} und nehmen Sie bei Bedarf folgende Anpassungen vor:
                    \begin{enumerate}
                        \item Port-Mapping der Services
                        \item Pfad zu einem Klassifizierungsmodell (Zeile 20)
                        \item Nutzername und Passwort des Datenbanksystems (Zeile 24, 35, 36)
                        \item Pfad zum Speicherort der Datenbank (Zeile 30)
                    \end{enumerate}
            \item   Öffnen Sie eine Kommandozeile im Verzeichnis der entpackten Auslieferung
            \item   \textit{Bei der erstmaligen Verwendung: } Führen Sie das Kommando \texttt{docker-compose build} aus
            \item   Führen Sie das Kommando \texttt{docker-compose up} aus
        \end{enumerate}

    \subsection*{Verwendung des WordPress Crawlers und des Annotator Plugins}
        Der WordPress Crawler und das Annotator Plugin befinden sich in den
        Ordnern \textit{wccs-wordpress-crawler-1.0.0}
        bzw \textit{wccs-annotator-plugin-1.0.0} der Auslieferung.
        Ihre Nutzung ist bereits in den Kapiteln 4.6.2 und 4.8.1
        der Ausarbeitung beschrieben.

    \subsection*{Nutzung des Annotation Viewers}
        Der Annotation Viewer wurde zur Durchführung der Fallbeispiele entwickelt,
        ist aber trotzdem Teil der Auslieferung und kann für Testzwecke verwendet werden.
        Starten Sie das WCCS und rufen Sie in einem Webbrowser folgende URL auf:
        \texttt{http://<HOST>:<PORT>?url=<URL>}.
        Der Platzhalter \texttt{HOST} ist durch den Hostnamen des Servers zu ersetzen,
        auf dem das WCCS läuft. Die Portnummer ist standardmäßig 29136, kann aber in der
        Datei \texttt{docker-compose.yml} angepasst worden sein.
        Der Parameter \texttt{url} enthält die URL der klassifizierten Seite,
        die aufgerufen werden soll.

\end{document}
