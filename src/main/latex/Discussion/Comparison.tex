\section{Vergleich mit anderen Arbeiten}
    Die Diskussion der Ergebnisse wird in diesem Kapitel
    durch einen Vergleich der \gls{wccdl} und des Klassifizierungssystems
    mit anderen Arbeiten in ähnlichen Gebieten abgeschlossen.

    \subsection{Vergleich der Modellierungssprache}
        \label{section:discussionComparisonLanguage}
        Es existiert eine Vielzahl von Sprachen,
        die zur Datenmodellierung geeignet sind.
        Bei eigenen Recherchen wurde allerdings keine wissenschaftliche Arbeit über eine Sprache gefunden,
        die einen identischen Anwendungsfall adressiert und dieselbe enge Domäne umfasst wie die \gls{wccdl}.
        Es wurde lediglich ein
        GitHub Projekt\footnote{vgl. Kapitel \ref{section:discussionInterpretationLanguage} ; \url{https://github.com/sabrams/web-miner}} entdeckt,
        welches die Implementierung einer Sprache mit einer sehr ähnlichen Intention enthält.
        Diese Sprache ist eine in Ruby umgesetzte Internal \gls{dsl},
        die mittels XPath-Ausdrücken Informationen aus HTML-Dokumenten bezieht
        und in flachen Datenstrukturen speichert.
        Wie die \gls{wccdl} erfasst sie mehrfach auftretende Elemente in Listen,
        unterstützt offensichtlich aber nicht die Erzeugung komplexer Datenstrukturen.
        Neben der aufgezeigten besseren
        Ausdrucksstärke\footnote{vgl. Kapitel \ref{section:discussionInterpretationLanguage}},
        ist ein weiterer Vorteil,
        dass ihr prinzipiell sämtliche Sprachmittel und Funktion von Ruby
        zur Verfügung stehen.
        Gewonnene Informationen können dadurch beliebig weiterverarbeitet werden.
        Dieser Vorteil ist durch ihre Implementierung als Internal \gls{dsl} begründet.
        Neben deren Vorteile besitzt die Sprache natürlich auch ihre Nachteile,
        wie zum Beispiel die eingeschränkten syntaktischen Möglichkeiten,
        wodurch der Syntax bereits jetzt technischer und komplexer als der der \gls{wccdl} ist.
        Wie bereits angesprochen bietet sie außerdem keine komplexen Strukturen und verwendet lediglich XPath-Ausdrücke.

        Erweitert man die Domäne bei der Suche, findet man Sprachen,
        die Überschneidungen mit der \gls{wccdl} besitzen.
        Ein Beispiel ist die \gls{webml} \cite{ceri:webML}.
        Diese Sprache erlaubt die graphische Modellierung von Webseiten
        und bietet dazu fünf verschiedene Modelltypen:
        Structure, Composition, Navigation, Presentation und Personalization.
        Vor allem die beiden zuerst genannten Modelltypen eignen sich
        als Alternative zur \gls{wccdl}.
        Ein weiteres Beispiel ist \gls{uwe} \cite{koch:uwe} --
        ein Software Engineering Ansatz,
        welcher ein spezielles UML Profil für die Entwicklung von Webanwendung bietet.
        Es verwendet unter anderem Anwendungsfalldiagramme zur Beschreibung von Anforderungen,
        Klassendiagramme zur Modellierung der Domäne
        und mit speziellen Stereotypen versehende Klassendiagramme zur Modellierung der Navigation
        und Präsentation.
        Klassendiagramme sind ebenfalls geeignet, um die Problemstellungen des \glspl{wccs} zu beschreiben.
        Im Vergleich zur \gls{webml} und zum \gls{uwe} besitzt die \gls{wccdl} dennoch Stärken,
        wenn es um die Nutzung als Sprache zur Instrumentierung des \glspl{wccs} geht.
        Die liegen vor allem in ihrer begrenzteren Domäne,
        wodurch sie deren Konzepte in geeigneteren Sprachkonzepten umsetzen kann.
        Die anderen beiden Sprachen besitzen z. B. kein konkretes Konzept für
        Selektoren, weshalb dazu z. B. ein Attribut in einem Klassendiagramm genutzt werden müsste.
        Aus der Sicht eines Entwicklers ist dies weniger sprechend als das Konzept der \gls{wccdl}.
        Da sie eine kleinere Domäne abdeckt, besitzt sie weniger Konzepte
        und ist dadurch leichter zu erlernen.
        Die stärken von \gls{webml} und \gls{uwe} liegen dagegen in der vollständigen Konzeption
        ganzer Webseiten und Webanwendungen.
    
    \subsection{Vergleich des Klassifizierungssystems}
        \label{section:discussionComparisonClassificationSystem}
        Ein Vergleich des Klassifizierungssystems kann in verschiedenen Anwendungsbereichen erfolgen.
        Dieses Kapitel führt einen Vergleich in drei Kategorien durch.
        Dabei handelt es sich um "`\gls{cms}-Migration"', "`Web Mining"'
        und "`Schemavalidierung"'.

        \paragraph{\gls{cms}-Migrationen}
        Das \gls{wccs} wurde aufgrund des konkreten Falles der {\fernUni}
        entwickelt, bei der eine Migration von Inhalten aus {\wordpress} zu {\imperia} notwendig ist.
        Die Intention des \glspl{wccs} für diesen Fall ist die vorbereitende Strukturierung der
        Inhalte aus {\wordpress}.
        Es ist deshalb legitim das \gls{wccs} als einen Bestandteil eines Werkzeugkastens
        für \gls{cms}-Migrationen zu kategorisieren und vergleichebaren Produkten gegenüberzustellen.
        Im Internet finden sich mehrere Dienste, die eine einfache Migration versprechen.
        Ein interessanter Anbieter ist cms2cms\footnote{vgl. \url{https://cms2cms.com/}}.
        Bei diesem Anbieter muss zur Migration sowohl die Art eines Quellsystems als auch eines Zielsystems ausgewählt
        und auf beiden ein passendes Plugin installiert werden.
        Die Migration kann anschließend über cms2cms gestartet werden.
        Der Vorteil dieses Systems ist,
        dass eine aufwändige Vorbereitung der Inhalte nicht notwendig ist.
        Die speziellen Plugins erlauben prinzipiell außerdem eine Behandlung spezieller
        Eigenarten jedes unterstützten \glspl{cms}.
        Die Nachteile treten allerdings ebenso hervor: Ohne eine vorangegangene Strukturierung,
        müssen die Inhalte womöglich später im Zielsystem passend strukturiert werden,
        was ebenso aufwendig ist.
        Außerdem unterstützt cms2cms nur gewisse Systeme,
        wozu {\imperia} bspw. nicht zählt.
        Eine Transformation basierend auf der \gls{html}-Repräsentation einer Webseite wird zwar angeboten,
        ist aber nicht vollständig unterstützt.
        Genauso können die Inhalte eines Systems nicht durch Drittsysteme abgerufen werden.
        Beides sind elementare Ideen des \glspl{wccs}.

        \paragraph{Web Mining}
        Das \gls{wccs} betreibt im weitesten Sinne eine Form des Web Minings.
        Diese Disziplin ist in die Bereiche Web Structure Mining,
        Web Usage Mining und Web Content Mining unterteilt
        \cite{markov:webMining}.

        Links zwischen Webseiten erzeugen ein Netzwerk,
        welches im Fokus des Web Structure Minings steht.
        Basierend auf den eingehenden Verweisen kann dabei zum Beispiel
        ein Rang einer Seite berechnet werden,
        den Suchmaschinen zur Sortierung ihrer Ergebnisse verwenden
        \cite[Part I]{markov:webMining}.
        Ein bekannter Vertreter ist der von Google verwendete PageRank-Algorithmus
        \cite{page:pageRank}.
        Das \gls{wccs} ist selbst kein Werkzeug zum Web Structure Mining,
        da es weder einen Ranking-Algorithmus implementiert
        noch einen Webseitennutzer imitiert, der verschiedenen Links folgt
        und so automatische neue Webseiten auffindet.
        Stattdessen analysiert es nur vorgegebene Seiten.
        Da es das Netzwerk der klassifizierten Seiten in seiner Graphdatenbank speichert,
        taugt seine Datenbasis prinzipiell aber für die Ermittlung eines Rangs
        der klassifizierten Seiten.

        Beim Web Usage Mining wird das Verhalten eines Webseitenbesuchers analysiert,
        um darin Muster zu finden, Cluster zu bilden und darauf basierend z. B.
        einigen Besuchern spezielle Inhalte auf einer Webseite anzuzeigen.
        Dazu werden Log-Dateien analysiert oder spezielle JavaScript-Funktionen
        in Seiten eingebaut, die das Verhalten des Besuchers aufzeichnen
        \cite[Part III]{markov:webMining}.
        Das \gls{wccs} betreibt diese Form des Web Minings in keinster Weise
        und bietet durch seine Klassifikationen hierfür auch keine relevanten Informationen.

        Das Web Content Mining ist die größte Herausforderung.
        Es versucht Korrelationen zwischen den Inhalten von Webseiten zu finden
        und aus ihnen neues Wissen zu ziehen.
        Dadurch sollen Informationen im Internet besser gefunden werden können
        und Suchergebnisse auf mehr als dem Vorhandensein eines Suchbegriffes basieren.
        Dazu wendet es Techniken des Data Minings,
        des Text Minings und des Knowledge Discoveries auf Webdokumente an,
        wobei oftmals Systeme zum maschinellen Lernen zum Einsatz kommen.
        Zwei wichtige Ansätze sind Classification und Clustering.
        Beim ersten werden dem System klassifizierte Objekt präsentiert.
        Anschließend versucht es eigenständig nicht klassifizierte Objekte einzuordnen.
        Beim Clustering versucht das System hingegen eigenständig
        Übereinstimmungen und gemeinsame Muster unter Objekten zu finden
        und so Gruppen zu bilden
        \cite[Part II]{markov:webMining}.
        In gewissen Grenzen betreibt das \gls{wccs} eine simple Form des Web Content Minings,
        da es basierend auf einer vorhandenen Klassifizierungsregel
        Muster auf einer Webseite aufdeckt und Inhalte strukturiert.
        Durch die Struktur der Datenhaltung sind außerdem weiterführende Informationen ableitbar.
        Die Eignung des \glspl{wccs} als vollständiges Web Content Mining Werkzeugs ist
        aber nicht gegeben.
        Zum einen findet es nur Muster, die den Regeln exakt entsprechen.
        Des Weiteren ist es weniger mächtig als Werkzeuge,
        die auf maschinelles Lernen setzen.
        Beide Einschränkungen sind durch den angedachten Einsatzzweck des \glspl{wccs} begründet.
        Die Idee ist für eine definierte Menge an Webseiten ein Modell zu entwickeln,
        anhand dessen das \gls{wccs} dann eine Klassifizierung dieser Seiten durchführt.
        Das "`Training"' des Systems findet also für eine sehr kleine Teilmenge des \glspl{www} statt,
        wie zum Beispiel einigen Seiten der Fakultät \gls{ksw}.
        Aber nicht für das gesamte \gls{www}.
        Dadurch ist es nur auf wenigen Seiten anwendbar,
        kann aber sehr genau für die speziellen Anforderungen dieser Seiten instrumentiert werden.      

        \paragraph{Schemavalidierung von Webseiten}
        Das \gls{wccs} stellt außerdem eine Möglichkeit dar,
        Webseiten auf die Konformität bezüglich eines frei definierbaren
        inhaltlichen und strukturellen Schemas zu prüfen.
        Nämlich dann, wenn das Fehlen eines Features als Verletzung
        dieses Schemas interpretiert wird.
        Im ersten Fallbeispielen hätte diese Verwendungsart zum Beispielt aufgedeckt,
        dass bei zwei Kontakten die E-Mail-Adresse inkorrekt ist,
        da sie kein vorangestelltes \texttt{mailto:} enthalten.
        Andere Werkzeuge zur Validierung von \gls{html}-Dokumenten
        beschränken sich auf die Einhaltung von Standards des
        \glspl{w3c}\footnote{vgl. \url{https://validator.w3.org/} und \url{https://jigsaw.w3.org/css-validator/}}.
        Vorausgesetzt die Webseite ist ein valides XML-Dokument,
        ließe sich ein individuelles Schema auch über XSD formulieren und validieren.
        Mit XSD können komplexe Bedingungen formuliert werden und es existieren
        zahlreiche Applikationen zur Auswertung.
        Da Webseiten selten valide XML-Dokumente sind,
        besitzt das \gls{wccs} an dieser Stelle durchaus
        ein Alleinstellungsmerkmal.
