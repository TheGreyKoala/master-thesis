\subsection{Übertrakbarkeit eines Klassifizierungsmodells}
    Ein Ziel der Fallbeispiele ist herauszufinden,
    wie gut sich ein anhand einer einzelnen konkreten Seite
    erstelltes Klassifizierungsmodell auf andere Seiten anwenden lässt,
    die augenscheinlich konzeptionell identisch aufgebaut sind.
    Je besser diese Übertragbarkeit ist,
    desto höher ist der Nutzen der Automatisierung.

    Eine erste naheliegende Erkenntnis,
    die nicht direkt aus den präsentierten Ergebnissen hervorgeht,
    aber schon bei der Entwicklung des Klassifizierungsmodells
    für das Portal "`BaBw"' hervortrat, ist,
    das Klassen iterativ zu definieren sind.
    Ausnahmen innerhalb einer Seite werden oft erst
    bei der Prüfung des Klassifizierungsergebnis entdeckt,
    woraufhin die Ausnahme analysiert und die Klassendefinitionen
    entsprechend angepasst werden müssen.
    Ein Beispiel ist die Faxnummer, die nur bei einem Mitarbeiter auftritt,
    deren Erfassung aber auch eine Anpassung des Selektors der Telefonnummer
    erforderlich machte.

    Diese Erkenntnis wiegt natürlich noch schwerer,
    wenn eine Menge an Klassen auf viele verschiedene Seiten angewandt werden soll.
    Wie das erste Fallbeispiel zeigt,
    können auch vermeintlich sehr ähnliche Seiten deutliche Unterschiede
    sowohl im Modell als auch in der HTML-Struktur haben,
    die nicht alle vorausgedacht werden können.
    Es müssen demnach mehrere Seiten bei der Entwicklung der Klassendefinitionen
    betrachtet werden.
    Andernfalls werden Informationen nicht klassifiziert,
    wie der Verweis auf die Detailseite eines Mitarbeiters im Falle des
    Portals "`BaPVS"' oder es kommt zu unerwarteten Ergebnissen,
    wie die doppelte Klassifizierungen bei "`BaPVS"'.

    Deutlich wurde aber auch, dass viele und zu große Unterschiede
    die Auflösung eines Konfliktes sehr komplex machen.

    Insgesamt lassen sich fast alle Unregelmäßigkeiten in den Klassifikationen
    auf Unterschiede in den Seiten und dadurch unzutreffende Selektoren zurückführen.
    Eine Schlussfolgerung könnte sein, auf ein anderes Vorgehen beim
    Schreiben der Klassendefinitionen zu setzen,
    welches nicht versucht alle Seiten durch eine Definition abzudecken.
    Eine Alternative ist zum Beispiel ein Modell für eine Seite zu schreiben,
    sodass es diese korrekt und vollständig klassifiziert.
    Es dann auf eine weitere Seite anzuwenden,
    das Ergebnis zu prüfen und anzupassen,
    bis diese zweite Seite korrekt klassifiziert wird.
    Die Klassifikation findet dabei nur auf den betrachteten Seiten statt,
    sodass bereits korrekt klassifizierte Seiten unberührt bleiben.
    Dadurch entstehen natürlich Klassifikationen,
    die ein anderes konkretes Schema besitzen,
    was Drittsysteme, die diese Informationen nutzen,
    beachten müssen.
    Außerdem sollte nicht die Möglichkeit missachtet werden,
    bewusst einzelne falsche Klassifizierungen in Kauf zu nehmen
    und über das Annotator Plugin im Nachhinein zu korrigieren.

    Nicht erfasste Features und unzutreffende Selektoren können
    aber auch auf Fehler in der Webseite hinweisen,
    wie der E-Mail-Adresse ohne vorangestelltes "`mailto"'
    bei "`BscPsy"' oder einem nicht erwähnten Fall,
    beim dem die angebliche E-Mail-Adresse ein Link auf eine andere Webseite war.
    Das \gls{wccs} fungiert indirekt also als ein Werkzeug
    zu Validierung eines individuell definierbaren Schemas für Webseiten.