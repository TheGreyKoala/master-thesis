\subsection{Übertrakbarkeit eines Klassifizierungsmodells}
    Ein Ziel der Fallbeispiele ist herauszufinden,
    wie gut sich ein anhand einer einzelnen konkreten Seite
    erstelltes Klassifizierungsmodell auf andere Seiten anwenden lässt,
    die augenscheinlich konzeptionell identisch aufgebaut sind.
    Je besser diese Übertragbarkeit ist,
    desto höher ist der Nutzen der Automatisierung.

    Eine erste naheliegende Erkenntnis,
    die nicht direkt aus den präsentierten Ergebnissen hervorgeht,
    aber schon bei der Entwicklung des Klassifizierungsmodells
    für das Portal "`BaBw"' hervortrat, ist,
    das Klassen iterativ zu definieren sind.
    Ausnahmen innerhalb einer Seite werden oft erst
    bei der Prüfung des Klassifizierungsergebnis entdeckt,
    woraufhin die Ausnahme analysiert und die Klassendefinitionen
    entsprechend angepasst werden müssen.
    Ein Beispiel ist die Faxnummer, die nur bei einem Mitarbeiter auftritt,
    deren Erfassung aber auch eine Anpassung des Selektors der Telefonnummer
    erforderlich machte.

    Diese Erkenntnis wiegt natürlich noch schwerer,
    wenn eine Menge an Klassen auf viele verschiedene Seiten angewandt werden soll.
    Wie das erste Fallbeispiel zeigt,
    können auch vermeintlich sehr ähnliche Seiten deutliche Unterschiede
    sowohl im Modell als auch in der HTML-Struktur haben,
    die nicht alle vorausgedacht werden können.
    Es müssen demnach mehrere Seiten bei der Entwicklung der Klassendefinitionen
    betrachtet werden.
    Andernfalls werden Informationen nicht klassifiziert,
    wie der Verweis auf die Detailseite eines Mitarbeiters im Falle des
    Portals "`BaPVS"' oder es kommt zu unerwarteten Ergebnissen,
    wie die doppelte Klassifizierungen bei "`BaPVS"'.

    Deutlich wurde aber auch, dass viele und zu große Unterschiede
    die Auflösung eines Konfliktes sehr komplex machen.

    Diese Überlegungen lassen die Schlussfolgerung zu,
    dass die Erstellung einer auf mehreren Seiten nutzbaren
    Klassendefinition aufgrund der großen Unterschiede eine komplexe Aufgabe ist.
    Eine kritische, aber berechtigte Frage ist deshalb,
    wie groß der Nutzen durch die Automatisierung tatsächlich ist.
    Vor allem dann, wenn zur Erfassung einiger Inhalte Anpassungen an der Seite notwendig sind,
    was im nächsten Kapitel behandelt wird.

    Dem ist entgegenzusetzen, dass die Effektivität durch ein anderes Vorgehen beim
    Schreiben der Klassendefinitionen gesteigert werden kann,
    welches nicht versucht alle Seiten durch eine Definition abzudecken.
    Eine Alternative ist zum Beispiel ein Modell für eine Seite zu schreiben,
    sodass es diese korrekt und vollständig klassifiziert.
    Dieser Schritt ist einfach und bringt zweifelsfrei Vorteile,
    was bei der initialen Klassifikation der Lehrenden des Portals "`BaBw"' ersichtlich wurde,
    wobei genau dieser Schritt stattfand.
    Dann sollte das Klassifizierungsmodell auf eine weitere Seite angewandt werden,
    das Ergebnis geprüft und angepasst werden,
    bis diese zweite Seite korrekt klassifiziert wird.
    Die Klassifikation findet in diesem Schritt ausschließelich auf der zweiten Seite statt
    sodass die erste bereits korrekt klassifizierte Seite unberührt bleibt.
    Dieser zweite Schritt wird für alle Seiten wiederholt.
    Dieses Vorgehen erlaubt die Wiederverwendung global gültiger Klassendefinitionen
    und erfordert nur die Anpassung einzelner Klassen an die jeweiligen Anforderungen.
    Insgesamt dürfte der Prozess dadurch deutlich effektiver werden.

    Es sollte außerdem nicht die Möglichkeit missachtet werden,
    bewusst einzelne falsche Klassifizierungen in Kauf zu nehmen
    und diese über das Annotator Plugin im Nachhinein zu korrigieren.

    Nicht erfasste Features und unzutreffende Selektoren können
    aber auch auf Fehler in der Webseite hinweisen,
    wie der E-Mail-Adresse ohne vorangestelltes "`mailto"'
    bei "`BscPsy"' oder einem nicht erwähnten Fall,
    beim dem die angebliche E-Mail-Adresse ein Link auf eine andere Webseite war.
    Das \gls{wccs} fungiert indirekt also als ein Werkzeug
    zu Validierung eines individuell definierbaren Schemas für Webseiten.