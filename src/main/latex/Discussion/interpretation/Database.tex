\subsection{Eignung einer Graphdatenbank}
    Die Nutzung einer Graphdatenbank für das \gls{wccs} hat mehrere Beweggründe.
    Dazu zählen die einfache, natürliche und schemalose Datenmodellierung
    und die Möglichkeit Beziehungen sehr einfach auszuwerten.
    Die Wiederverwendung von Knoten für mehrere Klassifikationen
    bietet außerdem die Möglichkeit, weiterführende Analysen auf den
    klassifizierten Inhalten durchzuführen.
    Dieses Kapitel bewertet die tatsächliche Eignung einer Graphdatenbank,
    des verwendeten Datenmodells sowie des Algorithmus
    zum Schreiben von Klassifikationen, der das Teilen von Knoten ermöglicht.

    \paragraph{Referenzen zwischen Webseiten}
    Das zweite Fallbeispiel hat gezeigt,
    wie Referenzen zwischen Webseiten auf sehr einfache Art und Weise
    Reihenfolgen und Navigationspfade (der Nachrichtenseiten) explizit machen.
    Eine Auswertung ist trivial, da lediglich ein- und ausgehenden Kanten gefolgt werden muss.
    Andere Datenbankmodelle hätten komplexere Konstrukte
    zur Speicherung und zur Auswertung erfordert.

    \paragraph{Wiederverwendung von \texttt{Text}- und \texttt{Resource}-Knoten}
    Betrachtet man Tabelle \ref{table:findingsTeachersFiguresNodesByLabel} wird deutlich,
    dass \texttt{Text}- und \texttt{Resource}-Knoten verhältnismäßig am meisten von der
    Wiederverwendbarkeit gebrauch machen.
    Das ist nicht verwunderlich, da sie jeweils genau einen Wert speichern,
    der sie identifiziert.
    Gleichzeitig besitzen sie keine ausgehenden Kanten,
    die seitenspezifische Informationen enthalten.
    Wie Tabelle \ref{table:findingsTeachersFiguresSharedNodes} belegt,
    steigt ihre Wiederverwendung bei mehreren Klassifikationen in einer Datenbank,
    was eine logische Konsequenz der größeren Datenmenge ist.
    Für \texttt{Resource}-Knoten ist das aber nicht sofort ersichtlich, da
    die Summe der mehrfach verwendeten \texttt{Resource}-Knoten von Lehrgebieten
    im Falle einzelner Datenbanken höher ist, als im Fall einer gemeinsamen Datenbank (36 vs. 30).
    Gleichzeitig ist die Zahl der geteilten \texttt{SubjectArea}-Knoten aber höher,
    die jeder einen \texttt{Resource}-Knoten eines Lehrgebietes referenzieren.
    Es wird also ein größerer Teil des Graphen wiederverwendet.
    Außerdem enthielten die einzelnen Datenbanken identische \texttt{Resource}-Knoten,
    die in der gemeinsamen natürlich zusammengefasst werden konnten.

    \paragraph{Entbehrlichkeit von \texttt{Text}-Knoten}
    Innerhalb einer Datenbank macht es aus semantischen Gründen keinen Sinn
    \texttt{Resource}-Knoten zu duplizieren,
    da sie eine Entität der Domäne darstellen.
    {\resources} existieren genau ein mal, was die Datenbank widerspiegeln sollte
    und außerdem die Möglichkeiten eines Graphen besser ausschöpft.
    Nicht so eindeutig ist dies allerdings bei \texttt{Text}-Knoten.
    Aus Tabelle \ref{table:findingsTeachersFiguresSharedNodes} geht hervor,
    dass im ersten Fallbeispiel niemals ein \texttt{Text}-Knoten geteilt wurde.
    Stattdessen konnte immer der zugehöroge \texttt{Content}-Knoten geteilt werden.
    Deshalb ist die Zahl beider Knotentypen gesunken und die der geteilten
    \texttt{Content}-Knoten dafür gestiegen.
    Tabelle \ref{table:findingsNewsFiguresSharedNodes}
    zeigt eine andere Situation beim zweiten Fallbeispiel.
    Dort werden zwei \texttt{Text}-Knoten geteilt,
    weil es mehrere Nachrichten mit derselben Überschrift gibt.
    Sie referenzieren aber unterschiedliche Detailseiten,
    weshalb die Nachrichten unterschiedliche \texttt{Content}-Knoten besitzen.
    Der tatsächliche Nutzen der \texttt{Text}-Knoten ist in beiden Beispielen trotzdem sehr gering.
    Der textuelle Wert eines {\contentFeature}s ist deshalb im \texttt{Content}-Knoten selbst besser aufgehoben.
    Für die Schnittstellen des \glspl{wccs} ändert sich durch eine entsprechende Anpassung nichts,
    allerdings vereinfacht sich das Datenmodell der Datenbank sowie
    die generierten Datenbankanweisungen.

    \paragraph{Hierarchieebene wiederverwendeter Teilgraphen}
    Bezüglich der Wiederverwendung von Teilgraphen ist außerdem ersichtlich,
    dass sie häufig erst auf einer tiefen Ebene des Graphen stattfindet.
    \texttt{Content}-Knoten der Klasse \texttt{SubjectAreaName} werden bspw.
    sehr viel häufiger von mehreren Klassifikationen verwendet als solche der
    Klasse \texttt{Teacher}.
    Betrachtet man die klassifizierten Webseiten,
    ist eine entgegengesetzte Erwartung gerechtfertigt.
    Der Kopfbereich wiederholt sich z. B. auf allen Seiten
    und mehrere Mitarbeiter werden auf unterschiedlichen Webseiten identisch aufgeführt.
    Trotzdem wird bei solchen Mitarbeitern und dem Kopfbereich
    nicht der \texttt{Teacher}- bzw. \texttt{Header}-Knoten geteilt,
    sondern nur ihre Unterknoten.
    Dafür gibt es zwei Gründe.
    Zum einen müssen Features inklusive ihrer {\childFeature}s inhaltlich
    exakt übereinstimmen, damit ihr \texttt{Content}-Knoten mehrfach verwendet werden kann.
    Die kleinste Abweichung macht dies unmöglich,
    da die Inhalte sich letztendlich unterscheiden und nur Teilaspekte übereinstimmen.
    Das muss die Datenbank widerspiegeln.
    Im Falle des Kopfbereiches im ersten Beispiel ist der \gls{url} des Logos
    auf jeder Seite unterschiedlich.
    Das ist im zweiten Beispiel nicht der Fall,
    da alle Seiten zu einem Portal in {\wordpress} gehören
    und deshalb das gleiche Bild referenzieren.
    Deshalb existiert in der Datenbank nur ein \texttt{Header}-Knoten,
    den alle Klassifikationen referenzieren.
    Die zweite Ursache sind unterschiedliche Positionen identischer Inhalte
    auf verschiedenen Webseiten.
    Diese Positionen werden in Form des eindeutigen Selektors in der eingehenden
    Kante eines \texttt{Content}-Knotens gespeichert\footnote{vgl. Kapitel \ref{section:solutionDetailsPersistenceDataModel}}.
    Diese Knoten sind prinzipiell also unabhängig von der konkreten Position
    von mehreren Klassifikationen gleichzeitig nutzbar.
    Dies ändert sich allerdings, sobald ein Inhalt {\childFeature}s besitzt.
    Die ausgehenden Kanten zu diesen Features speichern ihre \textit{absoluten} Selektoren.
    Befindet sich ein Mitarbeiter auf zwei Seiten also nicht an exakt der gleichen Position,
    sind die Selektoren seines Namens, seines Lehrgebietes etc. unterschiedlich.
    Der \texttt{Teacher}-Knoten kann dann nicht von beiden Klassifikationen referenziert
    werden\footnote{vgl. Kapitel \ref{section:solutionDetailsClassificationStorageAPIUpdatePage}}.
    
    \paragraph{Semantischer Inhalt und physische Struktur einer Webseite}
    Genau betrachtet spiegelt die Datenbank lediglich die Struktur der Webseiten exakt wieder.
    Trotzdem lässt sich argumentieren, dass es fachlich konsistenter wäre,
    wenn die \texttt{Teacher}-Knoten im beschriebenen Fall geteilt werden würden.
    Ein Beispiel verdeutlicht das:
    Schon jetzt lässt sich die Frage beantworten,
    welche Mitarbeiter in einem gewissen Lehrgebiet arbeiten.
    Dazu muss lediglich den eingehenden Kanten eines
    \texttt{SubjectAreaName}-Knotens über alle \texttt{SubjectArea}-Knoten
    bis zu \texttt{Teacher}-Knoten gefolgt werden.
    Aus fachlicher Sicht ist es aber inkonsistent,
    dass man diese Suche nicht beim \texttt{SubjectArea}-Knoten
    starten kann, weil mehrere dieser Knoten denselben
    \texttt{SubjectAreaName}-Knoten verwenden.
    Aus fachlicher Sicht existieren laut Datenbank also mehrere Lehrgebiete,
    die denselben Namen tragen.
    Hier wird ein Konflikt zwischen dem akkuraten Widerspiegeln der Strukturen der Webseiten
    und dem semantischen Inhalt deutlich.

    \paragraph{Alternativen}
    Ein erster Schritt zur Auflösung dieses Konfliktes ist anstatt absolute Selektoren
    lediglich zum {\parentFeature} relative Selektoren in den Kanten zu speichern.
    Allerdings hilft dies nicht, wenn die HTML-Struktur sich unterscheidet,
    weil dann auch die relative Position eines Elementes anders ist.
    Außerdem hätte dies zur Folge, dass zur Ermittlung der eindeutigen Position
    eines Features, der gesamte Vaterpfad bis zur Seite abgelaufen werden muss.
    Eine weitere Alternative zum aktuellen Vorgehen,
    die auch diesen Nachteil umgeht,
    ist die Speicherung der absoluten Selektoren in Kanten,
    die vom \texttt{Page}-Knoten direkt zu \texttt{Content}-Knoten führen.
    Die Teilbarkeit von Content Knoten hinge dann nur noch von den Inhalten und ihrer Klasse ab,
    da die Position komplett unabhängig gespeichert wird.

    \paragraph{Änderung von Klassifikationen}
    Durch den entwickelten Algorithmus zum Schreiben von Klassifikationen\footnote{vgl. Kapitel \ref{section:solutionDetailsStorageAPI}},
    wurde das Teilen von Knoten ermöglicht.
    Eine nachteilige Folge ist allerdings,
    dass Änderungen immer über diesen Algorithmus eingearbeitet werden müssen,
    weil sonst Inkonsistenzen riskiert werden.
    Falls die Analysemöglichkeiten keine praktische Relevanz haben,
    ist deshalb ein anderer Datenbanktyp, wie zum Beispiel ein einfacher Document Store,
    in Betracht zu ziehen.

    Insgesamt scheint eine Graphdatenbank trotz der genannten Herausforderungen geeignet
    für den Anwendungsfall des \glspl{wccs}.
