\subsection{Teilen von Knoten in der Datenbank}
    Der erste Beweggrund für die Verwendung einer Graphdatenbank
    ist die natürliche Modellierung und Speicherung von Verweisen
    zwischen Seiten und auf sonstige {\resources}.
    Des Weiteren sollte durch die Vermeidung von Duplikaten in der Datenbank
    und der Möglichkeit einen Knoten mehrfach zu referenzieren,
    die Möglichkeit entstehen weiterführende Analysen auf den klassifizierten
    Inhalten auszuführen.

    Das zweite Beispiel hat gezeigt,
    wie Referenzen zwischen Seiten auf sehr einfache Art und Weise
    die Reihenfolge der Übersichtsseiten und Navigationspfade explizit machen.
    Eine Auswertung ist trivial, da lediglich ein- und ausgehenden Kanten gefolgt werden muss.
    Andere Datenbankmodelle hätten komplexere Konstrukte erfordert,
    um diese Informationen zu speichern oder hätten komplexere
    Aggregierungsschritte zur Auswertung erfordert.

    Betrachtet man Tabele \ref{table:findingsTeachersFiguresNodesByLabel} wird deutlich,
    dass Text- und Resource-Knoten verhältnismäßig am meisten von der
    Möglichkeit Knoten wiederzuverwenden profitieren.
    Das ist nicht verwunderlich, da sie jeweils genau einen Wert speichern,
    der sie identifiziert und keine ausgehenden Kanten besitzen,
    die seitenspezifische Informationen enthalten.
    Wie Tabelle \ref{table:findingsTeachersFiguresSharedNodes} belegt,
    steigt ihre Wiederverwendung bei mehreren Klassifikationen in einer Datenbank,
    was eine logische Konsequenz der größeren Informationsmenge ist.

    Für Resource-Knoten ist dies aber nicht sofort ersichtlich,
    zum Beispiel die Zahl der mehrfach referenzierten
    Lehrgebiets-Resource-Knoten im Falle einer gemeinsamer Datenbank niedriger ist,
    als die Summe der geteilten Knoten in einzelnen Datenbanken (30 vs. 36).
    Gleichzeitig ist die Zahl der geteilten SubjectArea-Knoten aber höher,
    die jeder einen Lehrgebietsknoten referenzieren.
    Es wird also lediglich ein größerer Teilbaum geteilt.
    Außerdem enthielten die einzelnen Datenbanken identische Resource-Knoten,
    die in der gemeinsamen natürlich zusammengefasst werden konnten.

    Resource-Knoten innerhalb einer Datenbank zu duplizieren
    macht aus semantischen Gründen keinen Sinn,
    da sie eine Entität der Domäne darstellen,
    die genau ein mal existiert, was die Datenbank wiederspiegeln sollte
    und außerdem die Möglichkeiten eines Graphens besser ausschöpft.

    Nicht so eindeutig ist dies allerdings bei Text-Knoten.
    Aus Tabelle \ref{table:findingsTeachersFiguresSharedNodes} geht hervor,
    dass im ersten Fallbeispiel niemals ein Text-Knoten geteilt wurde.
    Stattdessen konnte immer der Content-Knoten,
    der ihn referenziert geteilt werden.
    Deshalb ist die Zahl beider Knoten-Typen gesunken und die der geteilten
    Content Knoten dafür gesteigen.
    Das ist möglich, wenn die entsprechenden Content-Knoten
    sich sowohl in ihrer Klasse als auch in ihren Features nicht unterscheiden.
    Betrachtet man die Klassen der geteilten Content Knoten wird deutlich,
    dass diese keine Features haben.

    Anders ist das beim zweiten Fallbeispiel.
    Dort wurden Text-Knoten geteilt, weil es mehrere News mit derselben ÜBerschrift
    gab, die aber jeweils eine eigene Detailseite referenzieren.
    % TODO: Referenz auf TABELLE!

    
% Oder: Verwendung einer Graphdatenbank