\subsection{Eignung der domänenspezifischen Sprache}
    % TODO: Vorteile aus SolutionConcept/DSL.tex hier hin? Oder zu Grundlagen über DSLs bzw. im Solution Details Abschnitt der DSL?
    \label{section:discussionInterpretationLanguage}
    Die Analyse der \gls{wccdl} orientiert sich
    an den Design Dimensionen von \cite{voelter:DslEngineering}.

    \paragraph*{Ausdrucksstärke}
    Die Ausdrucksstärke einer Sprache beschreibt,
    wie kompakt Programme einer Problemdomäne in dieser Sprache
    formuliert werden können.
    Zur Vergleichbarkeit zweier Sprachen bezüglich ihrer Ausdrucksstärke
    stellen \citet[Kapitel 4.1]{voelter:DslEngineering} die folgende
    Definition auf, wobei
    $P_D$ die Menge aller Programme der Domäne,
    $P_L$ die Menge aller in einer Sprache formulierbaren Programme und
    $p_L$ ein konkretes Programm in einer Sprache bezeichnet.

    \begin{quote}
        A language $L_1$ is \textit{more expressive in domain D}
        than a language $L_2 (L_1 {\prec}_D L_2)$,
        if for each $p \in P_D \cap {P_L}_1 \cap {P_L}_2, |{p_L}_1| < |{p_L}_2|$.
    \end{quote}

    Eine Aussage über die Ausdrucksstärke der \gls{wccdl} kann anhand dieser Formel
    nur relativ zu anderen Sprachen gemacht werden,
    weshalb zusätzlich zu den gesammelten Ergebnissen und
    Kapitel \ref{section:discussionComparisonLanguage}
    an dieser Stelle ein kurzer Vergleich mit zwei anderen Sprachen stattfindet.
    Die obige Definition ist schwer anzuwenden,
    wenn eine Sprache textuell und die andere grafischer Natur ist.
    Viele Modellierungssprachen sind deshalb schwer mit der \gls{wccdl} zu vergleichen.
    Allerdings konnten zwei geeignete Kandidaten gefunden werden.
    Der Erste ist PlantUML\footnote{vgl. \url{http://plantuml.com/}},
    ein Werkzeug und eine Sprache zur textuellen Beschreibung und anschließender
    Generierung grafischer UML-Diagramme.
    Die zweite Sprache wurde bei Recherchen auf
    GitHub\footnote{vgl. \url{https://github.com/sabrams/web-miner}} gefunden
    und verfolgt ein sehr ähnliches Ziel wie die \gls{wccdl}.
    Sie ist eine Internal Ruby DSL, die trotz ihrer etwaigen Unvollständigkeit sowie der fehlenden
    offiziellen wissenschaftlichen Arbeit oder Dokumentation
    für einen Vergleich mit der \gls{wccdl} geeignet ist.
    Sie vermittelt nämlich, wie das Vorhaben der \gls{wccdl}
    in einer anderen Sprache aussehen kann.
    Als Vergleich dient ein Beispiel im
    Testcode\footnote{vgl. \url{https://github.com/sabrams/web-miner/blob/master/features/mine_according_to_command_file.feature}}
    dieser DSL, welches in Listing \ref{listing:discussionExpressivityExampleRubyDsl}
    zu sehen ist.

    \lstinputlisting[
        label=listing:discussionExpressivityExampleRubyDsl,
        caption=Ein Programm in einer Ruby DSL zum Vergleich der Ausdrucksstärke,
        style=pseudo,
        language=Ruby
    ]{../resources/discussion/interpretation/language/expressivity-example.rb}

    Ein äquivalentes Konstrukt in PlantUML zeigt Listing
    \ref{listing:discussionExpressivityExamplePlantUml}.

    \lstinputlisting[
        label=listing:discussionExpressivityExamplePlantUml,
        caption=Ein Programm in PlantUML zum Vergleich der Ausdrucksstärke,
        style=plantuml
    ]{../resources/discussion/interpretation/language/expressivity-example.plantuml}

    Nicht zuletzt zeigt Listing \ref{listing:discussionExpressivityExampleWccs},
    wie das Beispiel in der \gls{wccdl} aussieht.

    \lstinputlisting[
        label=listing:discussionExpressivityExampleWccs,
        caption=Ein Programm in der \acrshort{wccdl} zum Vergleich der Ausdrucksst\"arke,
        inputencoding=utf8/latin1,
        style=wccdl
    ]{../resources/discussion/interpretation/language/expressivity-example.wctd}

    Die \gls{wccdl} braucht in diesem Beispiel mehr Code als die anderen beiden Sprachen.
    Auch ohne genaue Analyse sollte diese geringere Ausdrucksstärke
    allgemeingültig für die \gls{wccdl} sein.
    Anders als ihre Kontrahenten benötigt sie z. B. für jedes Feature eine definierte Klasse
    (im Beispiel \texttt{Object}).
    Außerdem ist die Definition einer einzelnen Klasse,
    eines einzelnen Features oder eines Selektors in etwa gleich lang oder länger
    als in den Vergleichssprachen.
    Der Grund sind die langen Schlüsselwortsequenzen,
    die auch im Selektor der Seitenklasse \texttt{Teacher}
    erkennbar sind.
    An dieser Klasse und ihrem Pendant \texttt{News}
    wird außerdem deutlich, dass der Sprache ein Vererbungskonzept fehlt.
    Identische Features müssen nämlich wiederholt werden,
    wodurch sich die Ausdrucksstärke der Sprache verringert.
    Eine Kritik der Ausdrucksstärke der \gls{wccdl} ist berechtigt,
    allerdings muss beachtet werden,
    dass dieses Merkmal in Konflikt mit einigen nun folgenden
    Eigenschaften steht, die positiver bewertet werden.

    % TODO: Namespace?

    \paragraph*{Abdeckung der Domäne}
    Die Abdeckung der Domäne einer Sprache beschreiben
    \citet[Kapitel 4.2]{voelter:DslEngineering} als den
    prozentualen Anteil an allen Programmen der Domäne,
    die durch die Sprache formuliert werden können.
    Die formale Definition lautet

    \begin{quote}
        $C_D(L) = \frac{number\ of\ P_D\ programs\ expressable\ by\ L}{number\ of\ programs\ in\ domain\ D}$.
    \end{quote}

    Eine volle Abdeckung ist demnach gegeben,
    wenn jedes Programm der Domäne in der Sprache ausgedrückt werden kann.
    Im Falle der \gls{wccdl} kann zumindest die Frage beantwortet werden,
    ob sie ihre Domäne voll abdeckt, das heißt, ob jede beliebige Struktur auf einer Webseite
    durch sie erfasst werden kann.
    Wie das zweite Fallbeispiel zeigte, ist das nicht der Fall,
    da kein allgemeiner Selektor gefunden werden konnte,
    der die Absätze einer Nachricht korrekt erfasst.
    Spezielle Klassen pro Meldung können dieses Problem zwar lösen,
    sind aber nicht im eigentlichen Sinne des \glspl{wccs}.
    Es wurden aber auch einige nicht triviale Strukturen durch die Sprache abgedeckt.
    Selbstverständlich spielen bei dieser Frage auch die Fähigkeiten der Selektoren
    eine Rolle.

    \paragraph*{Vollständigkeit}
    Eine Sprache ist nach \citet[Kapitel 4.5]{voelter:DslEngineering}
    vollständig, wenn ihr Generierungsergebnis keine manuell geschriebenen Codefragmente in niedrigeren Sprachen,
    Konfigurationsdateien o. Ä. benötigt,
    um die gleiche Semantik zu besitzen, wie das Programm in der \gls{dsl}.
    Formaler beschreiben sie diese Eigenschaft wie folgt:

    \begin{quote}
        Let us introduce a function $G$ ("code generator") that transforms
        a program $p$ in $L_D$ to a program $q$ in $L_{D-1}$.
        For a complete language, $p$ and $q$ have the same semantics, i.e.
        $OB(p) == OB(G(p)) == OB(q)$ [...]. For incomplete languages
        where $OB(G(p)) \subset OB(p)$ we have to write additional
        code in $L_{D-1}$ to obtain a program in $D_{-1}$ that has the same semantics
        as intended by the original program in $L_D$.
    \end{quote}

    Die \gls{wccdl} ist vollständig, da das generierte JSON-Dokument eines
    {\classificationModel}s mit diesem semantisch übereinstimmt.

    \paragraph*{Konkreter Syntax}
    Für die konkrete Syntax einer Sprache schlagen \citet[Kapitel 4.7]{voelter:DslEngineering}
    verschiedene Bewertungskriterien vor: Writeability, Readability, Learnability, Effectivness.

    Der Nachteil der langen Sequenzen von Schlüsselwörtern wird beim Schreiben des Codes
    deutlich, da auch einfache Konzepte vergleichsweise viel Code erfordern.
    Allerdings relativiert sich dieser Nachteil durch die Unterstützung der Entwicklungsumgebung.
    Ein weiterer Nachteil beim Schreiben des Codes ist die fehlende Überprüfung der
    Selektoren, wodurch ihre Formulierung unnötig schwer und fehleranfällig wird.
    Das gilt vor allem bei komplexeren Selektoren.
    Die Zeichen zur Einklammerung von Selektoren sind auf vielen Tastaturlayouts nicht zu finden.
    Das ist nachteilig, weil man auf die Autovervollständigung der Entwicklungsumgebung angewiesen ist.
    Gleichzeitig erlauben sie aber die Nutzung vieler Zeichen, ohne sie speziell kennzeichnen zu müssen,
    wie z. B. Anführungszeichen.
    Im Vergleich zu anderen Sprachen ist das ein wichtiger Vorteil.

    Die vielen Schlüsselwörter sind aber auch ein Vorteil,
    da sich Programme nahezu wie ein Fließtext lesen lassen
    und ihre Verständlichkeit dadurch steigt.
    Das ist positiv zu bewerten.
    Beide Fallbeispiele zeigen aber auch Konstrukte,
    in denen die Lesbarkeit besser sein könnte.
    Zum Beispiel bei der Klasse \texttt{Portal},
    bei der nicht offensichtlich ist,
    dass sie selbst den Namen des Studienportals speichert,
    aber der Link in einem {\childFeature} gespeichert wird.
    Eingeschränkt wird die Lesbarkeit auch durch die Selektoren,
    die in den Fallbeispielen schnell eine hohe Komplexität angenommen haben,
    wie zum Beispiel bei der Klasse \texttt{Phone}.
    Bezüglich der Selektoren und ihrer Lesbarkeit ist aber nochmals
    positiv hervorzuheben, dass spezielle Zeichen nicht codiert werden müssen.

    Die Konzepte der Sprache sind leicht zu verstehen.
    Durch die gebotene Autovervollständigung und einige
    semantische Prüfungen ist das Erlernen der Sprache einfach.
    Hinzu kommen allerdings die benötigten Kenntnisse von CSS, XPath und regulären Ausdrücken,
    wodurch das Formulieren von Selektoren erschwert wird.

    Für eine fundierte Bewertung der Effektivität der Sprache,
    also der Frage, wie gut typische Probleme der Domäne ausgedrückt werden können,
    sind weitere Anwendungen auf noch mehr Webseiten notwendig.
    Bei den Fallbeispielen lassen sich
    viele Features und Klassen durch simple Selektoren erfassen,
    einige erfordern jedoch komplexe Selektoren.
    Dabei dürfte zumindest das bei der Klasse \texttt{Portal} angewandte Konstrukt,
    um ein HTML-Element sowohl als {\contentFeature} als auch als {\referenceFeature} zu nutzen,
    eine häufige Anforderung sein und sollte besser unterstützt werden.

    \paragraph*{Struktur}
    Beim Entwurf der \gls{wccdl} wurde sich bewusst gegen
    eine logische Strukturierung der Klassen in Namensräume,
    dafür aber für eine globale Sichtbarkeit aller Klassen
    in allen Quelldateien entschieden.
    Die hier vorgestellten Fallbeispiele profitieren von diesen Entscheidungen,
    da die Klassen leicht aufgeteilt und wiederverwendet werden konnten.
    Als Argument gegen Namensräume wurde die höhere Komplexität für
    Nicht-Programmierer genannt, was ein valider Punkt ist.
    Allerdings steigt die Einstiegshürde für diese Nutzergruppe bereits
    durch die Selektoren. Namensräume würden diese Hürde womöglich nicht noch weiter anheben.
    Es ist außerdem nicht auszuschließen, dass in größeren Projekten
    Namensräume und eine eingeschränkte Sichtbarkeit von Vorteil wären.
    Anders als die Beispielklassen \texttt{NewsDate} und \texttt{TeacherName},
    müssten Klassen mit abstrakten Namen dann nicht mit einem Präfix versehen werden.

    Insgesamt ist festzuhalten, dass die \gls{wccdl} viele Problemstellungen der Domäne
    mit einfachen Konzepten abdeckt, was wichtiger ist als jeden Sonderfall zu berücksichtigen
    und deshalb positiv zu bewerten ist.