\subsection{Eignung der domänenspezifischen Sprache}
    \label{section:discussionInterpretationLanguage}
    Die Analyse der entwickelten Sprache orientiert sich
    an den Design Dimensionen von \cite{voelter:DslEngineering}.

    \paragraph{Ausdrucksstärke}
    Die Ausdrucksstärke einer Sprache beschreibt,
    wie kompakt Programme einer Problemdomäne in dieser Sprache
    formuliert werden können.
    Zur Vergleichbarkeit zweier Sprachen bezüglich ihrer Ausdrucksstärke
    stellt \citet[Kapitel 4.1]{voelter:DslEngineering} die folgende
    Definition auf, wobei
    $P_D$ die Menge aller Programme der Domäne,
    $P_L$ die Menge aller in einer Sprache formulierbaren Programme und
    $p_L$ ein konkretes Programm in einer Sprache bezeichnet.

    \begin{quote}
        A language $L_1$ is \textit{more expressive in domain D}
        than a language $L_2 (L_1 {\prec}_D L_2)$,
        if for each $p \in P_D \cap {P_L}_1 \cap {P_L}_2, |{p_L}_1| < |{p_L}_2|$.
    \end{quote}

    Eine Aussage über die Ausdrucksstärke der \gls{wccdl} kann also
    nur relativ zu anderen Sprachen gemacht werden,
    weshalb zusätzlich zu den gesammelten Ergebnissen und
    Kapitel \ref{section:discussionComparisonLanguage}
    an dieser Stelle ein kurzer Vergleich stattfindet.

    Die obige Definition ist schwer anzuwenden,
    wenn eine Sprache textuell und die andere graphischer Natur ist.
    Viele Modellierungssprachen sind deshalb schwer mit der \gls{wccdl} zu vergleichen.
    Allerdings konnten zwei geeignete Kandidaten gefunden werden.
    Das erste ist PlantUML\footnote{vgl. \url{http://plantuml.com/}},
    ein Werkzeug und eine Sprache zur textuellen Beschreibung und anschließender
    graphischer Generierung von UML-Diagrammen.
    Die zweite Sprache wurde bei Recherchen auf
    GitHub\footnote{\url{https://github.com/sabrams/web-miner}} gefunden
    und verfolgt ein sehr ähnliches Ziel wie die \gls{wccdl}.
    Sie ist eine interne Ruby DSL.
    Auch wenn keine offizielle Arbeit oder eine Dokumentation dieser Sprache
    gefunden wurde und sie möglicherweise unvollständig ist,
    vermittelt sie trotzdem, wie das Vorhaben der \gls{wccdl}
    in einer anderen Sprache aussehen kann und ist deshalb für einen
    Vergleich geeignet.

    Als Vergleich dient ein Beispiel in
    Testcode\footnote{vgl. \url{https://github.com/sabrams/web-miner/blob/master/features/mine_according_to_command_file.feature}}
    dieser DSL, welches in Listing \ref{listing:discussionExpressivityExampleRubyDsl}
    zu sehen ist.

    \lstinputlisting[
        label=listing:discussionExpressivityExampleRubyDsl,
        caption=Beispiel zum Vergleich der Ausdrucksstärke in einer Ruby DSL
    ]{../resources/discussion/interpretation/language/expressivity-example.rb}

    Ein äquivalentes Konstrukt in PlantUML zeigt Listing
    \ref{listing:discussionExpressivityExamplePlantUml}.

    \lstinputlisting[
        label=listing:discussionExpressivityExamplePlantUml,
        caption=Beispiel zum Vergleich der Ausdrucksstärke in PlantUML
    ]{../resources/discussion/interpretation/language/expressivity-example.plantuml}

    Nicht zuletzt zeigt Listing \ref{listing:discussionExpressivityExampleWccs}
    wie das Beispiel in der \gls{wccdl} aussieht.

    \lstinputlisting[
        label=listing:discussionExpressivityExampleWccs,
        caption=Beispiel zum Vergleich der Ausdrucksstärke in WCCDL,
        language=wccdl,
        inputencoding=utf8/latin1,
        style=wccdl
    ]{../resources/discussion/interpretation/language/expressivity-example.wctd}

    Obwohl sich auch die \gls{wccdl} in diesem Beispiel kompakt ausdrückt,
    braucht sie mehr Code als die anderen beiden Sprachen.
    Auch ohne genaue Analyse sollte diese geringere Ausdrucksstärke
    allgemeingültig für die \gls{wccdl} sein,
    da sie anders als ihre Kontrahenten für jedes Feature eine definierte Klasse
    benötigt (im Beispiel "`Object"').
    Außerdem ist die Definition einer einzelnen Klasse,
    eines einzelnen Features und eines Selektors in etwa gleich lang oder länger.

    Der Grund sind die langen Schlüsselwortsequenzen,
    die zum Beispiel auch im Selektor der Seitenklasse "`Teacher"'
    im ersten Fallbeispiel erkennbar sind.
    An dieser Klasse und ihrem Pendant "`News"' aus dem zweiten Beispiel
    wird außerdem deutlich, dass der Sprache ein Vererbungskonzept fehlt,
    wodurch identische Features wiederholt werden müssen
    und sich die Ausdrucksstärke der Sprache verringert.

    Eine Kritik der Ausdrucksstärke ist berechtigt,
    allerdings muss beachtet werden,
    dass dieses Merkmal in Konflikt mit einigen nun folgenden
    Eigenschaften steht, die positiver bewertet werden.

    % TODO: Namespace?

    \paragraph{Abdeckung der Domäne}
    Die Abdeckung ihrer Domäne einer Sprache beschreibt
    \citet[Kapitel 4.2]{voelter:DslEngineering} als den
    prozentualen Anteil an allen Programmen der Domäne,
    die durch die Sprache formuliert werden können.
    Seine formale Definition lautet:

    \begin{quote}
        $C_D(L) = \frac{number of P_D programs expressable by L}{number of programs in domain D}$
    \end{quote}

    Eine volle Abdeckung ist demnach gegeben,
    wenn jedes Programm der Domäne in der Sprache ausgedrückt werden kann.

    Im Falle der \gls{wccdl} kann zumindest die Frage beantwortet werden,
    ob sie ihre Domäne voll abdeckt, das heißt, ob jede beliebige Struktur auf einer Webseite
    durch sie erfasst werden kann.
    Wie das zweite Fallbeispiel zeigte, ist das nicht der Fall.
    Gleichzeitig sollte nicht missachtet werden,
    dass auch einige nicht triviale Strukturen durch sie abgedeckt sind.
    Allerdings besteht auch eine gewisse Abhängigkeit zu den Fähigkeiten der Selektoren,
    was in Kapitel \ref{section:discussionInterpretationSelectors}
    nochmals aufgegriffen wird.

    \paragraph{Vollständigkeit}
    Eine Sprache ist nach \citet[Kapitel 4.5]{voelter:DslEngineering}
    vollständig wenn sie keine Code-Fragmente in niedrigeren Sprachen,
    Konfigurationsdateien oder ähnliches benötigt,
    um ausführbare Programme zu formulieren.
    Es kann argumentiert werden, dass die \gls{wccdl} zur Definition von
    Selektoren auf niedrigere Sprachen angewiesen ist,
    nämlich CSS, XPath und reguläre Ausdrücke.
    Zum einen, weil sie keine eigenen Konzepte für diese Sprachen enthält,
    sie weder einer syntaktischen noch semantischen Prüfungen unterzieht
    und sie während der Generierung nicht auswertet,
    sondern als simple Zeichenketten 1:1 übernimmt.
    All dies geschieht erst durch das Klassifizierungssystem.

    Die Unvollständigkeit der Sprache an dieser Stelle ist nachvollziehbar,
    weil warum das Rad neu erfinden?
    % TODO: Buch nochmal genau lesen, ggf. Zitat.

    %Die \gls{wccdl} ist vollständig, da zur Generierung keine Abhängigkeiten benötigt.
    %Die Werte von Selektoren sollen CSS, XPath oder reguläre Ausdrücke sein,
    %allerdings werden sie weder einer syntaktischen noch einer semantischen Prüfung unterzogen,
    %sondern als beliebige Zeichenketten behandelt.
    %Sie stellen also keine Verwendung von niedrigeren Sprachen dar.
    % TODO: Stimmt das so, oder könnte die spätere Verwendung sie doch dazu machen?
    % Vermutlich schon, aber das ist nicht schlimm, weil warum das Rad neu erfinden?

    \paragraph{Konkreter Syntax}
    Für den konkreten Syntax einer Sprache schlägt \citet[Kapitel 4.7]{voelter:DslEngineering}
    verschiedene Bewertungskriterien vor: Writeability, Readability, Learnability, Effectivness.

    Ein Nachteil der langen Sequenzen von Schlüsselwörtern wird beim Schreiben des Codes
    deutlich, allerdings relativiert sich dieser Nachteil durch die Unterstützung der Entwicklungsumgebung.
    Ein weiterer Nachteil beim Schreiben des Codes ist die fehlende Prüfung der
    Werte der Selektoren, wodurch ihre Formulierung unnötig schwer und fehleranfällig wird.
    Das gilt vor allem bei komplexeren Selektoren.

    Die vielen Schlüsselwörter sind aber auch ein Vorteil,
    da sich Programme nahezu wie ein Fließtext lesen lassen
    und ihre Verständlichkeit dadurch steigt.
    Da Quelltext in der Regel häufiger gelesen als geschrieben wird,
    ist das sehr positiv zu bewerten.
    Beide Fallbeispiele zeigen aber auch Konstrukte,
    in denen die Lesbarkeit besser sein könnte.
    Zum Beispiel bei der Klasse "`Portal"',
    bei der nicht offensichtlich ist,
    dass sie selbst den Namen des Studienportals speichert,
    der Link aber in einem Child Feature gespeichert wird.
    Eingeschränkt wird die Lesbarkeit aber auch durch die Selektoren,
    die in den Fallbeispielen schnell eine hohe Komplexität angenommen haben,
    wie zum Beispiel bei der Klasse "`Phone"'.

    Die Konzepte der Sprache sind leicht zu verstehen
    und durch die gebotene Autovervollständigung und einige
    semantische Prüfungen, ist das Erlernen der Sprache vergleichsweise einfach.
    Hinzu kommen allerdings die benötigten CSS-, XPath und RegEx-Kentnisse,
    wodurch das formulieren von Selektoren erschwert wird.

    Für eine fundierte Bewertung der Effektivität der Sprache,
    also der Frage, wie gut typische Probleme der Domäne ausgedrückt werden können,
    sind weitere Anwendungen auf noch mehr Seiten notwendig.
    Das durch die Fallbeispiele erhaltene Bild ist zweiseitig:
    Viele Features und Klassen lassen sich durch simple Selektoren erfassen,
    einige erfordern jedoch komplexe Selektoren.
    Dabei dürfte zumindest das bei der Klasse "`Portal"' angewandte Konstrukt,
    um ein HTML-Node sowohl als Content als auch als Reference Feature zu nutzen,
    eine häufige Anforderung sein und sollte besser unterstützt werden.

    \paragraph{Struktur}
    Beim Entwurf der \gls{wccdl} wurde sich bewusst gegen
    eine logische Strukturierung der Klassen in Namensräume,
    dafür aber für eine globale Sichtbarkeit aller Klassen
    in allen Quelldateien eines Projektes entschieden.
    Die hier vorgestellten Fallbeispiele profitieren von diesen Entscheidungen,
    da die Klassen leicht aufgeteilt und wiederverwendet werden konnten.
    Als Argument gegen Namensräume wurde die höhere Komplexität für
    Nicht-Programmierer genannt, was ein valider Punkt ist.
    Allerdings steigt die Einstiegshürde für diese Gruppe wie erläutert bereits
    durch die Selektoren und Namensräume würden diese Hürde womöglich nicht noch weiter anheben.

    Es ist außerdem nicht auszuschließen, dass in größeren Projekten
    Namensräume und eine eingeschränkte Sichtbarkeit von Vorteil wäre,
    damit, anders als die Beispielklassen "`NewsDate"' und "`TeacherName"',
    Klassen mit abstrakten Namen nicht mit einem Präfix versehen werden müssen,
    was durchaus aus schlechter Stil angesehen werden kann.

    Insgesamt ist festzuhalten, dass die \gls{wccdl} viele Problemstellungen der Domäne
    mit einfachen Konzepten abdeckt, was wichtiger ist als jeden Sonderfall zu berücksichtigen
    und deshalb positiv zu bewerten ist.