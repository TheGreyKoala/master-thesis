\section{Interpretation und Bewertung}
    % TODO: Wird erwähnt, dass <<>> kein Problem ist dank IDE FEatures?
    % Und, dass ohne escaping die selektoren viel leichter zu lesen sind.
    \label{section:findingsInterpretation}
    Beide Fallbeispiele haben die prinzipielle Funktionsfähigkeit des \gls{wccs} bewiesen,
    da sowohl die domänenspezifische Sprache als auch der Klassifizierungsansatz
    erfolgreich angewandt werden konnten.
    Aufgrund der hohen Zahl an Lehrenden und Betreuenden,
    verdeutlicht das erste Beispielt außerdem,
    dass eine Automatisierung der Klassifizierung sinnvoll ist.

    Die Ergebnisse lassen aber selbstverständlich noch wietergehende Schlüsse zu,
    von denen einige in den folgenden Unterkapiteln diskutiert werden.

    \subsection{Eignung der domänenspezifischen Sprache}
    % TODO: Vorteile aus SolutionConcept/DSL.tex hier hin? Oder zu Grundlagen über DSLs bzw. im Solution Details Abschnitt der DSL?
    \label{section:discussionInterpretationLanguage}
    Die Analyse der \gls{wccdl} orientiert sich
    an den Design Dimensionen von \cite{voelter:DslEngineering}.

    \paragraph{Ausdrucksstärke}
    Die Ausdrucksstärke einer Sprache beschreibt,
    wie kompakt Programme einer Problemdomäne in dieser Sprache
    formuliert werden können.
    Zur Vergleichbarkeit zweier Sprachen bezüglich ihrer Ausdrucksstärke
    stellen \citet[Kapitel 4.1]{voelter:DslEngineering} die folgende
    Definition auf, wobei
    $P_D$ die Menge aller Programme der Domäne,
    $P_L$ die Menge aller in einer Sprache formulierbaren Programme und
    $p_L$ ein konkretes Programm in einer Sprache bezeichnet.

    \begin{quote}
        A language $L_1$ is \textit{more expressive in domain D}
        than a language $L_2 (L_1 {\prec}_D L_2)$,
        if for each $p \in P_D \cap {P_L}_1 \cap {P_L}_2, |{p_L}_1| < |{p_L}_2|$.
    \end{quote}

    Eine Aussage über die Ausdrucksstärke der \gls{wccdl} kann anhand dieser Formel
    nur relativ zu anderen Sprachen gemacht werden,
    weshalb zusätzlich zu den gesammelten Ergebnissen und
    Kapitel \ref{section:discussionComparisonLanguage}
    an dieser Stelle ein kurzer Vergleich mit zwei anderen Sprachen stattfindet.
    Die obige Definition ist schwer anzuwenden,
    wenn eine Sprache textuell und die andere graphischer Natur ist.
    Viele Modellierungssprachen sind deshalb schwer mit der \gls{wccdl} zu vergleichen.
    Allerdings konnten zwei geeignete Kandidaten gefunden werden.
    Der erste ist PlantUML\footnote{vgl. \url{http://plantuml.com/}},
    ein Werkzeug und eine Sprache zur textuellen Beschreibung und anschließender
    Generierung graphischer UML-Diagramme.
    Die zweite Sprache wurde bei Recherchen auf
    GitHub\footnote{vgl. \url{https://github.com/sabrams/web-miner}} gefunden
    und verfolgt ein sehr ähnliches Ziel wie die \gls{wccdl}.
    Sie ist eine interne Ruby DSL, die trotz ihrer etwaigen Unvollständigkeit sowie der fehlenden
    offiziellen wissenschaftlichen Arbeit oder Dokumentation
    für einen Vergleich mit der \gls{wccdl} geeignet ist.
    Sie vermittelt nämlich, wie das Vorhaben der \gls{wccdl}
    in einer anderen Sprache aussehen kann.
    Als Vergleich dient ein Beispiel im
    Testcode\footnote{vgl. \url{https://github.com/sabrams/web-miner/blob/master/features/mine_according_to_command_file.feature}}
    dieser DSL, welches in Listing \ref{listing:discussionExpressivityExampleRubyDsl}
    zu sehen ist.

    \lstinputlisting[
        label=listing:discussionExpressivityExampleRubyDsl,
        caption=Programm in einer Ruby DSL zum Vergleich der Ausdrucksstärke,
        style=pseudo,
        language=Ruby
    ]{../resources/discussion/interpretation/language/expressivity-example.rb}

    Ein äquivalentes Konstrukt in PlantUML zeigt Listing
    \ref{listing:discussionExpressivityExamplePlantUml}.

    \lstinputlisting[
        label=listing:discussionExpressivityExamplePlantUml,
        caption=Programm in PlantUML zum Vergleich der Ausdrucksstärke,
        style=plantuml
    ]{../resources/discussion/interpretation/language/expressivity-example.plantuml}

    Nicht zuletzt zeigt Listing \ref{listing:discussionExpressivityExampleWccs},
    wie das Beispiel in der \gls{wccdl} aussieht.

    \lstinputlisting[
        label=listing:discussionExpressivityExampleWccs,
        caption=Programm in der \acrshort{wccdl} zum Vergleich der Ausdrucksst\"arke,
        inputencoding=utf8/latin1,
        style=wccdl
    ]{../resources/discussion/interpretation/language/expressivity-example.wctd}

    Die \gls{wccdl} braucht in diesem Beispiel kompakt ausdrückt,
    braucht sie mehr Code als die anderen beiden Sprachen.
    Auch ohne genaue Analyse sollte diese geringere Ausdrucksstärke
    allgemeingültig für die \gls{wccdl} sein.
    Anders als ihre Kontrahenten benötigt sie z. B. für jedes Feature eine definierte Klasse
    (im Beispiel \texttt{Object}).
    Außerdem ist die Definition einer einzelnen Klasse,
    eines einzelnen Features oder eines Selektors in etwa gleich lang oder länger,
    als in der Vergleichssprachen.
    Der Grund sind die langen Schlüsselwortsequenzen,
    die auch im Selektor der Seitenklasse \texttt{Teacher}
    erkennbar sind.
    An dieser Klasse und ihrem Pendant \texttt{News}
    wird außerdem deutlich, dass der Sprache ein Vererbungskonzept fehlt.
    Identische Features müssen nämlich wiederholt werden,
    wodurch sich die Ausdrucksstärke der Sprache verringert.
    Eine Kritik der Ausdrucksstärke der \gls{wccdl} ist berechtigt,
    allerdings muss beachtet werden,
    dass dieses Merkmal in Konflikt mit einigen nun folgenden
    Eigenschaften steht, die positiver bewertet werden.

    % TODO: Namespace?

    \paragraph{Abdeckung der Domäne}
    Die Abdeckung der Domäne einer Sprache beschreiben
    \citet[Kapitel 4.2]{voelter:DslEngineering} als den
    prozentualen Anteil an allen Programmen der Domäne,
    die durch die Sprache formuliert werden können.
    Die formale Definition lautet

    \begin{quote}
        $C_D(L) = \frac{number\ of\ P_D\ programs\ expressable\ by\ L}{number\ of\ programs\ in\ domain\ D}$.
    \end{quote}

    Eine volle Abdeckung ist demnach gegeben,
    wenn jedes Programm der Domäne in der Sprache ausgedrückt werden kann.
    Im Falle der \gls{wccdl} kann zumindest die Frage beantwortet werden,
    ob sie ihre Domäne voll abdeckt, das heißt, ob jede beliebige Struktur auf einer Webseite
    durch sie erfasst werden kann.
    Wie das zweite Fallbeispiel zeigte, ist das nicht der Fall,
    da kein allgemeiner Selektor gefunden werden konnte,
    der die Absätze einer Nachricht korrekt erfasst.
    Spezielle Klassen pro Meldung können dieses Problem zwar lösen,
    sind aber nicht im eigentlichen Sinne des \glspl{wccs}.
    Es wurden aber auch einige nicht triviale Strukturen durch die Sprache abgedeckt.
    Selbstverständlich spielen bei dieser Frage auch die Fähigkeiten der Selektoren
    eine Rolle\footnote{vgl. Kapitel \ref{section:discussionInterpretationSelectors}}.

    \paragraph{Vollständigkeit}
    Eine Sprache ist nach \citet[Kapitel 4.5]{voelter:DslEngineering}
    vollständig, wenn ihr Generierungsergebnis keine manuell geschriebenen Codefragmente in niedrigeren Sprachen,
    Konfigurationsdateien o. Ä. benötigt,
    um die gleiche Semantik zu besitzen, wie das Programm in der \gls{dsl}.
    Formaler beschreiben sie diese Eigenschaft wie folgt:

    \begin{quote}
        Let us introduce a function $G$ ("code generator") that transforms
        a program $p$ in $L_D$ to a program $q$ in $L_{D-1}$.
        For a complete language, $p$ and $q$ have the same semantics, i.e.
        $OB(p) == OB(G(p)) == OB(q)$ [...]. For incomplete languages
        where $OB(G(p)) \subset OB(p)$ we have to write additional
        code in $L_{D-1}$ to obtain a program in $D_{-1}$ that has the same semantics
        as intended by the original program in $L_D$.
    \end{quote}

    Die \gls{wccdl} ist vollständig, da das generierte JSON-Dokument eines
    {\classificationModel}s mit diesem semantisch übereinstimmt.

    \paragraph{Konkreter Syntax}
    Für den konkreten Syntax einer Sprache schlagen \citet[Kapitel 4.7]{voelter:DslEngineering}
    verschiedene Bewertungskriterien vor: Writeability, Readability, Learnability, Effectivness.

    Der Nachteil der langen Sequenzen von Schlüsselwörtern wird beim Schreiben des Codes
    deutlich, da auch einfache Konzepte vergleichsweise viel Code erfordern.
    Allerdings relativiert sich dieser Nachteil durch die Unterstützung der Entwicklungsumgebung.
    Ein weiterer Nachteil beim Schreiben des Codes ist die fehlende Überprüfung der
    Selektoren, wodurch ihre Formulierung unnötig schwer und fehleranfällig wird.
    Das gilt vor allem bei komplexeren Selektoren.
    Die Zeichen zur Einklammerung von Selektoren sind auf vielen Tastaturlayouts nicht zu finden.
    Das ist nachteilig, weil man auf die Autovervollständigung der Entwicklungsumgebung angewiesen ist.
    Gleichzeitig erlauben sie aber die Nutzung vieler Zeichen, ohne sie speziell kennzeichnen zu müssen,
    wie z. B. Anführungszeichen.
    Im Vergleich zu anderen Sprachen ist das ein wichtiger Vorteil.

    Die vielen Schlüsselwörter sind aber auch ein Vorteil,
    da sich Programme nahezu wie ein Fließtext lesen lassen
    und ihre Verständlichkeit dadurch steigt.
    Das ist positiv zu bewerten.
    Beide Fallbeispiele zeigen aber auch Konstrukte,
    in denen die Lesbarkeit besser sein könnte.
    Zum Beispiel bei der Klasse \texttt{Portal},
    bei der nicht offensichtlich ist,
    dass sie selbst den Namen des Studienportals speichert,
    aber der Link in einem {\childFeature} gespeichert wird.
    Eingeschränkt wird die Lesbarkeit aber durch die Selektoren,
    die in den Fallbeispielen schnell eine hohe Komplexität angenommen haben,
    wie zum Beispiel bei der Klasse \texttt{Phone}.
    Bezüglich der Selektoren und ihrer Lesbarkeit ist aber nochmals
    positiv hervorzuheben, dass spezielle Zeichen nicht kodiert werden müssen.

    Die Konzepte der Sprache sind leicht zu verstehen.
    Durch die gebotene Autovervollständigung und einige
    semantische Prüfungen, ist das Erlernen der Sprache einfach.
    Hinzu kommen allerdings die benötigten Kenntnisse von CSS, XPath und regulären Ausdrücken,
    wodurch das formulieren von Selektoren erschwert wird.

    Für eine fundierte Bewertung der Effektivität der Sprache,
    also der Frage, wie gut typische Probleme der Domäne ausgedrückt werden können,
    sind weitere Anwendungen auf noch mehr Webseiten notwendig.
    Das durch die Fallbeispiele erhaltene Bild ist zweiseitig:
    Viele Features und Klassen lassen sich durch simple Selektoren erfassen,
    einige erfordern jedoch komplexe Selektoren.
    Dabei dürfte zumindest das bei der Klasse \texttt{Portal} angewandte Konstrukt,
    um ein HTML-Element sowohl als {\contentFeature} als auch als {\referenceFeature} zu nutzen,
    eine häufige Anforderung sein und sollte besser unterstützt werden.

    \paragraph{Struktur}
    Beim Entwurf der \gls{wccdl} wurde sich bewusst gegen
    eine logische Strukturierung der Klassen in Namensräume,
    dafür aber für eine globale Sichtbarkeit aller Klassen
    in allen Quelldateien entschieden.
    Die hier vorgestellten Fallbeispiele profitieren von diesen Entscheidungen,
    da die Klassen leicht aufgeteilt und wiederverwendet werden konnten.
    Als Argument gegen Namensräume wurde die höhere Komplexität für
    Nicht-Programmierer genannt, was ein valider Punkt ist.
    Allerdings steigt die Einstiegshürde für diese Nutzergruppe bereits
    durch die Selektoren. Namensräume würden diese Hürde womöglich nicht noch weiter anheben.
    Es ist außerdem nicht auszuschließen, dass in größeren Projekten
    Namensräume und eine eingeschränkte Sichtbarkeit von Vorteil wären.
    Anders als die Beispielklassen \texttt{NewsDate} und \texttt{TeacherName},
    müssten Klassen mit abstrakten Namen dann nicht mit einem Präfix versehen werden.

    Insgesamt ist festzuhalten, dass die \gls{wccdl} viele Problemstellungen der Domäne
    mit einfachen Konzepten abdeckt, was wichtiger ist als jeden Sonderfall zu berücksichtigen
    und deshalb positiv zu bewerten ist.
    \subsection{Übertragbarkeit eines {\classificationModel}s}
    Ein Ziel der Fallbeispiele war herauszufinden,
    wie gut sich ein anhand einer einzelnen konkreten Seite
    erstelltes {\classificationModel} auf andere Seiten anwenden lässt,
    die augenscheinlich konzeptionell identisch aufgebaut sind.
    Je besser diese Übertragbarkeit ist,
    desto höher ist der Nutzen der Automatisierung.

    Eine erste naheliegende Erkenntnis ist,
    das Klassen iterativ zu definieren sind,
    was schon bei der Entwicklung des {\classificationModel}s
    für das Portal \gls{babw} deutlich wurde.
    Ausnahmen innerhalb einer Webseite werden oft erst
    bei der Prüfung des Klassifikation entdeckt,
    woraufhin die Ausnahme analysiert und das Modell
    entsprechend angepasst werden muss.
    Ein Beispiel ist die Faxnummer, die nur bei einem Mitarbeiter auftritt,
    deren Erfassung aber auch eine Anpassung des Selektors der Telefonnummer
    erforderlich machte.

    Diese Erkenntnis wiegt natürlich noch schwerer,
    wenn eine Menge an Klassen auf viele verschiedene Seiten angewandt werden soll.
    Wie das erste Fallbeispiel zeigt,
    können auch vermeintlich sehr ähnliche Seiten deutliche Unterschiede
    im konzeptionellen Modell und in der HTML-Struktur haben.
    Diese können nicht alle vorausgedacht werden.
    Es müssen demnach mehrere Webseiten bei der Entwicklung der Klassendefinitionen
    betrachtet werden.
    Andernfalls werden Informationen nicht klassifiziert oder es kommt zu unerwarteten Ergebnissen.
    Beispiele sind der Verweis auf die Detailseite eines Mitarbeiters
    und die doppelt klassifizierten Mitarbeiter im Portal \gls{bapvs}.
    Deutlich wurde aber auch, dass viele und zu große Unterschiede
    die Auflösung eines Konfliktes sehr komplex machen.

    Diese Überlegungen lassen die Schlussfolgerung zu,
    dass die Erstellung eines {\classificationModel}s,
    welches auf viele Webseiten passt,
    aufgrund der großen Unterschiede eine komplexe Aufgabe ist.
    Eine kritische, aber berechtigte Frage ist deshalb,
    wie groß der Nutzen durch die Automatisierung tatsächlich ist.
    Vor allem dann, wenn zur Erfassung einiger Inhalte Anpassungen an der Seite notwendig sind,
    was im nächsten Kapitel behandelt wird.
    Dem ist entgegenzusetzen, dass die Effektivität durch ein anderes Vorgehen beim
    Schreiben der Klassendefinitionen gesteigert werden kann.
    Eine solche Alternative zum Versuch alle Seiten durch ein Modell abzudecken
    besteht aus mehreren Schritten.
    Zunächst wird ein Modell für eine Webseite geschrieben,
    sodass es diese korrekt und vollständig klassifiziert.
    Dieser Schritt ist einfach und bringt zweifelsfrei Vorteile,
    was bei der initialen Klassifizierung der Mitarbeiter des Portals \texttt{BaBw} ersichtlich wurde.
    Dann sollte das {\classificationModel} auf eine weitere Seite angewandt werden und
    das Ergebnis geprüft und angepasst werden,
    bis diese zweite Seite korrekt klassifiziert wird.
    Die Klassifizierung findet in diesem Schritt ausschließlich auf der zweiten Seite statt,
    sodass die erste bereits korrekt klassifizierte Seite unberührt bleibt.
    Dieser zweite Schritt wird für alle Seiten wiederholt.
    Dieses Vorgehen erlaubt die Wiederverwendung global gültiger Klassendefinitionen
    und erfordert nur die Anpassung einzelner Klassen an die jeweiligen Anforderungen.
    Insgesamt dürfte der Prozess dadurch deutlich effektiver werden.
    Es sollte außerdem nicht die Möglichkeit missachtet werden,
    bewusst einzelne falsche Klassifizierungen in Kauf zu nehmen
    und diese über das {\annotatorPlugin} im Nachhinein zu korrigieren.

    Nicht erfasste Features und unzutreffende Selektoren können
    aber auch auf Fehler in der Webseite hinweisen.
    Wie z. B. die E-Mail-Adresse ohne vorangestelltes \texttt{mailto}
    im Portal \gls{bscpsy}.
    Das \gls{wccs} fungiert indirekt also als ein Werkzeug
    zur Validierung eines individuell definierbaren Schemas für Webseiten.
    \subsection{Mächtigkeit der Selektoren}
    \label{section:discussionInterpretationSelectors}
    Zusammen mit der Schachtelung von Features und dem
    damit verbundenen eingeschränkten Kontexts bei der
    Auswertung von Selektoren,
    spielen deren Fähigkeiten eine zentrale Rolle bei den
    Möglichkeiten der Klassifizierung, die das \gls{wccs} bietet.
    Betrachtet man die {\classificationModel}e der beiden Fallbeispiele wird klar,
    dass ohne den {\xpathSelector} die Möglichkeiten weitaus eingeschränkter wären.
    Die Klasse \texttt{Portal}, durch die ein Feature das Kontextelement
    als {\referenceFeature} \texttt{homepage} klassifiziert,
    wäre beispielsweise weder mit dem {\cssSelector}
    noch mit dem {\urlSelector} möglich.
    Beide können ausschließlich untergeordnete Elemente erfassen.
    Die Selektoren und die Art ihrer Verwendung hat allerdings auch Grenzen.
    Ein Beispiel sind die Mitarbeiter, deren Namen laut Klassifikation
    "`Prof."' bzw. "`Dr."' sind.
    Ein ähnliches Problem gibt es im zweiten Beispiel bei den Absätzen einer Meldung,
    von denen es unterschiedlich viele geben kann.
    In beiden Fällen lässt sich argumentieren,
    dass mehrere HTML-Elemente zusammen eine Einheit darstellen
    und dass ihr textueller Inhalt deshalb
    in einem skalaren Feature gespeichert werden sollten.
    Anders also als {\collectionFeature}s,
    bei denen jedes Element für sich eine eigenständige Einheit darstellt.
    Die beschriebene Anforderung lässt sich bisher nur umsetzen,
    wenn die übergeordnete Einheit der Elemente durch ein gemeinsames
    Vaterelement dargestellt wird, welches nur genau diese Element enthält.
    Für das erste Beispiele hieße die Anforderung, dass der Selektor des Namens eines Mitarbeiters
    alle relevanten Elemente erfasst und
    das \gls{wccs} die textuellen Werte dieser Elemente
    in einem skalaren Feature speichert.
    Der eindeutige Selektor dieses Features würde alle Elemente umfassen,
    was prinzipiell schon jetzt möglich ist,
    da der Selektor Start- und Endelemente unterscheidet.

    Das zweite Fallbeispiel zeigt eine weitere Grenze des Systems,
    da es nicht möglich ist einen allgemeinen Selektor zu formulieren,
    der die Absätze pro Nachricht korrekt
    erfasst.
    Allgemein ausgedrückt ist eine sinnvolle Klassifizierung nicht möglich,
    wenn ein Element eines {\collectionFeature}s selbst ein {\collectionFeature} besitzt
    und weder das Parent- noch das {\childFeature} alleine in einem Element gekapselt ist.
    Diese Einschränkung ist durch die verwendeten Selektortypen begründet.
    Eine Erweiterung des Systems durch andere Selektortypen, kann diese Lücke
    füllen\footnote{vgl. Kapitel \ref{section:endingOutlook}}.

    Die Entscheidung als Ergebnis eines {\xpathSelector}s auch reinen Text und nicht nur
    HTML-Elemente zu akzeptieren, bringt deutliche Vorteile für das \gls{wccs}.
    Es erlaubt z. B. die Telefonnummer eines Mitarbeiters zu selektieren,
    was sonst nicht möglich wäre, da sie nicht alleine in einem Element steht.
    Allerdings darf in diesem Fall die Bestimmung der Versatzangabe des Startelementes nicht auf
    Basis der Eigenschaft \texttt{innerText} geschehen.
    Wie bei der falsch annotierten Telefonnummer im Portal \gls{bapvs} deutlich
    wird,
    muss der durch XPath ermittelte Text nicht in derselben Form im \texttt{innerText} eines Elementes auftauchen.
    Der zu durchsuchende Text muss deshalb ebenfalls über XPath ermittelt werden.
    Für Drittsystemen wird dadurch die Auswertung eines eindeutigen Selektors allerdings komplexer.

    \subsection{Eignung einer Graphdatenbank}
    Die Nutzung einer Graphdatenbank für das \gls{wccs} hat mehrere Beweggründe.
    Dazu zählen die einfache, natürliche und schemalose Datenmodellierung
    und die Möglichkeit Beziehungen sehr einfach auszuwerten.
    Die Wiederverwendung von Knoten für mehrere Klassifikationen
    bietet außerdem die Möglichkeit, weiterführende Analysen auf den
    klassifizierten Inhalten durchzuführen.
    Dieses Kapitel bewertet die tatsächliche Eignung einer Graphdatenbank,
    des verwendeten Datenmodells sowie des Algorithmus
    zum Schreiben von Klassifikationen, der das Teilen von Knoten ermöglicht.

    \paragraph*{Referenzen zwischen Webseiten}
    Das zweite Fallbeispiel hat gezeigt,
    wie Referenzen zwischen Webseiten auf sehr einfache Art und Weise
    Reihenfolgen und Navigationspfade (der Nachrichtenseiten) explizit machen.
    Eine Auswertung ist trivial, da lediglich ein- und ausgehenden Kanten gefolgt werden muss.
    Andere Datenbankmodelle hätten komplexere Konstrukte
    zur Speicherung und zur Auswertung erfordert.

    \paragraph*{Wiederverwendung von Text- und Resource-Knoten}
    Betrachtet man Tabelle \ref{table:findingsTeachersFiguresNodesByLabel}, wird deutlich,
    dass \texttt{Text}- und \texttt{Resource}-Knoten verhältnismäßig am meisten von der
    Wiederverwendbarkeit Gebrauch machen.
    Das ist nicht verwunderlich, da sie jeweils genau einen Wert speichern,
    der sie identifiziert.
    Gleichzeitig besitzen sie keine ausgehenden Kanten,
    die seitenspezifische Informationen enthalten.
    Wie Tabelle \ref{table:findingsTeachersFiguresSharedNodes} belegt,
    steigt ihre Wiederverwendung bei mehreren Klassifikationen in einer Datenbank,
    was eine logische Konsequenz der größeren Datenmenge ist.
    Für \texttt{Resource}-Knoten ist das aber nicht sofort ersichtlich, da
    die Summe der mehrfach verwendeten \texttt{Resource}-Knoten von Lehrgebieten
    im Falle einzelner Datenbanken höher ist, als im Fall einer gemeinsamen Datenbank (36 vs. 30).
    Gleichzeitig ist die Zahl der geteilten \texttt{SubjectArea}-Knoten aber höher,
    die jeder einen \texttt{Resource}-Knoten eines Lehrgebietes referenzieren.
    Es wird also ein größerer Teil des Graphen wiederverwendet.
    Außerdem enthielten die einzelnen Datenbanken identische \texttt{Resource}-Knoten,
    die in der gemeinsamen natürlich zusammengefasst werden konnten.

    \paragraph*{Entbehrlichkeit von Text-Knoten}
    Innerhalb einer Datenbank macht es aus semantischen Gründen keinen Sinn
    \texttt{Resource}-Knoten zu duplizieren,
    da sie eine Entität der Domäne darstellen.
    {\resources} existieren genau ein mal, was die Datenbank widerspiegeln sollte
    und außerdem die Möglichkeiten eines Graphen besser ausschöpft.
    Nicht so eindeutig ist dies allerdings bei \texttt{Text}-Knoten.
    Aus Tabelle \ref{table:findingsTeachersFiguresSharedNodes} geht hervor,
    dass im ersten Fallbeispiel niemals ein \texttt{Text}-Knoten geteilt wurde.
    Stattdessen konnte immer der zugehörige \texttt{Content}-Knoten geteilt werden.
    Deshalb ist die Zahl beider Knotentypen gesunken und die der geteilten
    \texttt{Content}-Knoten dafür gestiegen.
    Tabelle \ref{table:findingsNewsFiguresSharedNodes}
    zeigt eine andere Situation beim zweiten Fallbeispiel.
    Dort werden zwei \texttt{Text}-Knoten geteilt,
    weil es mehrere Nachrichten mit derselben Überschrift gibt.
    Sie referenzieren aber unterschiedliche Detailseiten,
    weshalb die Nachrichten unterschiedliche \texttt{Content}-Knoten besitzen.
    Der tatsächliche Nutzen der \texttt{Text}-Knoten ist in beiden Beispielen trotzdem sehr gering.
    Der textuelle Wert eines {\contentFeature}s ist deshalb im \texttt{Content}-Knoten selbst besser aufgehoben.
    Für die Schnittstellen des \glspl{wccs} ändert sich durch eine entsprechende Anpassung nichts,
    allerdings vereinfacht sich das Datenmodell der Datenbank sowie
    die generierten Datenbankanweisungen.

    \paragraph*{Hierarchieebene wiederverwendeter Teilgraphen}
    Bezüglich der Wiederverwendung von Teilgraphen ist außerdem ersichtlich,
    dass sie häufig erst auf einer tiefen Ebene des Graphen stattfindet.
    \texttt{Content}-Knoten der Klasse \texttt{SubjectAreaName} werden bspw.
    sehr viel häufiger von mehreren Klassifikationen verwendet als solche der
    Klasse \texttt{Teacher}.
    Betrachtet man die klassifizierten Webseiten,
    ist eine entgegengesetzte Erwartung gerechtfertigt.
    Der Kopfbereich wiederholt sich z. B. auf allen Seiten
    und mehrere Mitarbeiter werden auf unterschiedlichen Webseiten identisch aufgeführt.
    Trotzdem wird bei solchen Mitarbeitern und dem Kopfbereich
    nicht der \texttt{Teacher}- bzw. \texttt{Header}-Knoten geteilt,
    sondern nur ihre Unterknoten.
    Dafür gibt es zwei Gründe.
    Zum einen müssen Features inklusive ihrer {\childFeature}s inhaltlich
    exakt übereinstimmen, damit ihr \texttt{Content}-Knoten mehrfach verwendet werden kann.
    Die kleinste Abweichung macht dies unmöglich,
    da die Inhalte sich letztendlich unterscheiden und nur Teilaspekte übereinstimmen.
    Das muss die Datenbank widerspiegeln.
    Im Falle des Kopfbereiches im ersten Beispiel ist die \gls{url} des Logos
    auf jeder Seite unterschiedlich.
    Das ist im zweiten Beispiel nicht der Fall,
    da alle Seiten zu einem Portal in {\wordpress} gehören
    und deshalb das gleiche Bild referenzieren.
    Deshalb existiert in der Datenbank nur ein \texttt{Header}-Knoten,
    den alle Klassifikationen referenzieren.
    Die zweite Ursache sind unterschiedliche Positionen identischer Inhalte
    auf verschiedenen Webseiten.
    Diese Positionen werden in Form des eindeutigen Selektors in der eingehenden
    Kante eines \texttt{Content}-Knotens gespeichert\footnote{vgl. Kapitel \ref{section:solutionDetailsPersistenceDataModel}}.
    Diese Knoten sind prinzipiell also unabhängig von der konkreten Position
    von mehreren Klassifikationen gleichzeitig nutzbar.
    Dies ändert sich allerdings, sobald ein Inhalt {\childFeature}s besitzt.
    Die ausgehenden Kanten zu diesen Features speichern ihre \textit{absoluten} Selektoren.
    Befindet sich ein Mitarbeiter auf zwei Seiten also nicht an exakt der gleichen Position,
    sind die Selektoren seines Namens, seines Lehrgebietes etc. unterschiedlich.
    Der \texttt{Teacher}-Knoten kann dann nicht von beiden Klassifikationen referenziert
    werden.
    
    \paragraph*{Semantischer Inhalt und physische Struktur einer Webseite}
    Genau betrachtet spiegelt die Datenbank lediglich die Struktur der Webseiten exakt wider.
    Trotzdem lässt sich argumentieren, dass es fachlich konsistenter wäre,
    wenn die \texttt{Teacher}-Knoten im oben beschriebenen Fall geteilt werden würden.
    Ein Beispiel verdeutlicht das:
    Schon jetzt lässt sich die Frage beantworten,
    welche Mitarbeiter in einem gewissen Lehrgebiet arbeiten.
    Dazu muss lediglich den eingehenden Kanten eines
    \texttt{SubjectAreaName}-Knotens über alle \texttt{SubjectArea}-Knoten
    bis zu \texttt{Teacher}-Knoten gefolgt werden.
    Aus fachlicher Sicht ist es aber inkonsistent,
    dass man diese Suche nicht beim \texttt{SubjectArea}-Knoten
    starten kann, weil mehrere dieser Knoten denselben
    \texttt{SubjectAreaName}-Knoten verwenden.
    Aus fachlicher Sicht existieren laut Datenbank also mehrere Lehrgebiete,
    die denselben Namen tragen.
    Hier wird ein Konflikt zwischen dem akkuraten Widerspiegeln der Strukturen der Webseiten
    und dem semantischen Inhalt deutlich.

    \paragraph*{Alternativen}
    Ein erster Schritt zur Auflösung dieses Konfliktes ist, anstatt absolute Selektoren
    lediglich zum {\parentFeature} relative Selektoren in den Kanten zu speichern.
    Allerdings hilft dies nicht, wenn die HTML-Struktur sich unterscheidet,
    weil dann auch die relative Position eines Elementes anders ist.
    Außerdem hätte dies zur Folge, dass zur Ermittlung der eindeutigen Position
    eines Features, der gesamte Vaterpfad bis zur Seite abgelaufen werden muss.
    Eine weitere Alternative zum aktuellen Vorgehen,
    die auch diesen Nachteil umgeht,
    ist die Speicherung der absoluten Selektoren in Kanten,
    die vom \texttt{Page}-Knoten direkt zu \texttt{Content}-Knoten führen.
    Die Teilbarkeit von Content Knoten hinge dann nur noch von den Inhalten und ihrer Klasse ab,
    da die Position komplett unabhängig gespeichert wird.

    \paragraph*{Änderung von Klassifikationen}
    Durch den entwickelten Algorithmus zum Schreiben von Klassifikationen
    wurde das Teilen von Knoten ermöglicht.
    Eine nachteilige Folge ist allerdings,
    dass Änderungen immer über diesen Algorithmus eingearbeitet werden müssen,
    weil sonst Inkonsistenzen riskiert werden.
    Falls die Analysemöglichkeiten keine praktische Relevanz haben,
    ist deshalb ein anderer Datenbanktyp, wie zum Beispiel ein einfacher Document Store,
    in Betracht zu ziehen.

    Insgesamt scheint eine Graphdatenbank trotz der genannten Herausforderungen geeignet
    für den Anwendungsfall des \glspl{wccs}.

    \subsection{Annotationen}
