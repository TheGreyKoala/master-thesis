\section{Interpretation}
    \label{section:findingsInterpretation}
    Beide Fallbeispiele haben die prinzipielle Funktionsfähigkeit des \gls{wccs} bewiesen,
    da sowohl die domänenspezifische Sprache als auch der Klassifizierungsansatz
    erfolgreich angewandt werden konnten.
    Aufgrund der hohen Zahl an Lehrenden und Betreuenden,
    verdeutlicht das erste Beispielt außerdem,
    dass eine Automatisierung der Klassifizierung sinnvoll ist.

    Die Ergebnisse lassen aber selbstverständlich noch wietergehende Schlüsse zu,
    von denen einige in den folgenden Unterkapiteln diskutiert werden.

    \subsection{Eignung der domänenspezifischen Sprache}
    % TODO: Vorteile aus SolutionConcept/DSL.tex hier hin? Oder zu Grundlagen über DSLs bzw. im Solution Details Abschnitt der DSL?
    \label{section:discussionInterpretationLanguage}
    Die Analyse der \gls{wccdl} orientiert sich
    an den Design Dimensionen von \cite{voelter:DslEngineering}.

    \paragraph{Ausdrucksstärke}
    Die Ausdrucksstärke einer Sprache beschreibt,
    wie kompakt Programme einer Problemdomäne in dieser Sprache
    formuliert werden können.
    Zur Vergleichbarkeit zweier Sprachen bezüglich ihrer Ausdrucksstärke
    stellen \citet[Kapitel 4.1]{voelter:DslEngineering} die folgende
    Definition auf, wobei
    $P_D$ die Menge aller Programme der Domäne,
    $P_L$ die Menge aller in einer Sprache formulierbaren Programme und
    $p_L$ ein konkretes Programm in einer Sprache bezeichnet.

    \begin{quote}
        A language $L_1$ is \textit{more expressive in domain D}
        than a language $L_2 (L_1 {\prec}_D L_2)$,
        if for each $p \in P_D \cap {P_L}_1 \cap {P_L}_2, |{p_L}_1| < |{p_L}_2|$.
    \end{quote}

    Eine Aussage über die Ausdrucksstärke der \gls{wccdl} kann also
    nur relativ zu anderen Sprachen gemacht werden,
    weshalb zusätzlich zu den gesammelten Ergebnissen und
    Kapitel \ref{section:discussionComparisonLanguage}
    an dieser Stelle ein kurzer Vergleich stattfindet.

    Die obige Definition ist schwer anzuwenden,
    wenn eine Sprache textuell und die andere graphischer Natur ist.
    Viele Modellierungssprachen sind deshalb schwer mit der \gls{wccdl} zu vergleichen.
    Allerdings konnten zwei geeignete Kandidaten gefunden werden.
    Der erste ist PlantUML\footnote{vgl. \url{http://plantuml.com/}},
    ein Werkzeug und eine Sprache zur textuellen Beschreibung und anschließender
    Generierung graphischer UML-Diagramme.
    Die zweite Sprache wurde bei Recherchen auf
    GitHub\footnote{vgl. \url{https://github.com/sabrams/web-miner}} gefunden
    und verfolgt ein sehr ähnliches Ziel wie die \gls{wccdl}.
    Sie ist eine interne Ruby DSL, die trotz ihrer etwaigen Unvollständigkeit sowie der fehlenden
    offiziellen wissenschaftlichen Arbeit oder Dokumentation
    für einen Vergleich mit der \gls{wccdl} geeignet ist.
    Sie vermittelt nämlich, wie das Vorhaben der \gls{wccdl}
    in einer anderen Sprache aussehen kann.
    Als Vergleich dient ein Beispiel im
    Testcode\footnote{vgl. \url{https://github.com/sabrams/web-miner/blob/master/features/mine_according_to_command_file.feature}}
    dieser DSL, welches in Listing \ref{listing:discussionExpressivityExampleRubyDsl}
    zu sehen ist.

    \lstinputlisting[
        label=listing:discussionExpressivityExampleRubyDsl,
        caption=Programm in einer Ruby DSL zum Vergleich der Ausdrucksstärke,
        style=pseudo,
        language=Ruby
    ]{../resources/discussion/interpretation/language/expressivity-example.rb}

    Ein äquivalentes Konstrukt in PlantUML zeigt Listing
    \ref{listing:discussionExpressivityExamplePlantUml}.

    \lstinputlisting[
        label=listing:discussionExpressivityExamplePlantUml,
        caption=Programm in PlantUML zum Vergleich der Ausdrucksstärke,
        style=plantuml
    ]{../resources/discussion/interpretation/language/expressivity-example.plantuml}

    Nicht zuletzt zeigt Listing \ref{listing:discussionExpressivityExampleWccs}
    wie das Beispiel in der \gls{wccdl} aussieht.

    \lstinputlisting[
        label=listing:discussionExpressivityExampleWccs,
        caption=Programm in der \acrshort{wccdl} zum Vergleich der Ausdrucksstärke,
        inputencoding=utf8/latin1,
        style=wccdl
    ]{../resources/discussion/interpretation/language/expressivity-example.wctd}

    Obwohl sich auch die \gls{wccdl} in diesem Beispiel kompakt ausdrückt,
    braucht sie mehr Code als die anderen beiden Sprachen.
    Auch ohne genaue Analyse sollte diese geringere Ausdrucksstärke
    allgemeingültig für die \gls{wccdl} sein.
    Anders als ihre Kontrahenten benötigt sie z. B. für jedes Feature eine definierte Klasse
    (im Beispiel \texttt{Object}).
    Außerdem ist die Definition einer einzelnen Klasse,
    eines einzelnen Features oder eines Selektors in etwa gleich lang oder länger,
    als in der Vergleichssprachen.

    Der Grund sind die langen Schlüsselwortsequenzen,
    die zum Beispiel auch im Selektor der Seitenklasse \texttt{Teacher}
    erkennbar sind.
    An dieser Klasse und ihrem Pendant \texttt{News}
    wird außerdem deutlich, dass der Sprache ein Vererbungskonzept fehlt.
    Identische Features müssen nämlich wiederholt werden,
    wodurch sich die Ausdrucksstärke der Sprache verringert.

    Eine Kritik der Ausdrucksstärke der \gls{wccdl} ist berechtigt,
    allerdings muss beachtet werden,
    dass dieses Merkmal in Konflikt mit einigen nun folgenden
    Eigenschaften steht, die positiver bewertet werden.

    % TODO: Namespace?

    \paragraph{Abdeckung der Domäne}
    Die Abdeckung der Domäne einer Sprache beschreiben
    \citet[Kapitel 4.2]{voelter:DslEngineering} als den
    prozentualen Anteil an allen Programmen der Domäne,
    die durch die Sprache formuliert werden können.
    Die formale Definition lautet

    \begin{quote}
        $C_D(L) = \frac{number of P_D programs expressable by L}{number of programs in domain D}$.
    \end{quote}

    Eine volle Abdeckung ist demnach gegeben,
    wenn jedes Programm der Domäne in der Sprache ausgedrückt werden kann.

    Im Falle der \gls{wccdl} kann zumindest die Frage beantwortet werden,
    ob sie ihre Domäne voll abdeckt, das heißt, ob jede beliebige Struktur auf einer Webseite
    durch sie erfasst werden kann.
    Wie das zweite Fallbeispiel zeigte, ist das nicht der Fall,
    da kein allgemeiner Selektor gefunden werden konnte,
    der die Absätze einer Nachricht korrekt erfasst.
    Spezielle Klassen pro Meldung können dieses Problem zwar lösen,
    sind aber nicht im eigentlichen Sinne des \glspl{wccs}.
    Es wurden aber auch einige nicht triviale Strukturen durch die Sprache abgedeckt.
    Selbstverständlich spielen bei dieser Frage auch die Fähigkeiten der Selektoren
    eine Rolle\footnote{vgl. Kapitel \ref{section:discussionInterpretationSelectors}}.

    \paragraph{Vollständigkeit}
    Eine Sprache ist nach \citet[Kapitel 4.5]{voelter:DslEngineering}
    vollständig wenn ihr Generierungsergebnis keine manuell geschriebenen Codefragmente in niedrigeren Sprachen,
    Konfigurationsdateien oder ähnliches benötigt,
    um die gleiche Semantik zu besitzen wie das Programm in der \gls{dsl}.
    Formaler beschreiben sie diese Eigenschaft wie folgt:

    \begin{quote}
        Let us introduce a function $G$ ("code generator") that transforms
        a program $p$ in $L_D$ to a program $q$ in $L_{D-1}$.
        For a complete language, $p$ and $q$ have the same semantics, i.e.
        $OB(p) == OB(G(p)) == OB(q)$ [...]. For incomplete languages
        where $OB(G(p)) \subset OB(p)$ we have to write additional
        code in $L_{D-1}$ to obtain a program in $D_{-1}$ that has the same semantics
        as intended by the original program in $L_D$.
    \end{quote}

    Die \gls{wccdl} ist vollständig, da das generierte JSON-Dokument eines
    {\classificationModel}s mit diesem semantisch übereinstimmt.

    \paragraph{Konkreter Syntax}
    Für den konkreten Syntax einer Sprache schlagen \citet[Kapitel 4.7]{voelter:DslEngineering}
    verschiedene Bewertungskriterien vor: Writeability, Readability, Learnability, Effectivness.

    Der Nachteil der langen Sequenzen von Schlüsselwörtern wird beim Schreiben des Codes
    deutlich, da auch einfache Konzepte vergleichsweise viel Code erfordern.
    Allerdings relativiert sich dieser Nachteil durch die Unterstützung der Entwicklungsumgebung.
    Ein weiterer Nachteil beim Schreiben des Codes ist die fehlende Prüfung der
    Selektoren, wodurch ihre Formulierung unnötig schwer und fehleranfällig wird.
    Das gilt vor allem bei komplexeren Selektoren.
    Die Zeichen zur Einklammerung von Selektoren sind auf vielen Tastaturlayouts nicht zu finden.
    Das ist nachteilig, weil man auf die Autovervollständigung der Entwicklungsumgebung angewiesen ist.
    Gleichzeitig erlauben sie die Nutzung vieler Zeichen ohne sie speziell Kennzeichnen zu müssen,
    wie z. B. Anführungszeichen.
    Im Vergleich zu anderen Sprachen ist das ein wichtiger Vorteil.

    Die vielen Schlüsselwörter sind aber auch ein Vorteil,
    da sich Programme nahezu wie ein Fließtext lesen lassen
    und ihre Verständlichkeit dadurch steigt.
    Da Quelltext in der Regel häufiger gelesen als geschrieben wird,
    ist das sehr positiv zu bewerten.
    Beide Fallbeispiele zeigen aber auch Konstrukte,
    in denen die Lesbarkeit besser sein könnte.
    Zum Beispiel bei der Klasse \texttt{Portal},
    bei der nicht offensichtlich ist,
    dass sie selbst den Namen des Studienportals speichert,
    aber der Link in einem {\childFeature} gespeichert wird.
    Eingeschränkt wird die Lesbarkeit aber durch die Selektoren,
    die in den Fallbeispielen schnell eine hohe Komplexität angenommen haben,
    wie zum Beispiel bei der Klasse \texttt{Phone}.
    Bezüglich der Selektoren und ihrer Lesbarkeit ist aber nochmals
    die fehlende Notwendigkeit, spezielle Zeichen zu kodieren, positiv hervorzuheben.

    Die Konzepte der Sprache sind leicht zu verstehen
    und durch die gebotene Autovervollständigung und einige
    semantische Prüfungen, ist das Erlernen der Sprache einfach.
    Hinzu kommen allerdings die benötigten Kenntnisse von CSS, XPath und regulären Ausdrücken,
    wodurch das formulieren von Selektoren erschwert wird.

    Für eine fundierte Bewertung der Effektivität der Sprache,
    also der Frage, wie gut typische Probleme der Domäne ausgedrückt werden können,
    sind weitere Anwendungen auf noch mehr Webseiten notwendig.
    Das durch die Fallbeispiele erhaltene Bild ist zweiseitig:
    Viele Features und Klassen lassen sich durch simple Selektoren erfassen,
    einige erfordern jedoch komplexe Selektoren.
    Dabei dürfte zumindest das bei der Klasse \texttt{Portal} angewandte Konstrukt,
    um ein HTML-Element sowohl als {\contentFeature} als auch als {\referenceFeature} zu nutzen,
    eine häufige Anforderung sein und sollte besser unterstützt werden.

    \paragraph{Struktur}
    Beim Entwurf der \gls{wccdl} wurde sich bewusst gegen
    eine logische Strukturierung der Klassen in Namensräume,
    dafür aber für eine globale Sichtbarkeit aller Klassen
    in allen Quelldateien eines Projektes entschieden.
    Die hier vorgestellten Fallbeispiele profitieren von diesen Entscheidungen,
    da die Klassen leicht aufgeteilt und wiederverwendet werden konnten.
    Als Argument gegen Namensräume wurde die höhere Komplexität für
    Nicht-Programmierer genannt, was ein valider Punkt ist.
    Allerdings steigt die Einstiegshürde für diese Gruppe bereits
    durch die Selektoren. Namensräume würden diese Hürde womöglich nicht noch weiter anheben.
    Es ist außerdem nicht auszuschließen, dass in größeren Projekten
    Namensräume und eine eingeschränkte Sichtbarkeit von Vorteil wären.
    Anders als die Beispielklassen \texttt{NewsDate} und \texttt{TeacherName},
    müssten Klassen mit abstrakten Namen dann nicht mit einem Präfix versehen werden,
    was durchaus aus schlechter Stil angesehen werden kann.

    Insgesamt ist festzuhalten, dass die \gls{wccdl} viele Problemstellungen der Domäne
    mit einfachen Konzepten abdeckt, was wichtiger ist als jeden Sonderfall zu berücksichtigen
    und deshalb positiv zu bewerten ist.
    \subsection{Übertragbarkeit eines {\classificationModel}s}
    Ein Ziel der Fallbeispiele ist herauszufinden,
    wie gut sich ein anhand einer einzelnen Webseite
    erstelltes {\classificationModel} auf andere Seiten anwenden lässt,
    die augenscheinlich identisch aufgebaut sind.
    Je besser diese Übertragbarkeit ist,
    desto höher ist der Nutzen der Automatisierung.
    Eine erste naheliegende Erkenntnis ist,
    das Klassen iterativ zu definieren sind,
    was schon bei der Entwicklung des {\classificationModel}s
    für das Portal \gls{babw} deutlich wurde.
    Ausnahmen innerhalb einer Webseite werden oft erst
    bei der Prüfung der Klassifikation entdeckt,
    woraufhin die Ausnahme analysiert und das Modell
    entsprechend angepasst werden muss.
    Ein Beispiel ist die Faxnummer, die nur bei einem Mitarbeiter auftritt,
    deren Erfassung aber auch eine Anpassung des Selektors der Telefonnummer
    erforderlich machte.
    Diese Erkenntnis wiegt natürlich noch schwerer,
    wenn eine Menge an Klassen auf viele Webseiten angewandt werden soll.
    Wie das erste Fallbeispiel zeigt,
    können auch vermeintlich sehr ähnliche Seiten deutliche Unterschiede
    im konzeptionellen Modell und in der \gls{html}-Struktur besitzen.
    Diese können nicht alle vorausgedacht werden.
    Es müssen demnach mehrere Webseiten bei der Entwicklung der Klassendefinitionen
    betrachtet werden.
    Andernfalls werden Informationen nicht klassifiziert oder es kommt zu unerwarteten Ergebnissen.
    Beispiele hierfür sind der Verweis auf die Detailseite eines Mitarbeiters
    und die doppelt klassifizierten Mitarbeiter im Portal \gls{bapvs}.
    Deutlich wurde aber auch, dass viele und zu große Unterschiede
    die Auflösung eines Konfliktes sehr komplex machen.
    Diese Überlegungen lassen die Schlussfolgerung zu,
    dass die Erstellung eines allgemeingültigen {\classificationModel}s
    aufgrund der großen Unterschiede eine komplexe Aufgabe ist.
    Eine kritische, aber berechtigte Frage ist deshalb,
    wie groß der Nutzen durch die Automatisierung tatsächlich ist.
    Vor allem dann, wenn zur Erfassung einiger Inhalte Anpassungen an der Seite notwendig sind.
    Dem ist entgegenzusetzen, dass die Effektivität durch ein anderes Vorgehen beim
    Schreiben der Klassendefinitionen gesteigert werden kann.
    Eine solche Alternative zum Versuch alle Seiten durch ein Modell abzudecken
    besteht aus mehreren Schritten.
    Zunächst wird ein Modell für eine Webseite geschrieben,
    sodass es diese korrekt und vollständig klassifiziert.
    Dieser Schritt ist einfach und bringt zweifelsfrei Vorteile,
    was bei der initialen Klassifizierung der Mitarbeiter des Portals \gls{babw} ersichtlich wurde.
    Dann sollte das {\classificationModel} auf eine weitere Seite angewandt werden und
    das Ergebnis geprüft und angepasst werden,
    bis diese zweite Seite korrekt klassifiziert wird.
    Die Klassifizierung findet in diesem Schritt ausschließlich auf der zweiten Seite statt,
    sodass die erste Klassifikation unberührt bleibt.
    Dieser zweite Schritt wird für alle Seiten wiederholt.
    Dieses Vorgehen erlaubt die Wiederverwendung global gültiger Klassendefinitionen
    und erfordert nur die Anpassung einzelner Klassen an die jeweiligen Anforderungen.
    Insgesamt dürfte der Prozess dadurch deutlich effektiver werden.
    Es sollte außerdem nicht die Möglichkeit missachtet werden,
    bewusst einzelne falsche Klassifizierungen in Kauf zu nehmen
    und diese über das {\annotatorPlugin} im Nachhinein zu korrigieren.
    Nicht erfasste Features und unzutreffende Selektoren können
    aber auch auf Fehler in der Webseite hinweisen.
    Wie z. B. die E-Mail-Adresse ohne vorangestelltes \texttt{mailto}
    im Portal \gls{bscpsy}.
    Das \gls{wccs} fungiert indirekt also als ein Werkzeug
    zur Validierung eines individuell definierbaren Schemas für Webseiten.
    \subsection{Mächtigkeit der Selektoren}
    \label{section:discussionInterpretationSelectors}
    Zusammen mit der Schachtelung von Features und dem
    damit verbundenen eingeschränkten Kontexts bei der
    Auswertung von Selektoren,
    spielen deren Fähigkeiten eine zentrale Rolle bei den
    Möglichkeiten der Klassifizierung, die das \gls{wccs} bietet.
    Betrachtet man die {\classificationModel}e der beiden Fallbeispiele wird klar,
    dass ohne den {\xpathSelector} die Möglichkeiten weitaus eingeschränkter wären.
    Die Klasse \texttt{Portal}, durch die ein Feature das Kontextelement
    als {\referenceFeature} \texttt{homepage} klassifiziert,
    wäre beispielsweise weder mit dem {\cssSelector}
    noch mit dem {\urlSelector} möglich.
    Beide können ausschließlich untergeordnete Elemente erfassen.
    Die Selektoren und die Art ihrer Verwendung hat allerdings auch Grenzen.
    Ein Beispiel sind die Mitarbeiter, deren Namen laut Klassifikation
    "`Prof."' bzw. "`Dr."' sind.
    Ein ähnliches Problem gibt es im zweiten Beispiel bei den Absätzen einer Meldung,
    von denen es unterschiedlich viele geben kann.
    In beiden Fällen lässt sich argumentieren,
    dass mehrere HTML-Elemente zusammen eine Einheit darstellen
    und dass ihr textueller Inhalt deshalb
    in einem skalaren Feature gespeichert werden sollten.
    Anders also als {\collectionFeature}s,
    bei denen jedes Element für sich eine eigenständige Einheit darstellt.
    Die beschriebene Anforderung lässt sich bisher nur umsetzen,
    wenn die übergeordnete Einheit der Elemente durch ein gemeinsames
    Vaterelement dargestellt wird, welches nur genau diese Element enthält.
    Für das erste Beispiele hieße die Anforderung, dass der Selektor des Namens eines Mitarbeiters
    alle relevanten Elemente erfasst und
    das \gls{wccs} die textuellen Werte dieser Elemente
    in einem skalaren Feature speichert.
    Der eindeutige Selektor dieses Features würde alle Elemente umfassen,
    was prinzipiell schon jetzt möglich ist,
    da der Selektor Start- und Endelemente unterscheidet.

    Das zweite Fallbeispiel zeigt eine weitere Grenze des Systems,
    da es nicht möglich ist einen allgemeinen Selektor zu formulieren,
    der die Absätze pro Nachricht korrekt
    erfasst.
    Allgemein ausgedrückt ist eine sinnvolle Klassifizierung nicht möglich,
    wenn ein Element eines {\collectionFeature}s selbst ein {\collectionFeature} besitzt
    und weder das Parent- noch das {\childFeature} alleine in einem Element gekapselt ist.
    Diese Einschränkung ist durch die verwendeten Selektortypen begründet.
    Eine Erweiterung des Systems durch andere Selektortypen, kann diese Lücke
    füllen\footnote{vgl. Kapitel \ref{section:endingOutlook}}.

    Die Entscheidung als Ergebnis eines {\xpathSelector}s auch reinen Text und nicht nur
    HTML-Elemente zu akzeptieren, bringt deutliche Vorteile für das \gls{wccs}.
    Es erlaubt z. B. die Telefonnummer eines Mitarbeiters zu selektieren,
    was sonst nicht möglich wäre, da sie nicht alleine in einem Element steht.
    Allerdings darf in diesem Fall die Bestimmung der Versatzangabe des Startelementes nicht auf
    Basis der Eigenschaft \texttt{innerText} geschehen.
    Wie bei der falsch annotierten Telefonnummer im Portal \gls{bapvs} deutlich
    wird,
    muss der durch XPath ermittelte Text nicht in derselben Form im \texttt{innerText} eines Elementes auftauchen.
    Der zu durchsuchende Text muss deshalb ebenfalls über XPath ermittelt werden.
    Für Drittsystemen wird dadurch die Auswertung eines eindeutigen Selektors allerdings komplexer.

    \subsection{Teilen von Knoten in der Datenbank}
    Der erste Beweggrund für die Verwendung einer Graphdatenbank
    ist die natürliche Modellierung und Speicherung von Verweisen
    zwischen Seiten und auf sonstige {\resources}.
    Des Weiteren sollte durch die Vermeidung von Duplikaten in der Datenbank
    und der Möglichkeit einen Knoten mehrfach zu referenzieren,
    die Möglichkeit entstehen weiterführende Analysen auf den klassifizierten
    Inhalten auszuführen.

    Das zweite Beispiel hat gezeigt,
    wie Referenzen zwischen Seiten auf sehr einfache Art und Weise
    die Reihenfolge der Übersichtsseiten und Navigationspfade explizit machen.
    Eine Auswertung ist trivial, da lediglich ein- und ausgehenden Kanten gefolgt werden muss.
    Andere Datenbankmodelle hätten komplexere Konstrukte erfordert,
    um diese Informationen zu speichern oder hätten komplexere
    Aggregierungsschritte zur Auswertung erfordert.

    Betrachtet man Tabele \ref{table:findingsTeachersFiguresNodesByLabel} wird deutlich,
    dass Text- und Resource-Knoten verhältnismäßig am meisten von der
    Möglichkeit Knoten wiederzuverwenden profitieren.
    Das ist nicht verwunderlich, da sie jeweils genau einen Wert speichern,
    der sie identifiziert und keine ausgehenden Kanten besitzen,
    die seitenspezifische Informationen enthalten.
    Wie Tabelle \ref{table:findingsTeachersFiguresSharedNodes} belegt,
    steigt ihre Wiederverwendung bei mehreren Klassifikationen in einer Datenbank,
    was eine logische Konsequenz der größeren Informationsmenge ist.

    Für Resource-Knoten ist dies aber nicht sofort ersichtlich,
    zum Beispiel die Zahl der mehrfach referenzierten
    Lehrgebiets-Resource-Knoten im Falle einer gemeinsamer Datenbank niedriger ist,
    als die Summe der geteilten Knoten in einzelnen Datenbanken (30 vs. 36).
    Gleichzeitig ist die Zahl der geteilten SubjectArea-Knoten aber höher,
    die jeder einen Lehrgebietsknoten referenzieren.
    Es wird also lediglich ein größerer Teilbaum geteilt.
    Außerdem enthielten die einzelnen Datenbanken identische Resource-Knoten,
    die in der gemeinsamen natürlich zusammengefasst werden konnten.

    Resource-Knoten innerhalb einer Datenbank zu duplizieren
    macht aus semantischen Gründen keinen Sinn,
    da sie eine Entität der Domäne darstellen,
    die genau ein mal existiert, was die Datenbank wiederspiegeln sollte
    und außerdem die Möglichkeiten eines Graphens besser ausschöpft.

    Nicht so eindeutig ist dies allerdings bei Text-Knoten.
    Aus Tabelle \ref{table:findingsTeachersFiguresSharedNodes} geht hervor,
    dass im ersten Fallbeispiel niemals ein Text-Knoten geteilt wurde.
    Stattdessen konnte immer der Content-Knoten,
    der ihn referenziert geteilt werden.
    Deshalb ist die Zahl beider Knoten-Typen gesunken und die der geteilten
    Content Knoten dafür gesteigen.
    Das ist möglich, wenn die entsprechenden Content-Knoten
    sich sowohl in ihrer Klasse als auch in ihren Features nicht unterscheiden.
    Betrachtet man die Klassen der geteilten Content Knoten wird deutlich,
    dass diese keine Features haben.

    Anders ist das beim zweiten Fallbeispiel.
    Dort wurden Text-Knoten geteilt, weil es mehrere News mit derselben ÜBerschrift
    gab, die aber jeweils eine eigene Detailseite referenzieren.
    % TODO: Referenz auf TABELLE!

    
% Oder: Verwendung einer Graphdatenbank
    \subsection{Visualisierung der Klassifikation durch Annotationen}
    Eine wichtige Funktion des \glspl{wccs}
    ist die Visualisierung einer Klassifikation durch Webannotationen
    auf der klassifizierten Webseite.
    Aus diesem Grund folgt eine Übersicht der Annotationen
    des Studienportals \gls{babw},
    welches stellvertretend auch für die restlichen klassifizierten Portale steht.
    Abbildung \ref{image:findingTeachersAnnotationsOverview}
    zeigt einen Ausschnitt der annotierten
    Seite\footnote{Die Darstellungsfehler oben rechts im Kopfbereich
    sowie am Anfang der Brotkrümelnavigation unter dem Portal
    sind dem in Kapitel \ref{section:findingsMethod} beschriebenen
    Annotation Viewer geschuldet.
    Durch die Zwischenschaltung dieser Komponente
    führt der Browser Cross-Origin-Requests durch,
    um die genutzte Bibliothek für Symbole zu beziehen.
    Diese Aufrufe werden allerdings unterbunden,
    weshalb die Symbole nicht korrekt dargestellt werden.
    Bei einer direkten Einbindung des Plugins wäre dies nicht der Fall.
    Der fünfte Link im Kopfbereich wurde richtig klassifiziert.}.
    Bis auf wenige Ausnahmen, die in
    Kapitel \ref{section:findingsTeachersAbnormalitiesBabw} besprochen werden,
    wurden alle klassifizierten Elemente korrekt hervorgehoben.
    Eine detaillierte Ansicht einer beispielhaften Annotation zeigt
    Abbildung \ref{image:findingTeachersSubjectAreaAnnotations}.
    Das Lehrgebiet besitzt korrekterweise zwei Annotationen,
    da das HTML-Element sowohl den Namen als auch den Link enthält
    und deshalb doppelt klassifiziert wurde.

    \begin{figure}[htb]
        \centering
        \includegraphics[width=\textwidth]{../resources/findings/case-study-1/babw/annotations/overview.png}
        \caption{Die annotierte Webseite über Mitarbeiter des Portals \acrshort{babw}}
        \label{image:findingTeachersAnnotationsOverview}
    \end{figure}

    \begin{figure}[htb]
        \centering
        \includegraphics[scale=\screenshotScaleFactor]{../resources/findings/case-study-1/babw/annotations/double-lg-annotation.png}
        \caption{Die Annotationen eines Lehrgebietes}
        \label{image:findingTeachersSubjectAreaAnnotations}
    \end{figure}

