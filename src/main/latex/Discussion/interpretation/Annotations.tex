\subsection{Annotationen}
    Die Fallbeispiele zeigen, dass die Visualisierung einer Klassifikation
    über Webannotationen ein nützliches Mittel ist, um festzustellen,
    ob eine Klassifizierung erfolgreich und den Erwartungen entsprechend durchgeführt wurde.

    Die falsch oder nicht markierten Inhalte schränken dies allerdings
    auf eine erste oberflächliche Sichtung ein
    und können zu falschen Annahmen und Aussagen über die Klassifikation führen.
    Für eine genaue Prüfung sind sie deshalb ungeeignet,
    weshalb hierzu auf die Webanwendung zurückgegriffen werden sollte.

    Das Annotator Plugin ist außerdem hilfreich,
    um die Klasse einzelner Inhalte zu korrigieren,
    falls diese falsch einsortiert wurden.
    Für umfangreichere Änderungen ist sie aufgrund des manuellen Aufwandes
    und der Tatsache, dass lediglich Leaf Features dargestellt werden,
    ungeeignet.
    Bei den festgestellten Fehlern in den Klassifikationen hätte diese
    Funktion allerdings nicht weitergeholfen,
    da nie der Fall eingetreten ist, dass Inhalte lediglich die falsche Klasse erhalten haben.
    Stattdessen wurden Inhalte gar nicht klassifiziert,
    umfassten zu viele oder zu wenige Informationen
    (z. B. Telefonnumern mit E-Mail-Adressen, Name nur "`Prof."')
    oder irrelevante Informationen wurden klassifiziert.
    Um diese Art von Fehlern korrigieren zu können,
    muss das Annotator Plugin um zwei Funktionen erweitert werden:

    \begin{enumerate}
        \item Nicht klasifizierte Inhalte der Seite manuell annotieren und dadurch einer Klasse zuordnen
        \item Annotationen löschen und dadurch die Klassifizierung eines Inhaltes entfernen
    \end{enumerate}