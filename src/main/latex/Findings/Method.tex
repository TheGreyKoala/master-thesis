\section{Vorgehen}
    Vor den eigentlichen Ergebnissen soll an dieser Stelle
    das Vorgehen beschrieben werden.

    Einge Studienportale der Fakultät \gls{ksw} der {\fernUni}
    besitzen Übersichtsseiten aller Lehrenden und Betreuenden
    in dem jeweligen Portal.
    Das erste Fallbeispiel klassifiziert diese Seiten.
    Auf Basis der Seite des Studienportal "`B.A. Bildungswissenschaft"'
    wird dazu ein Modell der Seite erstellt,
    welches dann in eine Menge von Klassendefinitionen in der \gls{wccdl} überführt wird,
    sodass alle relevanten Inhalte dieser Seite erfasst werden.
    Ziel ist es zu zeigen, dass das System erfolgreich auf einer einzelnen Seite angewandt werden kann.
    Andernfalls wäre es zu nichts nutze.

    Anschließend wird die Klassendefinition einzeln auf die Pendants der restlichen Sites angewandt,
    ohne im Vorfeld irgendwelche Anpassungen vorzunehmen,
    da dies ein Vorgehen in der Praxis darstellt.
    Dieser Schritt legt dar, wie einfach sich Konfigurationen auf mehrere vermeintlich
    sehr ähnliche Webseiten anwenden lässt und welcher zusätzliche Aufwand für Anpassungen entsteht.
    Die Zahl der Unterschiede im Modell der Seiten und der unzutreffenden Selektoren ist ein Indikator hierfür.

    Der letzte Abschnitt des ersten Fallbeispiels wiederholt den vorherigen,
    unterscheidet sich aber dahingehend, dass alle Klassifikationen
    gemeinsam in einer Datenbank gespeichert werden.
    Die Anzahl der resultierenden Knoten und Kanten in der Datenbank
    gibt Aufschluss darüber, wie effektiv die Möglichkeit Knoten in mehreren Seiten
    zu verwenden, in der Praxis ist und welche Informationen daraus gezogen werden können.

    Das zweite Fallbeispiel klassifiziert Übersichtsseiten mit aktuellen Nachrichten
    der Fakultät und der einzelnen Studienportale.
    Wie später gezeigt wird, gibt es auf diesen Seiten im Vergleich zum ersten Beispiel
    wichtige Unterschiede in der HTML-Struktur -- auf der die Klassifizierung letztendlich basiert --,
    weshalb mit diesem Beispiel gezeigt werden soll, inwieweit das System mit dieser Struktur zurecht kommt.
    Gleichzeitig verwendet es einige Klassendefinitionen aus dem ersten Beispiel wieder,
    und definiert lediglich spezielle Klassen selbst.
    Nach der Klassifizierung einer einzelnen Seite des Portals "`B.A. Bildungswissenschaft"'
    werden alle Nachrichtenübersichtsseiten dieser Site gemeinsam klassifiziert,
    was wiederum neue Resultate ergeben wird.

    Die Ergebnisse bestehen in beiden Beispielen aus
    den entwickelten Klassendefinitionen,
    Kennzahlen über die angelegten Koten und Kanten in der Datenbank,
    Auffälligkeiten in der Klassifikation und der Visualisierung durch Annotationen
    inkl. einer Erklärung, wodurch diese begründet sind.
