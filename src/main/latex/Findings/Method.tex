\section{Vorgehen}
    \label{section:findingsMethod}
    Einige Studienportale der Fakultät \gls{ksw} der {\fernUni}
    besitzen Übersichtsseiten aller Lehrenden und Betreuenden des Portals,
    die im Folgenden zusammengefasst auch als "`Mitarbeiter"' bezeichnet werden.
    Das erste Fallbeispiel klassifiziert diese Seiten.
    Auf Basis der Seite des Studienportals
    \gls{babw}\footnote{vgl. \url{http://www.fernuni-hagen.de/KSW/portale/babw/einstieg/lehrende-und-betreuende-im-b-a-bildungswissenschaft/}}
    wird dazu ein konzeptionelles Modell solcher Seiten erstellt,
    welches dann in ein {\classificationModel} überführt wird,
    sodass alle relevanten Inhalte dieser Seite erfasst werden.
    Ziel ist es zu zeigen, dass das System erfolgreich auf einer einzelnen Seite angewandt werden kann.

    Anschließend wird das {\classificationModel} einzeln auf die Pendants der restlichen Portale angewandt,
    ohne im Vorfeld irgendwelche Anpassungen vorzunehmen.
    Dieser Schritt legt dar, wie einfach sich Konfigurationen auf mehrere vermeintlich
    sehr ähnliche Webseiten anwenden lässt und welcher zusätzliche Aufwand für Anpassungen entsteht.
    Die Zahl der Unterschiede im konzeptionellen Modell der Webseiten und die Zahl unzutreffender Selektoren sind ein Indikator hierfür.

    Der letzte Schritt des ersten Fallbeispiels wiederholt den vorangegangenen,
    unterscheidet sich aber dahingehend, dass alle Klassifikationen
    gemeinsam in einer Datenbank gespeichert werden.
    Die Anzahl der resultierenden Knoten und Kanten in der Datenbank
    gibt Aufschluss darüber, wie effektiv die Möglichkeit, Knoten in mehreren Klassifikationen
    zu verwenden, in der Praxis ist und welche Informationen daraus gezogen werden können.

    Das zweite Fallbeispiel klassifiziert Übersichtsseiten mit aktuellen Nachrichten
    der Fakultät und der einzelnen Studienportale.
    Die HTML-Struktur dieser Seiten besitzt im Vergleich zum ersten Beispiel wichtige Unterschiede.
    Das zweite Fallbeispiel soll deshalb herausfinden, inwieweit das System mit dieser Struktur zurecht kommt.
    Zur Klassifizierung verwendet es einen Teil des {\classificationModel}s aus dem ersten Beispiel
    und ergänzt lediglich spezielle Klassen.
    Nach der Klassifizierung einer einzelnen Seite des Portals \gls{babw}
    werden alle Nachrichtenübersichtsseiten dieses Portals gemeinsam klassifiziert,
    was wiederum neue Resultate ergeben wird.

    Die Ergebnisse bestehen in beiden Beispielen aus:

    \begin{enumerate}
        \item dem verwendeten {\classificationModel},
        \item Kennzahlen über Graphen der Datenbank,
        \item Auffälligkeiten in Klassifikationen,
        \item Auffälligkeiten in der Visualisierung der Klassifikationen (Webannotationen) und
        \item Ursachen dieser Unregelmäßigkeiten.
    \end{enumerate}

    Während der Versuche bestand kein schreibender Zugriff auf die Quellen der klassifizierten Webseiten.
    Zur Anzeige der Annotationen wurde deshalb ein zusätzlicher Service implementiert -- der Annotation Viewer.
    Dieser lädt die HTML-Repräsentation einer Webseite herunter und ergänzt sie,
    sodass sie das Annotator Plugin einbindet\footnote{vgl. Kapitel \ref{section:solutionDetailsAnnotatorPluginIntegration}}.
    Das Ergebnis gibt sie an den Aufrufer -- sinnvollerweise ein Webbrowser -- weiter.
