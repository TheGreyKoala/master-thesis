\subsection{Konzeptionelles Modell der Webseite}
    \label{section:findingsTeachersConceptualModel}
    Das konzeptionelle Modell wird in diesem Beispiel anhand
    der Übersichtsseite des Studienportals \gls{babw} beschrieben.
    Abbildung \ref{image:findingTeachersModelOverview} zeigt einen
    Ausschnitt dieser Seite, auf dem die wichtigsten Bereiche zu sehen sind.
    Eine Darstellung des Modells ist in Abbildung
    \ref{image:findingTeachersModelUml} zu sehen.
    Die Webseite lässt sich in verschiedene Bereiche aufteilen.
    Zunächst einen Kopfbereich, der auf der linken Seite das Logo
    der {\fernUni} enthält.
    Dieses Bild ist gleichzeitig ein Link zur Hauptseite der Universität. 
    Auf der rechten Seite enthält der Kopfbereich einige Links zu den verschiedenen
    Bereichen der Site.
    Direkt unter dem Kopfbereich befindet sich der Name des Studienportals,
    welcher gleichzeitig ein Link auf die Einstiegsseite des Portals ist.
    Darunter befindet sich auf der linken Seite ein weiterer Navigationsbereich,
    der Verweise auf die Unterseiten des aktuellen Bereichs enthält.
    Rechts daneben findet sich zunächst der Titel der Seite,
    gefolgt von einem kurzen einleitenden Absatz.
    Alle bisher genannten Elemente der Seite finden sich in sehr ähnlicher Form
    auch auf anderen Seiten wieder.
    Die dem einleitenden Absatz folgende Liste aller Mitarbeiter unterscheidet die Seite hingegen von anderen.
    Für jeden Mitarbeiter ist eine Reihe an Informationen dargestellt.
    Neben einem Bild ist das der Name des Lehrgebiets, in dem er tätig ist.
    Dieser Name ist außerdem ein Link auf eine Seite über dieses Gebiet.
    Es folgen der Name des Mitarbeiters
    und einige Kontaktinformationen.
    Dies können eine E-Mail-Adresse, eine Telefonnummer
    und bei einigen wenigen Kontakten auch eine Faxnummer und ein Raum sein.

    \begin{figure}[t]
        \centering
        \includegraphics[width=\textwidth]{../resources/findings/case-study-1/model/overview.png}
        \caption{Die Webseite über Mitarbeiter des Portals \acrshort{babw}}
        \label{image:findingTeachersModelOverview}
    \end{figure}

    \begin{figure}[htb]
        \centering
        \includegraphics[scale=\imageScalingFactor]{../resources/findings/case-study-1/model/model.png}
        \caption{Das konzeptionelle Modell einer Webseite über Mitarbeiter}
        \label{image:findingTeachersModelUml}
    \end{figure}