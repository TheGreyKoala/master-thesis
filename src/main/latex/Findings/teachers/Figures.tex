\subsection{Kennzahlen}
    Nach jeder Klassifizierung wurden einige Kennzahlen des
    resultierenden Graphs ermittelt.
    Dies geschah im Fall einzelner Datenbanken und im Fall
    einer gemeinsamen.
    Zum besseren Verständnis der präsentierten Zahlen wird
    zunächst die konkrete Struktur der Graphen beschrieben.

    \paragraph{Struktur eines Graphs}
    Abbildung \ref{image:findingTeachersFiguresDbModel1}
    und \ref{image:findingTeachersFiguresDbModel2} zeigen,
    wie der Graph einer Klassifikation in diesem Fallbeispielt aufgebaut ist.
    Beide sind durch den \texttt{Content}-Knoten mit der Klasse \texttt{Teacher} verbunden.
    Referenzen sind in den Abbildungen nicht zu sehen,
    da sie lediglich Kanten von einem Knoten zu einer {\resource} darstellen
    und die Darstellungen unnötig vergrößern würden.
    Anhand des {\classificationModel}s und der beiden Abbildungen ist
    offensichtlich, welche der zu sehenden Knoten Referenzen besitzen.
    Bei {\collectionFeature}s ist immer ein stellvertretendes Element zu sehen.

    \begin{figure}[htb]
        \centering
        \includegraphics[scale=\imageScalingFactor]{../resources/findings/case-study-1/dbmodel/dbmodel1.png}
        \caption{Struktur des Graphs einer Seite über Lehrende und Betreuende (1)}
        \label{image:findingTeachersFiguresDbModel1}
    \end{figure}

    \begin{figure}[htb]
        \centering
        \includegraphics[scale=\imageScalingFactor]{../resources/findings/case-study-1/dbmodel/dbmodel2.png}
        \caption{Struktur des Graphs einer Seite über Lehrende und Betreuende (2)}
        \label{image:findingTeachersFiguresDbModel2}
    \end{figure}

    \paragraph{Präsentation der Kennzahlen}
    Die folgenden Tabellen präsentieren die gesammelten Kennzahlen.
    Sie sind folgendermaßen aufgebaut:
    Die erste Spalte benennt die Kennzahl.
    Dann folgen vier Spalten, die den Wert der Kennzahl für die einzelnen Sites enthalten.
    Diese Zahlen werden in der Spalte "`Summe"' aufaddiert,
    um sie mit der letzten Spalte zu vergleichen.
    Diese gibt den Wert der Kennzahl für den Fall der gemeinsam genutzten Datenbank an.
    Eine ausführliche Interpretation dieser Zahlen geschieht in Kapitel \ref{section:findingsInterpretation}.
    Trotzdem wird hier schon die Bedeutung einiger Kennzahlen kurz hervorgehoben.

    Tabelle \ref{table:findingsTeachersFiguresNodesByLabel}
    gruppiert die Knoten der Datenbank nach ihren Labels und zeigt,
    wie oft jedes Label oder jede Kombination von Labels Verwendung fand.   
    Für die einzelnen Klassifikationen gibt "`Content"' außerdem an,
    wie viele {\contentFeature}s sie enthalten.

    \begin{table}[htb]
        \centering
        \begin{tabular}{|l|c|c|c|c|c|c|}
            \hline
            \multicolumn{1}{|c|}{\textbf{Label}} & \textbf{\gls{babw}} & \textbf{\gls{bapvs}} & \textbf{\gls{bscpsy}} & \textbf{\gls{mabm}} & \textbf{Summe} & \textbf{Alle} \\ \hline
            Content                                     & 270           & 275            & 176             & 128           & 849            & 824           \\ \hline
            Page + Resource                             & 1             & 1              & 1               & 1             & 4              & 4             \\ \hline
            Resource                                    & 133           & 159            & 89              & 71            & 452            & 430           \\ \hline
            Site                                        & 1             & 1              & 1               & 1             & 4              & 4             \\ \hline
            Text                                        & 105           & 69             & 75              & 56            & 305            & 284           \\ \hline
            \hline
            \textbf{Summe}                              & 510           & 505            & 342             & 257           & 1614           & 1546          \\ \hline
        \end{tabular}
        \caption{Knoten gruppiert nach Labels für Seiten über Lehrende und Betreuende}
        \label{table:findingsTeachersFiguresNodesByLabel}
    \end{table}

    Die Knoten mit dem Label "`Content"' lassen sich nach der ihnen zugewiesenen Klasse
    weiter aufschlüssen, was in Tabelle \ref{table:findingsTeachersFiguresContentNodesByClass} geschieht.
    Eine Kennzahl in dieser Tabelle ist demnach gleichzusetzen mit der Häufigkeit der Verwendung
    der genannten Klasse in der Klassifikation.

    \begin{table}[htb]
        \centering
        \begin{tabular}{|l|c|c|c|c|c|c|}
        \hline
            \textbf{Klasse}  & \multicolumn{1}{l|}{\textbf{\gls{babw}}} & \multicolumn{1}{l|}{\textbf{\gls{bapvs}}} & \multicolumn{1}{l|}{\textbf{\gls{bscpsy}}} & \multicolumn{1}{l|}{\textbf{\gls{mabm}}} & \multicolumn{1}{l|}{\textbf{Summe}} & \multicolumn{1}{l|}{\textbf{Alle}} \\ \hline
            Brand              & 1                                  & 1                                   & 1                                    & 1                                  & 4                                   & 4                                  \\ \hline
            ContactInformation & 55                                 & 67                                  & 33                                   & 24                                 & 179                                 & 178                                \\ \hline
            Fax                & 1                                  & 0                                   & 0                                    & 1                                  & 2                                   & 1                                  \\ \hline
            Header             & 1                                  & 1                                   & 1                                    & 1                                  & 4                                   & 4                                  \\ \hline
            Introduction       & 1                                  & 0                                   & 1                                    & 1                                  & 3                                   & 2                                  \\ \hline
            PageHeading        & 1                                  & 1                                   & 1                                    & 1                                  & 4                                   & 4                                  \\ \hline
            Phone              & 35                                 & 38                                  & 29                                   & 21                                 & 123                                 & 119                                \\ \hline
            Portal             & 1                                  & 1                                   & 1                                    & 1                                  & 4                                   & 4                                  \\ \hline
            Room               & 2                                  & 0                                   & 0                                    & 0                                  & 2                                   & 2                                  \\ \hline
            SubjectArea        & 53                                 & 67                                  & 33                                   & 22                                 & 175                                 & 172                                \\ \hline
            SubjectAreaName    & 9                                  & 18                                  & 11                                   & 7                                  & 45                                  & 39                                 \\ \hline
            Teacher            & 55                                 & 70                                  & 33                                   & 24                                 & 182                                 & 182                                \\ \hline
            TeacherName        & 55                                 & 11                                  & 32                                   & 24                                 & 122                                 & 113                                \\ \hline
            \hline
            \textbf{Summe}     & 270                                & 275                                 & 176                                  & 128                                & 849                                 & 824                                \\ \hline
        \end{tabular}
        \caption{Content Knoten aufgeteilt nach Klasse}
        \label{table:findingsTeachersFiguresContentNodesByClass}
    \end{table}

    Auch über die Kanten des Graphens lassen sich eine Zahlen ermitteln.
    Tabelle \ref{table:findingTeachersFiguresEdgesByLabel} beginnt dazu
    mit der Aufschlüsselung der Kanten nach ihrem Label.
    Die Kennzahl "`References"' spiegelt die Anzahl der Referenzen innerhalb der Klassifikation wieder.

    \begin{table}[htb]
        \centering
        \begin{tabular}{|l|c|c|c|c|c|c|}
            \hline
            \multicolumn{1}{|c|}{\textbf{Kanten-Label}} & \textbf{\gls{babw}} & \textbf{\gls{bapvs}} & \textbf{\gls{bscpsy}} & \textbf{\gls{mabm}} & \textbf{Summe} & \textbf{Alle} \\ \hline
            Reads                                       & 105           & 69             & 75              & 56            & 305            & 284           \\ \hline
            References                                  & 180           & 209            & 110             & 86            & 585            & 582           \\ \hline
            Owns                                        & 320           & 335            & 200             & 144           & 999            & 996           \\ \hline
            \hline
            \textbf{Summe}                              & 605           & 613            & 385             & 286           & 1889           & 1862          \\ \hline
        \end{tabular}
        \caption{Kanten nach Label}
        \label{table:findingTeachersFiguresEdgesByLabel}
    \end{table}

    Neben dem Label ist in Bezug auf Kanten auch die Frage interessant,
    welche Knoten sie verbinden.
    Tabelle \ref{table:findingsTeachersFiguresEdgesByStartEndNodeLabel}
    zeigt, welche Arten von Knoten wie oft verbunden wurden.
    Die Beziehung eines Content Knotens zu einem Text-Knoten ist nicht enthalten,
    da diese äquivalent zum oben gezeigten Reads-Label ist.
    Diese Tabelle liefert unter anderem Informationen darüber,
    wie viele Referenzen die Seite selbst hat und wie viele zu Content Features gehören.

    \begin{table}[htb]
        \centering
        \begin{tabular}{|l|c|c|c|c|c|c|}
            \hline
            \multicolumn{1}{|c|}{\textbf{Start-, Zielknoten-Label}} & \textbf{\gls{babw}} & \textbf{\gls{bapvs}} & \textbf{\gls{bscpsy}} & \textbf{\gls{mabm}} & \textbf{Summe} & \textbf{Alle} \\ \hline
            (:Content) $\rightarrow$ (:Content)                           & 260           & 260            & 162             & 115           & 797            & 794           \\ \hline
            (:Content) $\rightarrow$ (:Resource)                         & 172           & 201            & 102             & 78            & 553            & 550           \\ \hline
            (:Page) $\rightarrow$ (:Content)                              & 59            & 74             & 37              & 28            & 198            & 198           \\ \hline
            (:Page) $\rightarrow$ (:Resource)                             & 8             & 8              & 8               & 8             & 32             & 32            \\ \hline
            (:Site) $\rightarrow$ (:Page)                                 & 1             & 1              & 1               & 1             & 4              & 4             \\ \hline
            \hline
            \textbf{Summe}                                          & 500           & 544            & 310             & 230           & 1584           & 1578          \\ \hline
        \end{tabular}
        \caption{Kanten nach Start und Zielknoten}
        \label{table:findingsTeachersFiguresEdgesByStartEndNodeLabel}
    \end{table}

    Eine letzte zu beantwortende Frage ist,
    wie viele Knoten in der Datenbank mehr als eine eingehende Kante haben,
    d.h., wie oft sie an verschiedenen Stellen einer oder mehrerer Klassifikationen Verwendung finden.

    \begin{table}[htb]
        \centering
        \begin{tabular}{|l|c|c|c|c|c|c|}
            \hline
            \multicolumn{1}{|c|}{\textbf{Knoten}} & \textbf{\gls{babw}} & \textbf{\gls{bapvs}} & \textbf{\gls{bscpsy}} & \textbf{\gls{mabm}} & \textbf{Summe} & \textbf{Alle} \\ \hline
            Bild                                  & 0             & 1              & 1               & 0             & 2              & 2             \\ \hline
            ContactInformation                    & 0             & 3              & 0               & 0             & 3              & 3             \\ \hline
            E-Mail-Adresse                        & 0             & 1              & 1               & 0             & 2              & 14            \\ \hline
            Fax                                   & 0             & 0              & 0               & 0             & 0              & 1             \\ \hline
            Hauptseite "`Studium"'                & 1             & 0              & 0               & 0             & 1              & 1             \\ \hline
            Homepage der FU                       & 0             & 0              & 0               & 0             & 0              & 1             \\ \hline
            Introduction                          & 0             & 0              & 0               & 0             & 0              & 1             \\ \hline
            Lehrgebietsseiten                     & 7             & 16             & 8               & 5             & 36             & 30            \\ \hline
            Phone                                 & 4             & 1              & 0               & 0             & 5              & 9             \\ \hline
            SubjectArea                           & 0             & 2              & 0               & 0             & 2              & 5             \\ \hline
            SubjectAreaName                       & 8             & 16             & 7               & 5             & 36             & 31            \\ \hline
            Teacher                               & 0             & 1              & 0               & 0             & 1              & 1             \\ \hline
            TeacherName                           & 0             & 3              & 1               & 0             & 4              & 13            \\ \hline
            \hline
            \textbf{Summe}                        & 20            & 44             & 18              & 10            & 92             & 112           \\ \hline
        \end{tabular}
        \caption{Mehrfach referenzierte Knoten}
        \label{table:findingsTeachersFiguresSharedNodes}
    \end{table}
