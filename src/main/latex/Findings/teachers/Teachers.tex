\section{Fallbeispiel 1 -- Lehrende und Betreuende}
    \label{section:findingsCaseStudy1}
    In diesem Fallbeispiel werden Übersichtsseiten
    über die Lehrenden und Betreuenden (Mitarbeiter) eines Studienportals
    der Fakultät \gls{ksw} der {\fernUni} klassifiziert.
    Anhand der Seite des Portals \gls{babw} wird dazu zunächst ein
    konzeptionelles Modell dieser Seiten sowie ein {\classificationModel} erstellt.
    Nach der einzelnen Klassifizierung dieser Seite folgt die separate Klassifizierung
    der Übersichtsseiten der Portale

    \begin{itemize}
        \item \gls{bapvs}\footnote{vgl. \url{http://www.fernuni-hagen.de/KSW/portale/bapvs/einstieg/lehrende-und-betreuende-im-b-a-pvs/}},
        \item \gls{bscpsy}\footnote{vgl. \url{http://www.fernuni-hagen.de/KSW/portale/bscpsy/einstieg/lehrende-und-betreuende-im-b-sc-psychologie/}} und
        \item \gls{mabm}\footnote{vgl. \url{http://www.fernuni-hagen.de/KSW/portale/mabm/einstieg/lehrende-und-betreuende-im-m-a-eeducation/}}.
    \end{itemize}

    Jede Klassifikation wird dabei in separaten Datenbanken gespeichert,
    um die Graphen isoliert betrachten zu können.
    Anschließend werden die Seiten aller Sites erneut klassifiziert und in
    einer gemeinsamen Datenbank gespeichert,
    um zu sehen, welchen Einfluss das auf die Graphen hat.
    Zu jeder Klassifikation werden einige Kennzahlen der Datenbank präsentiert.
    Anschließend erfolgt eine Betrachtung von Auffälligkeiten in den Klassifikationen und in den Annotationen.

    \subsection{Modell der Seite}
    % TODO: UML Diagramm der Seite, dann ist es auch ein Modell
    Abbildung \ref{image:findingNewsModelOverview} zeigt einen
    Ausschnitt der zu klassifizierenden Seite,
    anhand dessen das Modelld er Seite erläutert wird.
    Eine schematische Darstellung ist außerdem in Abbildung
    \ref{image:findingNewsModelUml} zu sehen.
    Aufgrund der nachfolgend beschriebenen Überschneidungen zum
    ersten Fallbeispiel, verzichtet dieses auf die Wiederholung
    identischer Elemente.

    \begin{figure}[htb]
        \centering
        \includegraphics[width=\textwidth]{../resources/findings/case-study-2/news-overview.png}
        \caption{Ausschnitt aus einer Seite: Lehrende und Betreuende}
        \label{image:findingNewsModelOverview}
    \end{figure}

    Es ist leicht zu erkennen, dass es auf dieser Seite einige Überschneidungen
    mit der des vorherigen Beispiels gibt.
    Das betrifft den Kopfbereich, den Namen des Portals,
    die seitlichen Navigationspunkte und die Überschrift der Seite.
    Im Vergleich zum ersten Beispiel gibt es hier nur wenige inhaltliche,
    keine konzeptionellen Unterschide.

    Diese werden erst im mittleren Bereich ersichtlich,
    wo die Auflistung der einzelnen Nachrichten erfolgt.
    Jede Nachricht besitzt ein Datum, eine Überschrift und beliebig viele Absätze,
    die Text, Links etc. enthalten.
    Die Überschrift ist gleichzeitig auch ein Link auf die Einzelseite der Meldung.
    Diese Seite enthält nicht alle Meldungen.
    Stattdessen werden sie auf mehrere Seiten verteilt.
    Zwei Links, die in Abbildung \ref{image:findingNewsModelOverview} nicht zu sehen ist,
    führen einen Besucher der Webseite zur nächsten bzw. vorherigen Übersichtsseite.
    Dabei handelt es sich um eigenständige Seiten mit individuellen \glspl{url},
    die aber alle dem beschriebenen Modell folgen.

    \begin{figure}[htb]
        \centering
        \includegraphics[width=\textwidth]{../resources/findings/case-study-2/model.png}
        \caption{Schematische Darstellung einer Nachrichtenübersichtsseite}
        \label{image:findingNewsModelUml}
    \end{figure}
    \subsection{Klassifizierungsmodell}
    Aus der obigen Analyse der Seite ist ein Klassifizierungsmodell hervorgegangen,
    welches sich in allgemeine und spezielle Klassen aufteilen lässt.
    Die Definition wurde entsprechend auf zwei Dateien aufgeteilt,
    sodass sich die allgemeinen Klassen leicht übertragen lassen.
    Zur Definition der Selektoren wurde diente die HTML-Struktur
    eines Mitarbeiters, die beispielhaft in Listing
    \ref{listing:findingsTeachersHtmlSource} dargestellt ist.
    Konkrete Inhalte wurden aus Gründen der Übersichtlichkeit entfernt.
    % TODO: Erwähnen, dass min. 1 Kontakt b anstatt strong verwendet?
    Wie später ersichtlich wird, spielen die Zeilenumbrüche im p-Element
    im Quelltext eine relevante Rolle, weshalb sie unverändert übernommen wurden.

    \lstinputlisting[
        label=listing:findingsTeachersHtmlSource,
        caption=HTML-Struktur eines Lehrenden,
        language=HTML
    ]{../resources/findings/case-study-1/babw/teacher.html}

    Die Klassen der allgemeinen Bereiche sind in Listing
    \ref{listing:findingsTeachersCommon} aufgeführt.
    Es folgt eine kurze Erläuterung,
    welche Elemente der Seite die einzelnen Klassen darstellen.

    \lstinputlisting[
        label=listing:findingsTeachersCommon,
        caption=Klassendefinition für allgemeine Bereiche,
        language=wccdl,
        inputencoding=utf8/latin1,
        style=wccdl
    ]{../resources/findings/case-study-1/classification-model/Common.wctd}

    Die Klasse "`Header"' repräsentiert den Kopfbereich der Seite
    und besitzt entsprechend Features,
    die das Logo sowie die Links klassifizieren.
    Das Logo wird durch die Klase "`Brand"' modelliert,
    die sowohl ein Feature für das Bild selbst als auch den Link enthält.
    Für den Namen des Portals und den zugehörige Link existiert die Klasse "`Portal"'.
    Die Überschrift der Seite wird als "`PageHeading"' klassifiziert und
    der einleitende Absatz als "`Introduction"', die außerdem Links auf andere
    Seiten der {\fernUni} erkennt.
    Wie die Links im Kopfbereich werden sie als "`FernUniInternalLink"' klassifiziert.

    Zusätzlich wird in dieser Datei für Bilder die Klasse "`Image"' definiert,
    die in der zweiten Datei Verwendung findet.
    Aus dieser geht auch hervor, wie die Navigationspunkte im linken Bereich erfasst werden.
    Sie ist in Listing \ref{listing:findingsTeachersSpecial} zu sehen.

    \lstinputlisting[
        label=listing:findingsTeachersSpecial,
        caption=Klassendefinition für Lehrende und Betreuende,
        language=wccdl,
        inputencoding=utf8/latin1,
        style=wccdl
    ]{../resources/findings/case-study-1/classification-model/Teachers.wctd}

    Zunächst wird die Seitenklasse "`Teachers"' definiert,
    inklusive einiger Features für die allgemeinen Bereiche der Seite,
    wofür die Klassen der vorangegangenen Datei genutzt werden.
    Eines dieser Features ist "`sidebarNavigationLinks"',
    welches die Navigationslinks auf der linken Seite als "`FernUniInternalLink"'
    klassifiziert.
    Ein einzelner Mitarbeiter wird durch die Klasse "`Teacher"' dargestellt
    sowie sein Name durch "`TeacherName"'.
    Ein Lehrgebiet mit Namen und Link wird als "`SubjectArea"' bzw. "`SubjectAreaNem"' klassifiziert.
    Die Kontaktinformationen eines Mitarbeiters kapselt die Klasse "`ContactInformation"',
    wobei die einzelnen Angaben durch die Klassen "`Phone"', "`Email"', "`Fax"' und "`Room"'
    repräsentiert werden.

    % TODO: Erwähnen, dass die Klassen iterativ erstellt wurden?
    \subsection{Kennzahlen}
    Nach jeder Klassifizierung wurden einige Kennzahlen des
    resultierenden Graphs ermittelt.
    Dies geschah im Fall einzelner Datenbanken und im Fall
    einer gemeinsamen.
    Zum besseren Verständnis der präsentierten Zahlen wird
    zunächst die konkrete Struktur der Graphen beschrieben.

    \paragraph{Struktur eines Graphs}
    Abbildung \ref{image:findingTeachersFiguresDbModel1}
    und \ref{image:findingTeachersFiguresDbModel2} zeigen,
    wie der Graph einer Klassifikation in diesem Fallbeispielt aufgebaut ist.
    Beide sind durch den \texttt{Content}-Knoten mit der Klasse \texttt{Teacher} verbunden.
    Referenzen sind in den Abbildungen nicht zu sehen,
    da sie lediglich Kanten von einem Knoten zu einer {\resource} darstellen
    und die Darstellungen unnötig vergrößern würden.
    Anhand des {\classificationModel}s und der beiden Abbildungen ist
    offensichtlich, welche der zu sehenden Knoten Referenzen besitzen.
    Bei {\collectionFeature}s ist immer ein stellvertretendes Element zu sehen.

    \begin{figure}[htb]
        \centering
        \includegraphics[scale=\imageScalingFactor]{../resources/findings/case-study-1/dbmodel/dbmodel1.png}
        \caption{Struktur des Graphs einer Seite über Lehrende und Betreuende (1)}
        \label{image:findingTeachersFiguresDbModel1}
    \end{figure}

    \begin{figure}[htb]
        \centering
        \includegraphics[scale=\imageScalingFactor]{../resources/findings/case-study-1/dbmodel/dbmodel2.png}
        \caption{Struktur des Graphs einer Seite über Lehrende und Betreuende (2)}
        \label{image:findingTeachersFiguresDbModel2}
    \end{figure}

    \paragraph{Präsentation der Kennzahlen}
    Die folgenden Tabellen präsentieren die gesammelten Kennzahlen.
    Sie sind folgendermaßen aufgebaut:
    Die erste Spalte benennt die Kennzahl.
    Dann folgen vier Spalten, die den Wert der Kennzahl für die einzelnen Sites enthalten.
    Diese Zahlen werden in der Spalte "`Summe"' aufaddiert,
    um sie mit der letzten Spalte zu vergleichen.
    Diese gibt den Wert der Kennzahl für den Fall der gemeinsam genutzten Datenbank an.
    Eine ausführliche Interpretation dieser Zahlen geschieht in Kapitel \ref{section:findingsInterpretation}.
    Trotzdem wird hier schon die Bedeutung einiger Kennzahlen kurz hervorgehoben.

    Tabelle \ref{table:findingsTeachersFiguresNodesByLabel}
    gruppiert die Knoten der Datenbank nach ihren Labels und zeigt,
    wie oft jedes Label oder jede Kombination von Labels Verwendung fand.   
    Für die einzelnen Klassifikationen gibt "`Content"' außerdem an,
    wie viele {\contentFeature}s sie enthalten.

    \begin{table}[htb]
        \centering
        \begin{tabular}{|l|c|c|c|c|c|c|}
            \hline
            \multicolumn{1}{|c|}{\textbf{Label}} & \textbf{\gls{babw}} & \textbf{\gls{bapvs}} & \textbf{\gls{bscpsy}} & \textbf{\gls{mabm}} & \textbf{Summe} & \textbf{Alle} \\ \hline
            Content                                     & 270           & 275            & 176             & 128           & 849            & 824           \\ \hline
            Page + Resource                             & 1             & 1              & 1               & 1             & 4              & 4             \\ \hline
            Resource                                    & 133           & 159            & 89              & 71            & 452            & 430           \\ \hline
            Site                                        & 1             & 1              & 1               & 1             & 4              & 4             \\ \hline
            Text                                        & 105           & 69             & 75              & 56            & 305            & 284           \\ \hline
            \hline
            \textbf{Summe}                              & 510           & 505            & 342             & 257           & 1614           & 1546          \\ \hline
        \end{tabular}
        \caption{Knoten gruppiert nach Labels für Seiten über Lehrende und Betreuende}
        \label{table:findingsTeachersFiguresNodesByLabel}
    \end{table}

    Die Knoten mit dem Label "`Content"' lassen sich nach der ihnen zugewiesenen Klasse
    weiter aufschlüssen, was in Tabelle \ref{table:findingsTeachersFiguresContentNodesByClass} geschieht.
    Eine Kennzahl in dieser Tabelle ist demnach gleichzusetzen mit der Häufigkeit der Verwendung
    der genannten Klasse in der Klassifikation.

    \begin{table}[htb]
        \centering
        \begin{tabular}{|l|c|c|c|c|c|c|}
        \hline
            \textbf{Klasse}  & \multicolumn{1}{l|}{\textbf{\gls{babw}}} & \multicolumn{1}{l|}{\textbf{\gls{bapvs}}} & \multicolumn{1}{l|}{\textbf{\gls{bscpsy}}} & \multicolumn{1}{l|}{\textbf{\gls{mabm}}} & \multicolumn{1}{l|}{\textbf{Summe}} & \multicolumn{1}{l|}{\textbf{Alle}} \\ \hline
            Brand              & 1                                  & 1                                   & 1                                    & 1                                  & 4                                   & 4                                  \\ \hline
            ContactInformation & 55                                 & 67                                  & 33                                   & 24                                 & 179                                 & 178                                \\ \hline
            Fax                & 1                                  & 0                                   & 0                                    & 1                                  & 2                                   & 1                                  \\ \hline
            Header             & 1                                  & 1                                   & 1                                    & 1                                  & 4                                   & 4                                  \\ \hline
            Introduction       & 1                                  & 0                                   & 1                                    & 1                                  & 3                                   & 2                                  \\ \hline
            PageHeading        & 1                                  & 1                                   & 1                                    & 1                                  & 4                                   & 4                                  \\ \hline
            Phone              & 35                                 & 38                                  & 29                                   & 21                                 & 123                                 & 119                                \\ \hline
            Portal             & 1                                  & 1                                   & 1                                    & 1                                  & 4                                   & 4                                  \\ \hline
            Room               & 2                                  & 0                                   & 0                                    & 0                                  & 2                                   & 2                                  \\ \hline
            SubjectArea        & 53                                 & 67                                  & 33                                   & 22                                 & 175                                 & 172                                \\ \hline
            SubjectAreaName    & 9                                  & 18                                  & 11                                   & 7                                  & 45                                  & 39                                 \\ \hline
            Teacher            & 55                                 & 70                                  & 33                                   & 24                                 & 182                                 & 182                                \\ \hline
            TeacherName        & 55                                 & 11                                  & 32                                   & 24                                 & 122                                 & 113                                \\ \hline
            \hline
            \textbf{Summe}     & 270                                & 275                                 & 176                                  & 128                                & 849                                 & 824                                \\ \hline
        \end{tabular}
        \caption{Content Knoten aufgeteilt nach Klasse}
        \label{table:findingsTeachersFiguresContentNodesByClass}
    \end{table}

    Auch über die Kanten des Graphens lassen sich eine Zahlen ermitteln.
    Tabelle \ref{table:findingTeachersFiguresEdgesByLabel} beginnt dazu
    mit der Aufschlüsselung der Kanten nach ihrem Label.
    Die Kennzahl "`References"' spiegelt die Anzahl der Referenzen innerhalb der Klassifikation wieder.

    \begin{table}[htb]
        \centering
        \begin{tabular}{|l|c|c|c|c|c|c|}
            \hline
            \multicolumn{1}{|c|}{\textbf{Kanten-Label}} & \textbf{\gls{babw}} & \textbf{\gls{bapvs}} & \textbf{\gls{bscpsy}} & \textbf{\gls{mabm}} & \textbf{Summe} & \textbf{Alle} \\ \hline
            Reads                                       & 105           & 69             & 75              & 56            & 305            & 284           \\ \hline
            References                                  & 180           & 209            & 110             & 86            & 585            & 582           \\ \hline
            Owns                                        & 320           & 335            & 200             & 144           & 999            & 996           \\ \hline
            \hline
            \textbf{Summe}                              & 605           & 613            & 385             & 286           & 1889           & 1862          \\ \hline
        \end{tabular}
        \caption{Kanten nach Label}
        \label{table:findingTeachersFiguresEdgesByLabel}
    \end{table}

    Neben dem Label ist in Bezug auf Kanten auch die Frage interessant,
    welche Knoten sie verbinden.
    Tabelle \ref{table:findingsTeachersFiguresEdgesByStartEndNodeLabel}
    zeigt, welche Arten von Knoten wie oft verbunden wurden.
    Die Beziehung eines Content Knotens zu einem Text-Knoten ist nicht enthalten,
    da diese äquivalent zum oben gezeigten Reads-Label ist.
    Diese Tabelle liefert unter anderem Informationen darüber,
    wie viele Referenzen die Seite selbst hat und wie viele zu Content Features gehören.

    \begin{table}[htb]
        \centering
        \begin{tabular}{|l|c|c|c|c|c|c|}
            \hline
            \multicolumn{1}{|c|}{\textbf{Start-, Zielknoten-Label}} & \textbf{\gls{babw}} & \textbf{\gls{bapvs}} & \textbf{\gls{bscpsy}} & \textbf{\gls{mabm}} & \textbf{Summe} & \textbf{Alle} \\ \hline
            (:Content) $\rightarrow$ (:Content)                           & 260           & 260            & 162             & 115           & 797            & 794           \\ \hline
            (:Content) $\rightarrow$ (:Resource)                         & 172           & 201            & 102             & 78            & 553            & 550           \\ \hline
            (:Page) $\rightarrow$ (:Content)                              & 59            & 74             & 37              & 28            & 198            & 198           \\ \hline
            (:Page) $\rightarrow$ (:Resource)                             & 8             & 8              & 8               & 8             & 32             & 32            \\ \hline
            (:Site) $\rightarrow$ (:Page)                                 & 1             & 1              & 1               & 1             & 4              & 4             \\ \hline
            \hline
            \textbf{Summe}                                          & 500           & 544            & 310             & 230           & 1584           & 1578          \\ \hline
        \end{tabular}
        \caption{Kanten nach Start und Zielknoten}
        \label{table:findingsTeachersFiguresEdgesByStartEndNodeLabel}
    \end{table}

    Eine letzte zu beantwortende Frage ist,
    wie viele Knoten in der Datenbank mehr als eine eingehende Kante haben,
    d.h., wie oft sie an verschiedenen Stellen einer oder mehrerer Klassifikationen Verwendung finden.

    \begin{table}[htb]
        \centering
        \begin{tabular}{|l|c|c|c|c|c|c|}
            \hline
            \multicolumn{1}{|c|}{\textbf{Knoten}} & \textbf{\gls{babw}} & \textbf{\gls{bapvs}} & \textbf{\gls{bscpsy}} & \textbf{\gls{mabm}} & \textbf{Summe} & \textbf{Alle} \\ \hline
            Bild                                  & 0             & 1              & 1               & 0             & 2              & 2             \\ \hline
            ContactInformation                    & 0             & 3              & 0               & 0             & 3              & 3             \\ \hline
            E-Mail-Adresse                        & 0             & 1              & 1               & 0             & 2              & 14            \\ \hline
            Fax                                   & 0             & 0              & 0               & 0             & 0              & 1             \\ \hline
            Hauptseite "`Studium"'                & 1             & 0              & 0               & 0             & 1              & 1             \\ \hline
            Homepage der FU                       & 0             & 0              & 0               & 0             & 0              & 1             \\ \hline
            Introduction                          & 0             & 0              & 0               & 0             & 0              & 1             \\ \hline
            Lehrgebietsseiten                     & 7             & 16             & 8               & 5             & 36             & 30            \\ \hline
            Phone                                 & 4             & 1              & 0               & 0             & 5              & 9             \\ \hline
            SubjectArea                           & 0             & 2              & 0               & 0             & 2              & 5             \\ \hline
            SubjectAreaName                       & 8             & 16             & 7               & 5             & 36             & 31            \\ \hline
            Teacher                               & 0             & 1              & 0               & 0             & 1              & 1             \\ \hline
            TeacherName                           & 0             & 3              & 1               & 0             & 4              & 13            \\ \hline
            \hline
            \textbf{Summe}                        & 20            & 44             & 18              & 10            & 92             & 112           \\ \hline
        \end{tabular}
        \caption{Mehrfach referenzierte Knoten}
        \label{table:findingsTeachersFiguresSharedNodes}
    \end{table}

    \subsection{Visualisierung der Klassifikation durch Annotationen}
    Eine wichtige Funktion des \glspl{wccs}
    ist die Visualisierung einer Klassifikation durch Webannotationen
    auf der klassifizierten Webseite.
    Aus diesem Grund folgt eine Übersicht der Annotationen
    des Studienportals \gls{babw},
    welches stellvertretend auch für die restlichen klassifizierten Portale steht.
    Abbildung \ref{image:findingTeachersAnnotationsOverview}
    zeigt einen Ausschnitt der annotierten
    Seite\footnote{Die Darstellungsfehler oben rechts im Kopfbereich
    sowie am Anfang der Brotkrümelnavigation unter dem Portal
    sind dem in Kapitel \ref{section:findingsMethod} beschriebenen
    Annotation Viewer geschuldet.
    Durch die Zwischenschaltung dieser Komponente
    führt der Browser Cross-Origin-Requests durch,
    um die genutzte Bibliothek für Symbole zu beziehen.
    Diese Aufrufe werden allerdings unterbunden,
    weshalb die Symbole nicht korrekt dargestellt werden.
    Bei einer direkten Einbindung des Plugins wäre dies nicht der Fall.
    Der fünfte Link im Kopfbereich wurde richtig klassifiziert.}.
    Bis auf wenige Ausnahmen, die in
    Kapitel \ref{section:findingsTeachersAbnormalitiesBabw} besprochen werden,
    wurden alle klassifizierten Elemente korrekt hervorgehoben.
    Eine detaillierte Ansicht einer beispielhaften Annotation zeigt
    Abbildung \ref{image:findingTeachersSubjectAreaAnnotations}.
    Das Lehrgebiet besitzt korrekterweise zwei Annotationen,
    da das HTML-Element sowohl den Namen als auch den Link enthält
    und deshalb doppelt klassifiziert wurde.

    \begin{figure}[htb]
        \centering
        \includegraphics[width=\textwidth]{../resources/findings/case-study-1/babw/annotations/overview.png}
        \caption{Die annotierte Webseite über Mitarbeiter des Portals \acrshort{babw}}
        \label{image:findingTeachersAnnotationsOverview}
    \end{figure}

    \begin{figure}[htb]
        \centering
        \includegraphics[scale=\screenshotScaleFactor]{../resources/findings/case-study-1/babw/annotations/double-lg-annotation.png}
        \caption{Die Annotationen eines Lehrgebietes}
        \label{image:findingTeachersSubjectAreaAnnotations}
    \end{figure}

    \subsection{Unregelmäßigkeiten in der Klassifikation des Portals "`Babw"'}
    \label{section:findingsTeachersAbnormalitiesBabw}
    Dieses Kapitel stellt die Auffälligkeiten in der Klassifikation
    der Seite der Lehrenden und Betreuenden im Studienportal
    "`B.A. Bildungswissenschaft"' vor.
    Zusätzlich erfolgt eine Erklärung,
    wodurch die jeweilige Unregelmäßigkeit begründet ist.

    \paragraph{Zwei Mitarbeiter ohne Lehrgebiet}
    Die Klassifikation enthält für zwei Mitarbeiter kein Lehrgebiet.
    Im ersten Fall nennt die Webseite kein Lehrgebiet,
    weshalb die Klassifikation an dieser Stelle korrekt ist.
    Der zweite Kontakt ist ein Mitarbeiter einer fremden Universität,
    weshalb dessen Lehrgebiet kein Verweis auf eine andere Seite,
    sondern einfacher Text ist.
    Der verwendete Selektor "`div.team-member-des > p > a:first-child"'
    hat diesen Text nicht erfasst, da er einen Link sucht.

    \paragraph{Unvollständige Namen zweier Mitarbeiter}
    Zwei Kontakte besitzen laut Klassifikation den Namen "`Prof."' bzw. "`Dr."'.
    In diesen Fällen stehen Titel und Name in getrennten strong-Elementen.
    Da der Name eines Mitarbeiters ein skalares Feature ist,
    wurde nur das erste Element vom System erfasst.

    \paragraph{Falsche und fehlende Hervorhebungen durch Annotationen}
    Einige Elemente der Seite wurden durch das Annotator inkorrekt oder nicht hervorgehoben.
    Aus Abbildung \ref{image:findingTeachersAnnotationsOverview} geht bereits hervor,
    dass Bilder hiervon betroffen sind und Annotator diese markiert.
    Abbildung \ref{image:findingTeachersBaBwWrongAnnotations}
    veranschaulicht, dass Telefonnummern und Raumangaben oftmals verschoben annotiert werden.
    Der Grund ist die in Kapitel \ref{section:solutionDetailsClassificationServiceClassification}
    beschriebene Konflikt bei der Bestimmung eines eindeutigen Selektor.
    Allerdings ist auch zu beobachten, dass einige Angaben,
    wie die E-Mail-Adresse in Abbildung \ref{image:findingTeachersBaBwWrongAnnotations}
    überhaupt nicht markiert werden.

    \begin{figure}[htb]
        \centering
        \includegraphics[width=0.5\textwidth]{../resources/findings/case-study-1/babw/annotations/missing-annotation.png}
        \caption{Fehlerhafte Hervorhebung durch Annotationen}
        \label{image:findingTeachersBaBwWrongAnnotations}
    \end{figure}

    \subsection{Unregelmäßigkeiten in der Klassifikation des Portals "`\acrshort{bapvs}"'}
    \label{section:findingsTeachersAbnormalitiesBaPVS}
    Die Klassifikation des Lehrgebietes \gls{bapvs}
    weist ebenfalls einige Unregelmäßigkeiten auf,
    die dieses Kapitel beschreibt.

    \paragraph{Doppelt klassifizierte Mitarbeiter}
    Drei Mitarbeiter des Portals wurden doppelt klassifiziert.
    Der Grund ist eine abweichende HTML-Struktur bei diesen Mitarbeitern,
    die in Listing \ref{listing:findingsTeachersBaPVSHtmlSource} zu sehen ist.

    \lstinputlisting[
        label=listing:findingsTeachersBaPVSHtmlSource,
        caption=Abweichende HTML-Struktur eines Mitarbeiters im Portal BaPVS,
        style=html
    ]{../resources/findings/case-study-1/bapvs/teacher.html}

    Im Vergleich zur erwarteten Struktur\footnote{vgl. Listing \ref{listing:findingsTeachersHtmlSource}} wiederholt sich
    das \texttt{div}-Element mit der Klasse \texttt{grid},
    weshalb der verwendete Selektor \texttt{section\#content div.grid}
    betroffene Mitarbeiter doppelt erfasst.
    Im Fall von zwei Mitarbeitern kann das System bei der zweiten Klassifizierung
    (beim inneren \texttt{div}-Element) kein Bild finden,
    weshalb sie von der ersten abweicht und ein zweiter \texttt{Teacher}-Knoten angelegt wird.
    Diese teilen sich aber die restlichen Informationen,
    also Name, Lehrgebiet und Kontaktinformationen.
    Im dritten Fall besitzt der Mitarbeiter kein Bild,
    weshalb der \texttt{Teacher}-Knoten vollständig wiederverwendet werden kann.

    \paragraph{Mitarbeiter ohne Lehrgebiet}
    Das Lehrgebiet eines einzelnen Mitarbeiters wurde vom System nicht erfasst.
    Anders als beim Lehrgebiet \gls{babw} ist der Grund allerdings,
    dass vor dem Lehrgebiet der Text "`Auskunft erteilt auch:"' platziert ist.
    Der Selektor \texttt{div.team-member-des > p > a:first-child} findet
    kein Lehrgebiet, weil er nach einem \texttt{a}-Element sucht,
    welches das erste Kindelement seines Vaterelementes ist.
    Eine naheliegende Lösung ist die Änderung des Selektors,
    sodass er mit \texttt{a:first-of-type} endet.
    Bei der Erstellung des {\classificationModel}s,
    was auf Basis des Portals \gls{babw} geschah,
    wurde sich allerdings bewusst gegen diese Variante entschieden.
    Bei dem Mitarbeiter im \gls{babw}, für den kein Lehrgebiet erfasst
    wurde\footnote{vgl. Kapitel \ref{section:findingsTeachersAbnormalitiesBabw}},
    hätte dieser Selektor nämlich dazu geführt,
    dass seine E-Mail-Adresse als Lehrgebiet erkannt wird.

    \paragraph{Falscher Text als Name klassifiziert}
    Der Name eines gewissen Mitarbeiters ist laut Klassifikation
    "`Auskunft erteilt auch:"'.
    Der Grund ist, dass dieser Text in einem \texttt{strong}-Element steht,
    welches vor dem Namen auftaucht.
    Das System hat in diesem Fall erwartungskonform nur den ersten
    Treffer für das skalare Feature klassifiziert.
        
    \paragraph{28 Mitarbeiter ohne Telefonnummer}
    Einige Kontakte besitzen in der Klassifikation keine Telefonnummer.
    Bei neun ist auch auf der Webseite keine zu finden.
    Bei den restlichen ist der Nummer nicht "`Tel.: "'
    sondern "`Telefon: "' oder "`Tel:"' vorangestellt.
    Der Selektor hat die Telefonnummern deshalb nicht erkannt.
    
    \paragraph{56 Mitarbeiter ohne Namen}
    Des Weiteren wurde für eine Vielzahl der Mitarbeiter kein Name erkannt,
    da sie weder in einem \texttt{strong} noch einem \texttt{b} Element stehen.
    Stattdessen befinden sie sich wie die Telefonnummer als reiner Text im Element
    oder in einem \texttt{a}-Element.

    \paragraph{Inkorrekte Telefonnummern}
    Die Klassifikation enthält einige Telefonnummern,
    die neben der eigentlichen Nummer auch weitere Informationen enthalten.
    Ein Beispiel ist "`02331/987-4315 email: Lisa.Schaefer Sprechstunde: nach Vereinbarung via e-mail"'.
    Die Telefonnummer wird über einen {\xpathSelector} erfasst,
    der auf dem Seitenquelltext ausgeführt wird.
    Anders als beim Portal \gls{babw} existieren in \gls{bapvs} Mitarbeiter,
    bei denen die Telefonnummer weder die letzte Angabe ist
    noch durch einen physischen Zeilenumbruch im Quelltext gefolgt wird.
    Der Selektor der Telefonnummer erkennt sie deshalb
    nicht\footnote{vgl. Kapitel \ref{section:findingsTeachersClassificationModel}}.

    \paragraph{Unterschiede im konzeptionellen Modell der Seite}
    Im Vergleich zur klassifizierten Seite des Portals \gls{babw}
    sind auch zwei Unterschiede im konzeptionellen Modell der Seite deutlich geworden.
    Der Name eines Mitarbeiters ist in einigen Fällen
    auch ein Link auf eine Detailseite.
    Außerdem besitzen einige Mitarbeiter Sprechzeiten.

    \paragraph{Falsch annotierte Telefonnummern}
    Bei der Betrachtung der Annotationen des Portals \gls{babw}
    ist bereits aufgefallen, dass sie für Telefonnummern
    mehrmals verschoben sind.
    Dies ist im Fall des Portals \gls{bapvs} ebenfalls zu beobachten,
    allerdings in einem deutlicheren Ausmaß,
    wie Abbildung \ref{image:findingTeachersBaPVSWrongPhone} zeigt.

    \begin{figure}[htb]
        \centering
        \includegraphics[scale=\screenshotScaleFactor]{../resources/findings/case-study-1/bapvs/annotations/triple-annotation.png}
        \caption{Verschobene Annotation einer Telefonnummer im \acrshort{bapvs}}
        \label{image:findingTeachersBaPVSWrongPhone}
    \end{figure}

    Bei der Bestimmung des eindeutigen Selektors,
    wird die Position des klassifizierten Textes in der Eigenschaft
    \texttt{innerText} als Versatz im Startelement
    verwendet\footnote{vgl. Kapitel \ref{section:solutionDetailsClassificationServiceClassification}}.
    In den betroffenen Fällen befindet sich zwischen der Telefonnummer
    und der E-Mail-Adresse nur ein \texttt{br}-Element,
    aber kein physischer Zeilenumbruch im
    Seitenquelltext\footnote{vgl. Listing \ref{listing:findingsTeachersHtmlSource}}.
    Die Telefonnummer, die über einen {\xpathSelector} ermittelt wird,
    enthält deshalb zu viele Informationen (siehe oben),
    die wiederum nur durch ein Leerzeichen, aber nicht durch einen Zeilenumbruch getrennt sind.
    In \texttt{innerText} wird das \texttt{br}-Element aber zu einem Umbruch,
    weshalb der klassifizierte Text in dieser Eigenschaft nicht gefunden wird.
    Als Versatz wird deshalb $-1$ gespeichert.
    Annotator setzt die Annotation deshalb an den Anfang des umschließenden Elementes.

    \subsection{Unregelmäßigkeiten in der Klassifikation des Portals \acrshort{bscpsy}}
    Auch bei der Klassifizierung des Portals \gls{bscpsy}
    sind im Vergleich zu den vorherigen Portalen zwei neue
    Auffälligkeiten aufgetreten.

    \paragraph*{Leere Einleitung}
        Die Klassifikation der Webseite enthält eine Einleitung,
        die aber nur ein Leerzeichen enthält.
        Die Webseite besitzt einen Absatz,
        auf den der Selektor zutrifft.
        Dieser Absatz enthält aber nur das Zeichen \texttt{\&nbsp;}.

    \paragraph*{Zwei Mitarbeiter ohne E-Mail-Adresse}
        Des Weiteren wurden zwei Mitarbeiter ohne E-Mail-Adresse klassifiziert.
        Bei einem ist dies korrekt, da die Webseite ebenfalls keine Adresse enthält.
        Im anderen Fall wurde sie nicht erkannt,
        da die \gls{uri} kein vorangestelltes \texttt{mailto:} enthält
        und deshalb nicht vom Selektor erfasst wurde.
        
    \subsection{Unregelmäßigkeiten in der Klassifikation des Portals "`MaBm"'}
    Nicht zuletzt enthält auch die Klassifikation des Portals
    "`M.A. Bildung und Medien: eEducation"'
    zwei Auffälligkeiten,
    die hier kurz aufgezeigt werden.

    \paragraph{Zwei Lehrende ohne Lehrgebiet}
    Im Falle zweier Mitarbeiter wurde kein Lehrgebiet erkannt.
    Der Grund ist eine erneute andere Struktur des \glspl{html}.
    In diesen Fällen ist der Link auf das Lehrgebiet
    in ein strong Element eingebettet, weshalb es nicht erkannt wurde.

    \paragraph{Inkorrekte Nabem zweier Mitarbeiter}
    Bei den gleichen Mitarbeitern enthält ihr Name außerdem ihr Lehrgebiet.
    Wie beschrieben enthält das strong Element das Lehrgebiet,
    aber auch den Namen des Mitarbeiters.
    Der Selektor des Namens des Mitarbeiters erfasst genau dieses strong Element.

