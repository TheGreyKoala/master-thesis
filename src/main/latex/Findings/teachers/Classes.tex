\subsection{Das {\classificationModel}}
    \label{section:findingsTeachersClassificationModel}
    Aus der obigen Analyse der Webseite ist hervorgegangen,
    dass sie allgemeine und spezielle Bereiche enthält.
    Das entstandene {\classificationModel} wurde deshalb in zwei Dateien aufgeteilt,
    sodass die Klassen der allgemeinen Bereiche leicht wiederverwendet werden können.
    Listing \ref{listing:findingsTeachersCommon} zeigt diesen allgemeinen
    Teil des {\classificationModel}s.

    \lstinputlisting[
        label=listing:findingsTeachersCommon,
        caption=Das {\classificationModel} der Seite \"uber Lehrende und Betreuende (1),
        language=wccdl,
        inputencoding=utf8/latin1,
        style=wccdl
    ]{../resources/findings/case-study-1/classification-model/Common.wctd}

    Die Klasse \texttt{Header} repräsentiert den Kopfbereich der Seite
    und besitzt entsprechende Features,
    die das Logo sowie die Links klassifizieren.
    Das Logo wird durch die Inhaltsklasse \texttt{Brand} modelliert,
    die sowohl ein {\referenceFeature} für das Bild selbst als auch für den Link enthält.
    Für den Namen des Portals und den zugehörige Link existiert die Klasse \texttt{Portal}.
    Die Überschrift der Seite wird als \texttt{PageHeading} klassifiziert und
    der einleitende Absatz als \texttt{Introduction}.
    Diese Klasse erkennt durch das Feature \texttt{links} außerdem Verweise auf andere Seiten der {\fernUni}.
    Wie die Links im Kopfbereich werden sie als \texttt{FernUniInternalLink} klassifiziert.
    Für Bilder wird in dieser Datei die Referenzklasse \texttt{Image} definiert,
    die auch in der zweiten Datei Verwendung findet.
    Aus dieser Datei, die in Listing \ref{listing:findingsTeachersSpecial} zu sehen,
    geht auch hervor, wie die Navigationspunkte im linken Bereich erfasst werden.

    \lstinputlisting[
        label=listing:findingsTeachersSpecial,
        caption=Das {\classificationModel} der Seite \"uber Lehrende und Betreuende (2),
        language=wccdl,
        inputencoding=utf8/latin1,
        style=wccdl
    ]{../resources/findings/case-study-1/classification-model/Teachers.wctd}

    Dieser Teil des Modells beginnt mit der Seitenklasse \texttt{Teachers},
    die Features für die allgemeinen Bereiche der Webseite enthält.
    Diese verwenden die Klassen der vorangegangenen Datei.
    Eines dieser Features ist \texttt{sidebarNavigationLinks},
    welches die Verweise auf der linken Seite als \texttt{FernUniInternalLink}
    klassifiziert.
    Ein einzelner Mitarbeiter wird durch die Klasse \texttt{Teacher} dargestellt,
    sein Name durch \texttt{TeacherName}.
    Ein Lehrgebiet wird als \texttt{SubjectArea} klassifiziert,
    ihr Name wiederum als \texttt{SubjectAreaName}.
    Die Kontaktinformationen eines Mitarbeiters kapselt die Klasse \texttt{ContactInformation},
    wobei die einzelnen Angaben als \texttt{Phone}, \texttt{Email}, \texttt{Fax} oder \texttt{Room}
    klassifiziert werden.

    Für spätere Erläuterungen ist es interessant zu wissen,
    wie die HTML-Repräsentation eines Mitarbeiters ist.
    Diese ist in Listing \ref{listing:findingsTeachersHtmlSource} zu sehen.
    Konkrete Inhalte wurden aus Gründen der Übersichtlichkeit entfernt.
    Physische Zeilenumbrüche (keine \texttt{br}-Elemente) im \texttt{p}-Element
    spielen später allerdings eine Rolle,
    weshalb sie unverändert übernommen wurden.

    \lstinputlisting[
        label=listing:findingsTeachersHtmlSource,
        caption=HTML-Repräsentation eines Mitarbeiters des \acrshort{babw},
        style=html
    ]{../resources/findings/case-study-1/babw/teacher.html}