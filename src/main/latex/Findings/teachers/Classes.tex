\subsection{Klassifizierungsmodell}
    Aus der obigen Analyse der Seite ist ein Klassifizierungsmodell hervorgegangen,
    welches sich in allgemeine und spezielle Klassen aufteilen lässt.
    Die Definition wurde entsprechend auf zwei Dateien aufgeteilt,
    sodass sich die allgemeinen Klassen leicht übertragen lassen.

    Die Klassen der allgemeinen Bereiche sind in Listing
    \ref{listing:findingsTeachersCommon} aufgeführt.
    Es folgt eine kurze Erläuterung,
    welche Elemente der Seite die einzelnen Klassen darstellen.

    \lstinputlisting[
        label=listing:findingsTeachersCommon,
        caption=Klassendefinition für allgemeine Bereiche,
        language=wccdl,
        inputencoding=utf8/latin1,
        style=wccdl
    ]{../resources/findings/case-study-1/classification-model/Common.wctd}

    Die Klasse "`Header"' repräsentiert den Kopfbereich der Seite
    und besitzt entsprechend Features,
    die das Logo sowie die Links klassifizieren.
    Das Logo wird durch die Klase "`Brand"' modelliert,
    die sowohl ein Feature für das Bild selbst als auch den Link enthält.
    Für den Namen des Portals und den zugehörige Link existiert die Klasse "`Portal"'.
    Die Überschrift der Seite wird als "`PageHeading"' klassifiziert und
    der einleitende Absatz als "`Introduction"', die außerdem Links auf andere
    Seiten der {\fernUni} erkennt.
    Wie die Links im Kopfbereich werden sie als "`FernUniInternalLink"' klassifiziert.

    Zusätzlich wird in dieser Datei für Bilder die Klasse "`Image"' definiert,
    die in der zweiten Datei Verwendung findet.
    Aus dieser geht auch hervor, wie die Navigationspunkte im linken Bereich erfasst werden.
    Sie ist in Listing \ref{listing:findingsTeachersSpecial} zu sehen.

    \lstinputlisting[
        label=listing:findingsTeachersSpecial,
        caption=Klassendefinition für Lehrende und Betreuende,
        language=wccdl,
        inputencoding=utf8/latin1,
        style=wccdl
    ]{../resources/findings/case-study-1/classification-model/Teachers.wctd}

    Zunächst wird die Seitenklasse "`Teachers"' definiert,
    inklusive einiger Features für die allgemeinen Bereiche der Seite,
    wofür die Klassen der vorangegangenen Datei genutzt werden.
    Eines dieser Features ist "`sidebarNavigationLinks"',
    welches die Navigationslinks auf der linken Seite als "`FernUniInternalLink"'
    klassifiziert.
    Ein einzelner Mitarbeiter wird durch die Klasse "`Teacher"' dargestellt
    sowie sein Name durch "`TeacherName"'.
    Ein Lehrgebiet mit Namen und Link wird als "`SubjectArea"' bzw. "`SubjectAreaNem"' klassifiziert.
    Die Kontaktinformationen eines Mitarbeiters kapselt die Klasse "`ContactInformation"',
    wobei die einzelnen Angaben durch die Klassen "`Phone"', "`Email"', "`Fax"' und "`Room"'
    repräsentiert werden.

    % TODO: Erwähnen, dass die Klassen iterativ erstellt wurden?