\subsection{Unregelmäßigkeiten}
    Im Vergleich zum ersten Fallbeispiel
    wurden keine neuen Auffälligkeiten in den Klassifikationen entdeckt.
    Aufgefallen ist aber das Ergebnis des {\wordpressCrawler}s,
    der zum Auffinden der Seiten und zum Starten der Klassifizierung
    in beiden Fallbeispielen Anwendung fand.   
    Jede Meldung ist in {\wordpress} ein Post der Kategorie "`Aktuelles"'.
    Die Übersichtsseite dieser Kategorie enthält eine {\wordpress}-eigene
    Funktion \cite[Kapitel "`Pagination"']{wordpress:codex},
    durch die nur eine Teilmenge der Einträge der Kategorie angezeigt wird.
    Ihre Parameterbelegung zieht die Funktion dynamisch aus einer Erweiterung der
    \gls{url} der Kategorieseite.
    Ein Webseitenbesucher kann so durch die Meldungen navigieren.
    {\wordpress} besitzt also keine dedizierten Beiträge oder Seiten
    für die einzelnen Nachrichtenseiten eines Portals.
    Stattdessen wird eine parametrisierbare Kategorieseite verwendet.
    Der {\wordpressCrawler} findet deshalb nur die allgemeine Übersichtsseite der Kategorie "`Aktuelles"',
    die den ersten Block an Meldungen enthält.
    Weitere Nachrichtenseiten findet er nicht,
    da {\wordpress} ihn nicht über die Aufteilung informiert.
