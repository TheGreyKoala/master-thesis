\subsection{Unregelmäßigkeiten}
    Im Vergleich zum ersten Fallbeispiel
    wurden keine neuen Auffälligkeiten in den Klassifikationen entdeckt.
    Aufgefallen ist aber das Ergebnis des {\wordpressCrawler}s,
    der zum Auffinden der Seiten und zum Starten der Klassifizierung
    in beiden Fallbeispielen Anwendung fand.   

    Jede Meldung ist in {\wordpress} ein Beitrag der Kategorie "`Aktuelles"'.
    Die Übersichtsseite dieser Kategorie enthält eine {\wordpress}-eigene
    Funktion\footnote{vgl. \url{https://codex.wordpress.org/Pagination}},
    durch die nur eine Teilmenge der Einträge der Kategorie eingezeigt werden.
    Ihre Parameterbelegung zieht die Funktion dynamisch aus einer Erweiterung der
    \gls{url} der Kategorieseite.
    Ein Webseitenbesucher kann so durch die Meldungen navigieren.
    {\wordpress} besitzt also keine dedizierten Beiträge oder Seiten
    für die einzelnen Nachrichtenseiten einer Site.
    Stattdessen wird eine parametrisierbare Kategorieseite verwendet.
    Der {\wordpressCrawler} findet deshalb nur die allgemeine Übersichtsseite der Kategorie "`Aktuelles"',
    die den ersten Block Meldungen enthält.
    Weitere Nachrichtenseiten findet er hingegen nicht,
    da {\wordpress} ihn nicht über deren Existenz informiert.
