\subsection{Unregelmäßigkeiten}
    Im Vergleich zum ersten Fallbeispiel in Kapitel \ref{section:findingsCaseStudy1}
    wurden keine neuen Auffälligkeiten ermittelt in den Klassifikationen entdeckt.

    Allerdings ist dies der Fall für den {\wordpress}-Crawler,
    der zum Auffinden der Seiten und zum Starten der Klassifizierung
    in beiden Fallbeispielen Anwendung fand.

    Wie bereits erwähnt sind die verschiedenen Meldungen auf mehrere Seiten verteilt.
    Dazu wird eine {\wordpress}-eigene Funktion verwendet
    \footnote{vgl. \url{https://codex.wordpress.org/Pagination}}.
    Die einzelnen Meldungen sind der Kategorie "`Aktuelles"' zugewiesen
    und eine Übersichtsseite ist eine Sicht auf alle Einträge
    dieser Kategorie, die mithilfe der genannten Pagination-Funktionen
    parametrisiert wird und so verschiedene Seiten von Meldungen erzeugt.

    Das bedeutet, dass {\wordpress} für die einzelnen Übersichtsseiten keine Posts oder Pages enthält.
    Der {\wordpress}-Crawler, der genau diese Einträge abfragt,
    findet die verschiedenen Übersichtsseiten deshalb nicht.
    Stattdessen teilt ihm {\wordpress} lediglich die
    generelle \gls{url} der Kategorieseite "`Aktuelles"' mit.
    Dadurch findet er zumindest die erste Nachrichtenübersichtsseite.