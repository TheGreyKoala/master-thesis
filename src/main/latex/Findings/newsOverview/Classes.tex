\subsection{Klassifizierungsmodell}
    Bei der Analyse der Nachrichtenseite wurde deutlich,
    dass einige Teile konzetionell identisch mit denen aus dem ersten Beispiel sind.
    Die allgemeinen Klassendefinitionen aus Listing \ref{listing:findingsTeachersCommon}
    werden deshalb hier wiederverwendet.

    Die speziellen Definitionen für die Nachrichtenseite sind in
    Listing \ref{listing:findingsNewsSpecial}
    aufgeführt\footnote{Der Selektor in den Zeile \ref{line:newsWctdInvalidSelectorStart}
    bis \ref{line:newsWctdInvalidSelectorEnd} wurde aus Gründen der Lesbarkeit umgebrochen,
    was syntaktisch nicht erlaubt ist.}.

    \lstinputlisting[
        label=listing:findingsNewsSpecial,
        caption=Klassendefinition für die Nachrichtenübersichtsseite,
        language=wccdl,
        inputencoding=utf8/latin1,
        style=wccdl,
        escapechar=|
    ]{../resources/findings/case-study-2/classification-model/NewsOverview.wctd}

    Die Seitenklasse "`NewsOverview"' enthält neben Features für die
    allgemeinen Bereiche der Seite außerdem welche für die vorherige und die nächste
    Seite, die als "`FernUniInternalLink"' klassifiziert werden,
    sowie für die Liste der Meldungen.

    Eine Nachricht wird durch die Klasse "`News"' repräsentiert,
    ihr Datum durch "`NewsDate"', ihre Überschrift durch "`NewsHeading"'
    und ihre Absätze durch "`NewsSection"'.
    Da die Überschrift auch einen Link enthält,
    wird dieser ebenfalls aus "`FernUniInternalLink"' klassifiziert.

    Die Klasse "`NewsSection"' besitzt keinen Selektor und wurde wie das
    korrespondierende Feature in "`News"' auskommentiert.
    Der Grund ist, dass für die Absätze einer Nachricht kein Selektor ermittelt werden konnte,
    mit dem eine sinnvolle Klassifizierung möglich wäre.
    Um den Grund hierfür zu erklären, ist eine Betrachtung eines
    Ausschnittes der \gls{html}-Struktur notwendig.
    Ein solcher ist in Listing \ref{listing:findingsNewsEvilHtmll}
    zu sehen, wobei die konkreten textuellen Inhalte entfernt wurden,
    da sie für das Verständnis der Problematik irrelevant sind.

    \lstinputlisting[
        label=listing:findingsNewsEvilHtmll,
        caption=HTML-Struktur einer Nachricht,
        language=HTML
    ]{../resources/findings/case-study-2/news.html}

    Wie aus diesem HTML-Fragment ersichtlich wird,
    werden Nachrichten nicht in einzelne Nodes gekapselt.
    Stattdessen befinden sich die Knoten alle auf einer Ebene.
    Eine Trennung der Nachrichten geschieht nur durch das hr-Element,
    welches es ermöglicht einzelne News zu identifizieren.
    Aufgrund der fehlenden Kapselung ist es aber mit den zur Verfügung
    stehenden CSS- und XPath-Selektoren nicht möglich festzustellen,
    welche Elemente als "`NewsSection"' zu klassifizieren sind.
    Der CSS-Selektor "`p"' würde beispielsweise alle Absätze aller Nachrichten selektieren.
    Erschwerend kommt hinzu, dass Absätze nicht nur p,
    sondern auch andere Elemente sein können und dass jede Nachricht eine andere Anzahl an Absätzen hat.
    Informell ausgedrückt wäre ein Selektor notwendig,
    der ausgehend vom aktuellen Kontext-Element (das hr-Element) alle Elemente
    ab dem ersten h4-Element bis zum nächsten hr selektiert.
    Das ist mit CSS nicht möglich, da es keine Sibling-Selektoren gibt.
    XPath besitzt diese zwar, allerdings konnte auch bei diesen keine
    passende Filterung umgesetzt werden, die nur die Absätze der aktuellen Nachricht matcht.

    Die einzige Lösung sind spezielle Klassen für jede einzelne Nachricht,
    die speziell auf sie zugeschnitten sind und deshalb Selektoren verwenden,
    die einzelne Elemente im HTML-Dokument eindeutig selektieren.
