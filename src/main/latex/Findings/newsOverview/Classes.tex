\subsection{Das {\classificationModel}}
    \label{section:findingsNewsClasses}
    Bei der Analyse der Nachrichtenseite wurde deutlich,
    dass einige Teile konzeptionell identisch mit denen aus dem ersten Beispiel sind.
    Die allgemeinen Klassendefinitionen aus Listing \ref{listing:findingsTeachersCommon}
    werden deshalb hier wiederverwendet.
    Die speziellen Definitionen für Nachrichtenseiten sind in
    Listing \ref{listing:findingsNewsSpecial}
    aufgeführt\footnote{Der Selektor in den Zeilen \ref{line:newsWctdInvalidSelectorStart}
    bis \ref{line:newsWctdInvalidSelectorEnd} wurde aus Gründen der Lesbarkeit umgebrochen,
    was syntaktisch nicht erlaubt ist.}.

    \lstinputlisting[
        label=listing:findingsNewsSpecial,
        caption=Das {\classificationModel} der Webseite \"uber aktuelle Meldungen,
        language=wccdl,
        inputencoding=utf8/latin1,
        style=wccdl,
        escapechar=|
    ]{../resources/findings/case-study-2/classification-model/NewsOverview.wctd}

    Die Seitenklasse \texttt{NewsOverview} enthält Features für die
    allgemeinen Bereiche der Seite sowie für die Referenzen auf die vorherige und die nächste
    Nachrichtenseite. Diese {\referenceFeature}s werden alle als \texttt{FernUniInternalLink} klassifiziert.
    Nicht zuletzt besitzt \texttt{NewsOverview} ein {\collectionFeature} für die Meldungen.
    Eine Nachricht wird durch die Klasse \texttt{News} repräsentiert,
    ihr Datum durch \texttt{NewsDate}, ihre Überschrift durch \texttt{NewsHeading}
    und ihre Absätze durch \texttt{NewsSection}.
    Die Überschrift enthält auch einen Link,
    der ebenfalls als \texttt{FernUniInternalLink} klassifiziert wird.
    Das Feature \texttt{sections} wurde auskommentiert.
    Der Grund ist, dass für die Absätze einer Nachricht kein Selektor ermittelt werden konnte,
    mit dem eine sinnvolle Klassifizierung möglich ist.
    Um dies zu erklären, ist eine Betrachtung der \gls{html}-Struktur der Nachrichtenliste notwendig.
    Diese Struktur ist in Listing \ref{listing:findingsNewsEvilHtmll}
    zu sehen, wobei konkrete textuelle Inhalte entfernt wurden,
    da sie für das Verständnis der Problematik irrelevant sind.

    \lstinputlisting[
        label=listing:findingsNewsEvilHtmll,
        caption=Die HTML-Repräsentation der Nachrichtenliste der Seiten über aktuelle Meldungen,
        style=html
    ]{../resources/findings/case-study-2/news.html}

    Aus diesem Fragment geht hervor,
    dass Nachrichten nicht in einzelne \gls{html}-Elemente gekapselt sind.
    Stattdessen befinden sich alle Elemente aller Nachrichten auf derselben Hierarchieebene.
    Nachrichten werden nur durch ein \texttt{hr}-Element voneinander getrennt,
    was sich das {\classificationModel} zunutze machen kann.
    Aufgrund der fehlenden Kapselung ist es aber mit den zur Verfügung
    stehenden CSS- und XPath-Selektoren nicht möglich festzustellen,
    welche Elemente als \texttt{NewsSection} zu klassifizieren sind.
    Der {\cssSelector} \texttt{p} würde bspw. alle Absätze aller Nachrichten selektieren.
    Erschwerend kommt hinzu, dass Absätze nicht nur \texttt{p},
    sondern auch andere Elemente sein können und dass jede Nachricht eine andere Anzahl an Absätzen besitzt.
    Informell ausgedrückt wäre ein Selektor notwendig,
    der ausgehend vom aktuellen Kontextelement (das \texttt{hr}-Element) alle Elemente
    ab dem ersten \texttt{h4}-Element bis zum nächsten \texttt{hr} selektiert.
    Das ist mit CSS nicht möglich, da es keine Selektoren für Geschwister des Kontextelementes gibt.
    XPath besitzt diese zwar, allerdings konnte auch mit diesen kein
    Ausdruck umgesetzt werden, der nur die Absätze der aktuellen Nachricht herausfiltert.
    Die einzige Lösung -- ohne das HTML anzupassen -- ist für jede Nachricht eine spezielle Klasse zu entwickeln.
    Durch eine solche 1:1-Beziehung kann jede Klasse individuelle und gleichzeitig
    eindeutige Selektoren für die Absätze enthalten.
