\subsection{Funktionen und Schnitstellen}
    \label{section:solutionDetailsClassificationServiceFunctions}
    Die Beschreibung des {\classificationService}s beginnt mit einem
    Überblick über seine Funktionen und Schnittstellen.
    Die Schnittstellenbeschreibungen beschränken sich auf die wichtigsten
    Parameter und betrachten keine Fehlerfälle,
    da eine entsprechende Beschreibung zu umfangreich wäre.
    % TODO: OpenAPI Beschreibung?

    \paragraph{Klassifizierung}
    Die Hauptaufgabe des {\classificationService}s ist die Klassifizierung einer Webseite
    und die Speicherung der resultierenden Klassifikation.
    Diese Funktion kann über die Schnittstelle in Tabelle \ref{table:startClassificationInterface} genutzt werden.

    \begin{table}[htb]
        \centering
        \begin{tabular}{|l|l|}
        \hline
        \textbf{Endpunkt} & \texttt{http://<HOST>:44284/classifications}\\
        \hline
        \textbf{Methode} & \texttt{POST}\\
        \hline
        \textbf{Eingabedokument} & \lstinputlisting[title=ClassificationServiceTasks]{../resources/classification-service/tasks.json}\\
        \hline
        \textbf{Statuscode} & \texttt{202 (Accepted)}\\
        \hline
        \end{tabular}
        \caption{Schnittstelle zum Starten einer Klassifizierung}
        \label{table:startClassificationInterface}
    \end{table}

    Das Eingabedokument spezifiziert die Webseiten,
    die vom {\classificationService} strukturiert werden sollen.
    Neben den \glspl{url} der Webseiten enthält es außerdem Informationen über ihre Site,
    damit in der Webanwendung\footnote{vgl. Kapitel \ref{section:solutionConceptWebApp}}
    eine Zuordnung der klassifizierten Seiten zu ihrer Site möglich ist.
    In einem Dokument können beliebig viele Sites und Webseiten spezifiziert werden.

    \paragraph{{\classificationModel} abfragen}
    Der Service bietet außerdem die Möglichkeit die ihm bekannten Klassen abzufragen,
    was das {\annotatorPlugin}\footnote{vgl. Kapitel \ref{section:conceptWebAnnotations}}
    für seine Korrekturfunktion nutzt.
    Zur Verwendung dieser Funktion stehen drei Schnittstellen bereit,
    die sich nur in einem Teil ihrer \gls{url} unterscheiden und deshalb gemeinsam in
    Tabelle \ref{table:getClassesInterface} dokumentiert sind.

    \begin{table}[htb]
        \centering
        \begin{tabular}{|l|l|}
        \hline
        \textbf{Endpunkt} & \texttt{http://<HOST>:44284/[page | content | reference]-classes}\\
        \hline
        \textbf{Methode} & \texttt{GET}\\
        \hline
        \textbf{Statuscode} & \texttt{200 (OK)}\\
        \hline
        \textbf{Ausgabedokument} & \lstinputlisting[title=ClassificationServiceClasses]{../resources/classification-service/classes.json}\\
        \hline
        \end{tabular}
        \caption{Schnittstelle zum Abfragen des {\classificationModel}s}
        \label{table:getClassesInterface}
    \end{table}

    Das Ausgabedokument kapselt die Klassen in einem neuen Dokument,
    welches neben den eigentlichen Klassen in \texttt{classes}
    auch die Anzahl aller Klassen in \texttt{total} enthält.
    Dadurch kann die Schnittstelle zu einem späteren Zeitpunkt z. B. um eine
    seitenweise Abfrage erweitert werden,
    ohne das Ausgabedokument inkompatibel zu früheren Version zu verändern.
    Die Klassen verwenden das Format des generierten
    {\classificationModel}s\footnote{vgl. Kapitel \ref{section:solutionDetailsDslGeneration}}.
