\subsection{Funktionen und Schnitstellen}
    \label{section:solutionDetailsClassificationServiceFunctions}
    Der Classification Service biete verschiedene Funktionen,
    die über eine REST-Schnittstelle angesprochen werden können.
    In diesem Abschnitt werden die Funktionen und die zugehörigen
    REST-Endpunkte beschrieben.
    Diese Beschreibung enthält die \gls{url}, die HTTP-Methode,
    verallgemeinerte Ein- bzw. Ausgabedokumente und den HTTP-Status
    der Antwort im Erfolgsfall.

    \paragraph{Klassifizierung}
    Die Hauptaufgabe des Classification Services ist die Klassifizierung einer Webseite
    und die Speicherung der resultierenden Klassifikation.
    Dazu bietet sie den Endpunkt in Tabelle \ref{table:startClassificationInterface}.

    \begin{table}[htb]
        \centering
        \begin{tabular}{|l|l|}
        \hline
        \textbf{Endpunkt} & \texttt{http://<HOST>:44284/classifications}\\
        \hline
        \textbf{Methode} & \texttt{POST}\\
        \hline
        \textbf{Eingabedokument} & \lstinputlisting{../resources/classification-service/tasks.json}\\
        \hline
        \textbf{Status} & \texttt{202 (Accepted)}\\
        \hline
        \end{tabular}
        \caption{Schnittstelle zum Starten einer Klassifizierung}
        \label{table:startClassificationInterface}
    \end{table}

    Das Eingabedokument kann sowohl mehrere Sites als auch mehrere Seiten innerhalb einer Sites enthalten.
    Dadurch können dem Service in einer Anfrage mehrere zu klassifizierende Seiten genannt werden.
    Die Site wird nur für die spätere Zuordnung z. B. in der WebApp benötigt und kann beliebig sein.

    \paragraph{Klassifizierungsmodell abfragen}
    Der Service biete außerdem die Möglichkeit die aktuellen Klassen,
    die er für eine Klassifizierung verwendet abzufragen.
    Diese Funktion wird später zum Beispiel vom Annotator Plugin genutzt,
    um die Auswahlboxen zu füllen. Siehe Kapitel \ref{section:solutionDetailsAnnotatorPlugin}.
    Dazu bietet der Service drei Endpunkte, die sich nur in ihrer \gls{url} unterscheiden.
    Tabelle \ref{table:getClassesInterface} zeigt die Schnittstelle.

    \begin{table}[htb]
        \centering
        \begin{tabular}{|l|l|}
        \hline
        \textbf{Endpunkt} & \texttt{http://<HOST>:44284/[page | content | reference]-class}\\
        \hline
        \textbf{Methode} & \texttt{GET}\\
        \hline
        \textbf{Status} & \texttt{200 (OK)}\\
        \hline
        \textbf{Ausgabedokument} & \lstinputlisting{../resources/classification-service/classes.json}\\
        \hline
        \end{tabular}
        \caption{Schnittstelle zum Abfragen des aktuellen Klassifizierungsmodells}
        \label{table:getClassesInterface}
    \end{table}

    Das Ausgabedokument kapselt die Klassen in einem neuen Dokument,
    welches neben den eigentlichen Klassen auch die Anzahl aller Klassen enthält.
    Das wurde gemacht, um schon jetzt kompatibel für eine Paginierung des Ergebnisses zu sein.
    Das Feld "`total"' enthält dann die Anzahl aller Klassen.
    Das Feld "`classes"' enthält die Klassen gesammelt in einem Array.
    Ein einzelnes Klassenobjekt folgt dem des DSL Generats.
    Siehe Kapitel \ref{section:solutionDetailsDslGeneration}.