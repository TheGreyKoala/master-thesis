\subsection{Konfiguration}
    Als Konfigurationsdatei dient das von der \gls{wccdl} genereirte Klassifizierungsmodell.
    Dieses wird beim Start des Services einmalig eingelesen.
    Zum Laden einer neuen Konfiguration muss der Service neugestartet werden.
    
    Eine Alternative wäre gewesen einer Klassifizierungsanfrage das zu nutzende Modell anzufügen.
    Dies hätte allerdings bedeutet, dass Nutzer des Services diese Konfiguration kennen müssen,
    was womöglich nicht der Fall ist, wenn diese z. B. von einer anderen Abteilung geschrieben wurde.
    Außerdem hätte es jede Anfrage unnötig vergrößert.
    Der eigentliche Nutzen dieses Vorgehen ist verschiedene Konfigurationen nutzen zu können.
    Ob dieses Anwendungsfall häufig eintritt bleibt abzuwarten und kann ggf. sogar unerwünscht sein.
    
    Eine spätere Kombination beider Ansätze ist denkbar.
    Genauso wie ein Endpunkt, über den man die Konfiguration setzen kann.