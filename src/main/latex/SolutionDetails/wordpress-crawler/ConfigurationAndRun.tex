\subsection{Konfiguration und Aufruf}
    Der {\wordpressCrawler} benötigt zur Ausführung einige Informationen,
    die ihm beim Start in einer Konfigurationsdatei mitgeteilt werden.
    Diese Parameter lassen sich in drei Kategorien einteilen:
    Classification Service, Sites und Crawling.

    \paragraph*{Parameter der Kategorie Classification Service}
    Da der {\wordpressCrawler} kein Service des \glspl{wccs} ist,
    muss ihm die \gls{url} eines laufenden {\classificationService}s mitgeteilt werden,
    an den er die Klassifizierungsaufträge richten soll.
    Die verfügbaren Parameter hierfür sind in Tabelle \ref{table:crawlerClassificationServiceParameter} aufgeführt.
    Jeder dieser Parameter ist optional und besitzt deshalb einen Standardwert.
    Der {\wordpressCrawler} baut die \gls{url} des {\classificationService}s nach
    folgendem Schema auf: \texttt{<scheme>://<host>:<port>:<path>}.

    \begin{table}[htb]
        \centering
        \begin{tabular}{|l|l|}
            \hline
            \textbf{Parameter} & \textbf{Standardwert}\\
            \hline
            \texttt{scheme} & \texttt{http} \\
            \hline
            \texttt{host} & \texttt{localhost} \\
            \hline
            \texttt{port} & \texttt{44284} \\
            \hline
            \texttt{path} & \texttt{/classifications} \\
            \hline
            \end{tabular}
        \caption{Die Konfigurationsparameter des {\wordpressCrawler}s für den {\classificationService}}
        \label{table:crawlerClassificationServiceParameter}
    \end{table}

    \paragraph*{Parameter der Kategorie Sites}
    Darüber hinaus muss das Werkzeug wissen,
    welche {\wordpress} Sites es durchsuchen soll.
    Diese Information wird ebenfalls in der Konfiguration gespeichert.
    Eine {\wordpress} Site wird darin durch
    ihre \gls{url}, einen Namen und eine frei wählbare eindeutige ID definiert.

    \paragraph*{Parameter der Kategorie Crawling}
    Diese Kategorie der Konfiguration kennt zwei weitere Parameter:
    \texttt{resultPageSize} und \texttt{maxConcurrentRequests}.
    Der erste Parameter legt die Anzahl der Beiträge und Seiten fest,
    die der {\wordpressCrawler} in einer Anfrage von {\wordpress} anfordert.
    Der Standardwert dieses Parameters ist 8.
    Der zweite Parameter bestimmt die Anzahl der gleichzeitigen Anfragen,
    die vom Crawler an {\wordpress} gerichtet werden.
    Hier liegt der Standardwert bei 5.
    Beide Parameter dienen der Erzeugung eines guten Flusses und stellen sicher,
    dass {\wordpress} durch die Anfragen des Crawlers
    nicht überlastet wird und keine anderen Anfragen mehr bearbeiten kann.

    \paragraph*{Beispielkonfiguration}
    Eine konkrete Konfiguration muss in YAML\footnote{vgl. \url{http://yaml.org}} geschrieben sein.
    Ein Beispiel ist in Listing \ref{listing:crawlerConfiguration} zu sehen.

    \lstinputlisting[
        label=listing:crawlerConfiguration,
        caption=Eine Beispielkonfiguration des {\wordpressCrawler}s,
        style=crawlerConfig
    ]{../resources/crawler/config.yaml}

    \paragraph*{Aufruf}
    Der Aufruf des {\wordpressCrawler}s erfolgt über automatisch generierte Startskripte,
    die dem Werkzeug beiliegen.
    Als Argument muss ihnen der Pfad zur Konfigurationsdatei übergeben werden.
    Ein beispielhafter Aufruf ist \texttt{./wccs-wordpress-crawler /home/wccs/crawler.yaml}.
