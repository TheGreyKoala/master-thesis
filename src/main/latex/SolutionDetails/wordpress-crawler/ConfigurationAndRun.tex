\subsection{Konfiguration und Aufruf}
    Der Crawler benötigt einige Informationen,
    die ihm beim Start in einer Konfigurationsdatei mitgeteilt werden.
    Die Konfigurationsparameter lassen sich in drei Kategorien einteilen:
    Classification Service, Sites und Crawling.

    \paragraph{Classification Service Parameter}
    Der Crawler soll die Klassifizierung der gefundenen Seiten anstoßen,
    weshalb er die \gls{url} des Classification Services benötigt.
    Die verfügbaren Parameter sind in Tabelle \ref{table:crawlerClassificationServiceParameter} aufgeführt.
    Jeder dieser Parameter ist optional und besitzt einen Standardwert,
    der genutzt wird, falls er nicht durch die Konfiguration überschrieben wird.

    \begin{table}[]
        \centering
        \begin{tabular}{|l|l|}
            \hline
            \textbf{Parameter} & \textbf{Standardwert}\\
            \hline
            scheme & http \\
            \hline
            host & localhost \\
            \hline
            port & 44284 \\
            \hline
            path & /classifications \\
            \hline
            \end{tabular}
        \caption{Konfigurationsparameter des Crawlers: Classification Service}
        \label{table:crawlerClassificationServiceParameter}
    \end{table}

    Der Crawler baut basierend auf den Angaben die \gls{url} des Classification Services nach
    folgendem Schema auf: \texttt{<scheme>://<host>:<port>:<path>}.

    \paragraph{Sites Parameter}
    Neben Informationen zum Classification Service benötigt der Crawler auch Informationen zu den Sites,
    die er durchsuchen soll.
    Es können deshalb beliebig viele Sites durch die Angabe einer Id,
    eines Namens und vor allem einer Basis-\gls{url} konfiguriert werden.

    \paragraph{Crawling Parameter}
    Die Kategorie "`Crawling"' enthält zwei Parameter:
    "`resultPageSize"' und "`maxConcurrentRequests"'.
    Der erste Parameter legt die Anzahl der Posts und Pages fest,
    die in einer Anfrage von {\wordpress} angefordert werden.
    Der Standardwert dieses Parameters ist 8.
    Der Crawler führt entsprechend viele Anfragen durch,
    um alle Seiten zu beziehen.
    Seiten die er ermittelt hat, leitet er direkt zum Classification Service weiter,
    anstatt erst alle Seiten zu sammeln.
    Der zweite Parameter bestimmt die Anzahl der gleichzeiten Anfragen,
    die an {\wordpress} gerichtet werden.
    Hier liegt der Standardwert bei 5.
    Beide Parameter dienen der Erzeugung eines guten Flusses zwischen Anfragen an
    {\wordpress} und den Classification Service.
    Außerdem stellen sie sicher, dass {\wordpress} durch die Anfragen des Crawlers
    nicht überlastet wird und keine anderen Anfragen mehr bearbeiten kann.

    \paragraph{Beispielkonfiguration}
    Eine Konfiguration muss in YAML\footnote{vgl. \url{http://yaml.org}} geschrieben.
    Ein Beispiel ist in Listing \ref{listing:crawlerConfiguration} zu sehen.

    \lstinputlisting[
        label=listing:crawlerConfiguration,
        caption=Beispielkonfiguration für den {\wordpress}-Crawler,
        style=pseudo
    ]{../resources/crawler/config.yaml}

    \paragraph{Aufruf}
    Der Aufruf erfolgt über automatisch generierte Start-Skripte,
    die dem Werkzeug beiliegen und denen als Argument
    der Pfad zur Konfigurationsdatei mitgeteilt werden muss.
    Ein beispielhafter Aufruf ist \texttt{./wccs-wordpress-crawler /home/wccs/crawler.yaml}.
