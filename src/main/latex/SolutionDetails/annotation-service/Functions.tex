\subsection{Funktionen und Schnittstellen}
    Der Annotation Service ist für die Umformung einer Klassifikation zu
    einer Menge von Annotationen zuständig und kann Änderungen an einer
    Annotation in eine Änderung an der Klassifikation übersetzen.
    Hierzu bietet er eine REST-Schnittstelle, die gleichzeitig auch die Annotator Storage API implementiert.
    % TODO: Link
    Dieser Schnittstelle und der verbunden Funktionen widmet sich dieses Kapitel.

    \paragraph{Format einer Annotation}
    Zunächst wird allerdings beschrieben, welches konkrete technische Format eine Annotation besitzt.
    Dieses folgt dem vorgegebenen Format von Annotator und ergänzt es um einige spezifische Angaben.
    Das vollständige Format, wie es vom Annotation Service erzeugt und akzeptiert wird,
    ist in Listing \ref{listing:annotationServiceAnnotationFormat} zu sehen.

    \lstinputlisting[
        label=listing:annotationServiceAnnotationFormat,
        caption=Format einer Annotation
    ]{../resources/annotation-service/annotation.json}

    Die ID einer Annotation wird vom Service automatisch generiert.
    Dieser Prozess wird in Kapitel \ref{section:solutionDetailsAnnotationServiceMapping}
    genau erklärt.
    Das Feld "`text"' bestimmt den Inhalt der Annotation, der dem Nutzer sichtbar ist.
    Die vom Service erzeugten Annotationen beinhalten immer den Namen des verknüpften Features und dessen Klasse.
    Die Angaben im Feld "`ranges"' mappen 1:1 auf den eindeutigen Selektor jedes Features.
    Durch das Feld "`user"' identifiziert Annotator den Ersteller der Annotation,
    der in diesem Fall immer der imaginäre technische Benutzer "`wccs"' ist.
    Da im Feld "`permissions"' lediglich das Recht "`admin"' auf diesen Nutzer beschränkt wurde,
    kann jeder die Annotation lesen und bearbeiten.
    Änderungen an den Rechten bleiben aber dem Nutzer "`wccs"' vorbehalten.
    Die Eigenschaft "`wccs"' ist nicht durch das Format von Annotator vorgegeben,
    sondern stellt eine Ergänzung des \gls{wccs} dar.
    Dieses speichert, ob es sich beim verknüpften Feature um ein Content oder ein Reference Feature handelt,
    und welche Klasse dieses besitzt.

    \paragraph{Bereitstellung von Metainformationen}
    Um der Annotator Storage API zu genügen,
    muss ein Endpunkt existieren, der ein JSON-Dokument liefert,
    welches Informationen über die Version der implementierten Schnittstelle enthält.

    \begin{table}[htb]
        \centering
        \begin{tabular}{|l|l|}
            \hline
            \textbf{Endpunkt} & \texttt{http://<HOST>:16612/pages/\{pageId\}}\\
            \hline
            \textbf{Methode} & \texttt{GET}\\
            \hline
            \textbf{Status} & \texttt{200 (OK)}\\
            \hline
            \textbf{Ausgabedokument} & \lstinputlisting[title=Test123]{../resources/annotation-service/meta.json}\\
            \hline
        \end{tabular}
        \caption{Schnittstelle zum Abfragen von Metainformationen über den Annotation Service}
        \label{my-label}
    \end{table}


    
 %   \begin{table}[htb]
 %       \centering
 %       \begin{tabular}{|l|l|}
 %       \hline
 %       \textbf{Endpunkt}     & /pages/\{pageId\}/annotations\\
 %       \hline
 %       \textbf{Methode}      & GET\\
 %       \hline
 %       \textbf{Beschreibung} & Liefert alle Annotationen einer Seite\\
 %       \hline
 %       \textbf{Status}       & 200\\
 %       \hline
 %       \textbf{Antwort}      & \{ ``name'': ``Annotator Store API'', ``version'': ``2.0.0'' \}\\
 %       \hline
 %       \end{tabular}
 %       \caption{My caption}
 %       \label{my-label}
 %   \end{table}

    % Schnittstellenbeschreibung:
    % - Endpunkt
    % - Methoden
    % -- Input-Dokument
    % -- Status Codes der Antwort
    % -- Return Dokument für jede Antwort

    % - Schnittstelle so, dass später auch anderes Format für Annotationen möglich