\subsection{Funktionen und Schnittstellen}
    \label{section:solutionDetailsAnnotationServiceFunctions}
    Der Annotation Service ist für die Umformung einer Klassifikation zu
    einer Menge von Annotationen zuständig und kann Änderungen an einer
    Annotation in eine Änderung an der Klassifikation übersetzen.
    Hierzu bietet er eine REST-Schnittstelle, die gleichzeitig auch die Annotator Storage API implementiert.
    % TODO: Link
    Dieser Schnittstelle und der verbunden Funktionen widmet sich dieses Kapitel.

    \paragraph{Format einer Annotation}
    Zunächst wird allerdings beschrieben, welches konkrete technische Format eine Annotation besitzt.
    Dieses folgt dem vorgegebenen Format von Annotator und ergänzt es um einige spezifische Angaben.
    Das vollständige Format, wie es vom Annotation Service erzeugt und akzeptiert wird,
    ist in Listing \ref{listing:annotationServiceAnnotationFormat} zu sehen.

    \lstinputlisting[
        label=listing:annotationServiceAnnotationFormat,
        caption=Format einer Annotation
    ]{../resources/annotation-service/annotation.json}

    Die ID einer Annotation wird vom Service automatisch generiert.
    Dieser Prozess wird in Kapitel \ref{section:solutionDetailsAnnotationServiceMapping}
    genau erklärt.
    Das Feld "`text"' bestimmt den Inhalt der Annotation, der dem Nutzer sichtbar ist.
    Die vom Service erzeugten Annotationen beinhalten immer den Namen des verknüpften Features und dessen Klasse.
    Die Angaben im Feld "`ranges"' mappen 1:1 auf den eindeutigen Selektor jedes Features.
    Durch das Feld "`user"' identifiziert Annotator den Ersteller der Annotation,
    der in diesem Fall immer der imaginäre technische Benutzer "`wccs"' ist.
    Da im Feld "`permissions"' lediglich das Recht "`admin"' auf diesen Nutzer beschränkt wurde,
    kann jeder die Annotation lesen und bearbeiten.
    Änderungen an den Rechten bleiben aber dem Nutzer "`wccs"' vorbehalten.
    Die Eigenschaft "`wccs"' ist nicht durch das Format von Annotator vorgegeben,
    sondern stellt eine Ergänzung des \gls{wccs} dar.
    Dieses speichert, ob es sich beim verknüpften Feature um ein Content oder ein Reference Feature handelt,
    und welche Klasse dieses besitzt.

    \paragraph{Bereitstellung von Metainformationen}
    Um der Annotator Storage API zu genügen,
    muss ein Endpunkt existieren, der ein JSON-Dokument liefert,
    welches Informationen über die Version der implementierten Schnittstelle enthält.
    % TODO: Referenz: http://docs.annotatorjs.org/en/v1.2.x/storage.html#root
    Die entsprechende Schnittstelle ist in Tabelle
    \ref{table:annotationServiceMetaInterface} dokumentiert.

    % TODO: Erwähnen, dass die Page ID die URL ist. Oder den Parameter gleich PAGE_URL nennen.

    \begin{table}[htb]
        \centering
        \begin{tabular}{|l|l|}
            \hline
            \textbf{Endpunkt} & \texttt{http://<HOST>:16612/pages/<PAGE\_ID>}\\
            \hline
            \textbf{Methode} & \texttt{GET}\\
            \hline
            \textbf{Status} & \texttt{200 (OK)}\\
            \hline
            \textbf{Ausgabedokument} & \lstinputlisting[title=AnnotationServiceMetaInfo]{../resources/annotation-service/meta.json}\\
            \hline
        \end{tabular}
        \caption{Schnittstelle zum Abfragen von Metainformationen über den Annotation Service}
        \label{tableannotationServiceMetaInterface}
    \end{table}

    \paragraph{Annotationen einer Seite abrufen}
    Zentrale Aufgabe des Services ist die Bereitstellung der Annotationen einer Seite.
    Auf diese Funktion greift zum Beispiel das Annotator Plugin zurück
    (siehe Kapitel \ref{section:solutionDetailsAnnotatorPlugin}).
    Hierfür stellt der Serivce den in Tabelle \ref{}
    dargestellten Endpunkt bereit, der die entsprechende Funktion der Annotator Storage API
    umsetzt.
    % TODO: Referenz: http://docs.annotatorjs.org/en/v1.2.x/storage.html#index
    
    \begin{table}[htb]
        \centering
        \begin{tabular}{|l|l|}
            \hline
            \textbf{Endpunkt} & \texttt{http://<HOST>:16612/pages/<PAGE\_ID>/annotations}\\
            \hline
            \textbf{Methode} & \texttt{GET}\\
            \hline
            \textbf{Status} & \texttt{200 (OK)}\\
            \hline
            \textbf{Ausgabedokument} & \texttt{Array von Annotationen}\\
            \hline
        \end{tabular}
        \caption{Schnittstelle zum Abfragen aller Annotationen einer Seite}
        \label{tablegetAllAnnotationsInterface}
    \end{table}

    Die \gls{url} der Seite, für die die Annotationen angefragt werden,
    wird in der Anfrage-\gls{url} codiert.
    Die Antwort ist ein JSON-Array von Annotationen, wie sie im ersten Abschnitt vorgestellt wurden.

    \paragraph{Einzelne Annotation lesen}
    Eine einzelne Annotation kann anhand ihrer ID beim Service angefragt werden.
    Der Endpunkt, der die entsprechende Storage API Funktion implementiert,
    % TODO: Referenz: http://docs.annotatorjs.org/en/v1.2.x/storage.html#read
    ist in Tabelle \ref{tablegetAnnotationInterface} zu sehen.

    \begin{table}[htb]
        \centering
        \begin{tabular}{|l|l|}
            \hline
            \textbf{Endpunkt} & \texttt{http://<HOST>:16612/pages/<PAGE\_ID>/annotations/<ANNOTATION\_ID>}\\
            \hline
            \textbf{Methode} & \texttt{GET}\\
            \hline
            \textbf{Status} & \texttt{200 (OK)}\\
            \hline
            \textbf{Ausgabedokument} & \texttt{Annotation}\\
            \hline
        \end{tabular}
        \caption{Schnittstelle zum Abfragen einer Annotation}
        \label{tablegetAnnotationInterface}
    \end{table}

    Wie zu sehen ist, wird an die Anfrage-\gls{url} einfach die ID der Annotation gehangen.
    Die Antwort besteht aus einer einzelnen Annotation.

    \paragraph{Einzelne Annotation aktualisieren}
    Eine einzelne Annotation kann auch aktualisiert werden,
    was zur Aktualisierung des entsprechenden Features der Seite führt
    (siehe Kapitel \ref{section:solutionDetailsAnnotationServiceMapping}).
    Der Endpunkt in Tabelle \ref{table:putAnnotationInterface} implementiert
    die entsprechende Storage API Funktion.
    % TODO: Referenz: http://docs.annotatorjs.org/en/v1.2.x/storage.html#update

    \begin{table}[htb]
        \centering
        \begin{tabular}{|l|l|}
            \hline
            \textbf{Endpunkt} & \texttt{http://<HOST>:16612/pages/<PAGE\_ID>/annotations/<ANNOTATION\_ID>}\\
            \hline
            \textbf{Methode} & \texttt{PUT}\\
            \hline
            \textbf{Eingabedokument} & \texttt{Annotation}\\
            \hline
            \textbf{Status} & \texttt{303 (See Other)}\\
            \hline
        \end{tabular}
        \caption{Schnittstelle zum Aktualisieren einer Annotation}
        \label{table:putAnnotationInterface}
    \end{table}

    In der Anfrage muss eine Annotation im oben beschriebenen Format enthalten sein.
    Die Antwort enthält im Header "`Location"' die \gls{url} zum Lesen der geschriebenen
    Annotation, wie es die Storage API vorschreibt.

    % TODO: Schnittstelle so, dass später auch anderes Format für Annotationen möglich