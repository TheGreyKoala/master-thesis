\subsection{Abbildung einer Klassifikation auf Annotationen}
    \label{section:solutionDetailsAnnotationServiceMapping}
    Dieses Kapitel erklärt, wie der {\annotationService}
    die Klassifikation einer Seite auf eine Menge von Annotationen abbildet.
    Dazu wird zunächst beschrieben, wie lesende Operationen Annotationen
    erzeugen und dazu eine IDs generieren.
    Anschließend erfolgt die Beschreibung schreibender Operationen.

    \paragraph{Lesen von Annotationen}
    Der {\annotationService} erzeugt Annotationen für alle Referenzen einer Klassifikation
    sowie für alle {\contentFeature}s ohne eigene
    {\contentFeature}s\footnote{vgl. Kapitel \ref{section:conceptWebAnnotations}}.
    Klassifikationen bezieht er von der {\classificationStorageAPI}.

    Eine Herausforderung ist die Generierung einer eindeutigen ID für jede Annotation.
    Beim Lesen oder Schreiben einzelner Annotationen muss das betroffene
    Feature, welches keine einzelne eindeutige Kennung besitzt, anhand dieser ID erkannt werden.
    Die Generierung muss deshalb deterministisch und umkehrbar sein.
    Bei zufällig generierten oder nicht umkehrbaren IDs müsste der Service
    die Zuordnung festhalten.
    Dadurch würde er seine Zustandslosigkeit
    verlieren\footnote{vgl. Kapitel \ref{section:conceptMicroServices}}.
    Bei einem unvorhergesehenen Neustart des Services ginge diese Zuordnung unter Umständen verloren.
    Annotationen müssten dann erst neu gelesen werden,
    bevor der Service Aktualisierungsanfragen bearbeiten kann.

    Zur Erzeugung einer ID konkateniert der {\annotationService} deshalb
    alle Navigationsschritte von der obersten Ebene der Klassifikation,
    bis zum Feature und trennt sie durch einen Doppelpunkt.
    Bei {\collectionFeature}s ist außerdem der Index des jeweiligen Elementes Teil dieser Kette.
    Für die Referenz \texttt{r1} aus dem Beispiel in Kapitel \ref{section:solutionDetailPersistenceDataModelExample}
    entsteht so die Zeichenkette \texttt{contents:c3:contents:c5:references:internalLinks:0}.
    Anschließend codiert der Service diese Kette noch mit Base64 \cite{rfc:4648}.
    So entsteht eine deterministische und leicht umkehrbare ID.

    \paragraph{Schreiben einer Annotationen}
    An verschiedenen Stellen wurde bereits erwähnt,
    dass das \gls{wccs} über Änderungen einer Annotation erlaubt die Klasse eines Features zu bearbeiten.
    Beim Schreiben einer Annotation muss der {\annotationService} deshalb lediglich
    die Klasse des betroffenen Features neu setzen und die Änderung persistieren.

    Dank der beschriebenen Erzeugung der ID einer Annotation,
    stellt die Abbildung einer Annotation auf ein Feature keine Herausforderung dar.
    Der Service muss die ID lediglich Base64 dekodieren,
    bei jedem Doppelpunkt splitten und die einzelnen Teile verwenden,
    um durch die Klassifikation bis zum Feature zu navigieren.
    Anschließend liest er die neue Klasse aus dem Feld \texttt{wccs.featureClass} der Annotation
    und aktualisiert die Klasse des Features entsprechend.
    Anschließend übermittelt er die neue Klassifikation zur {\classificationStorageAPI},
    die die Änderung persistiert.
