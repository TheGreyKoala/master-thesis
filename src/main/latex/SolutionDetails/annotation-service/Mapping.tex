\subsection{Abbildung einer Klassifikation auf Annotationen}
    \label{section:solutionDetailsAnnotationServiceMapping}
    Dieses Kapitel erklärt, wie der Annotation Service
    die Klassifikation einer Seite auf eine Menge von Annotationen abbildet.
    Dazu wird zunächst beschrieben, wie lesende Operationen Annotationen
    erzeugen und dazu eine ID für jede Annotation generieren.
    Anschließend wird vorgestellt, wie schreibende Operationen agieren.

    \paragraph{Lesen von Annotationen}
    Wie in Kapitel \ref{section:conceptWebAnnotations} beschrieben,
    erzeugt der Annotation Service Annotationen für alle Referenzen einer Seite,
    sowie für alle Content Features ohne eigene Child Features.
    Der Algorithmus fragt deshalb zunächst die Seite bei der Calssification Storage API an
    und traversiert anschließend über die Klassifikation und erzeugt
    für die beschriebenen Features eine Annotation im oben vorgestellten Format.

    Eine Herausforderung dabei ist die Generierung einer eindeutigen Id für jede Annotation,
    da anhand dieser beim Lesen oder Schreiben einzelner Annotationen das betroffene
    Feature der Seite erkannt werden muss.
    Features besitzen keine ID, weshalb eine solche ID erzeugt werden muss.
    Diese Generierung muss für ein Feature allerdings deterministisch und umkehrbar sein.
    Bei zufällig generierten oder nicht umkehrbaren IDs müsste der Service
    die Zuordnung festhalten.
    Dadurch würde er seine Zustandslosigkeit verlieren (siehe Kapitel \ref{section:conceptMicroServices}).
    Bei einem unvorhergesehenen Neustart des Services ginge diese Zuordnung unter Umständen verloren.
    Annotationen müssten dann erst neu gelesen werden,
    bevor der Service Aktualisierungsanfragen bearbeiten kann.

    Zur Erzeugung einer ID konkateniert der Annotation Service deshalb
    alle Navigationsschritte von der obersten Ebene der Klassifikation,
    bis zum aktuellen Feature und trennt sie durch einen Doppelpunkt.
    Bei Collection Features ist außerdem der Index des jeweiligen Elementes Teil dieser Kette.
    Für die Referenz r1 aus dem Beispiel in Kapitel \ref{section:solutionDetailPersistenceDataModelExample}
    enstünde demnach die Zeichenkette \texttt{contents:c3:contents:c5:references:internalLinks:0}.
    Anschließend codiert er diese Kette noch mit Base64.
    So entsteht eine deterministische und leicht umkehrbare ID.

    \paragraph{Schreiben einer Annotationen}
    An verschiedenen Stellen wurde bereits erwähnt,
    dass das \gls{wccs} über Änderungen einer Annotation erlaubt die Klasse eines Features zu bearbeiten.
    Beim Schreiben einer Annotation muss der Annotation Service deshalb lediglich
    die Klasse des betroffenen Features neu setzen und die Änderung persistieren.

    Dank der beschriebenen Erzeugung der ID einer Annotation,
    stellt die Abbildung von Annotation auf Feature keine Herausforderung dar.
    Der Service muss die ID lediglich Base64 dekodieren,
    bei jedem Doppelpunkt splitten und die einzelnen Teile verwenden,
    um durch die Klassifikation bis zum Feature zu navigieren.
    Anschließend liest er die neue Klasse aus dem Feld "`wccs.featureClass"' der Annotation
    (siehe Kapitel \ref{section:solutionDetailsAnnotationServiceFunctions})
    und aktualisiert die Klasse des Features entsprechend.
    Anschließend übermittelt er die neue Klassifikation zur Classification Storage API,
    die die Änderung persistiert.
