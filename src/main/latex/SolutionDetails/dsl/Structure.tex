\subsection{Strukturelles Design}
    Dieses Kapitel beschreibt einige strukturelle Aspekte der Sprache,
    die im wesentlichen auf den Ausführungen in \cite[Kapitel 4, 5.1]{voelter:DslEngineering}.

    \paragraph{Separation of Concerns}
    Eine Domäne kann aus verschiedenen Aspekten bestehen,
    die verschiedene Bereiche der Domäne abdecken.
    Alle diese Belange müssen beim Design der Sprache berücksichtigt werden.
    \citet[Kapitel 4.1]{voelter:DslEngineering}
    stellt dazu zwei Ansätze vor: Eine Sprache, die alle Aspekte adressiert
    oder mehrere Sprachen, die jeweils einen Aspekt bearbeiten.

    Für den konkreten Fall gibt es die folgenden Aspekte:
    Klassen deklarieren, d.h. bekannt machen.
    Klassen definieren, d.h. Features angeben.
    Definition von Selektoren für Klassen und Features.

    Diese Aspekte werden alle von einer Sprache, der \gls{wccdl}
    angesprochen.
    Gründe dafür sind die Enge Kopplung dieser Aspekte
    und das die Aspekte zu klein sind.
    Es würden winzige Sprachen dabei herausfallen
    und der Overhead beim Schreiben wäre viel zu groß.

    \paragraph{Modularisierung und Sichtbarkeit}
    Modularisierung beschreibt die logische Strukturierung
    von Programmen und sind in irgendeiner Form in jeder Sprache
    notwendig und vorhanden.
    Beispiele sind Namespaces
    \cite[Kapitel 5.1.1]{voelter:DslEngineering}.
    Die Referenzierung eines Elementes sollte immer auf der logischen
    Struktur erfolgen, weshalb Modularisierung eng mit der Sichtbarkeit
    verbunden ist.
    Auch dieses Konzept ist in jeder Sprache vertreten und legt
    die Möglichkeiten fest, die Sichtbarkeit eines Elementes im restlichen Programm zu steuern.
    Beispiele sind public oder private Modifier in Java.

    Die \gls{wccdl} verwendet ein sehr einfaches Konzept bzgl.
    der Modularisierung und Sichtbarkeit von Porgrammelementen.
    Da lediglich zu Klassen Cross-References gebildet werden können,
    sind diese Punkte nur für Klassen relevant.
    Klassen werden in einem globalen Namensraum abgelegt und sind überall sichtbar.
    D.h. jede Klasse kann für jedes Feature genutzt werden.
    Features und Selektore können nicht als Ziel von Cross-Referenzen definieren
    und sind deshalb gewissermaßen absolut privat. % TODO: Braucht man diese Info überhaupt?
    Komplexere Konzepte würden Hürde für Nicht-Programmierer darstellen.

    \paragraph{Partitionierung}
    Die Strukturierung des Programmes in physische Einheiten (Dateien)
    heißt Partitionierung.
    Diese muss der logischen Strukturierung nicht entsprechen.
    Vor allem, da Referenzierung nur auf Basis der logischen Struktur erfolgen sollte
    \cite[Kapitel 5.1.2]{voelter:DslEngineering}.

    Die \gls{wccdl} erlaubt die Klassendefinitionen auf verschiedenen Dateien
    aufzuteilen, wobei eine Klassendefinition immer vollständig in er Datei sein muss.
    Da alle Klassen logisch in einem globalen Namensraum gesammelt werden,
    sind auch alle Klassen aus anderen Dateien referenzierbar.

    \paragraph{Scoping und Linking}
    Der Scope einer Cross-Referenz in einem Programm ist die Menge der
    gültigen Ziele dieser Referenz
    \cite[Kapitel 8]{voelter:DslEngineering}.
    Wie beschrieben werden in der \gls{wccdl} lediglich für die Klasse
    eines Features Cross-References verwendet.
    Das gültige Ziel (der Scope) wird an dieser Stelle bereits durch das Sprachkonzept
    eingeschränkt, da nur FeatureClass erlaubt sind.
    Die Sichtbarkeit spielt keine Rolle, wie bereits beschrieben.
    Dieses Eigenschaft ist ein Beitrag zur automatischen Validitätsprüfung der Definition
    und erlaubt dank Xtext auch Code Completion.

    Syntaktisch stellt der abstrakte Syntax eines Programmes einen Baum dar.
    Semantisch wird er durch Querverweise allerdings zu einem Tree.
    Linking beschreibt den Transformationsprozess vom syntaktischen Baum zum
    semantischen Tree, der basierent auf den Namen der referenzierten Elementen passiert.
    Die \gls{wccdl} kann hierzu auf die Mittel von Xtext zurückgreifen,
    welches Querverweise automatisch auflöst
    \cite[Kapitel 8]{voelter:DslEngineering}. % TODO Hier noch ein Link aus der Xtext Doku?: https://www.eclipse.org/Xtext/documentation/303_runtime_concepts.html#linking

    \paragraph{Spezifikation und Implementierung}
    Sprachen können die Trennung von Spezifikation und Implementierung von
    Porgrammelementen unterstützen, um eine bessere Entkopplung und verschiedene
    Implementierungen zu ermöglichen \cite[Kapitel 5.1.3]{voelter:DslEngineering}.
    Beispiele aus anderen Sprachen sind abstrakte Klassen und Interfaces.

    Eine denkbare Anwendung dieser Idee in der \gls{wccdl} ist die Selektoren als
    Implementierung zu betrachten. Dann könnte man Klassen mit Features 1x spezifizieren
    und z. B. für verschiedene Sites verschiedene Selektoren angeben.

    Die \gls{wccdl} unterstützt eine solche Trennung aber nicht.
    D.h. Klassen müssen immer inkl. Features definiert werden, für die auch direkt ein Selektore
    ableitbar sein muss.
    Das widerspricht etwas der Idee, die Klassendefinition nur 1x schreiben zu müssen.
    Sonderlocken für Sites stellt Mehraufwand dar. % Aber wenn man doch eh alle Sites checken muss?

    \paragraph{Spezialisierung}
    In vielen Sprachen können Programmelemente andere erweitern und somit eine Spezialisierung formulieren.
    In diesen Fällen erbt das konkrete Element alle Eigenschaften des allgemeinen
    und kann deshalb anstelle des allgemeinen verwendet werden
    \cite[Kapitel 5.1.4]{voelter:DslEngineering}.

    Für die \gls{wccdl} wäre dieses Konzept für Klassen denkbar,
    sodass Klassen von anderen erben und diese erweitern können.
    Die \gls{wccdl} unterstützt diese Funktion allerdings nicht,
    da es ein komplexeres Konzept darstellt, welches für Nicht-Programmierer
    auf den ersten Blick unverständlich sein könnte.

    \paragraph{Typen und Instanzen}
    Die in einer Programmiersprache definierten Typen können bei ihrer Instanziierung
    häufig Parameter entgegen nehmen, wodurch durch ihre Wiederverwendbarkeit erhöht
    und allgemeine Teile gut wiederverwendet werden können.
    Beispiele sind Konstruktor-Parameter von Klassen in OOPs
    \cite[Kapitel 5.1.5]{voelter:DslEngineering}.

    Die \gls{wccdl} ist eine rein deklarative Sprache,
    die lediglich zur Definition von Klassen dient.
    Eine Instanz einer Klasse ist die konkrete Anwendung ihrer Features und
    deren Selektoren auf eine konkrete Seite,
    wodurch konkrete Inhalte und Referenzen klassifiziert und damit eine Instanz
    der Klasse entsteht.
    Innerhalb eines Programmes werden diese Klassen nie instanziiert,
    weshalb die Sprache die Definition von Typparametern nicht vorsieht.
    Eine 
    Auch der Selektor einer Klasse oder eines Features stellt keinen Parameter im
    Sinne von \cite[Kapitel 5.1.5]{voelter:DslEngineering} dar,
    da er für alle Instanzen der jeweligen Klasse/Feature identisch ist und nicht
    überschrieben werden kann.

    \paragraph{Superposition und Aspekte}
    \citet[Kapitel 5.1.6]{voelter:DslEngineering} beschreibt zwei
    allgemeine Sprachkonzepte, durch die Programmfragmente zusammengeführt
    oder verändert werden können.
    Superposition vereint demnach mehrere Fragmente anhand eines speziellen Operators.
    Aspekte bieten die Möglichkeit durch eine Abfrage verschiedene Code-Stellen
    auszuwählen und aspektspezifisch zu modifizieren.

    Die \gls{wccdl} unterstützt keines dieser Konzepte,
    da sie für ihren Anwendungsfall zu komplex sind und keine sinnvolle Anwendung finden.

    \paragraph{Sprachmodularität}
    Diese Konzept beschreibt die Möglichkeit (Teil-)Sprachen wiederzuverwenden
    und zu einer neuen zusammenzusetzen.
    Dadurch kann Konsistenz gewahrt und eine doppelte Implementierung vermieder werden.
    Die verschiedenen Ansätze die \citet[Kapitel 4.6]{voelter:DslEngineering} beschreibt
    sind Language Referencing, Extension, Reuse und Embedding.

    Für die \gls{wccdl} bietet sich die Einbettung der Sprachen der verschiedenen Selektorenarten an.
    D.h. eine Sprache für CSS, eine für XPath und eine für RegEx.
    Dadurch muss die \gls{wccdl} das nicht selbst machen und kann auf vorhandene
    und vermutlich bessere und vollständigere Implementierungen zurückgreifen.
    Dadurch wäre die Korrektheit der Selektoren sichergestellt.

    Bisher unterstützt die Sprache das aber nicht.

    % TODO Versionierung?