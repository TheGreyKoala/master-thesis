\subsection{Strukturelles Design}
    Dieses Kapitel beschreibt einige strukturelle Aspekte der \gls{wccdl},
    die im Wesentlichen auf den Ausführungen in \cite[Kapitel 4, 5.1]{voelter:DslEngineering}
    basieren.

    \paragraph{Separation of Concerns}
    Eine Domäne kann aus verschiedenen Aspekten bestehen,
    die verschiedene Bereiche der Domäne abdecken.
    Alle diese Belange müssen beim Design der Sprache berücksichtigt werden.
    \citet[Kapitel 4.1]{voelter:DslEngineering}
    stellen dazu zwei Ansätze vor: Eine Sprache, die alle Aspekte adressiert
    und mehrere Sprachen, die sich auf jeweils einen Aspekt fokussieren.
    Für den Anwendungsfall der \gls{wccdl} lassen sich die
    folgenden Aspekte identifizieren:

    \begin{enumerate}
        \item Die Deklaration von Klassen, d. h. sie lediglich bekannt zu machen.
        \item Die Definition von Klassen, d. h. die Angabe von Features.
        \item Die Definition von Selektoren für Klassen und Features.
    \end{enumerate}

    Wie aus den bisherigen Ausführungen bereits deutlich wurde,
    deckt die \gls{wccdl} alle diese Aspekte ab,
    da sie sehr eng gekoppelt sind und jeweils einen sehr geringen Umfang besitzen.

    \paragraph{Modularisierung und Sichtbarkeit}
    Die Modularisierung beschreibt die logische Strukturierung
    von Programmelementen und wird durch jede Sprache in irgendeiner Form adressiert.
    Ein Beispiel ist die Strukturierung der Elemente in Namensräume.
    Die Referenzierung eines Elementes sollte immer auf der logischen
    Struktur erfolgen, weshalb die Modularisierung eng mit der Sichtbarkeit
    von Programmelementen verbunden ist.
    Auch dieses Konzept ist in jeder Sprache vertreten und legt fest,
    von wo ein Element referenziert werden kann
    \cite[Kapitel 5.1.1]{voelter:DslEngineering}.
    In Java lässt sich dieser Raum z. B. über die Schlüsselwörter
    \texttt{private} und \texttt{protected} einschränken
    \cite[Kapitel 6.6]{oracle:javaSpec}.
    Die \gls{wccdl} verwendet ein sehr einfaches Konzept bez.
    der Modularisierung und Sichtbarkeit von Porgrammelementen.
    Klassen sind die einzigen Elemente, die in dieser Sprache das Ziel einer Referenz
    sein können, weshalb Features und Selektoren implizit nirgendwo sichtbar sind.
    Klassen sind hingegen global sichtbar,
    sodass sie für jedes Feature genutzt werden können.

    \paragraph{Partitionierung}
    Die Strukturierung eines Programmes in physische Einheiten (Dateien) heißt Partitionierung.
    Da die Referenzierung von Programmelementen nur auf Basis der logischen Struktur erfolgen sollte,
    muss die physische der logischen Strukturierung nicht entsprechen
    \cite[Kapitel 5.1.2]{voelter:DslEngineering}.
    Die \gls{wccdl} erlaubt die Klassendefinitionen auf verschiedenen Dateien
    aufzuteilen, wobei eine Klasse immer vollständig in einer Datei enthalten sein muss.
    Da alle Klassen logisch in einem globalen Namensraum gesammelt werden,
    sind auch Klassen anderer Dateien referenzierbar.

    \paragraph{Scoping und Linking}
    Der Scope eines Querverweises in einem Programm ist die Menge der
    gültigen Ziele dieser Referenz.
    Linking beschreibt den Transformationsprozess vom syntaktischen Baum des Programmes
    zum semantischen Graphen, bei dem Verweise basierend auf den Namen der referenzierten
    Elemente aufgelöst werden
    \cite[Kapitel 8]{voelter:DslEngineering}.
    Querverweise sind in der \gls{wccdl} nur zur Angabe der Klasse eines Features notwendig.
    Der Scope wird an dieser Stelle bereits durch das Sprachkonzept hinreichend
    eingeschränkt, da das Ziel vom Typ
    \texttt{FeatureClass}\footnote{vgl. Kapitel \ref{solutionDetails:dslConcepts}} sein muss.
    Das Linking wird vollständig durch Xtext übernommen
    \cite[Kapitel "`Language Implementation"']{xtext:documentation}.

    \paragraph{Spezifikation und Implementierung}
    Sprachen können die Trennung von Spezifikation und Implementierung von
    Porgrammelementen unterstützen, um eine bessere Entkopplung und verschiedene
    Implementierungen zu ermöglichen \cite[Kapitel 5.1.3]{voelter:DslEngineering}.
    Ein Beispiel aus Java sind Interfaces \cite[Kapitel 9]{oracle:javaSpec}.
    Eine denkbare Anwendung dieser Idee in der \gls{wccdl} ist Selektoren als
    Implementierung zu betrachten. Dann könnte man Klassen inkl. Features einmalig spezifizieren
    und z. B. für verschiedene Sites durch verschiedene Selektoren unterschiedlich "`implementieren"'.
    Die \gls{wccdl} unterstützt eine solche Trennung bisher allerdings nicht.

    \paragraph{Spezialisierung}
    In vielen Sprachen können Programmelemente andere erweitern und somit eine Spezialisierung formulieren.
    In diesen Fällen erbt das konkrete Element alle Eigenschaften des allgemeinen
    und kann deshalb an dessen Stelle verwendet werden
    \cite[Kapitel 5.1.4]{voelter:DslEngineering}.
    Für die \gls{wccdl} wäre dieses Konzept für Klassen denkbar,
    um ihre Selektoren, Features und deren Selektoren zu vererben.
    Sie unterstützt diese Funktion bisher allerdings nicht.

    \paragraph{Typen und Instanzen}
    Die in einer Programmiersprache definierten Typen können bei ihrer Instanziierung
    häufig Parameter entgegen nehmen, wodurch sich ihre Wiederverwendbarkeit steigert.
    Ein Beispiele sind die Parameter eines Konstruktors in objektorientierten Programmiersprachen
    \cite[Kapitel 5.1.5]{voelter:DslEngineering}.
    In der \gls{wccdl} können Features den Selektor ihrer Klasse überschreiben,
    was eine Anwendung dieses Konzeptes ist.

    \paragraph{Superposition und Aspekte}
    \citet[Kapitel 5.1.6]{voelter:DslEngineering} beschreiben zwei
    allgemeine Sprachkonzepte, durch die Programmfragmente zusammengeführt
    oder verändert werden können.
    Superposition vereint mehrere Fragmente anhand eines speziellen Operators.
    Aspekte bieten die Möglichkeit, durch eine Abfrage verschiedene Codestellen
    auszuwählen und aspektspezifisch zu modifizieren.
    Die \gls{wccdl} unterstützt keines dieser Konzepte,
    da sie für ihren Anwendungsfall zu komplex sind und kein sinnvoller Anwendungsfall existiert.

    \paragraph{Sprachmodularität}
    Dieses Konzept beschreibt die Möglichkeit (Teil-)Sprachen wiederzuverwenden
    und zu einer neuen zusammenzusetzen.
    Dadurch kann Konsistenz gewahrt und eine doppelte Implementierung vermieden werden.
    Die verschiedenen Ansätze die \citet[Kapitel 4.6]{voelter:DslEngineering} beschreiben
    sind Language Referencing, Extension, Reuse und Embedding.
    Für die \gls{wccdl} bietet sich die Einbettung anderer Sprachen zur Definition von Selektoren an,
    um Entwickler bei der Formulierung von CSS-, XPath und regulären Ausdrücken zu unterstützen
    und die syntaktische Korrektheit von Selektoren sicherzustellen.
    Bisher setzt die Sprache dies aber nicht um.
