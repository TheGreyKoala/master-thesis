\subsection{Integration in Eclipse}
    Die Nutzung einer Programmiersprache wird durch eine gute Integration in eine
    Entwicklungsumgebung effektiver.
    \citet[Kapitel 12.]{voelter:DslEngineering} zählen eine Reihe von
    Funktionen auf, die solche Werkzeuge bieten können.
    Dazu zählen bspw. die Autovervollständigung und farbliche Hervorhebung von Quelltext
    sowie Refaktorisierungsfunktionen.
    Für viele dieser Mittel generiert Xtext eine Integration
    in die Entwicklungsumgebung Eclipse,
    die bei Bedarf erweitert werden kann\footnote{vgl. Kapitel \ref{section:solutionDetailsDslXtext}}.

    Diese Möglichkeit wurde genutzt, um Selektoren farblich im Editor hervorzuheben.
    Des Weiteren wurde die Autovervollständigung für die Klammern um Selektoren ergänzt,
    da sie auf vielen Tastaturlayouts nicht vorhanden sind.
    Eine dritte Ergänzung sind sprechende Fehlermeldungen für die implementierten
    semantischen Validierungen\footnote{vgl. Kapitel \ref{section:solutionDetailsDslStaticSemantics}}.
    Damit Eclipse eine Datei als {\classificationModel} erkennt und diese Unterstützung bietet,
    muss sie die Dateiendung \texttt{.wcm} besitzen.
    % TODO: Ggf. Screenshots
