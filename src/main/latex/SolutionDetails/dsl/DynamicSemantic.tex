\subsection{Dynamische Semantik}
    Die dynamische Semantik eines Programmes beschreibt das Verhalten
    während der Laufzeit. Deshalb wird sie auch als "`Execution Semantics"' bezeichnet
    \cite[Kapitel 4.3]{voelter:DslEngineering}.

    Die \gls{wccdl} ist eine rein deklarative Sprache.
    Sie beschreibt keinen Kontrollfluss sondern lediglich
    Klassen, deren Features und Selektoren.
    Wie diese Informationen verarbeitet werden, ist für die Sprache unerheblich.
    Deshalb wird der Programmcode nicht in ein ausführbares Programm übersetzt,
    sondern in ein anderes Beschreibungsformat\footnote{vgl. \ref{section:conceptDslGeneration}}.
    Bezüglich des Laufzeitverhaltens trägt die \gls{wccdl}
    deshalb lediglich die Verantwortung, ein {\classificationModel}
    semantisch äquivalent in dieses Format zu übersetzen.