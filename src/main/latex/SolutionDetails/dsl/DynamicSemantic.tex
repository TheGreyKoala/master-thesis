\subsection{Dynamische Semantik}
    Die dynamische Semantik eines Programmes beschreibt sein Verhalten
    während der Laufzeit. Deshalb wird sie auch als Execution Semantics bezeichnet
    \cite[Kapitel 4.3]{voelter:DslEngineering}.

    Die \gls{wccdl} ist eine rein deklarative Sprache.
    Sie beschreibt keinen Kontrollfluss sondern lediglich
    Klassen, deren Struktur (Features) und Selektoren,
    anhand derer Klassifizierungswerkzeuge Klassen und Features erkennen.
    Wie diese Informationen verarbeitet werden, spielt für die Sprache keine Rolle.

    Entsprechend wird der Programmcode nicht in ein ausführbares Programm übersetzt,
    sondern lediglich in ein anderes Beschreibungsformat
    (siehe Kapitel \ref{section:conceptDslGeneration}).

    Bezüglich des Laufzeitverhaltens trägt die \gls{wccdl}
    deshalb lediglich die Verantwortung eine Klassendefinition
    korrekt zu übersetzen.