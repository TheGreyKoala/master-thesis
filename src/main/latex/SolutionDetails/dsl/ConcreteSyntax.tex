\subsection{Konkrete Syntax und deren Grammatik}
    Dieses Kapitel geht auf die konkrete Syntax der DSL
    und deren Grammatik ein.
    Die vollständige Grammatik ist in Anhang \ref{appendix:dsl}
    in Listing \ref{listing:dlsGrammar} zu finden.
    
    Zur Veranschaulichung zeigt Listing \ref{listing:dlsExample} dazu ein Beispiel.
    Da es sich um eine deklarative Sprache handelt,
    sind wie zu sehen viele Schlüsselwörter beschreibend formuliert und verwenden
    den Indikativ anstelle des Imperativs.
    Die Intention der konkreten Syntax ist es, eine Klassendefinition
    so zu formulieren, dass sie als englischer Satz gelesen werden kann.

    \lstinputlisting[
        label=listing:dlsExample,
        caption=Beispielhafte Klassendeklaration,
        language=wccdl,
        inputencoding=utf8/latin1]{../resources/dsl/example.wccd}
    
    Dieses Beispiel definiert die Seitenklasse Service mit einem \gls{url}-Pattern-Selektor
    und den beiden Features pageHeading und images.
    Erstere hat die Inhaltsklasse PageHeading, die ebenfalls im Beispiel definiert wird.
    Das Feature definiert einen eigenen CSS-Selektor.
    Das Feature image ist hingegen ein Referenzfeature vom Typ Image und überschreibt
    den Selektor der Klasse mit einem eigenen.

    Wie aus dem Beispiel hervor geht, ist ein Programm in der DSL eine Sammlung
    von Klassendefinitionen, die deklarativ beschrieben werden.
    Die formale Grammatik einer solchen Definition zeigt Listing \ref{listing:dlsGrammarClasses}.

    \lstinputlisting[
        label=listing:dlsGrammarClasses,
        caption=Klassen in der Grammatik der DSL,
        language=wccdl,
        inputencoding=utf8/latin1]{../resources/dsl/grammar/classes.xtext}

    Jede Klassendefinition beginnt mit einer Folge von Schlüsselwörtern,
    die festlegen, um welche Art von Klasse es sich handelt.
    Die Definition einer Seitenklassen wird demnach mit "`page class"',
    die einer Inhaltsklassen durch "`content class"' und die einer
    Referenzklassen durch "`reference class"' angefangen.
    Anschließend folgt der Name der Klasse, der ein Identifier gemäß
    der Xtext-Terminal Rule ID sein muss.

    Dem Namen kann die Definition des Selektors folgen,
    was durch die Schlüsselwortsequenz "`is recognized by"' geschieht.
    Jede Klasse verwendet dabei eine andere Regel,
    wodurch syntaktisch sichergestellt ist, dass nur semantisch korrekte Selektoren
    verwendet werden.
    Die Syntax eines Selektors wird weiter unten beschrieben.
    Bei Inhalts- und Referenzklassen ist der Selektor optional,
    was man im Beispiel an der Klasse Image sieht.
    In der Grammatik wird das durch die entsprechende Kardinalität deutlich.
    Bei Seitenklassen ist der Selektor nicht optional.
    
    Im Falle von Seiten- und Inhaltsklassen kann anschließend optional die Deklaration von Features folgen.
    Bei Referenzen ist da gemäß Konzept nicht möglich.
    Die Deklaration der Features wird durch das Schlüsselwort "`classifies"' eingeleitet.
    Dem muss mindestens eine Featuredeklaration folgen, es können aber beliebig viele sein.

    Die formale Grammatik eines Features ist in Listing \ref{listing:dlsGrammarFeatures} zu sehen.

    \lstinputlisting[
        label=listing:dlsGrammarFeatures,
        caption=Features in der Grammatik der DSL,
        language=wccdl,
        inputencoding=utf8/latin1]{../resources/dsl/grammar/features.xtext}

    Eine Deklaration eines Features beginnt mit der Angabe des Namens des Features
    (in der Grammatik durch die Regeln "`ScalarFeature"' und "`CollectionFeature"' realisiert),
    wobei wiederum auf die Terminal-Regel ID zurückgegriffen wird.
    Anschließend folgt das Schlüsselwort "`as"' welches die Angabe der Klasse
    des Features einleitet.
    Soll das Feature allerdings ein CollectionFeature sein, folgt zunächst das
    Schlüsselwort "`many"' und anschließend erst der Name der Klasse.
    Dies muss gemäß Konzept eine FeatureClass sein, was entweder eine
    Inhalts- oder eine Referenzklasse sein kann.
    Der Klasse kann ein Selektor folgen.
    Bei ScalarFeatures wird dies durch das Schlüsselwort "`by"',
    im Falle von CollectionFeatures hingegen durch "`each by"' begonnen.
    Der Selektor muss ein FeatureSelector sein.
    Darunter fallen alle Selektoren, die zum Selektieren von Contents oder References geeignet sind.
    Leider kann die Regel FeatureSelector nicht auf ContentSelector | ReferenceSelector verweisen,
    weil dadurch eine Doppeldeutigkeit entstünde, die Xtext nicht auflösen kann.
    Im Beispiel definieren beide Features der Seitenklasse einen Selektor.

    Die Syntax von Selektoren ist für Klassen und Features identisch
    und ist durch die Grammatik in Listing \ref{listing:dlsGrammarSelectors} definiert.

    \lstinputlisting[
        label=listing:dlsGrammarSelectors,
        caption=Selektoren in der Grammatik der DSL,
        language=wccdl,
        inputencoding=utf8/latin1]{../resources/dsl/grammar/selectors.xtext}

    Die Definition eines Selektors beginnt mit einem Schlüsselwort,
    welches die Art des konkreten Selektors bestimmt.
    In der Grammatik passiert das durch die Regeln
    CssSelector, UrlPatternSelector und XPathSelector.
    Demnach beginnt ein Selektor mit "`css"', "`url pattern"' oder "`xpath"'.
    Anschließend folgt in jedem Fall die Definition des Selektors,
    die durch {\flqq } und {\frqq } umschlossen wird.
    Dazwischen können beliebige Zeichen folgen, wobei die Definition mindestens ein Zeichen umfassen muss.
    Ausgenommen sind Zeichen für Backspace, Formfeed, Linebreak, Carriage Return und Tab.
    Kein Zeichen muss escaped werden, wie in anderen Sprachen.
    Das macht die Formulierung besonders einfach.
    Zum Beispiel im UrlPatternSelector muss der Back-Slash nicht escaped werden.
    Das ist gut, weil \glspl{url} enthalten oft einen Slash, der in Regex
    durch Backslash escaped werden muss.
    Der Backslash muss aber nicht escaped werden, was in anderen Sprachen notwendig wäre.
    Dies wird durch die Terminal Regel SELECTOR\_VALUE in der Grammatik umgesetzt.
    Hier wird bewusst nicht die Xtext-Regel STRING verwendet.
    Da müssen Backslash sowie einfache und doppelte Anführungszeichen escaped werden.
    Erlaubt außerdem Zeilenumbrüche innerhalb des Textes zu haben,
    was Selektoren ohne spezielle Nachbehandlung kaputt machen könnte (Regex oder XPath erauben kein Zeilenumbruch).
    Außerdem macht es die Selektoren schlechter lesbar.
    Es werden keine Anführungszeichen verwendet, weil damit viele Strings verbinden
    und ggf. gewisse Erwartungen nicht erfüllt werden. Wie z.B. Dinge escapen zu müssen.
    Außerdem escaped ein Backslash vor dem schließenden Anführungszeichen dieses Zeichen
    und der Text wird nicht abgeschlossen. Also auch ein technischer Grund.
    Und es ist semantisch nicht unbedingt ein String.
    Sprache könnte auch erweitert werden, sodass Code da drin syntaktisch geprüft wird.