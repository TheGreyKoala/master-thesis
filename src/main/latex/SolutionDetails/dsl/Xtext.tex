\subsection{Xtext}
    \label{section:solutionDetailsDslXtext}
    Xtext ist ein Framework zur Entwicklung von \glspl{dsl} und ihrer Integration
    in die Entwicklungsumgebung Eclipse.
    Dies beinhaltet die automatische Generierung eines Parsers,
    das Bauen und Speichern des abstrakten Syntaxbaumes eines Programmes,
    Hilfsmittel zum Schreiben eines Code-Generators oder Interpreters
    sowie eine automatische Umsetzung gängiger Funktionen von
    Entwicklungsumgebungen, wie Quelltexthervorhebung und Autovervollständigung
    \cite[Kapitel 1]{bettini:xtext}.

    Zur Erweiterung und Anpassung dieser Artefakte ist die Sprache Xtend vorgesehen.
    Dabei handelt es sich um eine Java-ähnliche \gls{gpl},
    die zusätzliche Sprachmittel bietet, wie zum Beispiel Typinferenz.
    Sie wurde als Proof-of-Concept selbst mit Xtext entwickelt
    \cite[Kapitel 3]{bettini:xtext}.

    Die Entwicklung einer \gls{dsl} mit Xtext erfordert lediglich die Definition
    einer Grammatik, auf dessen Basis das Framework unter anderem die genannten 
    Artefakte generiert.
    Regeln in dieser Grammatik werden über Anweisungen definiert,
    die der erweiterten Backus-Naur-Form ähneln.
    Zur Generierung eines Parsers setzt Xtext auf
    ANTLR\footnote{vgl. \url{http://www.antlr.org/}},
    welches einen LL(*)-Algorithmus implementiert
    \cite{xtext:documentation,parr:antlr}.
