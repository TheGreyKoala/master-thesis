\subsection{Statische Semantik}
    \label{section:solutionDetailsDslStaticSemantics}
    Die statische Semantik einer Sprache beschreibt alle Bedingungen,
    die zum Zeitpunkt der Kompilierung eines Programmes erfüllt sein müssen.
    \citet[Kapitel 4.3]{voelter:DslEngineering} unterteilen sie in die Kategorien
    Constraints und Type System Rules.
    Die \gls{wccdl} besitzt kein komplexes Typsystem,
    da an keiner Stelle Typen berechnet werden müssen.
    Ein Großteil der typbezogenen Bedingungen wird stattdessen über
    Sprachkonzepte
    sichergestellt.
    Allerdings gibt es einige Bedingungen, die so nicht gewährleistet werden können.
    Hierfür implementiert die Sprache folgende semantische Validierungen:

    \begin{enumerate}
        \item Der Name einer Klasse ist global eindeutig.
        \item Der Name eines Features ist lokal (innerhalb ihrer Klasse) eindeutig.
        \item   Falls die Klasse eines Features keinen Selektor spezifiziert,
                definiert das Feature einen Selektor.
        \item   Falls ein Feature einen Selektor spezifiziert,
                ist dieser mit der Klasse des Features kompatibel.
        \item Der Wert eines Selektors ist nicht leer und besteht nicht ausschließlich aus Leerzeichen.
    \end{enumerate}
