\subsection{Statische Semantik}
    \label{section:solutionDetailsDslStaticSemantics}
    Die statische Semantik einer Sprache beschreibt alle Bedingungen,
    die zum Zeitpunkt der Kompilierung eines Programmes erfüllt sein müssen.
    \citet[Kapitel 4.3]{voelter:DslEngineering} unterteilen sie in die Kategorien
    Constraints und Type System Rules.

    Die \gls{wccdl} besitzt kein komplexes Typsystem,
    da an keiner Stelle Typen berechnet werden müssen.
    Ein Großteil der typbezogenen Bedingungen wird stattdessen über
    Sprachkonzepte\footnote{vgl. Kapitel \ref{solutionDetails:dslConcepts}}
    sichergestellt.

    Allerdings gibt es einige Bedingungen, die so nicht gewährleistet werden können.
    Hierfür implementiert die Sprache einige semantische Validierungen:

    \begin{enumerate}
        \item Klassennamen sind global eindeutig
        \item Namen von Features sind lokal (innerhalb ihrer Klasse) eindeutig
        \item Für jedes Feature kann ein Selektor ermittelt werden
        \item Individuelle Selektoren passen zu den Klassen der jeweiligen Features
        \item Selektoren sind nicht leer und bestehen nicht nur aus Leerzeichen
    \end{enumerate}
