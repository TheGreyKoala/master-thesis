\subsection{Statische Semantik}
    Die statische Semantik einer Sprache beschreibt alle Bedingungen,
    die zum Zeitpunkt der Kompilierung eines Programmes erfüllt sein müssen.
    \citet[Kapitel 4.3]{voelter:DslEngineering} unterteilt sie in zwei Kategorien:
    Constraints und Type System Rules.

    Die \gls{wccdl} besitzt kein komplexes Typsystem,
    da an keiner Stelle Typen berechnet werden müssen.
    Ein Großteil der Typbezogenen Regeln wird über Sprachkonzepte und Syntax (siehe unten)
    sichergestellt.
    Der Selektor einer Klasse hat speziellen Typ.
    Als Feature dürfen nur bestimmte Klassen.

    Allerdings gibt es einige Bedingungen, die nicht über Sprachkonzepte und Syntax gewährleistet werden können.
    Hierfür implementiert die Sprache einige semantische Validierungen.
    Dies sind:
    Sicherstellen, dass Klassennamen global eindeutig sind.
    Sicherstellen, dass die Namen der Features innerhalb einer Klasse eindeutig sind.
    Sicherstellen, dass für jedes Feature ein Selektor ableitbar ist. Entweder über Klasse oder direkt über Feature.
    Falls ein Feature selbst einen Selektor spezifiziert, muss dieser zur Klasse des Features passen.
    Ein Selektor ist nicht leer und besteht nicht nur aus Leerzeichen.