\subsection{Domänenspezifische Sprachen}
    Domänenspezifische Sprachen (engl. Domain specific languages (DSLs))
    sind spezielle Sprachen zum Ausdrücken der Programme einer speziellen
    Problemdomäne \cite[Kapitel 2.2]{voelter:DslEngineering}.

    Im Gegensatz zu \glspl{gpl} sind \glspl{dsl} nicht zwangsläufig Turing-Vollständig
    und deshalb nicht generell austauschbar.
    Stattdessen ist eine \gls{dsl} spezialisiert und optimiert auf eine gegebene Domäne,
    weshalb sie eng mit deren Konzepten und Abstraktionen verbunden ist
    und in Form von speziellen Sprachkonzepten wiederspiegelt.
    Technische und unnötige Details blendet sie aus und überträgt diese Verantwortung
    auf die Execution Engine, d.h. einen Interpreter oder einer Repräsentation in einer
    niedrigeren Sprache, zu der sie transformiert wird.
    Dadurch kann eine \gls{dsl} die Programme ihrer Domäne kürzer und mit einer besseren Semantik
    als \glspl{gpl} ausdrücken \cite[Kapitel 2.2]{voelter:DslEngineering}.

    Durch die Verwendung einer \gls{dsl} steigert sich sowohl die Produktivität
    als auch die Qualität des Ergebnisses,
    da der Entwickler sich stärker auf die eigentliche Problemstellung und weniger
    auf unterstützende technische Aspekte konzentrieren kann.
    Der Quelltext wird dadurch bei gleichbleibender Semantik kürzer und besser les- sowie wartbar.
    Dies wird auch durch die eingeschränkten Möglichkeiten des Entwicklers,
    durch die er weniger Fehler produzieren kann, unterstützt.
    Die stärkere Semantik erleichtert die Implementierung von Analysen und die Formulierung
    hilfreicher Fehlermeldungen.
    Sowohl der Bau als auch Verwendung einer \gls{dsl} fördert das Verständnis der Domäne
    der Entwickler, steigert die Kommunikation und erlaubt die Einbeziehung von Domänenexperten
    \cite[Kapitel 2.5]{voelter:DslEngineering}.
    Aspekte, die auch in \citet{evans:DomainDrivenDesign} Domain Driven Design eine zentrale Rolle spielen.

    % TODO: Vorteile aus SolutionConcept/DSL.tex hier hin? Oder zu Interpretation der Ergebnisse?

    \glspl{dsl} sind natürlich auch mit einigen Herausforderungen verbunden.
    Dazu zählt zunächst der zusätzliche Aufwand, der durch den Entwurf und Bau einer \gls{dsl} entsteht,
    der durch moderne Werkzeuge allerdings verringert werden kann.
    Gleichzeitig erfordert der Entwurf einer guten Sprache Erfahrung,
    da die Domäne nicht nur analysiert und modelliert werden muss,
    sondern auch entschieden werden muss, welche Konzepte in welcher Form in die Sprache übernommen werden.
    Iterative Vorgehen sind hierbei zu empfehlen.
    Die Implementierung einer oftmals nur schwer portiert werden,
    wodurch eine Abhängigkeit zum verwendet Werkzeug entsteht.
    Nicht zuletzt müssen die Nutzer der Sprache diese erlernen,
    was ebenfalls Aufwand erzeugt.
    Gleichzeitig kann sich das parallele Lernen der Domäne und der Sprache positiv auf einander auswirken
    \cite[Kapitel 2.6]{voelter:DslEngineering}.

    \glspl{dsl} sollten generell nicht eingesetzt werden,
    wenn ein kein gemeinsames Verständnis der Domäne und keine Möglichkeit dieses zu erlangen
    existiert.
    Außerdem sollte die Erfahrung der Entwickler in diesem Bereich ausreichend sein,
    um erfolgreich zu sein.

    Eine Umsetzung einer \gls{dsl} kann generell in zwei Formen geschehen:
    Als "`Internal \gls{dsl}"' oder als "`External \gls{dsl}"'.
    Internal \glspl{dsl} nutzen die Sprachmöglichkeiten einer anderen Sprache
    -- meist einer \gls{gpl} --, wodurch sie sich in diese einbetten und
    Programme schnell den Eindruck machen eine ganz neue Sprache zu verwenden.
    Technisch stellen sie aber nur eine geschickte Nutzung der Sprachmöglichkeiten
    einer anderen Sprache, kombiniert mit einigen die Domäne modellierenden Programmierschnittstellen, dar.
    Sie besitzen keinen eigenen Compiler oder Interpreter und stellen deshalb immer gültige Programme
    in der Wirtssprache dar.
    Ihr Vorteil ist ihre vergleichsweise leichte Umsetzung sowie die Möglichkeit das Umfeld der Sprache,
    wie Standardbibliotheken leicht nutzen zu können
    \cite[Kapitel 2.8.1]{voelter:DslEngineering}.

    External \glspl{dsl} stellen hingegen komplett eigenständige Programmiersprachen dar,
    bei denen der Syntax komplett frei definierbar ist,
    wodurch die Domäne oftmals simpler wiedergespiegelt werden kann.
    Sie erfordern allerdings mehr Aufwand, zum Beispiel bei der Definition der Grammatik
    und der Implementierung eines Generators oder Interpreters.

    Bei der entwickelten Sprache handelt es sich um eine externe \gls{dsl}.