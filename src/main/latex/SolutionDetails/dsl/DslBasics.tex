\subsection{Domänenspezifische Sprachen}
    Domänenspezifische Sprachen
    sind spezielle Programmiersprachen zum Ausdrücken der Programme einer eingegrenzten
    Problemdomäne \cite[Kapitel 2.2]{voelter:DslEngineering}.

    Im Gegensatz zu \glspl{gpl} sind \glspl{dsl} nicht zwangsläufig Turing-Vollständig
    und deshalb nicht generell austauschbar.
    Stattdessen ist eine \gls{dsl} spezialisiert und optimiert auf eine gegebene Domäne,
    weshalb sie eng mit deren Konzepten und Abstraktionen verbunden ist
    und in Form von speziellen Sprachkonzepten wiederspiegelt.
    Technische und unnötige Details blendet sie aus und überträgt diese Verantwortung
    auf die Execution Engine.
    Dabei kann es sich konzeptionell um einen Interpreter oder eine Repräsentation in einer
    niedrigeren Sprache handeln.
    Dadurch kann eine \gls{dsl} die Programme ihrer Domäne kürzer und mit einer besseren Semantik
    als \glspl{gpl} ausdrücken \cite[Kapitel 2.2]{voelter:DslEngineering}.

    Durch die Verwendung einer \gls{dsl} steigert sich sowohl die Produktivität
    als auch die Qualität des Ergebnisses,
    da der Entwickler sich stärker auf die eigentliche Problemstellung und weniger
    auf unterstützende technische Aspekte konzentrieren kann.
    Der Quelltext wird dadurch bei gleichbleibender Semantik kürzer und besser les- sowie wartbar.
    Dies wird auch durch die eingeschränkten Möglichkeiten des Entwicklers unterstützt,
    durch die er weniger Fehler produzieren kann.
    Die stärkere Semantik erleichtert die Implementierung von Analysen und die Formulierung
    hilfreicher Fehlermeldungen.
    Sowohl der Bau als auch die Verwendung einer \gls{dsl} fördert das Verständnis der Domäne
    der Entwickler, steigert die Kommunikation und erlaubt die Einbeziehung von Domänenexperten
    \cite[Kapitel 2.5]{voelter:DslEngineering}.
    Aspekte, die auch in Evans' \cite{evans:DomainDrivenDesign} Domain Driven Design eine zentrale Rolle spielen.

    % TODO: Vorteile aus SolutionConcept/DSL.tex hier hin? Oder zu Interpretation der Ergebnisse?

    \glspl{dsl} sind natürlich auch mit einigen Herausforderungen verbunden.
    Dazu zählt zunächst der zusätzliche Aufwand, der durch den Entwurf und Bau einer \gls{dsl} entsteht.
    Dieser kann durch moderne Werkzeuge allerdings verringert werden.
    Gleichzeitig erfordert der Entwurf einer guten Sprache ein gewissess Maß an Erfahrung.
    Entwickler müssen die Domäne verstehen und modellieren und entscheiden,
    welche ihrer Konzepte in welcher Form in die Sprache übernommen werden.
    Iterative Vorgehen sind hierbei zu empfehlen.
    Die Implementierung kann oftmals nur schwer portiert werden,
    wodurch eine Abhängigkeit zum verwendeten Werkzeug entsteht.
    Nicht zuletzt müssen die Nutzer der Sprache diese erlernen,
    was ebenfalls Aufwand erzeugt
    \cite[Kapitel 2.6]{voelter:DslEngineering}.

    Die Umsetzung einer \gls{dsl} kann generell in zwei Formen geschehen:
    Als "`Internal \gls{dsl}"' oder als "`External \gls{dsl}"'.
    Internal \glspl{dsl} nutzen die Sprachmöglichkeiten einer anderen Sprache
    -- meist einer \gls{gpl} --, wodurch sie sich in diese einbetten.
    Genau genommen stellen sie keine neue Sprache,
    sondern nur eine geschickte Nutzung der Sprachmöglichkeiten
    der Wirtssprache dar, die mit einigen die Domäne modellierenden Programmierschnittstellen kombiniert wird.
    Sie besitzen keinen eigenen Compiler oder Interpreter und stellen deshalb immer gültige Programme
    der Wirtssprache dar.
    Ihr Vorteil ist ihre vergleichsweise leichte Umsetzung sowie die Möglichkeit das Umfeld der Sprache,
    wie Standardbibliotheken, leicht nutzen zu können
    \cite[Kapitel 2.8.1]{voelter:DslEngineering}.

    External \glspl{dsl} stellen hingegen komplett eigenständige Programmiersprachen dar,
    bei denen der Syntax komplett frei definierbar ist,
    wodurch die Domäne oftmals simpler wiedergespiegelt werden kann.
    Sie erfordern allerdings mehr Aufwand, zum Beispiel bei der Definition der Grammatik
    und der Implementierung eines Generators oder Interpreters.
    Bei der \gls{wccdl} handelt es sich um eine External \gls{dsl}.

    Konzepte der Domäne kann eine \gls{dsl} in Form von
    "`Linguistic Abstractions"' oder durch "`In-Language Abstractions"'
    wiederspiegeln \cite[Kapitel 4.1.2]{voelter:DslEngineering}.
    Die erste Methode setzt Domänenkonzepte als neue Sprachkonzepte um,
    wodurch sie eindeutig in einer Sprache erkennbar sind
    und dadurch leicht analysiert werden können.
    In-Language Abstractions sind allgemeine Sprachmittel,
    die dem Nutzer der Sprache erlauben Domänenkonzepte selbst umzusetzen.
    Beispiele aus \glspl{gpl} sind Klassen
    \cite[Kapitel 4.1.2]{voelter:DslEngineering}.
    Entwickler haben dadurch mehr Möglichkeiten,
    allerdings werden die Domänenkonzepte schwächer durch die \gls{dsl}
    repräsentiert.