\subsection{Generierung}
    \label{section:solutionDetailsDslGeneration}
    Ein {\classificationModel} wird in ein JSON-Dokument übersetzt,
    wozu ein entsprechender Generator implementiert wurde.

    \paragraph*{Klassen}
    Im generierten Zustand werden die Klassendefinitionen weiterhin nach
    Seiten-, Inhalts- und Referenzklassen unterteilt.
    Im generierten JSON-Dokument geschieht dies durch die Schlüssel \texttt{pageClasses},
    \texttt{contentClasses} und \texttt{referenceClasses}.
    Ihre Objekte enthalten wiederum Schlüssel-Wert-Paare,
    wobei der Schlüssel in diesem Fall der Name einer Klasse und der
    Wert die Klasse selbst ist.
    Klassen speichern ihren Namen zusätzlich in der Eigenschaft \texttt{name},
    sodass sie auch ohne Kontext identifizierbar sind.
    Listing \ref{listing:dlsGenerationPageClass} zeigt dies am Beispiel
    der Seitenklasse \texttt{Service}.
    Inhalts- und Referenzklassen sind sehr ähnlich aufgebaut.

    \lstinputlisting[
        label=listing:dlsGenerationPageClass,
        caption=Ein Beispiel einer Seitenklasse in einem generierten {\classificationModel},
        style=json
    ]{../resources/dsl/generation/page-class.json}

    \paragraph*{Selektoren}
    Unter \texttt{selector} ist in Listing \ref{listing:dlsGenerationPageClass}
    der Selektor der Klasse gespeichert,
    was seinen Typ und seinen Wert umfasst.
    Der Selektor einer Klasse ist niemals undefiniert,
    kann im Falle von Inhalts- oder Referenzklassen aber ein leeres Objekt sein.
    Dann ist der Selektor von den Features zu beziehen.
    Der Typ eines Selektors ist immer \texttt{CssSelector},
    \texttt{UrlPatternSelector} oder \texttt{XPathSelector}.
    Im Beispiel ist zu sehen, dass der {\urlSelector} \verb+\/service\/?$+ codiert wurde.
    Andernfalls würden JSON-Parser z. B. die Rückwärtsschrägstriche als besondere Zeichen interpretieren.
    Der JSON-Standard \cite[Kapitel 7]{rfc:8259} spezifiziert einige Zeichen,
    die innerhalb von Zeichenketten so codiert werden müssen.
    Diese Aufgabe übernimmt der entwickelte Generator,
    sodass Nutzer der \gls{wccdl} sich darum nicht kümmern müssen.
    Eine Ausnahme ist allerdings der {\xpathSelector}.
    XPath 1.0 kennt keine Codierung für besondere Zeichen,
    wie z. B. \texttt{\textbackslash{n}} für einen Zeilenumbruch \cite{w3c:xpath}.
    Außerdem ist in dieser Version die Funktion
    \texttt{codepoints-to-string} \cite[Kapitel 5.2.1]{w3c:xpathXquery} noch nicht enthalten,
    mit der Unicode Codes in Zeichenketten umgewandelt werden können.
    Um trotzdem Inhalte mit besonderen Zeichen beschreiben zu können,
    wird in {\xpathSelector}en der Rückwärtsschrägstrich während der
    Generierung nicht codiert.
    Das heißt, die Sequenz \texttt{\textbackslash{n}} wird genau so in das JSON-Dokument übernommen.
    Ein JSON-Parser setzt das Zeichen zu einem Zeilenumbruch um,
    wodurch der XPath-Ausdruck einen tatsächlichen Umbruch enthält,
    der als solcher interpretiert wird.
    Ein konkretes Anwendungsbeispiel für diese Ausnahme im Generator
    ist im ersten Fallbeispiel in Kapitel \ref{section:findingsTeachersClassificationModel} zu sehen.

    \paragraph*{Features}
    Seiten- und Inhaltsklassen können in den Eigenschaften
    \texttt{contents} und \texttt{references} Features speichern.
    Die Struktur eines generierten Features zeigt Listing
    \ref{listing:dlsGenerationFeature} am Beispiel des {\contentFeature}s \texttt{pageHeading}.

    \lstinputlisting[
        label=listing:dlsGenerationFeature,
        caption=Ein Beispiel eines Features in einem generierten {\classificationModel},
        style=json
    ]{../resources/dsl/generation/page-heading-feature.json}

    Wie schon bei Klassen wird der Name des Features sowohl als Schlüssel des Objektes
    als auch innerhalb des Objektes selbst verwendet.
    Außerdem besitzt ein Feature die Eigenschaft \texttt{class},
    die den Namen der Klasse des Features speichert. Dabei handelt es sich immer
    um den Namen einer Klasse aus \texttt{contentClasses} oder \texttt{referenceClasses}.
    Das JSON-Dokument ist also in sich geschlossen.
    Darüber hinaus besitzt jedes Feature die Eigenschaft \texttt{isCollection},
    die entweder \texttt{true} oder \texttt{false} sein kann.
    Sie definiert,
    ob es sich um ein {\scalarFeature} oder ein {\collectionFeature} handelt.
    Wie Klassen enthält auch ein Feature einen Selektor.
    Ist dieser leer, ist der Selektor der Klasse zu verwenden.
    {\childFeature}s werden innerhalb eines Features ebenfalls unter \texttt{contens} bzw.
    \texttt{references} abgelegt.