\subsection{Generierung}
    Alle Klassendefinitionen eines Projektes werden in ein JSON-Dokument generiert
    und in einer Datei gespeichert.
    Listing \ref{listing:dlsGenerationGlobal} zeigt die oberste Ebene dieses Dokumentes.

    \lstinputlisting[
        label=listing:dlsGenerationGlobal,
        caption=Erste Ebene
    ]{../resources/dsl/generation/global.json}

    Im generierten Zustand werden die Klassendefinitionen weiterhin nach
    Seiten-, Inhalts- und Referenzklassen unterteilt,
    was im JSON-Dokument durch die Eigenschaften "`pageClasses"',
    "`contentClasses"' und "`referenceClasses"' geschieht.
    Jedes Objekte in diesen Properties ist eine Klasse.
    Listing \ref{listing:dlsGenerationPageClass} zeigt dies am Beispiel
    der Seitenklasse "`Service"'.

    \lstinputlisting[
        label=listing:dlsGenerationPageClass,
        caption=Page Class
    ]{../resources/dsl/generation/page-class.json}

    Der Name der Klasse wird an zwei Stellen verwendet.
    Zum einen als Name des Objektes, sodass eine Klasse anhand
    ihres Namens direkt und eindeutig angesprochen werden kann,
    ohne alle Klassen nach dem Namen zu filtern.
    Außerdem wird der Name im Objekt selbst unter der Eigenschaft "`name"'
    gespeichert, sodass das Objekt alleine auch identifizierbar ist.
    Anschließend folgt die Eigenschaft "`selector"',
    die den Selektor der Klasse speichert.
    Dieses Objekt speichert den Typ des Selektors und seinen Wert.
    Die verwendeten Werte für den Typ entsprechen den Namen der Sprachkonzepte:
    "`CssSelector"', "`UrlPatternSelector"' und "`XPathSelector"'.
    Im Beispiel ist zu sehen, dass das verwendete \gls{url} Pattern
    escaped wurde, da einige Zeichen ansonsten von JSON-Interpretern
    als Steuerzeichen interpretiert würden. %TODO: LINK AUF JSON ESCAPE SPEC!
    Dieses escapen passiert automatisch während der Generierung und muss nicht
    vom Entwickler vorgenommen werden.
    Der Selektor einer Klasse ist niemals undefiniert, kann aber ein leeres Objekt sein.
    In diesem Fall ist der Selektor vom Feature zu beziehen.

    Seiten und Inhaltsklassen können Features enthalten.
    Im JSON-Dokument werden diese nach Content und Reference Features unterteilt
    und in den Eigenschaften "`contents"' und "`references"' gespeichert.
    Die Struktur eines Features im JSON-Dokument zeigt Listing
    \ref{listing:dlsGenerationFeature} am Beispiel des Content Features "`pageHeading"'.

    \lstinputlisting[
        label=listing:dlsGenerationFeature,
        caption=Feature
    ]{../resources/dsl/generation/page-heading-feature.json}

    Wie schon bei Klassen wird der Name des Features sowohl als Name des Objektes
    als auch innerhalb des Objektes selbst verwendet.
    Neben dem Namen enthält ein Feature die Eigenschaft "`class"',
    welche den Namen ihrer Klasse speichert. Dabei handelt es sich immer
    um den Namen einer Klasse aus "`contentClasses"' oder "`referenceClasses"'.
    Das JSON-Dokument ist also in sich geschlossen und hat keine Verweise auf andere {\resources}.
    Darüber hinaus besitzt jedes Feature die Eigenschaft "`isCollection"',
    die entweder "`true"' oder "`false"' sein kann und definiert,
    ob es sich um ein Collection Feature handelt oder nicht.
    Wie Klassen enthält auch ein Feature einen Selektor.
    Ist dieser leer, ist der Selektor der Klasse zu verwenden.