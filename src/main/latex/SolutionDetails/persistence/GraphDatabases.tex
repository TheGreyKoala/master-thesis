\subsection{Graphdatenbanken}
    Zur Familie der NoSQL-Datenbanken gehören auch Graphdatenbanken,
    die Informationen in Form eines gerichteten Graphens speichern.
    Anders als relationale Datenbanken besitzen diese Graphen kein Schema
    und basieren auf dem "`Property Graph Model"'.
    Bei diesem besteht der Graph klassisch aus Knoten und Kanten
    (in diesem Kontext auch "`Beziehungen"'),
    die aber beide eine beliebige Menge an Informationen in Form
    von Schlüssel-Wert-Paaren speichern können.
    Beziehungen sind außerdem benannt, stets gerichtet und haben immer einen
    Star- und einen Endknoten
    \cite[Kapitel 1]{robinson:graphdatabases}.
    In dem vom \gls{wccs} verwendeten Graphdatenbanksystem Neo4J sind Knoten ebenfalls benannt.
    Sie können dazu eine beliebige Menge von "`Labels"' besitzen
    \cite[Kapitel 1.2.1.4]{neo4j:documentation}.

    Anders als in relationalen und vielen anderen NoSQL-Datenbanken,
    sind Beziehungen in Graphdatenbanken also First-Class-Citiziens,
    wodurch ihre Abfrage und Auswertung ohne komplexe Aggregierungsfunktionen möglich ist.
    Dadurch eigenen sie sich besser für verbundene Daten,
    auf denen oft gemeinsam Abfragen geschehen
    \cite[Kapitel 2]{robinson:graphdatabases}
    \cite[Kapitel 11.2]{sadalage:nosql}.

    Weitere Stärken sind
    \cite[Kapitel 1]{robinson:graphdatabases}
    \cite[Kapitel 11.1]{sadalage:nosql}:
    
    \begin{enumerate}
        \item   Auch bei größer werdenden Datenmenge gleichbleibende Performanz,
                da Beziehungen nicht berechnet werden müssen.
        \item   Größere Flexibilität bei der Datenmodellierung da zum Beispiel
                neue Beziehungstypen einfach und ohne Risiko oder Anpassungen eingeführt werden können.
        \item   Gute Integration in agile Entwicklungsmethoden.
    \end{enumerate}
    
    Eine Herausforderung bei der Nutzung von Graphdatenbanken ist die Skalierung,
    da Knoten prinzipiell zu jedem anderen Knoten eine Beziehung erhalten können
    und das Aufteilen der Datenbank auf mehrere Server dadurch erschwert wird
    \cite[Kapitel 11.2.5]{sadalage:nosql}.

    Graphdatenbanken sind aus verschiedenen Gründen geeignet für die Anforderungen des \gls{wccs}.
    Innerhalb einer Datenbank werden verschiedene Klassifikationen gespeichert,
    die verschiedene Schema besitzen können, da sie verschiedene Seitenklassen besitzen können.
    Diese sind darüber hinaus frei definierbar und deshalb aus Sicht der Datenbank unvorhersehbar.
    Bei der Verwendung einer relationalen Datenbank hätte es deshalb zwei Alternativen gegeben:
    
    Pro Seiten-, Inhalts-, und Referenzklasse werden zur Laufzeit nach Bedarf Datenbanktabellen angelegt,
    die das Schema der jeweiligen Klasse wiederspiegeln und über Fremdschlüssel zum Beispiel die Beziehung
    zwischen Parent und Child Feature realisieren.
    Tiefe Klassenstrukturen erfordern bei diesem Ansatz aufwändige JOINS.
    Eine Änderung der Klasse eines Features hieße den Datensatz in eine andere Tabelle zu verschieben
    und alle Fremdschlüssel entsprechend zu aktualisieren.
    Vereinzelte Ausnahmen auf Seiten, wie zum Beispiel zusätzliche Informationen heißt bei diesem Ansatz
    eine Erweiterung der betroffenen Klasse und aller Datensätze.

    Eine Alternative bei relationalen Datenbank wäre die Speicherung der Daten in sehr abstrakten Tabellen
    wie "`Page"' und "`Feature"'.
    Die Beziehung zwischen Parent und Child Feature wäre hierbei in einer weiteren Tabelle gespeichert,
    die Paare aus Schlüsseln der Tabelle Feature speichert, wobei einer das Parent und der andere das
    Child Feature identifiziert.
    Die Auflösung dieser Beziehungen für eine ganze Seite würde sich sehr komplex erweisen.
    Beide Ansätze sind theoretisch denkbar, scheinen aber keine optimale Lösung darzustellen.
    
    Graphdatenbanken haben den Vorteil, dass sie das Netzwerk, welches aus den Verweisen zwischen
    Webseiten entsteht, sehr direkt und natürlich abbilden können.
    Außerdem sind Beziehungen, wie zum Beispiel zwischen Parent und Child Feature,
    sehr leicht auszuwerten, was oben bereits beschrieben wurde.
    Des Weiteren sind Ausnahmen in Seiten leicht zu realisieren,
    da der Graph nur um entsprechende Knoten und Beziehungen erweitert werden muss.

    Dies wird in den folgenden Kapiteln noch deutlicher.