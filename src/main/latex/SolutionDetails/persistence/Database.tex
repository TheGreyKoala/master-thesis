\subsection{Schnittstellen und Bedingungen}
    \label{section:solutionDetailsPersistenceDatabse}
    Das \gls{wccs} setzt auf das Graphdatenbanksystem Neo4J in der Version 3.3.1.
    Dieses bietet verschiedene Schnittstellen, die Nutzern des \glspl{wccs} ebenfalls zur Verfügung stehen.
    Dabei handelt es sich um einen RESTful Webservice \cite{neo4j:restDocumentation},
    eine HTTP Schnittstelle zur Ausführung von Cypher Queries \cite[Kapitel 5]{neo4j:documentation}
    und eine Schnittstelle, die ein Binärprotokoll verwendet \cite[Kapitel 4]{neo4j:documentation}.
    Das \gls{wccs} stellt beim Starten des {\classificationStorage}s automatisch sicher,
    das für Knoten mit speziellen Labels und Eigenschaften sogenannte
    Unique Node Property Constraints
    \cite[Kapitel 3.5.2.2]{neo4j:documentation} existieren.
    Dadurch ist zum Beispiel sichergestellt,
    dass jeder \texttt{Resource}-Knoten eine einmalige \gls{url} in der Datenbank besitzt.
    Tabelle \ref{table:solutionDetailsPersistenceDatabaseConstraints}
    listet die Kombinationen von Labels und Eigenschaften auf,
    für die eine solche Bedingung garantiert wird.

    \begin{table}[htb]
        \centering
        \begin{tabular}{|l|l|}
            \hline
            \textbf{Label} & \textbf{Eigenschaft} \\ \hline
            Content        & checksum             \\ \hline
            Resource       & url                  \\ \hline
            Site           & id                   \\ \hline
            Text           & value                \\ \hline
        \end{tabular}
        \caption{Unique Node Property Constraints im {\classificationStorage}}
        \label{table:solutionDetailsPersistenceDatabaseConstraints}
    \end{table}

    Durch das Anlegen einer Unique Node Property Constraint wird auch ein entsprechender Index in der Datenbank angelegt
    \cite[Kapitel 3.5.2.2]{neo4j:documentation}. 
