\subsection{Datenbank / Technologien}
    \label{section:solutionDetailsPersistenceDatabse}
    Das \gls{wccs} setzt auf das Graphdatenbanksystem Neo4J in der Version 3.3.1.
    Diese bietet verschiedenen Schnittstellen, die Nutzern des Systems ebenfalls über die jeweiligen Standardports offen stehen.
    Dabei handelt es sich um eine REST-Schnittstelle, % TODO: https://neo4j.com/docs/rest-docs/3.3/
    eine HTTP API zur Ausführung von Cypher Queries % TODO: https://neo4j.com/docs/developer-manual/3.3/http-api/
    und eine API, die ein Binärprotokoll verwendet. %TODO: https://neo4j.com/docs/developer-manual/3.3/drivers/

    Das System stellt beim Starten automatisch Unique Property Constraints % TODO http://neo4j.com/docs/developer-manual/3.3/cypher/schema/constraints/#query-constraint-unique-nodes
    für die folgenden Kombinationen aus Label und Property:
    :Text(value), :Content(checksum), :Resource(url), :Site(id).
    Dadurch werden automatisch auch Indexe auf diesen Kombinationen erzeugt.