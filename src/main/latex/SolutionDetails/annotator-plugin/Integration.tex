\subsection{Einbindung des Plugins}
    \label{section:solutionDetailsAnnotatorPluginIntegration}
    Das {\annotatorPlugin} kann in jede Webseite eingebunden werden,
    die Annotator verwendet.
    Dazu sind die folgenden Schritte notwendig.

    \paragraph*{Einbinden von Bibliotheken und Styledefinitionen}
    Zunächst muss gewährleistet sein, dass die
    Annotator Bibliothek
    und die Annotator Stylesheets
    \cite[Kapitel "`Getting started with Annotator"']{annotator:documentation}
    korrekt in die Seite eingebunden sind.
    Des Weiteren muss das Annotator Plugin über ein \texttt{script}-Element
    im \texttt{head}-Element der Webseite eingebunden werden.
    Anschließend sollten die Styledefinitionen aus Listing \ref{listing:annotatorCustomStyles}
    dem \texttt{head}-Element der Webseite hinzugefügt werden,
    da Annotator die Struktur der Seite bearbeitet und andernfalls Designprobleme auftreten können.

    \lstinputlisting[
        label=listing:annotatorCustomStyles,
        caption=Zusätzliche Styledefinitionen zur Nutzung des {\annotatorPlugin}s,
        style=css
    ]{../resources/annotator-plugin/custom-styles.css}

    \paragraph*{Annotator Initialisierung}
    Anschließend muss der Initialisierungsaufruf von Annotator ergänzt werden,
    sodass es das neue Plugin verwendet.
    Listing \ref{listing:annotatorInitialization} zeigt dazu ein Beispiel.

    \lstinputlisting[
        label=listing:annotatorInitialization,
        caption=Die Initialisierung von Annotator zur Nutzung des {\annotatorPlugin}s,
        style=js
    ]{../resources/annotator-plugin/annotator-initialize.js}

    Nachdem Annotator auf dem \texttt{body}-Element der Webseite initialisiert wurde,
    werden die verschiedenen Plugins registriert.
    Begonnen wird mit dem
    Storage Plugin \cite[Kapitel "`Plugins"']{annotator:documentation},
    welches für den {\annotationService} konfiguriert wird.
    Entsprechend der vorgestellten Schnittstelle dieser
    Komponente
    legt die Konfiguration des Plugins den Präfix von Anfrage-\glspl{url} fest.
    Anschließend wird das {\annotatorPlugin} über den Schlüssel \texttt{wccs} inkludiert.
    Zuletzt konfiguriert das Skript das
    Permissions Plugin \cite[Kapitel "`Plugins"']{annotator:documentation}.
