\subsection{Einbindung des Plugins}
    Das Plugin kann prinzipiell in jede beliebige Webseite eingebunden werden,
    die Annotator verwendet.
    Dazu sind die folgenden Schritte notwendig.

    \paragraph{Einbinden von Annotator}
    Zunächst muss gewährleistet sein, dass die
    Annotator Bibliothek\footnote{\url{http://assets.annotateit.org/annotator/v1.2.10/annotator-full.min.js}}
    und die Annotator Stylesheets\footnote{\url{http://assets.annotateit.org/annotator/v1.1.0/annotator.min.css}}
    korrekt in die Seite eingebunden sind.

    \paragraph{Einbinden des \gls{wccs}-Annotator-Plugins}
    Des Weiteren muss das Annotator Plugin des \gls{wccs} im head der Seite über ein
    script Tag engebunden werden.

    \paragraph{Einbinden weiterer Style-Definitionen}
    Anschließend sollten die Styledefinition aus Listing \ref{listing:annotatorCustomStyles}
    dem head der Webseite hinzugefügt werden,
    da Annotator das HTML der Seite bearbeitet und andernfalls Designprobleme auftreten können.

    \lstinputlisting[
        label=listing:annotatorCustomStyles,
        caption=Eigene Styles für Annotator,
        style=pseudo
    ]{../resources/annotator-plugin/custom-styles.html}

    \paragraph{Annotator Initialisierung}
    Anschließend muss der Initialisierungsaufruf von Annotator ergänzt werden,
    sodass Annotator das neue Plugin verwendet.
    Listing \ref{listing:annotatorInitialization} zeigt eine Beispielhafte
    Initialisierung von Annotator, die das \gls{wccs}-Plugin integriert.

    \lstinputlisting[
        label=listing:annotatorInitialization,
        caption=Angepasste Annotator-Initialisierung,
        style=pseudo
    ]{../resources/annotator-plugin/annotator-initialize.js}

    Nachdem Annotator auf dem body der Webseite initialisiert wurde,
    werden die verschiedenen Plugins registriert.
    Begonnen wird mit dem
    Storage Plugin\footnote{vgl. \url{http://docs.annotatorjs.org/en/v1.2.x/plugins/store.html}},
    welches für den Annotation Service konfiguriert wird.
    Entsprechend der vorgestellten Schnittstelle dieser
    Komponente\footnote{vgl. Kapitel \ref{section:solutionDetailsAnnotationServiceFunctions}}
    legt die Konfiguration des Plugins den Präfix von Anfrage-\glspl{url} fest.
    Diese beinhaltet die kodierte \gls{url} der aktuellen Seite,
    welche als Schlüssel der Seite fungiert.
    Anschließend wird das \gls{wccs}-Annotator-Plugin über den Schlüssel "`wccs"' inkludiert.
    Zuletzt konfiguriert das Skript das
    Permissions Plugin\footnote{vgl. \url{http://docs.annotatorjs.org/en/v1.2.x/plugins/permissions.html}}.
