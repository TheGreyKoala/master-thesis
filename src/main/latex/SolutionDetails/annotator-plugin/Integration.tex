\subsection{Einbindung des Plugins}
    \label{section:solutionDetailsAnnotatorPluginIntegration}
    Das {\annotatorPlugin} kann in jede Webseite eingebunden werden,
    die Annotator verwendet.
    Dazu sind die folgenden Schritte notwendig.

    \paragraph{Einbinden von Annotator}
    Zunächst muss gewährleistet sein, dass die
    Annotator Bibliothek\footnote{\url{http://assets.annotateit.org/annotator/v1.2.10/annotator-full.min.js}}
    und die Annotator Stylesheets\footnote{\url{http://assets.annotateit.org/annotator/v1.1.0/annotator.min.css}}
    korrekt in die Seite eingebunden sind.

    \paragraph{Einbinden des {\annotatorPlugin}s}
    Des Weiteren muss das Annotator Plugin über ein \texttt{script}-Tag
    im \texttt{head}-Element der Webseite engebunden werden.

    \paragraph{Einbinden weiterer Styledefinitionen}
    Anschließend sollten die Styledefinition aus Listing \ref{listing:annotatorCustomStyles}
    dem \texttt{head}-Element der Webseite hinzugefügt werden,
    da Annotator die Struktur der Seite bearbeitet und andernfalls Designprobleme auftreten können.

    \lstinputlisting[
        label=listing:annotatorCustomStyles,
        caption=Zusätzliche Styledefinitionen zur Nutzung des {\annotatorPlugin}s,
        style=html
    ]{../resources/annotator-plugin/custom-styles.css}

    \paragraph{Annotator Initialisierung}
    Anschließend muss der Initialisierungsaufruf von Annotator ergänzt werden,
    sodass es das neue Plugin verwendet.
    Listing \ref{listing:annotatorInitialization} zeigt dazu ein Beispiel.

    \lstinputlisting[
        label=listing:annotatorInitialization,
        caption=Initialisierung von Annotator zur Nutzung des {\annotatorPlugin}s,
        style=js
    ]{../resources/annotator-plugin/annotator-initialize.js}

    Nachdem Annotator auf dem \texttt{body}-Element der Webseite initialisiert wurde,
    werden die verschiedenen Plugins registriert.
    Begonnen wird mit dem
    Storage Plugin\footnote{vgl. \url{http://docs.annotatorjs.org/en/v1.2.x/plugins/store.html}},
    welches für den {\annotationService} konfiguriert wird.
    Entsprechend der vorgestellten Schnittstelle dieser
    Komponente\footnote{vgl. Kapitel \ref{section:solutionDetailsAnnotationServiceFunctions}}
    legt die Konfiguration des Plugins den Präfix von Anfrage-\glspl{url} fest.
    Anschließend wird das {\annotatorPlugin} über den Schlüssel \texttt{wccs} inkludiert.
    Zuletzt konfiguriert das Skript das
    Permissions Plugin\footnote{vgl. \url{http://docs.annotatorjs.org/en/v1.2.x/plugins/permissions.html}}.
