\section{Annotation Service}
    \label{section:solutionDetailsAnnotationService}
    \subsection{Funktionen und Schnittstellen}
        \begin{table}[htb]
            \centering
            \begin{tabular}{|l|l|}
            \hline
            \textbf{Endpunkt}     & /pages/\{pageId\}\\
            \hline
            \textbf{Methode}      & GET\\
            \hline
            \textbf{Beschreibung} & Liefert Annotator Storage API version\\
            \hline
            \textbf{Status}       & 200\\
            \hline
            \textbf{Antwort}      & \{ ``name'': ``Annotator Store API'', ``version'': ``2.0.0'' \}\\
            \hline
            & \\
            \hline
            \textbf{Endpunkt}     & /pages/\{pageId\}/annotations\\
            \hline
            \textbf{Methode}      & GET\\
            \hline
            \textbf{Beschreibung} & Liefert alle Annotationen einer Seite\\
            \hline
            \textbf{Status}       & 200\\
            \hline
            \textbf{Antwort}      & \{ ``name'': ``Annotator Store API'', ``version'': ``2.0.0'' \}\\
            \hline
            \end{tabular}
            \caption{My caption}
            \label{my-label}
        \end{table}
        % Schnittstellenbeschreibung:
        % - Endpunkt
        % - Methoden
        % -- Input-Dokument
        % -- Status Codes der Antwort
        % -- Return Dokument für jede Antwort

        % - Schnittstelle so, dass später auch anderes Format für Annotationen möglich

    \subsection{Abbildung einer Klassifikation auf Annotationen}
    \subsection{Technologien}