\subsection{Funktionen und Schnitstellen}
    Die Intention der {\classificationStorageAPI} ist es,
    fachliche Anfragen an das System in Anfragen an die Datenbank zu übersetzen,
    sodass deren technischen Details verborgen bleiben.
    Dieses Kapitel widmet sich der Schnittstelle der {\classificationStorageAPI},
    deren \glspl{url} hierarchisch aufgebaut sind.
    Die Beschreibung erfolgt anhand dieser Hierarchie
    und verwendet das bekannte Format\footnote{vgl. Kapitel \ref{section:solutionDetailsClassificationServiceFunctions}}.

    \paragraph{Format einer Klassifikation}
    Zunächst muss aber geklärt werden, wie die {\classificationStorageAPI} das Datenmodell einer
    Klassifikation\footnote{vgl. Kapitel \ref{section:conceptPageDataModel}} konkret als JSON-Dokument umsetzt.
    Listing \ref{listing:solutionDetailsStorageAPIPageFormatPage} zeigt die oberste Ebene,
    die die Informationen zur Seite enthält.
    Features speichert es in den Objekten \texttt{contents} und \texttt{references},
    deren Inhalt in Listing \ref{listing:solutionDetailsStorageAPIPageFormatConten}
    und \ref{listing:solutionDetailsStorageAPIPageFormatReference} ausgelagert wurde.

    \begin{lstlisting}[
        label=listing:solutionDetailsStorageAPIPageFormatPage,
        caption=Format einer Klassifikation in der {\classificationStorageAPI} (1),
        style=json
    ]
{
    "class": "Service",
    "url": "http://www.fernuni-hagen.de/KSW/portale/babw/service/",
    "status": "Classified",
    "contents": { ... },
    "references": { ... }
}
    \end{lstlisting}

    Listing \ref{listing:solutionDetailsStorageAPIPageFormatConten}
    zeigt ein beispielhaftes {\contentFeature},
    wobei der Selektor ausgelassen wurde.
    Dieser wird später vorgestellt.

    \begin{lstlisting}[
        label=listing:solutionDetailsStorageAPIPageFormatConten,
        caption=Format einer Klassifikation in der {\classificationStorageAPI} (2),
        style=json
    ]
"heading": {
    "class": "PageHeading",
    "content": "Service",
    "selector": { ... },
    "contents": { ... },
    "references": { ... }
}
    \end{lstlisting}

    Ein {\referenceFeature}, welches gleichzeitig ein {\collectionFeature} ist,
    zeigt Listing \ref{listing:solutionDetailsStorageAPIPageFormatReference}.
    Der Unterschied zu einem {\scalarFeature} (wie dem oberen) ist,
    dass das Feature kein Objekt, sondern ein Array ist,
    welches wiederum {\scalarFeature}s enthält.

    \begin{lstlisting}[
        label=listing:solutionDetailsStorageAPIPageFormatReference,
        caption=Format einer Klassifikation in der {\classificationStorageAPI} (3),
        style=json
    ]
"fernUniLinks": [{
    "class": "FernUniInternalLink",
    "destination": "http://[ ... ]/babw/service/kontakt/",
    "selector": { ... }
}]
    \end{lstlisting}

    Das Format des eindeutigen Selektors eines Features ist für {\contentFeature}s
    und {\referenceFeature}s identisch.
    Listting \ref{listing:solutionDetailsStorageAPIPageFormatSelector} zeigt ein entsprechendes Beispiel.

    \begin{lstlisting}[
        label=listing:solutionDetailsStorageAPIPageFormatSelector,
        caption=Format einer Klassifikation in der {\classificationStorageAPI} (4),
        style=json
    ]
"type": "RangeSelector",
"startSelector": {
    "type": "XPathSelector",
    "value": "/html[1]/body[1]/div[1]/[ ... ]/h3[1]",
    "offset": 0
},
"endSelector": {
    "type": "XPathSelector",
    "value": "/html[1]/body[1]/div[1]/[ ... ]/h3[1]",
    "offset": 7
}
    \end{lstlisting}

    \paragraph{Sites abfragen}
    Die {\classificationStorageAPI} erlaubt es alle bekannten Sites abzurufen.
    Dazu bietet sie die Schnittstelle in Tabelle \ref{table:getSitesInterface}.

    \begin{table}[htb]
        \centering
        \begin{tabular}{|l|l|}
        \hline
        \textbf{Endpunkt} & \texttt{http://<HOST>:52629/sites}\\
        \hline
        \textbf{Methode} & \texttt{GET}\\
        \hline
        \textbf{Statuscode} & \texttt{200 (OK)}\\
        \hline
        \textbf{Ausgabedokument} & \lstinputlisting[title=Test123]{../resources/classification-storage-api/sites.json}\\
        \hline
        \end{tabular}
        \caption{Schnittstelle zum Abfragen aller Sites}
        \label{table:getSitesInterface}
    \end{table}

    Das Ausgabedokument sammelt alle Sites im Array \texttt{sites}, wobei jedes Objekt die ID der jeweiligen Seite enthält.

    \paragraph{Seiten einer Site abfragen}
    Interessant an einer Site ist die Information, welche klassifizierten Webseiten zu ihr gehören.
    Diese lässt sich über die Schnittstelle in Tabelle \ref{table:getSitePagesInterface} für eine spezifische Site ermitteln.

    \begin{table}[htb]
        \centering
        \begin{tabular}{|l|l|}
        \hline
        \textbf{Endpunkt} & \texttt{http://<HOST>:52629/sites/<SITE\_ID>/pages}\\
        \hline
        \textbf{Methode} & \texttt{GET}\\
        \hline
        \textbf{Statuscode} & \texttt{200 (OK)}\\
        \hline
        \textbf{Ausgabedokument} & \lstinputlisting[title=PagesOfSite]{../resources/classification-storage-api/site-pages.json}\\
        \hline
        \end{tabular}
        \caption{Schnittstelle zum Abfragen aller klassifizierter Webseiten einer Site}
        \label{table:getSitePagesInterface}
    \end{table}

    Wie zu sehen ist, muss die Anfrage-\gls{url} die ID der Site beinhalten.
    Die Seiten einer Site sind im Ausgabedokument im Array \texttt{pages} abgelegt.
    Jede Seite enthält ihren \gls{url}, ihre Klasse und ihren Status.
    Informationen über ihre Features sind nicht enthalten,
    da der Fokus wie beschrieben auf der fachlichen Frage besteht,
    welche Webseiten zu einer Site gehören.

    \paragraph{Klassifikation einer Seite}
    Die interessanteste Information, die die {\classificationStorageAPI} liefern kann,
    ist selbstverständlich die vollständige Klassifikation einer Seite.
    Diese kann für eine spezifische Webseite über zwei Schnittstellen ermittelt werden.
    Beide werden in Tabelle \ref{table:getFullPageInterface} vorgestellt.

    \begin{table}[htb]
        \centering
        \begin{tabular}{|l|l|}
        \hline
        \textbf{Endpunkt} & \texttt{http://<HOST>:52629/[ sites/<SITE\_ID>/ ]pages/<PAGE\_ID>}\\
        \hline
        \textbf{Methode} & \texttt{GET}\\
        \hline
        \textbf{Statuscode} & \texttt{200 (OK)}\\
        \hline
        \textbf{Ausgabedokument} & Klassifikation\\
        \hline
        \end{tabular}
        \caption{Schnittstelle zum Abfragen der Klassifikation einer Webseite}
        \label{table:getFullPageInterface}
    \end{table}

    Die Anfrage-\gls{url} kann die Informationen der Site beinhalten,
    wodurch die bisher verfolgte Hierarchie der {\resources} fortgesetzt wird.
    Die \gls{url} einer Seite identifiziert sie allerdings eindeutig,
    weshalb die Site auch ausgelassen werden kann.
    Das macht sich zum Beispiel das {\annotatorPlugin} zunutze,
    welches die Site der aktuellen Webseite nicht bestimmen
    kann\footnote{vgl. Kapitel \ref{section:solutionDetailsAnnotatorPlugin}}.
    Die Antwort ist in beiden Fällen eine vollständige Klassifikation,
    wie sie am Anfang dieses Kapitels vorgestellt wurde.

    \paragraph{Klassifikation einer Seite speichern}
    Neben dem Lesen einer Klassifikation unterstützt die Schnittstelle auch das Schreiben einer Klassifikation,
    worauf der {\classificationService} und das {\annotatorPlugin} zurückgreifen.
    Auch hierfür bietet die Komponente zwei Schnittstellen, die in Tabelle \ref{table:writePageInterface} zu sehen sind.

    \begin{table}[htb]
        \centering
        \begin{tabular}{|l|l|}
        \hline
        \textbf{Endpunkt} & \texttt{http://<HOST>:52629/[ sites/<SITE\_ID>/ ]pages/<PAGE\_ID>}\\
        \hline
        \textbf{Methode} & \texttt{PUT}\\
        \hline
        \textbf{Eingabedokument} & Klassifikation\\
        \hline
        \textbf{Statuscode} & \texttt{201 (Created)} oder \texttt{204 (No Content)}\\
        \hline
        \end{tabular}
        \caption{Schnittstelle zum Schreiben der Klassifikation einer Webseite}
        \label{table:writePageInterface}
    \end{table}

    Wie beim Lesen einer Klassifikation kann die Anfrage-\gls{url} optional auch die ID der Site enthalten.
    Als Eingabe erwartet der Service eine vollständige Klassifikation im oben beschriebenen Format.
    Der Statuscode zeigt an, ob eine neue Klassifikation angelegt (201) oder eine vorhandenen aktualisiert wurde (204).
