\subsection{Funktionen und Schnitstellen}
    Die Intention der Classification Storage API ist es,
    fachliche Anfragen an die Datenbank in technische Datenbankanfragen zu übersetzen,
    sodass die technischen Details der Datenbank vor dem restlichen System verborgen bleiben.
    Zu diesem Zweck bietet der Service eine REST-Schnittstelle,
    die die unterstützen Operationen beschreibt.
    Dieser Schnitstelle und den fachlichen Funktionen widmet sich dieses Kapitel.
    Die \glspl{url} der Schnittstelle sind hierarchisch aufgebaut.
    Die folgenden Ausführungen folgen dieser Hierarchie und beginnen deshalb mit der obersten Funktion.

    \paragraph{Format einer Seite}
    Zunächst soll aber geklärt werden, wie die Classification Storage API das Datenmodell einer Seite
    (siehe Kapitel \ref{section:conceptPageDataModel}) konkret umsetzt.
    % TOODO: KONKRETES FORMAT EINER SEITE

    \paragraph{Sites abfragen}
    Die Classification Storage API erlaubt es alle bekannten Sites abzurufen.
    Dazu bietet sie die Schnittstelle in Tabelle \ref{table:getSitesInterface}.

    \begin{table}[htb]
        \centering
        \begin{tabular}{|l|l|}
        \hline
        \textbf{Endpunkt} & \texttt{http://<HOST>:52629/sites}\\
        \hline
        \textbf{Methode} & \texttt{GET}\\
        \hline
        \textbf{Status} & \texttt{200 (OK)}\\
        \hline
        \textbf{Ausgabedokument} & \lstinputlisting[title=Test123]{../resources/classification-storage-api/sites.json}\\
        \hline
        \end{tabular}
        \caption{Schnittstelle zum Holen aller Sites}
        \label{table:getSitesInterface}
    \end{table}

    Das Ausgabedokument sammelt alle Sites im Array "`sites"', wobei jedes Objekt die ID der jeweiligen Seite enthält.

    \paragraph{Seiten einer Site abfragen}
    Interessant an einer Site ist die Information, welche Seiten zu ihr gehören.
    Diese lässt sich über die Schnittstelle in Tabelle \ref{table:getSitePagesInterface} für eine spezifische Site abfragen.

    \begin{table}[htb]
        \centering
        \begin{tabular}{|l|l|}
        \hline
        \textbf{Endpunkt} & \texttt{http://<HOST>:52629/sites/<SITE\_ID>/pages}\\
        \hline
        \textbf{Methode} & \texttt{GET}\\
        \hline
        \textbf{Status} & \texttt{200 (OK)}\\
        \hline
        \textbf{Ausgabedokument} & \lstinputlisting{../resources/classification-storage-api/site-pages.json}\\
        \hline
        \end{tabular}
        \caption{Schnittstelle zum Holen aller Seiten einer Site}
        \label{table:getSitePagesInterface}
    \end{table}

    Wie zu sehen ist, muss die Anfrage-\gls{url} die ID der Site beinhalten.
    Die Seiten einer Site sind im Ausgabedokument im Array "`pages"' abgelegt.
    Jede Seite enthält ihre \gls{url}, ihre Klasse und ihren Status.
    Informationen über ihre Features sind nicht enthalten,
    da der Fokus wie beschrieben auf der fachlichen Frage besteht,
    welche Seiten zu einer Site gehören.
    Die vollständige Klassifikation einer Site kann über die folgende Schnittstelle bezogen werden.

    \paragraph{Klassifikation einer Seite}
    Die interessanteste Information die das \gls{wccs} liefern kann,
    ist selbstverständlich die vollständige Klassifikation einer Seite.
    Diese kann für eine spezifische Webseite über zwei Schnittstellen ermittelt werden.
    Beide werden in Tabelle \ref{table:getFullPageInterface} vorgestellt.

    \begin{table}[htb]
        \centering
        \begin{tabular}{|l|l|}
        \hline
        \textbf{Endpunkt} & \texttt{http://<HOST>:52629/[ sites/<SITE\_ID>/ ]pages/<PAGE\_ID>}\\
        \hline
        \textbf{Methode} & \texttt{GET}\\
        \hline
        \textbf{Status} & \texttt{200 (OK)}\\
        \hline
        \textbf{Ausgabedokument} & Page\\
        \hline
        \end{tabular}
        \caption{Schnittstelle zum Holen der Klassifikation einer Seite}
        \label{table:getFullPageInterface}
    \end{table}

    Die Anfrage-\gls{url} kann die Informationen der Site beinhalten,
    wodurch die bisher verfolgte Hierarchie der {\resources} fortgesetzt wird.
    Die ID eine Seite (ihre \gls{url}), die in jeden Fall enthalten sein muss, identifiziert sie allerdings eindeutig,
    weshalb die Site auch ausgelassen werden kann.
    Das macht sich zum Beispiel das Annotator Plugin zunutze,
    die die Site der Seite, auf dem es ausgeführt wird, nicht bestimmen kann
    (siehe Kapitel \ref{section:solutionDetailsAnnotatorPlugin}).
    Die Antwort ist in beiden Fällen eine vollständige Seite,
    wie am Anfang diese Kapitels vorgestellt wurde.

    \paragraph{Klassifikation einer Seite speichern}
    Neben dem Lesen einer Klassifikation unterstützt die Schnittstelle natürlich auch das Schreiben einer Klassifikation,
    worauf der Classification Service und das Annotator Plugin zurückgreifen.
    Auch hierfür bietet die Komponente zwei Schnittstellen, die in Tabelle \ref{table:writePageInterface} zu sehen sind.

    \begin{table}[htb]
        \centering
        \begin{tabular}{|l|l|}
        \hline
        \textbf{Endpunkt} & \texttt{http://<HOST>:52629/[ sites/<SITE\_ID>/ ]pages/<PAGE\_ID>}\\
        \hline
        \textbf{Methode} & \texttt{PUT}\\
        \hline
        \textbf{Eingabedokument} & Page\\
        \hline
        \textbf{Status} & \texttt{201 (Created)} oder \texttt{204 (No Content)}\\
        \hline
        \end{tabular}
        \caption{Schnittstelle zum Schreiben der Klassifikation einer Seite}
        \label{table:writePageInterface}
    \end{table}

    Wie beim Lesen einer Klassifikation kann die Anfrage-\gls{url} optional auch die Site-Informationen enthalten.
    Als Eingabe erwartet der Service eine Seite, wie sie oben beschrieben wurde.
    Der Status der Antwort zeigt an, ob eine neue Klassifikation angelegt oder eine vorhandenen aktualisiert wurde.
