\subsection{Erstmaliges Speichern einer Klassifikation}
    \label{section:solutionDetailsStorageAPIInitialWrite}
    Dieses Kapitel beschreibt, wie der Service eine neue Klassifikation in die Datenbank schreibt.

    Beim initialen Schreiben einer Klassifikation verwendet der Service einen
    einfach Bottom-Up-Algorithmus, durch denn alle Knoten und Beziehungen erstellt werden.
    Dieser ist in den Listing \ref{listing:storeClassificationPageFlow} und
    \ref{listing:storeClassificationContentReferenceFunctions} skizziert,
    wobei ersteres den globalen Ablauf aus der Sicht der Seite darstellt und
    letztere die Funktionen zum Speichern von Features erklärt.
    
    \lstinputlisting[
        label=listing:storeClassificationPageFlow,
        caption=Globaler Kontrollfluss zum Speichern,
        style=pseudo
    ]{../resources/classification-storage-api/store-classification-alg/page-flow.code}

    Da es sich um einen Bottom-Up-Algorithmus handelt,
    wird zunächst der Teilgraph für alle Features,
    das heißt für alle Inhalte und Referenzen der Seite, angelegt.
    Die entsprechenden Funktionen werden gleich anhand Listing \ref{listing:storeClassificationContentReferenceFunctions} erläutert.
    Anschließend erstellt der Service einen Knoten für die Seite selbst.
    Was dann noch fehlt sind die Kanten vom Knoten der Seite,
    zu denen der Content Features bzw. referenzierter {\resources}.
    Aus diesem Grund iteriert der Algorithmus nochmal über alle Features
    und letg die entsprechenden Beziehungen an.
    Dabei unterscheidet er zwischen skalaren und mehrelementigen Features,
    bei denen für jedes Element eine Beziehung angelegt wird.
    Die verwendeten Funktionen "`contentRelationship"' und "`referenceRelationship"',
    erzeugen lediglich konkrete Datenbankanweisungen, die im späteren Verlauf thematisiert werden.

    Wie aus Listing \ref{listing:storeClassificationContentReferenceFunctions} deutlich wird,
    arbeitet der Algorithmus bei der Erzeugung von Features sehr ähnlich.

    \lstinputlisting[
        label=listing:storeClassificationContentReferenceFunctions,
        caption=Funktionen zum Schreiben von Features,
        style=pseudo
    ]{../resources/classification-storage-api/store-classification-alg/content-reference-functions.code}

    Im Falle einer Referenz legt er lediglich einen passenden {\resource}-Knoten an,
    was er bei einem CollectionFeature durch einen rekursiven Aufruf für alle Element tut.
    
    Das Anlegen eines Content Features ist etwas komplexer,
    da diese selbst Features besitzen können.
    Dem Ansatz eines Bottom-Up-Algorithmus folgend, legt der Service
    zunächst die Teilgraphen der Child Features für das aktuelle Feature an,
    indem er sowohl über alle Contents als auch alle Referenzen iteriert und
    sich selbst für jedes rekursiv aufruft.
    Anschließend legt zunächst einen Text-Knoten und dann einen Content-Knoten
    für das aktulle ContentFeature an.
    Die im Listing verwendeten Funktionen erzeugen wiederum nur Datenbankanweisungen,
    die später erklärt werden.
    Danach iteriert er wiederum über alle Child Features, um die Beziehungen vom
    gerade erzeugten Content-Knoten, zu den Knoten der Features anzulegen.
    Abschließend legt er eine Reads-Kante zum Text-Knoten an.
    
    Nach dem vollständigen Durchlaufen dieses Algorithmus,
    ist die Klassifikation vollständig in der Datenbank gespeichert.

    Angewandt auf das Beispiel aus Kapitel \ref{section:solutionDetailPersistenceDataModelExample}
    legt der Algorithmus Knoten und Kanten in der folgenden Reihenfolge an:
    c1\_text, c1, rel4,
    c2\_text, c2, rel5,
    c4\_text, c4, rel8
    r1, r2, c5\_text, c5,
    rel9, rel10, rel11,
    c3, rel6, rel7,
    p, rel1, rel2, rel3.