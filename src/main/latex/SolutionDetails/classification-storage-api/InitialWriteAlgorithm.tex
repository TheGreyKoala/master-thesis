\subsection{Erstmaliges Speichern einer Klassifikation}
    \label{section:solutionDetailsStorageAPIInitialWrite}
    Das Schreiben einer Klassifikation in die Datenbank ist die
    komplexeste Aufgabe der {\classificationStorageAPI}.
    Um sich dieser Schrittweise zu nähern, beschreibt dieses Kapitel zunächst,
    wie der Service eine neue Klassifikation in die Datenbank schreibt.
    Dazu verwendet er einen
    einfach Bottom-Up-Algorithmus, durch den alle Knoten und Beziehungen
    in der Datenbank erstellt werden.
    Dieser ist in den Listing \ref{listing:storeClassificationPageFlow} und
    \ref{listing:storeClassificationContentReferenceFunctions} skizziert.
    Ersteres stellt den Ablauf auf einer höheren Ebene dar, wohingegen
    das zweite Listing die Funktionen zum Speichern von Features ergänzt.
    
    \lstinputlisting[
        label=listing:storeClassificationPageFlow,
        caption=Algorithmus zum Schreiben einer Klassifikation (1),
        style=pseudo
    ]{../resources/classification-storage-api/store-classification-alg/page-flow.code}

    Da es sich um einen Bottom-Up-Algorithmus handelt,
    werden in den ersten beiden Zeilen von Listing \ref{listing:storeClassificationPageFlow}
    zunächst die Teilgraphen für alle Features der Seite angelegt.
    Die entsprechenden Funktionen werden gleich anhand Listing \ref{listing:storeClassificationContentReferenceFunctions} erläutert.
    Anschließend erstellt der Service einen Knoten für die Webseite selbst.
    Was dann noch fehlt sind die Kanten vom Knoten der Seite,
    zu denen der {\contentFeature}s bzw. zu denen referenzierter {\resources}.
    Aus diesem Grund iteriert der Algorithmus nochmal über alle Features
    und legt die entsprechenden Beziehungen an.
    Dabei unterscheidet er zwischen skalaren und mehrelementigen Features.
    Im Falle eines {\collectionFeature}s legt er für jedes Element eine Beziehung an.
    Die verwendeten Funktionen \texttt{contentRelationship} und \texttt{referenceRelationship}
    erzeugen konkrete Datenbankanweisungen, die im späteren Verlauf thematisiert
    werden\footnote{vgl. Kapitel \ref{section:solutionDetailsClassificationStorageAPIUpdatePage}}.
    Wie aus Listing \ref{listing:storeClassificationContentReferenceFunctions} deutlich wird,
    arbeitet der Algorithmus bei der Erzeugung von Features sehr ähnlich.

    \lstinputlisting[
        label=listing:storeClassificationContentReferenceFunctions,
        caption=Algorithmus zum Schreiben einer Klassifikation (2),
        style=pseudo
    ]{../resources/classification-storage-api/store-classification-alg/content-reference-functions.code}

    Im Falle einer Referenz legt die Funktion \texttt{reference} einen passenden \texttt{Resource}-Knoten an,   
    was sie bei einem {\collectionFeature} durch einen rekursiven Aufruf für alle Elemente wiederholt.
    Das Anlegen eines {\contentFeature}s ist etwas komplexer,
    da es selbst Features besitzen kann.
    Dem Ansatz eines Bottom-Up-Algorithmus folgend, legt die Funktion \texttt{content}
    zunächst die Teilgraphen der {\childFeature}s an,
    indem sie sowohl über alle Inhalte als auch über alle Referenzen iteriert.
    Für {\contentFeature}s ruft sie sich rekursiv auf,
    {\referenceFeature}s gibt sie hingegen an die Funktion \texttt{reference} weiter.
    Anschließend legt sie -- falls erforderlich -- einen \texttt{Text}-Knoten und dann einen \texttt{Content}-Knoten
    für das Feature an.
    Danach iteriert sie erneut über alle {\childFeature}s, um die Beziehungen vom
    gerade erzeugten \texttt{Content}-Knoten zu den Knoten der {\childFeature}s anzulegen.
    Abschließend legt sie bei Bedarf eine \texttt{Reads}-Kante zum \texttt{Text}-Knoten an.

    Angewandt auf das Beispiel aus Kapitel \ref{section:solutionDetailPersistenceDataModelExample},
    legt dieser Algorithmus die Knoten und Kanten in der folgenden Reihenfolge an:
    \texttt{c1\_text}, \texttt{c1}, \texttt{rel4},
    \texttt{c2\_text}, \texttt{c2}, \texttt{rel5},
    \texttt{c4\_text}, \texttt{c4}, \texttt{rel8}
    \texttt{r1}, \texttt{r2}, \texttt{c5\_text}, \texttt{c5},
    \texttt{rel9}, \texttt{rel10}, \texttt{rel11},
    \texttt{c3}, \texttt{rel6}, \texttt{rel7},
    \texttt{p}, \texttt{rel1}, \texttt{rel2}, \texttt{rel3}.