\subsection{Aktualisierung einer Klassifikation}
    \label{section:solutionDetailsClassificationStorageAPIUpdatePage}
    Dieses Kapitel beschreibt, wie der Service eine Klassifikation in der Datenbank aktualisiert.
    Der verwendete Algorithmus muss einige Sonderfälle beachten,
    um zum Beispiel die Idempotenz der Operation sicherzustellen.
    Zur besseren Veranschaulichung verwendet dieses Kapitel das Beispiel
    aus Kapitel \ref{section:solutionDetailPersistenceDataModelExample}

    \paragraph{Fälle}
    Zur Beschreibung des Vorgehens bei einer Aktualisierung einer Klassifikation,
    sollte zunächst die Frage geklärt sein, wie eine Änderung an einer Klassifikation aussehen kann.
    Es lassen sich diese Fälle identifizieren:

    \begin{enumerate}
        \item Ein neues Content Feature wurde auf einer beliebigen Ebene hinzugefügt
        \item Eine Referenz zu einer bisher unbekannten {\resource} wurde auf einer beliebigen Ebene hinzugefügt
        \item Eine Referenz zu einer bereits bekannten {\resource} wurde auf einer beliebigen Ebene hinzugefügt
        \item Ein Feature inklusive seiner Child Features wurde auf einer beliebigen Ebene gelöscht
        \item Die Klasse einer Seite wurde verändert
    \end{enumerate}

    Wie im weiteren Verlauf deutlich wird,
    händelt der Service Änderungen an Features als eine Kombination aus
    Hinzufügen von Knoten für die neue Version und Löschen der Knoten der alten Version.
    Es wird außerdem deutlich, warum das für eine Seite nicht so gemacht wird.

    \paragraph{Herausforderung}
    Anstatt alle vorhandenen Knoten und Kanten einer Seite zu löschen
    und neu anzulegen, was sehr viel I/O verursachen würde,
    versucht der Algorithmus stattdessen den vorhandenen Graphen anzupassen,
    sodass er die übermittelte Klassifikation korrekt wiederspiegelt.
    
    Die Herausforderung dabei ist, dass der Service
    die einzelnen Schritte, die von der alten zur neuen Version der Klassifikation
    führen, nicht live verfolgen und entsprechende Anweisungen für die Datenbank generieren kann.
    Stattdessen erhält der Service die vollständige neue Version
    und muss die Änderungsschritte für das "`Object-Graph-Mapping"' selbst rekonstruieren.

    \paragraph{Lösungsidee}
    Anstatt die Differenz zwischen der alten und der neuen Klassifikation zu ermitteln,
    um entpsrechende Anpassungsanweisungen zu generieren,
    begegnet die Classification Storage API dieser Herausforderung,
    indem sie den Algorithmus aus Kapitel \ref{section:solutionDetailsStorageAPIInitialWrite}
    jedes Mal ausführt und dabei Datenbankanweisungen generiert,
    die nur bei Bedarf Änderungen an Knoten und Kanten vornehmen oder diese neu anlegen.
    Dadurch ist der Algorithmus unabhängig vom aktuellen Datenbankzustand.
    Zwei Beispiele sollen diese Idee verdeutlichen.

    \paragraph{Beispiel 1 -- Eine unveränderte Seite}
    Zunächst wird der Fall betrachtet, dass eine Klassifikation unverändert erneut in die Datenbank geschrieben wird.
    Im Detail wird dazu das Feature c1 aus dem Beispiel in Kapitel
    \ref{section:solutionDetailPersistenceDataModelExample} betrachtet.
    Zunächst wird für diesen Knoten festgestellt,
    dass ein Text-Knoten mit dem passenden Inhalt existiert und wiederverwendet werden kann.
    Ebenso existiert bereits ein passender Content-Knoten und eine Reads-Beziehung (rel4) zwischen diesen Knoten.
    Am Ende stellt der Algorithmus fest, dass sowohl der Page-Knoten als auch
    die Beziehung zu c1 nicht neu angelegt werden müssen.
    Insgesamt hat der Algorithmus also die gesamte zu schreibende Klassifikation abgearbeitet
    und hat keine Schreiboperationen in der Datenbank durchgeführt.
    Genau genommen wird diese Logik nicht durch den Kontrollfluss im Service umgesetzt,
    sondern durch die Datenbankanweisungen, die durch diesen erzeugt werden.
    % TODO: Erwähnen, welche Funktionen passende STatements erzeugen müssen.

    \paragraph{Beispiel 2 -- Eine veränderte Seite}
    Nun wird an einem zweiten Beispiel gezeigt,
    wie das Vorgehen bei einer Änderung ist.
    Dazu wird angenommen, dass sich die Klasse von c1 verändert hat.
    Wie zuvor wird der Text-Knoten wiederverwendet,
    da sich der Inhalt des Features nicht verändert hat.
    Anders als zuvor wird aber kein passender Content Knoten gefunden,
    weshalb ein neuer Knoten c1' erzeugt wird, der die neuen Informationen enthält.
    Genauso wird eine neue Kante rel4' ausgehend von c1' zu c1\_text erzeugt.
    Am Ende wird der Page-Knoten wiederverwendet,
    allerdings existiert keine Kante von p zu c1'.
    Diese wird deshalb in Form von rel1' angelegt,
    wohingegen rel1, die zum veralteten c1 führt, gelöscht wird.
    Da c1 ebenfalls verwaist ist, wird er inklusive rel4 gelöscht.

    Einer analogen Logik folgend geht der Service auch für neue Content oder
    Reference Features vor.
    
    \paragraph{Konsequenz für den Graphen}
    Das bis hierhin beschriebene Verfahren hat bereits eine wichtige
    Konsequenzen für den Graphen.
    Sowohl Text- als auch Content-Knoten können nämlich mehrere einkommende Kanten besitzen.
    Für Text-Knoten ist das der Fall, wenn es mehrere Features mit identischen Inhalt gibt
    und für Content-Knoten, wenn es mehrere identische Content Features gibt.
    Das soll wiederum am Beispiel von c1 verdeutlicht werden,
    welches die Überschrift der Seite p mit dem textuellen Inhalt "`Einstieg"' darstellt.
    Angenommen es existiert eine zweite Seite p',
    die ebenfalls diese Überschrift besitzt.
    Das beschriebene Verfahren legt dann keinen neuen Content-Knoten an,
    sondern lediglich eine Kante, die von p' zu c1 führt.
    Da der Selektor des Features an der Kante gespeichert wird, ist das ohne weiteres möglich.
    Genauso dürfen Name des Features oder isCollection anders sein.
    Eine Abzweigung findet immer an dem Punkt statt,
    ab dem sich zwei Seiten unterscheiden.
    Das heißt es ist auch möglich, dass nur Child Features tieferer Ebenen
    von verschiedenen Parent Features referenziert werden.
    Das wird im weiteren Verlauf noch deutlicher gezeigt.    
    Diese Eigenschaft ist nützlich für identische und sich oft wiederholende Bereiche,
    wie Header oder Footer, die dann nur einmalig gespeichert werden und außerdem
    Überschneidungen explizit und leicht auswertbar machen.

    \paragraph{Einschränkungen beim Teilen von Knoten}
    Das gerade beschriebene Teilen von Knoten ist nicht in jedem Fall gestattet.
    Es gibt drei Ausnahmen, die es zu behandeln gilt.

    Es kann der Fall eintreten, dass zwei identische Parent Features
    ein identisches Child Feature haben,
    welches innerhalb der Parents aber durch einen anderen Namen referenziert wird.
    In diesem Fall darf der Knoten des Parent Features nicht geteilt werden,
    was anhand eines Beispiels deutlich wird.
    Angenommen c5 und c3 existieren identisch auf den Seiten p und p',
    aber in p' ist der Name von c5 anders, nämlich c5'.
    Würde c3 wiederverwendet, existierten zwei Kanten von c3 zu c5,
    die sich nur im Namen des Features von c5 unterscheiden.
    Mit dem bisherigen Datenmodell wäre es nicht möglich festzustellen,
    welche Kante auf welcher Seite gültig ist.
    Anstelle die \gls{url} der Seite in den Kanten zu speichern,
    erzeugt der Algorithmus eine Kopie von c3,
    die eine Kante zu c5 besitzt und den richtigen Namen speichert.

    Durch die Änderung eines Features kann es außerdem passieren,
    dass ein zuvor geteilter Knoten nicht mehr geteilt werden darf.
    Zuvor wurde der Fall beschrieben, dass sich c1,
    also die Überschrift, auf zwei Seiten gleicht und c1 deshalb von beiden Seiten
    verwendet wird.
    Falls sich der Text der Überschrfit anschließend auf eine Seite ändert,
    ist c1\_text nicht mehr für beide Seiten zutreffend.
    Dieser Text-Knoten darf aber nicht angepasst werden,
    da er in dieser Form von der anderen Seite benötigt wird.
    Der Algorithmus legt deshalb eine Kopie von c1\_text und c1 an,
    was oben bereits beschrieben wurde.

    Der letzte Sonderfall tritt für ein Parent Feature ein,
    sobald sich irgendeins seiner Child Features verändert hat.
    Das heißt, es ist ein neues Child Feature hinzugekommen oder
    ein bestehendes wurde verändert bzw. gelöscht.
    Das betroffene Child Feature darf nicht angepasst werden,
    da es in der jetzigen Form noch von anderen Seiten benötigt werden könnte.
    Das Parent Feature muss in diesem Fall dupliziert werden,
    sodass die Kopie die Änderung am Child Feature beinhalten kann,
    ohne das Original zu beeinträchtigen.
    Auch dieser Fall lässt sich gut veranschaulichen.
    Angenommen die Seiten p und p' referenzieren beide c3 und teilen sich damit
    dieses Feature inklusive seiner Child Features.
    Ändert sich nun beispielsweise die Klasse von c5 auf p',
    ist das Vorgehen folgendermaßen:
    Nachdem festgestellt wurde, dass r1, r2 und c5\_text wiederverwendet werden können,
    wird ein neuer Knoten c5' angelegt, der die neue Klasse enthält
    und die zuvor genannten Knoten analog zu c5 referenziert.
    Anschließend erstellt der Algorithmus c3', eine exakte Kopie von c3,
    der ausgehende Kanten zu c4 und c5' erhält.
    Abschließend wird eine Kante von p' zu c3' erstellt und die Kante zu c3 dafür gelöscht.

    Durch diese letzte Prämisse ist auch der Fall abgedeckt,
    dass ein Feature gelöscht wurde.

    \paragraph{Abschließende Operationen}
    Der einzige offene Punkte der Liste am anfang des Kapitels ist die Änderung der Klasse der Seite.
    Dieser kann sehr einfach begegnet werden, indem die Klasse der Seite jedes mal neu gesetzt wird.
    Nach der Ausführung des beschriebenen Algorithmus kann die Datenbank
    Content- und {\resource}-Knoten enthalten, die keine eingehenden Knoten besitzen.
    Sie verletzen nicht die Konsistenz irgendeiner Klassifikation, da sie verwaist sind,
    sind aber trotzdem überflüssig.
    Aus diesem Grund setzt der Algorithmus abschließend eine Anweisung an die Datenbank,
    die Teilgraphen sucht, die von keiner Seite aus erreichbar sind, und löscht.
    Außerdem wird ein Knoten für die Site sowie eine Kante zur Seite sichergestellt.

    \paragraph{Datenbankanweisungen}
    Wie bereits erwähnt wurde, wird die oben beschriebene Logik nicht innerhalb
    des Services umgesetzt, sondern durch die Datenbankanweisungen, die es generiert.
    Deshalb sollen diese Anweisungen ebenfalls kurz erläutert werden.
    
    Eine Schlüsselrolle spielt das MERGE-Statement,
    welches Muster im Graphen sucht und bei Bedarf anlegt.
    %TODO: Referenz: https://neo4j.com/docs/developer-manual/3.3/cypher/clauses/merge/
    Die in Listing \ref{listing:storeClassificationContentReferenceFunctions} genutzten
    Funktionen textNode und resourceNode erzeugen deshalb simple Anweisungen:
    \texttt{MERGE (t:Text {value: \$t.value})} und \texttt{MERGE (r:Resource {url: \$r.destination})},
    wobei "`\$t"' und "`\$r"' Objekte sind, die der Datenbank inklusive der Anweisungen übergeben werden.
    Die Eigenschaften "`value"' bzw. "`url"' dienen also der eindeutigen Identifizierung von
    Text- und {\resource}-Knoten, was auch schon in Kapitel
    \ref{section:solutionDetailsPersistenceDataModel} erklärt wurde
    und weshalb diese Eigenschaft über Unique Property Constraints sichergestellt wird
    (siehe Kapitel \ref{section:solutionDetailsPersistenceDatabse}).

    Für Content Nodes wird die Anweisung
    \texttt{MERGE (c:Content {checksum: \$c.checksum, class: \$c.class})} erzeugt,
    wobei die Checksumme das eigentliche Kriterium der eindeutigen Identifizierung darstellt.
    Die Anweisung muss alle Child Features abdecken, damit im Falle einer Differenz eines Childs,
    auch die Beziehungen zu den restlichen Childs vom kopierten Parent ausgehend angelegt werden.
    Bei separaten Anweisungen für jedes Child Feature, wäre das nicht möglich.
    Die Checksumme wird deshalb über das gesamte Content Feature inklusive aller seiner Child Features gebildet.
    Andere Methoden, um dies zu erreichen, erweisen sich als zu komplex und ineffizient.
    Der Grund dafür ist die eindimensionale Natur eines MERGE-Statements.
    Dadurch ist es nicht möglich ein einzelnes Statement zu formulieren,
    dass n ausgehende Kanten zu n verschiedenen Knoten matcht.
    Mehrere Kanten und Knoten können nur über verallgemeinerte Statements gematcht werden,
    was für den vorliegenden Fall nicht ausreichend ist.
    Mit MERGE-Statements können aber auch Ketten der Form a-b-c formuliert werden,
    was auf den ersten Blick eine Alternative zur Prüfsumme darstellt.
    Nämlich dann, wenn neben der ausgehenden Kante vom Parent zum Child Feature,
    auch eine "`Rückkante"' vom Child zum Parent existieren würde.
    Dann könnte eine Kette der Form parent-child1-parent-child2-parent-child3 formuliert werden.
    Allerdings hilft das nicht im Falle eines gelöschten Child Features weiter,
    da dieses nicht in der Kette auftaucht, die aber trotzdem gefunden wird.
    Um diesen Fall abzufangen müssten die Child Features zusätzlich einen Ring bilden,
    wodurch allerdings ihre Position in der Kette relevant werden würde.
    Insgesamt ist dieser Ansatz nicht sinnvoll umsetzbar,
    weshalb die sich für die Prüfsumme entschieden wurde.

    Diese Prüfsumme und ihre Eindeutigkeit machen sich auch die Funktionen
    contentRelationship, referenceRelationship und textRelationship zunutze.
    Mit dieser Eigenschaft und der jeweiligen eindeutigen Eigenschaft des zweiten
    Knotens, können sie MATCH-Statements formulieren,
    die die beiden Knoten einer Beziehung finden.
    Anschließend wird zwischen diesen Knoten wiedrum ein MERGE generiert,
    der die gewünschte Beziehung zwischen ihnen beschreibt.
    Dies gilt analog auch für Seiten-Knoten, ihrer eindeutigen \gls{url}
    und ihre ausgehenden Beziehungen.
    Außerdem werden dadurch die oben beschriebenen Ausnahmen für das Teilen von Knoten sehr leicht händelbar.
