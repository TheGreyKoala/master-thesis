\subsection{Aktualisierung einer Klassifikation}
    \label{section:solutionDetailsClassificationStorageAPIUpdatePage}
    Beim Aktualisieren einer Klassifikation muss die {\classificationStorageAPI}
    einige einige Sonderfälle beachten,
    um zum Beispiel die Idempotenz der Operation sicherzustellen.
    Dieses Kapitel beschreibt die notwendigen Maßnahmen und verwendet
    zur Veranschaulichung das Beispiel
    aus Kapitel \ref{section:solutionDetailPersistenceDataModelExample}.

    \paragraph{Arten einer Aktualisierung}
    Zur Beschreibung des Vorgehens bei einer Aktualisierung einer Klassifikation,
    wird zunächst die Frage beantwortet, welche Arten einer Änderung
    die {\classificationStorageAPI} unterscheidet.
    Dabei handelt es sich um die folgenden Szenarien:

    \begin{enumerate}
        \item Ein neues {\contentFeature} wurde auf einer beliebigen Ebene hinzugefügt.
        \item Eine Referenz zu einer bisher unbekannten {\resource} wurde auf einer beliebigen Ebene hinzugefügt.
        \item Eine Referenz zu einer bereits bekannten {\resource} wurde auf einer beliebigen Ebene hinzugefügt.
        \item Ein Feature inklusive seiner Child Features wurde auf einer beliebigen Ebene gelöscht.
        \item Die Klasse einer Seite wurde verändert.
    \end{enumerate}

    Änderung an Features, wie das Setzen einer anderen Klasse,
    interpretiert der Service als Löschung eines Features und Hinzufügen eines neuen Features.
    Im weiteren Verlauf dieses Kapitels wird deutlich werden,
    warum er sich so verhält und wieso das bei Seiten nicht der Fall ist.

    \paragraph{Herausforderung}
    Bei jeder Änderung den gesamten Graphen zu löschen und neu anzulegen,
    ist in­ef­fi­zi­ent und verursacht viele Operationen auf Datenträgern.
    Stattdessen versucht die {\classificationStorageAPI} den vorhandenen Graphen anzupassen,
    sodass er die übermittelte Klassifikation korrekt wiederspiegelt.
    Die Herausforderung dabei ist, dass der Service
    die einzelnen Schritte von der alten zur neuen Version der Klassifikation nicht verfolgen konnte.
    Stattdessen erhält er das fertige Ergebnis und muss diese Schritte selbst rekonstruieren.

    \paragraph{Lösungsidee}
    Anstatt die Differenz zwischen der alten und der neuen Klassifikation zu ermitteln,
    um entsprechende Anpassungsanweisungen zu generieren,
    begegnet die {\classificationStorageAPI} dieser Herausforderung auf eine andere Art und Weise.
    Sie führt auch in diesem Fall den Algorithmus aus Kapitel \ref{section:solutionDetailsStorageAPIInitialWrite}
    aus und generiert dabei Datenbankanweisungen,
    die nur bei Bedarf Änderungen an Knoten und Kanten vornehmen oder diese neu anlegen.
    Teilgraphen gelöschter Features sind dadurch leicht zu erkennen,
    was im weiteren Verlauf ersichtlich wird.
    Durch diese Methode ist der Algorithmus unabhängig vom aktuellen Datenbankzustand.
    Zwei Beispiele sollen die Idee verdeutlichen.

    \paragraph{Beispiel 1 -- Eine unveränderte Klassifikation}
    Zunächst wird der Fall betrachtet, dass eine Klassifikation unverändert erneut in die Datenbank geschrieben wird.
    Im Detail wird dazu das Feature \texttt{c1} aus dem Beispiel in Kapitel
    \ref{section:solutionDetailPersistenceDataModelExample} betrachtet.
    Zunächst wird für diesen Knoten festgestellt,
    dass ein \texttt{Text}-Knoten mit dem passenden Inhalt existiert und wiederverwendet werden kann.
    Ebenso existiert bereits ein passender \texttt{Content}-Knoten und eine \texttt{Reads}-Beziehung (\texttt{rel4}) zwischen diesen Knoten.
    Am Ende stellt der Algorithmus fest, dass sowohl der \texttt{Page}-Knoten als auch
    die Beziehung zu \texttt{c1} nicht neu angelegt werden müssen.
    Insgesamt hat der Algorithmus also die gesamte Klassifikation abgearbeitet
    und hat keine Schreiboperationen in der Datenbank durchgeführt.
    Diese Logik wird nicht durch den Kontrollfluss im Service umgesetzt,
    sondern durch die Datenbankanweisungen, die durch die Funktionen
    \texttt{contentNode},
    \texttt{contentRelationship},
    \texttt{pageNode},
    \texttt{referenceRelationship},
    \texttt{resourceNode}
    \texttt{textNode} und
    \texttt{textRelationship} in den Listings
    \ref{listing:storeClassificationPageFlow} und
    \ref{listing:storeClassificationContentReferenceFunctions}
    erzeugt werden.

    \paragraph{Beispiel 2 -- Eine veränderte Klassifikation}
    Nun wird an einem zweiten Beispiel gezeigt,
    wie das Vorgehen bei einer Änderung ist.
    Dazu wird angenommen, dass sich die Klasse von \texttt{c1} verändert hat.
    Wie zuvor wird der \texttt{Text}-Knoten wiederverwendet,
    da sich der Inhalt des Features nicht verändert hat.
    Anders als zuvor wird aber kein passender \texttt{Content}-Knoten gefunden,
    weshalb ein neuer Knoten \texttt{c1'} erzeugt wird, der die neuen Informationen enthält.
    Genauso wird eine neue Kante \texttt{rel4'} ausgehend von \texttt{c1'} zu \texttt{c1\_text} erzeugt.
    Am Ende wird der \texttt{Page}-Knoten wiederverwendet.
    Allerdings existiert keine Kante von \texttt{p} zu \texttt{c1'}.
    Diese wird deshalb in Form von \texttt{rel1'} angelegt,
    wohingegen \texttt{rel1}, die zum veralteten \texttt{c1} führt, gelöscht wird.
    Der Knoten \texttt{c1} ist nun verwaist, da er von keinem \texttt{Page}-Knoten erreichbar ist.
    Er kann daher inklusive \texttt{rel4} gelöscht werden.
    Im Falle neuer Features verhält sich der Service analog.

    \paragraph{Konsequenz für den Graphen}
    Das bis hierhin beschriebene Verfahren hat bereits eine wichtige
    Konsequenz für den Graphen.
    Sowohl \texttt{Text}- als auch \texttt{Content}-Knoten können nämlich mehrere einkommende Kanten besitzen.
    Für \texttt{Text}-Knoten ist das der Fall, wenn mehrere {\contentFeature}s mit dem gleichen textuellen Inhalt existieren.
    Für \texttt{Content}-Knoten, wenn es mehrere komplett identische {\contentFeature}s gibt.
    Das soll wiederum am Beispiel von \texttt{c1} verdeutlicht werden,
    welches die Überschrift der Seite \texttt{p} mit dem textuellen Inhalt "`Einstieg"' darstellt.
    Angenommen es existiert eine zweite Seite \texttt{p'},
    die ebenfalls diese Überschrift besitzt und die identisch klassifiziert wird.
    Das beschriebene Verfahren legt dann keinen neuen \texttt{Content}-Knoten an,
    sondern lediglich eine Kante, die von \texttt{p'} zu \texttt{c1} führt.
    Da der eindeutige Selektor des Features an der Kante gespeichert wird, ist das ohne Weiteres möglich.
    Genauso dürfen der Name des Features oder die Eigenschaft \texttt{isCollection} anders sein.
    Teilgraphen können also auf beliebigen Ebenen von mehreren Klassifikationen geteilt werden.
    Eine Abzweigung findet dabei immer an dem Punkt statt,
    ab dem sich die klassifizierten Seiten unterscheiden.
    Das wird im weiteren Verlauf noch deutlicher gezeigt.    
    Diese Eigenschaft ist nützlich für identische Inhalte
    verschiedener Webseiten, wie Kopf- oder Fußbereiche.
    Diese brauchen dann nur einmalig gespeichert werden.
    Außerdem macht der Graph diese Überschneidungen explizit wodurch sie leicht
    ermittelt werden können.
    % TODO: Was ist mit Resource Knoten? Die doch auch. Steht das irgendwo?

    \paragraph{Einschränkungen beim Teilen von Knoten}
    Das gerade beschriebene Teilen von Knoten ist nicht in jedem Fall gestattet.
    Es gibt drei Ausnahmen, die es zu behandeln gilt.

    \begin{enumerate}
        \item Es kann der Fall eintreten, dass zwei Klassifikationen Features besitzen,
        die sich lediglich im Namen eines {\childFeature}s unterscheiden,
        welches ansonsten aber ebenfalls identisch ist.
        In diesem Fall darf der Knoten des {\parentFeature}s nicht geteilt werden,
        was anhand eines Beispiels deutlich wird.
        Angenommen \texttt{c3} existiert identisch auf den Seiten \texttt{p} und \texttt{p'}.
        Auf der zweiten Seite besitzt \texttt{c3} ebenfalls ein Feature,
        welches \texttt{c5} exakt gleich ist.
        Allerdings heißt es auf dieser Seite \texttt{c5'}.
        Eine Wiederverwendung von \texttt{c3} hätte in diesem Fall zur Folge,
        dass zwei Kanten zwischen \texttt{c3} und \texttt{c5} existieren würden,
        deren einziger Unterschied die Eigenschaft \texttt{name} wäre.
        Mit dem verwendeten Datenmodell wäre es nicht möglich festzustellen,
        welche Kante zu welcher Klassifikation gehört,
        d. h. welche Klassifikation das Feature \texttt{c5} und welche \texttt{c5'} nennt.
        Anstelle die jeweilige \gls{url} der Seite in den Kanten zu speichern,
        erzeugt der Algorithmus für \texttt{p'} eine Kopie von \texttt{c3}
        und eine Kante von diesem neuen Knoten zu \texttt{c5},
        welche den Namen \texttt{c5'} speichert.
        \item Durch die Änderung eines Features kann es außerdem passieren,
        dass ein zuvor geteilter Knoten nicht mehr geteilt werden darf.
        Oben wurde der Fall beschrieben, dass \texttt{c1}
        von zwei Klassifikationen verwendet wird.
        Falls sich der Text der Überschrfit auf \texttt{p'} ändert,
        ist \texttt{c1\_text} nicht mehr für beide Seiten zutreffend.
        Der Inhalt dieses \texttt{Text}-Knotens darf aber nicht angepasst werden,
        weil er für \texttt{p} weiterhin korrekt ist.
        Der Algorithmus legt deshalb eine Kopie von \texttt{c1\_text} und \texttt{c1} an
        und erzeugt entsprechende Verbindungen zwischen ihnen und zu \texttt{p'}.
        \item Der letzte Sonderfall tritt ein, wenn sich {\childFeature}s ändern,
        also wenn ein neues hinzukommt oder ein bestehendes verändert oder gelöscht wird.
        Sowohl der Knoten des {\parentFeature}s als auch der eines veränderten {\childFeature}s
        dürfen nicht angepasst werden, da sie in der jetzigen Form noch von anderen
        Klassifikationen benötigt werden könnten.
        Der Knoten des {\parentFeature}s muss in diesem Fall dupliziert werden,
        sodass die Änderung übernommen werden kann,
        ohne den jetzigen Stand für andere Klassifikationen zu beeinträchtigen.
        Auch dieser Fall lässt sich gut veranschaulichen.
        Angenommen die \texttt{Page}-Knoten \texttt{p} und \texttt{p'} referenzieren beide \texttt{c3} und teilen sich damit
        dieses Feature inklusive seiner {\childFeature}s.
        Ändert sich nun die Klasse von \texttt{c5} in \texttt{p'},
        darf \texttt{c5} nicht angepasst werden, weil sich seine Klasse in \texttt{p} nicht verändert hat.
        Das Vorgehen ist stattdessen folgendermaßen:
        Nachdem festgestellt wurde, dass \texttt{r1}, \texttt{r2} und \texttt{c5\_text} wiederverwendet werden können,
        wird ein neuer Knoten \texttt{c5'} angelegt, der eine Kopie von \texttt{c5} ist, aber die neue Klasse enthält.
        Die zuvor genannten Knoten referenziert \texttt{c5'} analog zu \texttt{c5}.
        Anschließend erstellt der Algorithmus \texttt{c3'}, eine exakte Kopie von \texttt{c3}.
        Für \texttt{c3'} werden außerdem Kanten zu \texttt{c4} und \texttt{c5'} angelegt.
        Abschließend wird eine Kante von \texttt{p'} zu \texttt{c3'} erstellt und die Kante zu \texttt{c3} dafür gelöscht.
        Jede Klassifikation enthält dann die jeweils korrekten Informationen.
        Wird ein Feature ergänzt oder entfernt, referenziert der kopierte Knoten des {\parentFeature}s
        entsprechend einen Knoten mehr oder weniger als das original.
    \end{enumerate}

    \paragraph{Abschließende Operationen}
    Von den zu Beginn dieses Kapitels beschriebenen Arten einer Aktualisierung wurde ein Fall bisher nicht betrachten.
    Das ist die Änderung der Klasse einer Seite.
    Anders als Features besitzen Seiten in der Datenbank eine global gültige Klasse,
    weshalb keine Kopie angelegt werden muss.
    Auch dann nicht, wenn die Seite Ziel verschiedener Referenzen ist,
    da die Klasse einer Referenz in einer Kante gespeichert wird.
    Stattdessen kann ein \texttt{Page}-Knoten direkt angepasst werden.
    Nach der Ausführung des beschriebenen Algorithmus kann die Datenbank
    Teilgraphen enthalten, die von keinem \texttt{Page}-Knoten erreicht werden können.
    Sie werden also von keiner Klassifikation mehr benötigt.
    Solche Bereiche verletzen nicht die Konsistenz irgendeiner Klassifikation,
    sind aber obsolet, weshalb sie automatisch gelöscht werden.
    Abschließend wird außerdem ein Knoten für die Site sowie eine Kante zum \texttt{Page}-Knoten sichergestellt.

    \paragraph{Datenbankanweisungen}
    Den beschriebenen Algorithmus realisiert die {\classificationStorageAPI}
    zu großen Teilen ausschließlich über Datenbankanweisungen,
    die an dieser Stelle deshalb ebenfalls kurz erläutert werden.
    Eine Schlüsselrolle spielt dabei Cyphers \texttt{MERGE}-Anweisung,
    die Muster im Graphen sucht und bei Bedarf anlegt
    \cite[Kapitel 3.3.16]{neo4j:documentation}.

    Die in Listing \ref{listing:storeClassificationContentReferenceFunctions} genutzten
    Funktionen \texttt{textNode} und \texttt{resourceNode} erzeugen simple Anweisungen,
    die nach einem \texttt{Text}-Knoten mit spezifischem Inhalt bzw.
    einem \texttt{Resource}-Knoten mit einer gewissen \gls{url} suchen,
    und bei Bedarf anlegen:

    \begin{lstlisting}[
        style=cypher,
        label=listing:solutionDetailsClassificationStorageAPIStatementsTextResource,
        caption=Datenbankanweisungen für \texttt{Text}- \texttt{Resource}-Knoten
    ]
MERGE (t:Text {value: $t.value}) // Merge a Text node
MERGE (r:Resource {url: $r.destination}) // Merge a Resource node
    \end{lstlisting}
    
    Die gesuchten Werte speichern die Objekte \verb+$t+ und \verb+$r+,
    die der Datenbank mit den Anweisungen genannt werden.

    Für jeden \texttt{Content}-Knoten wird die folgende Anweisung erzeugt,
    wobei die speziellen Informationen jedes Knotens im Objekt \verb+$c+
    abgelegt werden.

    \begin{lstlisting}[
        style=cypher,
        label=listing:solutionDetailsClassificationStorageAPIStatementsContent,
        caption=Datenbankanweisung für \texttt{Content}-Knoten
    ]
MERGE (c:Content {checksum: $c.checksum, class: $c.class})
    \end{lstlisting}

    Die Anweisung sucht nach Übereinstimmungen mit vorhandenen \texttt{Content}-Knoten,
    wobei sie sicherstellen muss, dass dieses auch die {\childFeature}s beinhaltet.
    Wie oben beschrieben muss sie nämlich auch bei einer Abweichung in dieser Menge einen neuen Knoten anlegen.
    Dazu wird im Algorithmus  eine Prüfsumme über das gesamte Feature inkl. seiner {\childFeature}s gebildet,
    die dann als Suchkriterium der \texttt{MERGE}-Anweisung dient.
    Ausgenommen von der Prüfsumme ist lediglich der eindeutige Selektor eines Features,
    da er in der eingehenden Kante des \texttt{Content}-Knotens gespeichert wird
    und deshalb nicht in allen Features übereinstimmen muss.

    Andere Methoden, um das mit der Prüfsumme verfolgte Ziel zu erreichen,
    sind zu komplex und ineffizient, um praktikabel zu sein.
    Der Grund dafür ist die eindimensionale Natur einer \texttt{MERGE}-Anweisung.
    Dadurch ist es nicht möglich ein einzelnes Statement zu formulieren,
    welches ausgehend von einem \texttt{Content}-Knoten für alle {\childFeature}s
    Kanten, Knoten und deren individuellen Eigenschaften umfasst.
    Das wäre im vorliegenden Fall aber notwendig.
    Allerdings können mit \texttt{Merge}-Anweisungen eindimensionale Pfade formuliert werden,
    was auf den ersten Blick eine Alternative zur Prüfsumme darstellt.
    Nämlich dann, wenn zwischen den Knoten eines Parent und eines {\childFeature}s
    zwei Kanten mit entgegengesetzten Richtungen existieren.
    Für das Beispiel aus Kapitel \ref{section:solutionDetailPersistenceDataModelExample} könnte dann
    die Kette \texttt{c3 -> c4 -> c3 -> c5 -> c3} formuliert werden,
    die alle für eine Übereinstimmung relevanten Informationen der Knoten und Kanten enthält.
    Allerdings können damit nicht alle Fälle abgedeckt werden.
    Wird zum Beispiel \texttt{c5} gelöscht, sähe die Kette folgendermaßen aus:
    \texttt{c3 -> c4 -> c3}.
    Die \texttt{MERGE}-Anweisung würde dieses Muster finden und keine Aktionen veranlassen,
    sodass die Löschung von \texttt{c5} im Graphen wiederspiegelt wird.
    Um diesen Fall abzufangen, müssten die Knoten der {\childFeature}s einen Ring bilden.
    Aus Sicht der \texttt{MERGE}-Answeisung wäre die Position der Knoten in diesem Ring
    für eine Übereinstimmung allerdings relevant.
    Insgesamt ist dieser Ansatz nicht sinnvoll umsetzbar,
    weshalb sich für die Einführung einer Prüfsumme entschieden wurde.
    Dadurch können kürzere Anweisungen generiert werden, die die Datenbank besser puffern
    und schneller ausführen kann.
    Außerdem besitzt der Graph weitaus weniger komplexe Strukturen.
    Auch die ersten beiden Ausnahmen beim Teilen von Knoten werden durch diesen Ansatz
    automatisch korrekt behandelt.

    Die Prüfsumme und ihre Eindeutigkeit machen sich auch die Funktionen
    \texttt{contentRelationship}, \texttt{referenceRelationship} und \texttt{textRelationship} zunutze.
    Zum Anlegen einer Referenz eines Features reichen bspw. folgende Anweisungen.

\begin{lstlisting}[
        style=cypher,
        label=listing:solutionDetailsClassificationStorageAPIReference,
        caption=Datenbankanweisungen für Referenzen
    ]
MATCH
    (c:Content {checksum: $c.checksum})
    (r:Resource {url: $r.destination})
MERGE (c)-[:References {
    name: $ref.name
    class: $ref.class
    isCollection: $ref.isCollection
    selectorType: $ref.selector.type
    startSelectorType: $ref.selector.startSelector.type
    startSelectorValue: $ref.selector.startSelector.value
    startSelectorOffset: $ref.selector.startSelector.offset
    endSelectorType: $ref.selector.endSelector.type
    endSelectorValue: $ref.selector.endSelector.value
    endSelectorOffset: $ref.selector.endSelector.offset
}]->(r)
    \end{lstlisting}

    Hier werden zunächst Start- und Endknoten über eine \texttt{MATCH}-Anweisung ermittelt.
    Für den \texttt{Resource}-Knoten geschieht dies über seine \gls{url},
    beim \texttt{Content}-Knoten kann die Prüfsumme genutzt werden.
    Über eine \texttt{MERGE}-Anweisung wird anschließend sichergestellt,
    dass eine Kante mit den entsprechenden Eigenschaften zwischen ihnen
    existiert\footnote{Ändern sich diese Informationen der Kante, wurde aufgrund der abweichenden
    Prüfsumme an dieser Stelle bereits ein neuer \texttt{Content}-Knoten angelegt,
    der noch keine Verbindung zum \texttt{Resource}-Knoten besitzt.
    Die alte Referenz ist dann Teil eines verwaisten Teilgraphs und wird gelöscht.}.
    Analog verfahren \texttt{contentRelationship} und \texttt{textRelationship}.

    Die Knoten und Kanten von Sites und Seiten werden über eine einfache \texttt{MERGE}-Anweisung erzeugt,
    die den gezeigten stark ähneln und deshalb hier nicht explizit erläutert werden.
